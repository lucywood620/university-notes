\documentclass{article}[18pt]
\usepackage{../../../../format}
\lhead{New Venture Creation}


\begin{document}
\begin{center}
\underline{\huge Significance and Place of Enterprise and Entrepreneurship}
\end{center}
\section{Place of the entrepreneur}
Entrepreneurs exist in both free market and government intervention markets. Entrepreneurship is one of the four factors of production: land, labour, capital and entrepreneurship.
\section{Theories}
\textbf{Say and Cantillon}: Organiser of factors of production and a catalyst for economic change\\
\textbf{Kirzner}: Spotter of opportunities "creative alertness"\\
\textbf{Schumpter}: The innovator, the hero\\
\textbf{Knight}: A risk taker - profit is reward for risk taking\\
\textbf{Casson:} The organiser of resources
\section{Other}
What happens to an entrepreneur if economic conditions are not right, will there be no entrepreneurship or will they change economic conditions to suit them?\\
\\
If entrepreneurs are not encouraged, they might find alternate ways to be entrepreneurial.\\
\\
Entrepreneurs are not always from leading classes, often from working classes. This happens in societies that do not value entrepreneurship. 
\section{Sociological understandings of entrepreneurs}
Sociological theories don't call the owner of a business the entrepreneur, things are thought of in households, so specified as a group. 
\section{Entrepreneurship today}
The entrepreneur is misunderstood, and sometimes ignored because of this. Over time state support has kicked in to promote entrepreneurship, giving grants and loans for entrepreneurs.\\
The power and influence of entrepreneurs also gives them lots of responsibilities.




\end{document}