\documentclass{article}[18pt]
\usepackage{../../../../../format}
\lhead{New Venture Creation}


\begin{document}
\begin{center}
\underline{\huge New Venture Creation - Chapter 2}
\end{center}

\section{Your Resources}
\begin{itemize}
\item Financial Capital
\begin{itemize}
\item May be limited
\item Using resources that you may not own is called bootstrapping
\item People only commit resources they can afford to lose
\item Find ways of using resources they do not own by partnering with others
\end{itemize}

\item Human Capital
\begin{itemize}
\item Education, training, managerial or industry experience
\item Vital as experience of industry brings a good source of business ideas. Also gives insight into problems that may be faced in business
\item Can increase credibility with financial backers
\end{itemize}

\item Social Capital
\begin{itemize}
\item Access to personal networks of friends and commercial contacts
\item This can be used to get information on new opportunities and threats
\item Can provide customers, workers or discounted office space
\end{itemize}
\end{itemize}
The more capital, of any kind, you bring to the business, the more likely you are to succeed
\section{Character Traits of entrepreneurs}
\begin{itemize}
\item Need for independence
\begin{itemize}
\item Entrepreneurs have a need to be their own boss, or an unwillingness to take orders
\item This may mean doing things differently, or being in challenging situations
\end{itemize}

\item Need for achievement
\begin{itemize}
\item Money is just a badge of success, validating their achievement
\item This does not mean that they are high achievers, but they do have a need for it
\end{itemize}

\item Internal locus of control
\begin{itemize}
\item This is a belief that they control their own destiny
\item They believe that their drive and determination will lead them to achieve the outcome they want
\item This is all driven by their self confidence in their ability to undertake a task
\item This can manifest itself in a desire to control everything and everyone around them
\item Because of this trait, entrepreneurs tend to be proactive, rather than reactive
\item Can be easily diverted by the latest market opportunity
\item Do things at twice the pace of others, work other people hard
\end{itemize}

\item Creativity, innovation and opportunism
\begin{itemize}
\item Creativity focused on commercial opportunities 
\item Spot an opportunity, then use creativity and innovation to exploit it
\end{itemize}

\item Acceptance of risk and uncertainty
\begin{itemize}
\item Willing to risk: money, reputation and personal standing
\item Try to avoid or minimise the risks taken
\item They have "inside information", real or imaginary, that reduces the risk and uncertainty in their minds
\item They never really believe that their business will fail and have complete faith that they will be able to influence the outcome
\end{itemize}
\end{itemize}
\section{Factors that influence the entrepreneurial character}
\begin{itemize}
\item Nationality
\begin{itemize}
\item Certain cultures promote entrepreneurial traits more than others
\item There are particular traits that vary by nationality:
\begin{itemize}
\item Individualism vs Collectivism - The degree to which people act as individuals, rather than groups
\item Power Distance - The degree of inequality among people that the community is willing to accept
\item Uncertainty avoidance - The degree to which people would like to avoid ambiguity and resolve uncertainty
\item Masculinity vs Femininity
\end{itemize}
\end{itemize}

\item Education
\begin{itemize}
\item There is a correlation between educational achievement and the probability of starting up in business
\end{itemize}

\item Age and Parenting
\begin{itemize}
\item Young to middle aged people are most likely to be associated with growth companies
\item Youth brings creativity. Age brings experience, knowledge and a larger network of contacts 
\end{itemize}

\item Immigration and ethnicity
\begin{itemize}
\item Immigration to a foreign country is positively associated with entrepreneurship
\item There are non uniform self employment rates for ethnic minorities in the UK. People from different countries have different rates of self employment
\end{itemize}

\item Gender
\begin{itemize}
\item Women are less likely to start a business than men
\item Statistics show that women owned businesses are likely to perform less well than male owned businesses
\end{itemize}
\end{itemize}
\section{How entrepreneurs manage}
\subsection{Relationships}
\begin{itemize}
\item Entrepreneurs are good at developing relationships with customers, staff, suppliers and all the stakeholders
\item \textbf{Relationship Marketing} - The ability to form loyal relationships with customers
\item Formality reduces flexibility so they manage informally
\item The ability to form strong personal relationships helps them to develop the partnerships and networks that are part of the social capital they create
\item The implication of all this is that entrepreneurs need strong \textbf{interpersonal skills}
\end{itemize}
\subsection{Strategy development}
\begin{itemize}
\item Entrepreneurs develop a \textbf{strong vision} of what they want their business to become
\item They have strong \textbf{strategic intent}
\item They use a loose or flexible plan with continuous strategising 
\item By creating more \textbf{strategic options} they improve their chances of successfully pursuing at least one opportunity and avoiding the most risks
\item They keep as many options open as possible for as long as possible
\end{itemize}
\subsection{Decision making}
\begin{itemize}
\item Entrepreneurs adopt an incremental approach to decision making. 
\item Only commit to costs after the opportunity has proved to be real
\item Often experiment with a limited launch to the market
\end{itemize}
\subsection{Risk Mitigation}
\begin{itemize}
\item Use knowledge and information coming from the network of close personal relationships they have developed
\item Use their network to form partnerships that help them spread the risk of the venture
\item Only commit limited resources
\item Compartmentalise risk by separating business ventures into separate legal entities
\end{itemize}


\end{document}