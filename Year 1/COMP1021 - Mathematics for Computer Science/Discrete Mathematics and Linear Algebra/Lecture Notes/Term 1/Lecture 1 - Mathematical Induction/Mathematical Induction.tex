\documentclass{article}[18pt]
\usepackage{../../../../../format}
\lhead{Mathematics for Computer Science - Andrei  Krokhin}


\begin{document}
\begin{center}
\underline{\huge Mathematical Induction}
\end{center}
\section{Proof By Induction}
Suppose we want to prove that the following statements are valid for every \textbf{positive integer} m:
\begin{itemize}
\item $n<2^n$
\item $1+2+..+n=n(n+1)/2$
\item $n^3-n$ is divisible by 3
\end{itemize}
Although these 3 cases look very different there is a \textbf{general approach} to prove such statements, called \textbf{proof by induction}
\subsection{The general principle of proof by induction}
Suppose we want to prove that a given statement $S(n)$ holds for all integers $n \geqslant j$ for a fixed integer j.\\
The simplest proof by induction works as follows:
\begin{itemize}
\item Step 1 (\textbf{Basis Step}): Check that S(n) is true for n=j (smallest possible value of n); If this is not the case, then the statement cannot be true. If S(j) is true, then proceed to Step 2
\item Step 2: (\textbf{Induction Step}): Prove the following \textbf{conditional} statement. If S(n) holds for a fixed value $n=k\geqslant j$ (\textbf{Induction Hypothesis or Assumption}) then it also holds for $n=k+1$
\end{itemize}
The two steps together then imply that $S(n)$ holds for $n=j$ (by Step 1), for $n=j+1$ (by Step 1 and Step 2 applied for k=j), for $n=j+2$ (by Step 2 applied for $k=j+1$, and so on, so it holds for all $n\geqslant j$
\subsection{Example 1: Proving $n< 2^n$ by induction}
\begin{enumerate}
\item Basis Step: Check that the statement is valid for n=1. It is, as $1<2$
\item Induction Step: Let $k\geqslant 1$ \begin{itemize}
\item Assume that the statement holds for $n=k$, that is, $k<s^k$
\item Need to use that to derive the statement for $n=k+1$, that is $k+1<2^{k+1}$
\item We have\\ $k+1<2^k+1<2^k+2^k=2^k+1$\\ The first inequality above is by the inductive assumption
\end{itemize}
\item Since the statement is \textbf{valid for n=1}, and that is \textbf{valid for n=k+1 if it is valid for n=k}, we conclude that it is \textbf{valid for all positive integers n}
\end{enumerate}
\subsection{Example 2: Proving that $n^3-n$ is divisible by 3}
\begin{enumerate}
\item Basis Step: Check that the statement is valid for $n=1$. It is as $1^3-1=0$ is divisible by 3
\item Induction Step: Let $k\geqslant 1$ be an integer\begin{itemize}
\item Assume that the statement holds for $n=k$, that is $k^3-k=3m$ for some integer $m\geqslant 0$
\item Need to use that to derive the statement for $n=k+1$, that is, $(k+1)^3-(k+1)=3m'$ for some integer $m'\geqslant0$
\item We have\\ $(k+1)^3-(k+1)=k^3+3k^2+3k+1-(k+1)=k^3-k+3k^2+3k=3(m+k^2+k)$ \\ The m appears due to replacing some of the terms in k with the assumption $k^3-k=3m$
\end{itemize}
\item Since both the basis and inductive step are completed, we conclude that the statement is valid for all positive integers $n$
\end{enumerate}
\subsection{Example 3: Proving that $1+2+...+n=(n+1)n/2$}
\begin{enumerate}
\item Basis Step: Check that the statement is valid for $n=1$. It is, as $1=2\cdot1/2$
\item Induction Step: Let $k\geqslant1$ be an integer\begin{itemize}
\item Assume that the statement holds for $n=k$, that is\\ $1+2+...+k=(k+1)k/2$
\item Need to use that to derive the statement for $n=k+1$, that is\\ $1+2+...+k+(k+1)=(k+2)(k+1)/2$
\item We have\\ $1+2+...+k+(k+1)=(k+1)k/2)+(k+1)=(k+2)(k+1)/2$ 
\end{itemize}
\item Since we know that the statement is valid for $n=1$, and that it is valid for $n=k+1$, if it is valid for $n=k$, we conclude that it is valid for all positive integers $n$
\end{enumerate}

\subsection{Variations}
\begin{itemize}
\item Sometimes a statement is valid only for $n\geqslant j$. Then the Basis Step is about $S(j)$, rather than just 1
\item Sometimes in the basis step we have to check a number of small cases, not just $n=j$
\item Sometimes in Induction Step we have to assume that $S(n)$ holds for all $n\leqslant k$, not just for $n=k$ (Strong Induction in Textbook)
\end{itemize}
\section{Geometry}
\subsection{Some Geometry}
\begin{itemize}
\item A \textbf{polygon} is a closed geometric figure formed by a series of line segments $s_1...,s_n$ called sides
\item Two consecutive sides share an endpoint, as do $s_1$ and $s_n$. The endpoints are called \textbf{vertices}
\item A polygon is \textbf{simple} if no two non-consecutive sides intersect
\item Each polygon divides the plane into two regions: \textbf{interior} and \textbf{exterior}
\item A polygon is \textbf{convex} if each line segment between two interior points is within the interior. Line segment joining points doesn't go outside shape
\item A \textbf{diagonal} is a line segment connecting two non-consecutive vertices
\item An \textbf{interior diagonal} is one that lies entirely inside the polygon\begin{itemize}
\item Lemma: Every simple polygon has an interior diagonal 
\end{itemize} 
\end{itemize}
\subsection{Triangulation in Computational Geometry}
\begin{itemize}
\item \textbf{Triangulation} is the process of dividing a simple polygon into triangles by adding non-intersecting diagonals
\item In order to computationally process a complicated surface, it is divided into simple polygons, which are then triangulated
\item Delaunay triangulations are especially popular

\end{itemize}
\subsection{Triangulation Proof By Strong Induction}
\textbf{Theorem}: Each simple polygon with $n\geqslant 3$ sides can be triangulated into $n-2$ triangles
\begin{enumerate}
\item Basis Step: The theorem trivially holds for $n=3$
\item Induction step: let $k\geqslant3$ be any integer\begin{itemize}
\item Assume that the theorem is true for all $n\leqslant k$
\item Take a simple polygon P with $n=k+1$ sides
\item By Lemma above, it has an interior diagonal
\item The diagonal splits P into two simple polygons. Q with s sides and R with t sides
\item We have $3\leqslant s\leqslant k$ and $3\leqslant t\leqslant k$. Moreover $k+1=s+t-2$. Using base assumption Q and R can also be split using a diagonal
\item By assumption Q and R can be triangulated into $s-2$ and $t-2$ triangles, respectively. Assume the theorem is true
\item This gives triangulation of P into $(s-2)+(t-2)=k-1=n-2$ triangles
\end{itemize}
\end{enumerate}
\subsection{Eye proof by induction}
Doesn't work for n=2







\end{document}