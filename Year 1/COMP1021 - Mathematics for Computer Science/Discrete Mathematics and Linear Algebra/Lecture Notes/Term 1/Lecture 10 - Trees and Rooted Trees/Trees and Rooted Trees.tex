\documentclass{article}[18pt]
\ProvidesPackage{format}
%Page setup
\usepackage[utf8]{inputenc}
\usepackage[margin=0.7in]{geometry}
\usepackage{parselines} 
\usepackage[english]{babel}
\usepackage{fancyhdr}
\usepackage{titlesec}
\hyphenpenalty=10000

\pagestyle{fancy}
\fancyhf{}
\rhead{Sam Robbins}
\rfoot{Page \thepage}

%Characters
\usepackage{amsmath}
\usepackage{amssymb}
\usepackage{gensymb}
\newcommand{\R}{\mathbb{R}}

%Diagrams
\usepackage{pgfplots}
\usepackage{graphicx}
\usepackage{tabularx}
\usepackage{relsize}
\pgfplotsset{width=10cm,compat=1.9}
\usepackage{float}

%Length Setting
\titlespacing\section{0pt}{14pt plus 4pt minus 2pt}{0pt plus 2pt minus 2pt}
\newlength\tindent
\setlength{\tindent}{\parindent}
\setlength{\parindent}{0pt}
\renewcommand{\indent}{\hspace*{\tindent}}

%Programming Font
\usepackage{courier}
\usepackage{listings}
\usepackage{pxfonts}

%Lists
\usepackage{enumerate}
\usepackage{enumitem}

% Networks Macro
\usepackage{tikz}


% Commands for files converted using pandoc
\providecommand{\tightlist}{%
	\setlength{\itemsep}{0pt}\setlength{\parskip}{0pt}}
\usepackage{hyperref}

% Get nice commands for floor and ceil
\usepackage{mathtools}
\DeclarePairedDelimiter{\ceil}{\lceil}{\rceil}
\DeclarePairedDelimiter{\floor}{\lfloor}{\rfloor}

% Allow itemize to go up to 20 levels deep (just change the number if you need more you madman)
\usepackage{enumitem}
\setlistdepth{20}
\renewlist{itemize}{itemize}{20}

% initially, use dots for all levels
\setlist[itemize]{label=$\cdot$}

% customize the first 3 levels
\setlist[itemize,1]{label=\textbullet}
\setlist[itemize,2]{label=--}
\setlist[itemize,3]{label=*}

% Definition and Important Stuff
% Important stuff
\usepackage[framemethod=TikZ]{mdframed}

\newcounter{theo}[section]\setcounter{theo}{0}
\renewcommand{\thetheo}{\arabic{section}.\arabic{theo}}
\newenvironment{important}[1][]{%
	\refstepcounter{theo}%
	\ifstrempty{#1}%
	{\mdfsetup{%
			frametitle={%
				\tikz[baseline=(current bounding box.east),outer sep=0pt]
				\node[anchor=east,rectangle,fill=red!50]
				{\strut Important};}}
	}%
	{\mdfsetup{%
			frametitle={%
				\tikz[baseline=(current bounding box.east),outer sep=0pt]
				\node[anchor=east,rectangle,fill=red!50]
				{\strut Important:~#1};}}%
	}%
	\mdfsetup{innertopmargin=10pt,linecolor=red!50,%
		linewidth=2pt,topline=true,%
		frametitleaboveskip=\dimexpr-\ht\strutbox\relax
	}
	\begin{mdframed}[]\relax%
		\centering
		}{\end{mdframed}}



\newcounter{lem}[section]\setcounter{lem}{0}
\renewcommand{\thelem}{\arabic{section}.\arabic{lem}}
\newenvironment{defin}[1][]{%
	\refstepcounter{lem}%
	\ifstrempty{#1}%
	{\mdfsetup{%
			frametitle={%
				\tikz[baseline=(current bounding box.east),outer sep=0pt]
				\node[anchor=east,rectangle,fill=blue!20]
				{\strut Definition};}}
	}%
	{\mdfsetup{%
			frametitle={%
				\tikz[baseline=(current bounding box.east),outer sep=0pt]
				\node[anchor=east,rectangle,fill=blue!20]
				{\strut Definition:~#1};}}%
	}%
	\mdfsetup{innertopmargin=10pt,linecolor=blue!20,%
		linewidth=2pt,topline=true,%
		frametitleaboveskip=\dimexpr-\ht\strutbox\relax
	}
	\begin{mdframed}[]\relax%
		\centering
		}{\end{mdframed}}
\lhead{MCS - DMLA}


\begin{document}
\begin{center}
\underline{\huge Trees and Rooted Trees}
\end{center}
\section{The number of leaves in a tree}
What is the minimum number of leaves in a tree with at least 2 vertices
\subsection{Lemma}
A tree with at least 2 vertices, $n_3$ of which have degree at least 3, has at least $n_3+2$ leaves
\subsection{Proof}
Let T be a tree on $n\geqslant$ 2 vertices. We use induction on n
\begin{itemize}
	\item Let $l(T)$ denote the number of leaves in T, and $n_3(T)$ denote the number of vertices of degree at least 3 in T
	\item Induction base: If n=2, then $n_3$=0 and T has 2 leaves
	\item Step: Now suppose that every tree on $< n$ vertices has at least $n_3+2$ leaves (induction hypothesis), and consider a tree T on $n\geqslant 3$ vertices
	\item Since T is a tree on at least 3 vertices, T has a leaf u
	\item Then T'=T-u is a tree on n-1 vertices. By the induction hypothesis we have $l(T')\geqslant n_3(T')+2$
	\item We have: a leaf u in T, a tree T'=T-u, $l(T')\geqslant n_3(T')+2$
	\item Let v be the (unique) neighbour of u in T
	\item T is connected and has at least 3 vertices, so v has at least 2 neightbours in T
	\item The rest of the proof is by \textbf{case analysis}
\end{itemize}
\begin{enumerate}
	\item Suppose that v has exactly 2 neighbours in T
	\begin{itemize}
		\item Then $n _ { 3 } \left( T ^ { \prime } \right) = n _ { 3 } ( T ) \text { and } \ell \left( T ^ { \prime } \right) = \ell ( T )$
		\item Hence, $\ell ( T ) = \ell \left( T ^ { \prime } \right) \geq n _ { 3 } \left( T ^ { \prime } \right) + 2 = n _ { 3 } ( T ) + 2$
	\end{itemize}
	\item Suppose that v has exactly 3 neighbours in T
	\begin{itemize}
		\item Then $n _ { 3 } \left( T ^ { \prime } \right) = n _ { 3 } ( T ) - 1 \text { and } \ell \left( T ^ { \prime } \right) = \ell ( T ) - 1$
		\item Hence, $\ell ( T ) = \ell \left( T ^ { \prime } \right) + 1 \geq n _ { 3 } \left( T ^ { \prime } \right) + 2 + 1 = n _ { 3 } ( T ) - 1 + 2 + 1 = n _ { 3 } ( T ) + 2$
	\end{itemize}
	\item Suppose that v has at least four neighbours in T
	\begin{itemize}
		\item Then, $n _ { 3 } ( T ) = n _ { 3 } \left( T ^ { \prime } \right) \text { and } \ell \left( T ^ { \prime } \right) = \ell ( T ) - 1$
		\item Hence, $\ell ( T ) = \ell \left( T ^ { \prime } \right) + 1 \geq n _ { 3 } \left( T ^ { \prime } \right) + 2 + 1 = n _ { 3 } ( T ) + 2 + 1 \geq n _ { 3 } ( T ) + 2$
	\end{itemize}
\end{enumerate}
This finishes the proof
\section{Every tree is a bipartite graph}
\subsection{Theorem}
Every tree is a bipartite graph
\subsection{Proof}
We give a \textbf{direct} proof. We can use the known result on unique paths in a tree T to define a bipartition of its vertex set V(T)
\begin{itemize}
	\item Choose any vertex v and put this vertex in the set $V_1$
	\item For every vertex $u\neq v$, there is a unique path from v to u in T, consider the length of this path
	\item If the length is odd, put u in $V_2$, otherwise put u in $V_1$
	\item We have to show that this is a valid bipartition
	\item $V_1$ and $V_2$ are disjoint and together make up $V(T)$
	\item Every edge has end vertices in both $V_1$ and $V_2$
	\item This completes the proof
\end{itemize}
\section{How to find and write down proofs?}
These are the questions to ask yourself to help finding a possible proof approach:
\begin{itemize}
	\item What do I have to \textbf{prove}? Is it one statement, or several; is it an implication or an equivalence; can I repharase it; does it resemble other statements?
	\item What do I \textbf{know}? What are the assumptions; do I know the relevant definitions; is there any known theory related to the statement
	\item Can I get more \textbf{insight}? Can I sketch the situation, the assumptions, the question; are there special (small) cases to check; can I break it into several subcases?
	\item How to \textbf{approach/attack} the question? Can I use induction; does a direct proof have any chance; or does it help to use contraposition, or a proof by contradiction?
	\item Is my solution \textbf{valid and convincing}? Write a draft first; check all the steps; critically examine the steps for errors or counterexamples; modify and revise the solution and write it down in a clear way
\end{itemize}
\subsection{The start: write down what you see}
We will consider the process of finding the proof on the following example:
\subsubsection{Lemma}
Let T be a tree on $n\geqslant 2$ vertices, and let $e\in E(T)$. The T-e is a forest consisting of precisely two trees.
\subsubsection{Proof}
\begin{itemize}
	\item Clearly, you have to know what a \textbf{tree} is, what a \textbf{forest} is, and what the \textbf{notations} $e\in E(T)$ and $T-e$ mean
	\item In fact, you have to prove \textbf{two}(or perhaps even \textbf{three}) statements: T-e is a forest, and this forest consists of precisely two trees (so not $\leqslant$ 1 and not $\geqslant 3$ trees)
	\item Here it (probably) helps to \textbf{draw a picture} that roughly sketches the situation and concepts
	\item If you draw the general situation, and know the definitions and notations, then you more or less \textbf{see the solution} in the picture
	\item The question is \textbf{how to write it down} (and check that the picture did not fool you)
	\item This requires certain \textbf{skills and experience}
	\item You can only learn this by \textbf{doing it yourself}
	\item A \textbf{tree} is a connected graph without cycles
	\item A forest is a graph without cycles
	\item Sine a tree is a connected graph, between any two vertices there is a path in a tree
	\item We know from the previous lecture that this path is \textbf{unique}
\end{itemize}
How to use (some of) the above facts to prove that T-e is a forest containing precisely two trees?\\
Let us consider the first part of the statement first. Can we prove that T-e is a forest\\
\\
There is an easy consequence of the definitions and so the observation that removing edges from a tree, we cannot introduce cycles. So if T is a tree, then T contains no cycles and T-e contains no cycles either, so T-e is a forest (This is a \textbf{direct proof})\\
\\
It remains to show that T-e consists of precisely 2 trees, i.e., at least 2 and at most 2 trees. How to prove this?\\
\\
At least 2: you have to show that T-e is not connected (not 1 tree). This is easy: if u and v are the end vertices of the edge e, then in T-e there is no path between u and v (This is also a direct proof)\\
\\
At most 2: you have to show that T-e does not consist of 3 or more trees. This is easy, using the observation that the edge e can only connect 2 trees into one. So, if T-e would consist of 3 or more trees, then T is not connected, a contradiction. (This is a proof by contradiction or contraposition)\\
\\
The proof seems to be complete. Now you have to write it down and \textbf{carefully check} the details
\subsection{A solution}
\subsubsection{Proof}
Since T is a tree, T - e has no cycles, so T - e is a forest. Since in T - e there
is no path between the two end vertices of e, T - e is not connected, hence T - e
consists of at least 2 trees.If T - e consists of at least 3 trees, then T cannot be
connected. Hence T - e is a forest consisting of precisely two trees.\\
\\
There are probably many different correct ways to prove the lemma. For instance for the last part you could use the fact that a tree on n vertices has n-1 edges.\\
\\
So suppose that T-e consists of trees of $n_1,...,n_k$ vertices for some integer $k\geqslant 1$. Now we count the number of edges of T in two ways: As T has $n_1+...+n_k$ vertices, T has $n_1+...+n_k-1$ edges. On the other hand, T has $(n_1-1)+...+(n_k-1)+1$ edges. The two expressions can only be equal if k=2, so T-e consists of precisely 2 trees
\section{Full m-ary trees}
\subsection{Definitions}
A rooted tree is called a \textbf{m-ary tree} if each vertex has at most m children. It is a \textbf{full} m-ary tree if each internal vertex has exactly m children\\
\\
A (full) 2-ary tree is usually called a (full) \textbf{binary tree}\\
\\
Often, the children of each node are assumed to be orderred
\subsection{Lemma} 
A full m-ary tree with i internal nodes has $n=n\cdot i+1$ vertices
\subsection{Proof}
Every node except the root is one of m children of a unique internal vertex\\
\\
Let $\ell$ be the number of leaves in a full m-ary tree. Since $n=i+\ell$ and $n=m\cdot i+1$, if we know any of $n,i,\ell$ then we can find all of them
\section{The height of a rooted tree}
\subsection{Definitions}
In a rooted tree, the \textbf{level} of a vertex u is the length of the (unique) path from the root to u. (The level of the root is 0)\\
The \textbf{height} of a rooted tree is the maximum level of a vertex in it
\subsection{Theorem}
There are at most $m^h$ leaves in a m-ary tree of height h
\subsection{Proof}
Induction on the height h
\begin{itemize}
	\item Base: If h=1 then the claim is obvious
	\item Step: Assume the claim is true for m-ary trees of height at most h-1
	\item Take an m-ary tree T of height $h\geqslant 2$, with root r
	\item Consider the subtrees of T rooted at children r
	\item There are at most m of them, and, by induction hypothesis, each has at most $m^{h-1}$ leaves
	\item Hence, T has at most $m\cdot m^{h-1}=m^h$ leaves
\end{itemize}
\section{Balanced m-ary trees}
\subsection{Definition}
An m-ary tree of height h is \textbf{balanced} if all leaves in it have height h-1 or h
\subsection{Theorem}
If an m-ary tree of height h has $\ell$ leaves then $h\geqslant \lceil log_m\ell$\\
If the tree is full and balances then $h=\lceil \log_m\ell\rceil$
\subsection{Proof}
\begin{itemize}
	\item The first part immediately follows from the previous theorem: We know that $\ell \leqslant m^h$, so $h\geqslant \log_m\ell$. Since h is an integer, $h\geqslant \lceil \log_m\ell \rceil$
	\item For the second part, note that there is at least one leaf of level h
	\item It follows that there are at least $m^{h-1}$ leaves
	\item So, we have $m^{h-1}<\ell\leqslant m^h$, or taking logarithm to the base m, $h-1<\log_m\ell \leqslant h$
	\item Since h is an integer, $h=\lceil \log_m\ell \rceil$
\end{itemize}
\section{Constructing trees}
Every tree $T\neq K_1$ has a leaf. We know that T-v is also a tree. This shows that T can be constructed from a smaller tree T'=T-v by adding a vertex to T' and joining it by one edge to a vertex in T'. This also proves the following statement
\subsection{Lemma}
We can construct all different trees on $n\geqslant 2$ vertices from all trees on n-1 vertices, by adding one vertex and joining it by one edge to a vertex in one of the trees, in all possible ways, and deleting multiple copies of the same trees
\begin{itemize}
	\item We can use the above result and procedure to obtain all different trees on n vertices, starting with $K_1$ (or we can give a \textbf{recursive definition} for the class of all trees)
	\item Check that there are, respectively, 1,1,1,2,3 and 6 different trees on 1,2,3,4,5 and 6 vertices
\end{itemize}
\end{document}