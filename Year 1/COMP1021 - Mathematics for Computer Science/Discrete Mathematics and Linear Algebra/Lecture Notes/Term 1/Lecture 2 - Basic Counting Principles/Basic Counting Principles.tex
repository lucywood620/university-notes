\documentclass{article}[18pt]
\ProvidesPackage{format}
%Page setup
\usepackage[utf8]{inputenc}
\usepackage[margin=0.7in]{geometry}
\usepackage{parselines} 
\usepackage[english]{babel}
\usepackage{fancyhdr}
\usepackage{titlesec}
\hyphenpenalty=10000

\pagestyle{fancy}
\fancyhf{}
\rhead{Sam Robbins}
\rfoot{Page \thepage}

%Characters
\usepackage{amsmath}
\usepackage{amssymb}
\usepackage{gensymb}
\newcommand{\R}{\mathbb{R}}

%Diagrams
\usepackage{pgfplots}
\usepackage{graphicx}
\usepackage{tabularx}
\usepackage{relsize}
\pgfplotsset{width=10cm,compat=1.9}
\usepackage{float}

%Length Setting
\titlespacing\section{0pt}{14pt plus 4pt minus 2pt}{0pt plus 2pt minus 2pt}
\newlength\tindent
\setlength{\tindent}{\parindent}
\setlength{\parindent}{0pt}
\renewcommand{\indent}{\hspace*{\tindent}}

%Programming Font
\usepackage{courier}
\usepackage{listings}
\usepackage{pxfonts}

%Lists
\usepackage{enumerate}
\usepackage{enumitem}

% Networks Macro
\usepackage{tikz}


% Commands for files converted using pandoc
\providecommand{\tightlist}{%
	\setlength{\itemsep}{0pt}\setlength{\parskip}{0pt}}
\usepackage{hyperref}

% Get nice commands for floor and ceil
\usepackage{mathtools}
\DeclarePairedDelimiter{\ceil}{\lceil}{\rceil}
\DeclarePairedDelimiter{\floor}{\lfloor}{\rfloor}

% Allow itemize to go up to 20 levels deep (just change the number if you need more you madman)
\usepackage{enumitem}
\setlistdepth{20}
\renewlist{itemize}{itemize}{20}

% initially, use dots for all levels
\setlist[itemize]{label=$\cdot$}

% customize the first 3 levels
\setlist[itemize,1]{label=\textbullet}
\setlist[itemize,2]{label=--}
\setlist[itemize,3]{label=*}

% Definition and Important Stuff
% Important stuff
\usepackage[framemethod=TikZ]{mdframed}

\newcounter{theo}[section]\setcounter{theo}{0}
\renewcommand{\thetheo}{\arabic{section}.\arabic{theo}}
\newenvironment{important}[1][]{%
	\refstepcounter{theo}%
	\ifstrempty{#1}%
	{\mdfsetup{%
			frametitle={%
				\tikz[baseline=(current bounding box.east),outer sep=0pt]
				\node[anchor=east,rectangle,fill=red!50]
				{\strut Important};}}
	}%
	{\mdfsetup{%
			frametitle={%
				\tikz[baseline=(current bounding box.east),outer sep=0pt]
				\node[anchor=east,rectangle,fill=red!50]
				{\strut Important:~#1};}}%
	}%
	\mdfsetup{innertopmargin=10pt,linecolor=red!50,%
		linewidth=2pt,topline=true,%
		frametitleaboveskip=\dimexpr-\ht\strutbox\relax
	}
	\begin{mdframed}[]\relax%
		\centering
		}{\end{mdframed}}



\newcounter{lem}[section]\setcounter{lem}{0}
\renewcommand{\thelem}{\arabic{section}.\arabic{lem}}
\newenvironment{defin}[1][]{%
	\refstepcounter{lem}%
	\ifstrempty{#1}%
	{\mdfsetup{%
			frametitle={%
				\tikz[baseline=(current bounding box.east),outer sep=0pt]
				\node[anchor=east,rectangle,fill=blue!20]
				{\strut Definition};}}
	}%
	{\mdfsetup{%
			frametitle={%
				\tikz[baseline=(current bounding box.east),outer sep=0pt]
				\node[anchor=east,rectangle,fill=blue!20]
				{\strut Definition:~#1};}}%
	}%
	\mdfsetup{innertopmargin=10pt,linecolor=blue!20,%
		linewidth=2pt,topline=true,%
		frametitleaboveskip=\dimexpr-\ht\strutbox\relax
	}
	\begin{mdframed}[]\relax%
		\centering
		}{\end{mdframed}}
\lhead{MCS - Discrete Maths and Linear Algebra}


\begin{document}
\begin{center}
\underline{\huge Basic Counting Principles}
\end{center}
\section{The Product Rule}
\subsection{Definition}
Suppose that a procedure can be broken down into a sequence of two tasks. If there are $n_1$ ways to do the first task and for each of these ways of doing the first task, there are $n_2$ ways to do the second task. Then there are $n_1\times n_2$ ways to do the procedure.
\subsection{Example}
\textit{How many different passwords can be constructed using two (or k) symbols from a set of N distinct symbols}\\
\\
For each of the N choices of the first symbol there are again N choices for the second symbol, so the answer is $N\times N$. For passwords consisting of k symbols the answer is $N\times N\times ...\times N=N^k$\\
\\
If we do not allow repetitions of symbols, the answer is $N\times(N-1)$, respectively $N\times (N-1)\times ...\times (N-k+1)$
\section{The sum rule}
\subsection{Definition}
If a task can be done either in one of $n_1$ ways or in one of $n_2$ ways, where none of the set of $n_1$ ways is the same as any of the set of $n_2$ ways, then there are $n_1+n_2$ ways to do the task
\subsection{Example}
\textit{Suppose we have a set of N characters and a set of M integers. In how many ways can we choose one symbol which is either a character or an integer?}\\
\\
It is clear that we can choose the symbol in $N+M$ ways\\
\\
In how many ways can we construct a sequence of three symbols where the first one is a character, the second one an integer, and the third one either of them?\\
\\
A combination of the product and sum rules gives the answer: $N\times M\times (N+M)$
\section{Example: Counting IP addresses}
\begin{itemize}
\item The internet is made up of interconnected physical networks of computers
\item Each computer (actually, each network connection of a computer) is assigned to an Internet address
\item Version 4 of the Internet Protocol is still in use

\begin{itemize}
\item An address in IPv4 is a string of 32 bits (looks like 172,16.254.1 in decimal)
\item It consists of \textbf{netid} (network number) and \textbf{hostid} (host number)
\item There are three classes of addresses: class A,B and C
\item Class A is for large networks, B for medium sized and C for small
\item Actually, there are also classes E and D, but for a separate purpose 
\end{itemize}
\item Class A address \textcolor{magenta}{0} \textcolor{blue}{7-bit netid } \textcolor{red}{24-bit hostid} (Short netid as small number of large networks, but lots of hosts on those networks)
\begin{itemize}
\item Technical restriction: Class A netid cannot be 111111
\end{itemize}
\item Class B address \textcolor{magenta}{10} \textcolor{blue}{14-bit netid }\textcolor{red}{16-bit hostid} 
\item Class C address \textcolor{magenta}{110} \textcolor{blue}{21-bit netid }\textcolor{red}{8-bit hostid} 
\begin{itemize}
\item Technical restriction: hostid in any class cannot be all 0 or 1
\end{itemize}
\end{itemize}
\newpage
\textit{How many different IPv4 addresses are available for a computer on the internet}\\
Let x be the number we want to compute
\begin{itemize}
\item Let $x_a$ be the number of class A addresses,
\item Let $n_a$ be the number of class A netids, and
\item Let $h_a$ be the number of class A hostids;
\item define $x_b, n_b, h_b$ and $x_c,n_c,h_c$ similarly
\end{itemize}
$ $
\begin{itemize}
\item By the sum rule $x=x_a+x_b+x_c$
\item By the product rule, $x_a=n_a\times h_a$
\begin{itemize}
\item By the product rule, $n_a=2^7-1=127$ (since 111111 is not available)
\item By the product rule, $h_a=2^{24}-2=16,777,124$ (since can't have all 0s or all 1s)
\item Hence $x_a=127\times 16,777,124=2,130,706,178$
\end{itemize}
\item Similarly $x_b=n_b\cdot h_b=2^{14}\cdot (2^16-2)=1,073,709,056$
\item Also, $x_c=n_c\cdot h_c=2^{21}\cdot(2^8-2)=532,676,608$
\item All in all, $x=x_a+x_b+x_c=3,737,091,842$ - this is a small number
\end{itemize}
IPv4 addresses are exhausted now and 128-bit IPv6 addresses are now in use
\section{Factorial Function}
\subsection{Definition}
The factorial of an integer $n\geqslant0$, denoted $n!$, is defined by:\\
$$0!=1$$
$$n!=1\cdot2\cdot...\cdot(n-1)\cdot n \quad n\geqslant 1$$
\subsection{Example}
\textit{How many different passwords of length 8 can we construct with the letters A,b,c,D,E,f,g,h if each symbol occurs exactly once?}\\
\\
The number is $8!$ (for the first symbol we have 8 possibilities; for each of these choices there are 7 possibilities for the second symbol, etc)
\subsection{Example 2}
The factorial function $n!$ grows extremely fast with increasing $n$\\
\\
\textit{If $n!>$ the age of the universe in seconds, how large should n be?}\\
n=20 is enough, $20!=2.43\cdot 10^{18}>age\approx 4.32\cdot 10^{17}$ sec
\section{Permutations}
\subsection{Definition}
The \textbf{permutation} of a set of distinct objects is an \textbf{ordered} arrangement of these objects\\

\subsection{r-Permutations}
\subsubsection{Definition}
An \textbf{ordered} arrangement of r elements from a set of at least r distinct objects is called an \textbf{r-permutation}.\\
This is different from a standard permutations as some elements from the set will be unused
\subsubsection{Theorem}
If n and r are integers with $1\leqslant r\leqslant n$ then there are
$$P(n,r)=n\cdot (n-1)\cdot...\cdot (n-r+1)$$
r-permutations of a set with n distinct elements\\
Easy to prove using the product rule:
\begin{itemize}
\item n different choices for the first position
\item for each of these choices, n-1 choices for the second position
\item and so on, until the final position, for which there are n-r+1 different choices given any of the choices for the first r-1 position
\item The product rule yields the given formula
\end{itemize}
\subsubsection{Corollary}
\textit{If n and r are integers with $1\leqslant r\leqslant n$ then}
$$P(n,r)=\frac{n!}{(n-r)!}$$
This is also easy to prove by using the definition of the factorial function and writing out the expansions of $n!$ and the product of $P(n,r)$ and $(n-r)!$.\\
\\
For any set of n distinct elements, there are n! permutations of the set
\subsubsection{Examples}
\textit{Suppose there are 8 runners in the final race, there can be no ties. How many ways are there to award the three medals if all outcomes are possible?}\\
Answer: $P(8,3)=8\cdot7\cdot6=336$\\
\\
\textit{How many permutations of the letters ABCDEFGH contain the string ABC?}\\
Answer: Since ABC must occur as a block, treat this string as one symbol. Then we need to count the number of permutations of symbols (ABC),D,E,F,G,H. It's 6!=720
\section{R combinations}
\subsection{Definition}
An r-combination of elements of a set of at least r elements is an \textbf{unordered} selection of r elements from the set\\
\\
The difference between a permutation and combination is that a permutation is ordered, whereas a combination is not.
\subsection{Theorem}
The number of r-combinations of a set with n elements, where n and r are integers with $0\leqslant r\leqslant n$, equals
$$C(n,r)=\frac{n!}{r!(n-r)!}$$
We can prove this as follows, using what we know about r-permutations:
\begin{itemize}
\item Each r-combination can be ordered in r! different ways to obtain an r-permutation
\item So $P(n,r)=r!\cdot C(n,r)$ Writing this out and dividing by $r!$ gives the above formula
\end{itemize}
\subsection{Example}
\textit{The department needs to form a committee by selecting 3 of 9 post-doctoral researchers and 4 of 11 PhD students. How many ways can this be done?}\\
\\
Answer: the ordering of the selecting committee members does not matter, so:
\begin{itemize}
\item $C(9,3)=\dfrac{9!}{3!6!}=84$ ways to select postdocs and
\item $C(11,4)=\frac{11!}{4!7!}=330$ ways to select PhD students
\item By the product rule, there are $84\cdot330=27,720$ ways to select the committee
\end{itemize}

\end{document}