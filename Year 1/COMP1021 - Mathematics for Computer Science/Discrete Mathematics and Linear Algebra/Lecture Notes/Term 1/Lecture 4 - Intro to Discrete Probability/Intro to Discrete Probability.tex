\documentclass{article}[18pt]
\usepackage{../../../../../format}
\lhead{MCS - Discrete Mathematics and Linear Algebra}
\usepackage{amsmath}

\begin{document}
\begin{center}
\underline{\huge Intro to Discrete Probability}
\end{center}

\section{Probability Theory}
Some applications in CS:
\begin{itemize}
	\item Monte Carlo algorithms - Faster algorithms that give correct answers with high probability
	\item Information Theory - Good models for data compression and error correction
	\item Computer Graphics - Automated random generation of objects
	\item Statistical analysis - empirical analysis of computer system performance 
	\item Machine learning - Probabilistic classification methods
	\item Bioinformatics - Finding genes associated with particular diseases
	\item ...
\end{itemize}
\section{Sample space, events and probability}
\begin{itemize}
	\item An \textbf{experiment} is a procedure that yields one of a given set of possible \textbf{outcomes}
	\item The \textbf{sample space} of the experiment is the set of possible outcomes
	\item An \textbf{event} is any subset of the sample space
\end{itemize}
For a finite sample space S and event E in it, let $|S|$ and $|E|$ denote the number of possible outcomes in S and E respectively.\\
$|S|$ represents the number of elements in S
\subsection{Definition}
If S is finite sample space of equally likely outcomes, and E is an event in it, then the \textbf{probability} of E is
$$p(E)=\dfrac{|E|}{|S|}$$
All of our sample spaces are finite, so we can use counting principles
\section{Examples}
\subsection{Example 1}
\textit{An urn contains four blue balls and five red balls. What is the probability that a ball chosen from the urn is blue?}
\begin{itemize}
	\item Sample Space: 9 balls. We are interested in the event: blue ball
	\item The number of successful outcomes (blue ball) is 4
	\item The total number of possible outcomes is 9
	\item Hence, the probability is $\frac{4}{9}$
\end{itemize}
\subsection{Example 2}
\textit{When two (fair) dice are rolled, which sum of the numbers on the dice is more probable: 7 or 8?}
\begin{itemize}
	\item Sample space: 36 combinations (x,y), x is on the 1st die and y is on the 2nd
	\item The successful outcomes for 7 are (1,6),(2,5),(3,4),(4,3),(5,2),(6,1)
	\item The number of successful outcomes for 7 is 6
	\item The total number of possible outcomes is 36
	\item Hence, the probability of sum 7 is 6/36=1/6
	\item What is the probability that the sum is 8? 5/36, so sum 7 is more probable
\end{itemize}
\subsection{Example 3}
\textit{What is the probability to win the big prize in the National Lottery, i.e. to correctly guess 6 numbers out of 49}
\begin{itemize}
	\item Sample Space: 6 combinations out of the set of 49 numbers $\binom{49}{6}=13,938,816$
	\item The number of successful outcomes is 1
	\item Hence, the probability is 1/13,938,816
\end{itemize}
\subsection{Example 4}
\textit{What is the probability that a hand of (5) cards in poker contains 4 cards of the same kind?}
\begin{itemize}
	\item Sample space: 5 combinations of the set of 52 numbers,\\
	C(52,5)=$\frac{52!}{47!5!}=2,598,960$ in total
	\item The number of successful outcomes is
	$$C(13,1)\cdot C(4,4)\cdot C(48,1)=13\cdot1\cdot48=624$$
	We choose
	\begin{itemize}
		\item 1 kind out of 13
		\item then all 4 cards of that kind, and
		\item then 1 card from the remaining 48
	\end{itemize}
	\item Hence, the probability os 624/2,598,960$\approx$0.00024
\end{itemize}

\subsection{Example 5}
\textit{What is the probability that a hand of (5) cards in poker contains 3 cards of 1 kind and 2 of another?}
\begin{itemize}
	\item Sample space: 5 combinations of the set of 52 numbers,\\
	C(52,5)=$\dfrac{52!}{47!5!}=2,598,960$ in total
	\item The number of successful outcomes is
	$$P(13,2)\cdot C(4,3) \cdot C(4,2)=13\cdot12\cdot4\cdot6=3774$$
	We choose
	\begin{itemize}
	\item One of a kind for three cards and a different one for 2,
	\item Then 3 out of 4 cards for the 1st kind, and
	\item 2 out of 4 for the other
	\end{itemize}
\item Hence, the probability is 3,744/2,598,960$\approx$0.0014
\end{itemize}

\subsection{Example 6}
\textit{What is the probability that number 11,4,17,39 and 23 are drawn in that order from a bin containing 50 balls labelled 1,2,...,50 if:\\
(a) the ball selected is not returned to the bin\\
(b) the ball selected is returned to the bin before the next one is drawn
}\\
\\
(a) "Sampling without replacement"
\begin{itemize}
\item Sample space S: 5 permutations from the set of 50 number P(50,5)=254,251,20
\item The number of successful outcomes is 1
\item Hence, the prob is...
\end{itemize}
(b) "Sampling with replacement
\section{The probability of the complementary event}
Let E be an event in sample space S. The probability of $\overline{E}$ is given by
$$P(\overline{E})=1-P(E)$$
\subsection{Example 1}
\textit{A sequence of 10 bits is randomly generated. What is the probability that at least one of these is zero}\\
\\
For this the complimentary value, which is that the sequence is all 1s is much easier to calculate
$$1-\frac{1}{1024}=\frac{1023}{1024}$$
\section{The probability of combinations of events}
Let $E_1$,$E_2$ be events in a sample space S then
$$P(E_1\cup E_2)=P(E_1)+P(E_2)-p(E_1\cap E_2)$$
\subsection{Example 1}
\textit{What is the probability that a number selected at random from the numbers from 1 to 100 is divisible by at least one of 2 and 5?}
\begin{itemize}
\item Sample space is 100
\item Prob divisible by 2 is 1/2
\item Prob divisible by 5 is 1/5
\item Prob divisible by 10 is 1/10
\item Therefore the probability is calculated by
$$\frac{1}{2}+\frac{1}{5}-\frac{1}{10}=\frac{3}{5}$$
\end{itemize}
\section{The 3 door puzzle}
\begin{itemize}
\item Asked to open 1 of 3 doors
\item Prize behind only 1
\item First select a door
\item One of other doors discounted
\item Should you change your choice or not?
\end{itemize}
\section{Assigning probabilities}
\begin{itemize}
\item There can be some sample space where the probabilities $s_1...s_n$ are not equally likely
\item This makes a probability distribution
\end{itemize}
\subsection{Example}
\begin{itemize}
\item Biased die such that 3 appears twice as often as any other number
\item Probability of 3 is 2/7, the probability of a different number is 1/7
\item The probability of odd numbers is 4/7
\end{itemize}



\end{document}