\documentclass{article}[18pt]
\ProvidesPackage{format}
%Page setup
\usepackage[utf8]{inputenc}
\usepackage[margin=0.7in]{geometry}
\usepackage{parselines} 
\usepackage[english]{babel}
\usepackage{fancyhdr}
\usepackage{titlesec}
\hyphenpenalty=10000

\pagestyle{fancy}
\fancyhf{}
\rhead{Sam Robbins}
\rfoot{Page \thepage}

%Characters
\usepackage{amsmath}
\usepackage{amssymb}
\usepackage{gensymb}
\newcommand{\R}{\mathbb{R}}

%Diagrams
\usepackage{pgfplots}
\usepackage{graphicx}
\usepackage{tabularx}
\usepackage{relsize}
\pgfplotsset{width=10cm,compat=1.9}
\usepackage{float}

%Length Setting
\titlespacing\section{0pt}{14pt plus 4pt minus 2pt}{0pt plus 2pt minus 2pt}
\newlength\tindent
\setlength{\tindent}{\parindent}
\setlength{\parindent}{0pt}
\renewcommand{\indent}{\hspace*{\tindent}}

%Programming Font
\usepackage{courier}
\usepackage{listings}
\usepackage{pxfonts}

%Lists
\usepackage{enumerate}
\usepackage{enumitem}

% Networks Macro
\usepackage{tikz}


% Commands for files converted using pandoc
\providecommand{\tightlist}{%
	\setlength{\itemsep}{0pt}\setlength{\parskip}{0pt}}
\usepackage{hyperref}

% Get nice commands for floor and ceil
\usepackage{mathtools}
\DeclarePairedDelimiter{\ceil}{\lceil}{\rceil}
\DeclarePairedDelimiter{\floor}{\lfloor}{\rfloor}

% Allow itemize to go up to 20 levels deep (just change the number if you need more you madman)
\usepackage{enumitem}
\setlistdepth{20}
\renewlist{itemize}{itemize}{20}

% initially, use dots for all levels
\setlist[itemize]{label=$\cdot$}

% customize the first 3 levels
\setlist[itemize,1]{label=\textbullet}
\setlist[itemize,2]{label=--}
\setlist[itemize,3]{label=*}

% Definition and Important Stuff
% Important stuff
\usepackage[framemethod=TikZ]{mdframed}

\newcounter{theo}[section]\setcounter{theo}{0}
\renewcommand{\thetheo}{\arabic{section}.\arabic{theo}}
\newenvironment{important}[1][]{%
	\refstepcounter{theo}%
	\ifstrempty{#1}%
	{\mdfsetup{%
			frametitle={%
				\tikz[baseline=(current bounding box.east),outer sep=0pt]
				\node[anchor=east,rectangle,fill=red!50]
				{\strut Important};}}
	}%
	{\mdfsetup{%
			frametitle={%
				\tikz[baseline=(current bounding box.east),outer sep=0pt]
				\node[anchor=east,rectangle,fill=red!50]
				{\strut Important:~#1};}}%
	}%
	\mdfsetup{innertopmargin=10pt,linecolor=red!50,%
		linewidth=2pt,topline=true,%
		frametitleaboveskip=\dimexpr-\ht\strutbox\relax
	}
	\begin{mdframed}[]\relax%
		\centering
		}{\end{mdframed}}



\newcounter{lem}[section]\setcounter{lem}{0}
\renewcommand{\thelem}{\arabic{section}.\arabic{lem}}
\newenvironment{defin}[1][]{%
	\refstepcounter{lem}%
	\ifstrempty{#1}%
	{\mdfsetup{%
			frametitle={%
				\tikz[baseline=(current bounding box.east),outer sep=0pt]
				\node[anchor=east,rectangle,fill=blue!20]
				{\strut Definition};}}
	}%
	{\mdfsetup{%
			frametitle={%
				\tikz[baseline=(current bounding box.east),outer sep=0pt]
				\node[anchor=east,rectangle,fill=blue!20]
				{\strut Definition:~#1};}}%
	}%
	\mdfsetup{innertopmargin=10pt,linecolor=blue!20,%
		linewidth=2pt,topline=true,%
		frametitleaboveskip=\dimexpr-\ht\strutbox\relax
	}
	\begin{mdframed}[]\relax%
		\centering
		}{\end{mdframed}}
\lhead{DMLA - Term 2}


\begin{document}
\begin{center}
\underline{\huge Divisibility and Primes}
\end{center}
\section{Divisibility and Perfect Numbers}
If a,b are integers and $a\neq 0$ then a \textbf{divides} b iff b=ak for some integer k\\
\\
a|b means "a is a divisor of b"/"a is a factor of b"/"b is a multiple of a"\\
\\
A positive integer $p>1$ is prime if its only positive divisors are 1 and p
\section{Properties of divisibility}
\subsection{Theorem}
The following statements about divisibility hold
\begin{enumerate}
	\item if a|b then a|(bc) for all c
	\item if a|b and b|c then a|c
	\item If a|b and a|c then a|(sb+tc) for all s,t
	\item For all $c\neq 0$, a|b iff $(ca)|(cb)$
\end{enumerate}
\subsection{Proof}
Let's prove item 2:
\begin{itemize}
	\item Since a|b, there is $k_1$ such that $b=ak_1$
	\item Since b|c there is $k_2$ such that $c=bk_2$
	\item Then $c=a(k_1k_2)$ so a|c
\end{itemize}
\section{The division algorithm}
\subsection{Theorem}
Let a be an integer and d a positive integer. Then there exists unique numbers q and r, with $0\leqslant r<d$, such that $a=qd+r$
\subsection{Definition}
In the equality in the division algorithm:
\begin{itemize}
	\item q is the quotient, denoted by $qent(a,d)$ or a div d
	\item r is the remainder, denoted by $rem(a,d)$ or a mod d
\end{itemize}
\section{Fundamental properties of primes}
\subsection{Theorem}
Every positive integer $n>1$ can be uniquely represented as $n=p_1\cdot p_2 \cdots p_k$ where the numbers $p_1\leqslant p_2 \leqslant ... \leqslant p_k$ are all prime
\subsection{Theorem}
There are infinitely many prime numbers
\subsection{Proof}
Assume that there are finitely many primes, say $p_1,...,p_n$ then consider the number $q=p_1 \cdots p_n+1$\\
By the fundamental theorem, q is either prime, or can be written as the product of primes. Hence $p_i|q$ for some i, say $p_1|q$\\
But then $p_1$ divides $q+(-p_2\cdots p_n)p_1=1$, a contradiction
\subsection{Theorem}
The number of primes not exceeding x approaches $x\ln x$ as x grows infinitely.
\section{The greatest common divisor}
Let $gcd(a,b)$ denote the greatest common divisor of a and b\\
A linear combination of a and b is any number of the form $sa+tb$
\subsection{Theorem}
$gcd(a,b)$ is equal to the smallest linear combination of a and b
\subsection{Proof}
Let $m=sa+tb$ be smallest positive. We prove that $m=gcd(a,b)$ by showing that $gcd(a,b)\leqslant m$ and $m\leqslant gcd(a,b)$\\
Any common divisor of $a,b$ divides m, hence $gcd(a,b)|m$ and $gcd(a,b)\leqslant m$\\
\\
Now show that $m\leqslant gcd(a,b)$. We show that $m|a$\\
By division algorithm, we have $a=qm+r$ where $0\leqslant r<m$\\
As $m=sa+tb$ we have $a=q(sa+tb)+r$, or $r=(1-qs)a+(-qt)b$\\
Since m is the smallest positive linear combination of a and b, and $0\leqslant r< m$ we must have $r=0$ and hence $m|a$\\
Similarly one shows m|b and so $m\leqslant gcd(a,b)$
\section{Properties of the GCD}
\subsection{Lemma}
The following statements hold:
\begin{itemize}
	\item gcd(ka,kb)=$k\cdot gcd (a,b)$ for all $k>0$
	\item If gcd(a.b)=1 and gcd(a,c)=1 then gcd(a,bc)=1
	\item if a|bc and gcd(a,b)=1 then a|c
\end{itemize}
\subsection{Proof}
We prove item 2, the other parts are similar\\
Since gcd(a,b)=1 and gcd(a,c)=1, there are number s,t,u,c such that $sa+tb=1$ and $ua+vc=1$\\
Multiplying these together gives $(sa+tb)(ua+vc)=1$\\
Rewrite LHS as $a\cdot(sau+tbu+svc)+bc(tv)$\\
This is a linear combination of a and bc, and is equal to 1\\
Hence $gcd(a,bc)=1$
\section{Euclid's Algorithm}
\subsection{Lemma}
If $a=qb+r$ then $gcd(a,b)=gcd(b,r)$
\subsection{Proof}
Suppose d|a and d|b. Then d|r because $r=a-qb$ and so d|gcd(b,r).\\
Conversely, if d|b and d|r then d|a and so d|gcd(a,b)\\
Then gcd(a,b) and gcd(b,r) divide each other, so gcd(a,b)=gcd(b,r)
\subsection{Method}
Suppose $a>b$ are positive numbers. Euclid's algorithm finds gcd(a,b) as follows
\begin{itemize}
	\item let $r_0=a$ and $r_1=b$. Recursively compute numbers $r_2,r_3...$
	\item Use division algorithm $(r_i=r_{i+1}q_1+r_{i+2})$ to find $r_{i+2}=rem(r_i,r_{i+1})$
	\item Note that $0\leqslant r_{i+2} <r_{i+1}$. Therefore, for some n, $r_n>0$ and $r_{n+1}=0$
	\item We know that $gcd(r_i,r_{i+1})=gcd(r_{i+1},r_{i+2})$ for all i (by the above lemma)
	\item $\operatorname { gcd } ( a , b ) = \operatorname { gcd } \left( r _ { 0 } , r _ { 1 } \right) = \operatorname { gcd } \left( r _ { 1 } , r _ { 2 } \right) = \ldots = \operatorname { gcd } \left( r _ { n - 1 } , r _ { n } \right) = \operatorname { gcd } \left( r _ { n } , 0 \right) = r _ { n }$
\end{itemize}
\subsection{Example}
Find gcd(414,662)
$$\begin{array} { l } { 662 = 414 \cdot 1 + 248 } \\ { 414 = 248 \cdot 1 + 166 } \\ { 248 = 166 \cdot 1 + 82 } \\ { 166 = 82 \cdot 2 + 2 } \\ { 82 = 2 \cdot 41 } \end{array}$$
The last non-zero remainder is 2, so gcd(414,662)=2
\subsection{Example 2}
How do we modify Euclid's algorithm to express gcd(a,b) as a linear combination of a and b?
In every line, express the current remainder as a linear combination of a and b
$$\begin{array} { l l l} { 662 = 414 \cdot 1 + 248 } & { 248 = 662 + ( - 1 ) \cdot 414 } & \\
 { 414 = 248 \cdot 1 + 166 } & { 166 = 414 + ( - 1 ) \cdot 248} &{= ( - 1 ) \cdot 662 + 2 \cdot 414 } \\
  { 248 = 166 \cdot 1 + 82 } & { 82 = 248 + ( - 1 ) \cdot 166}  &{= 2 \cdot 662 + ( - 3 ) \cdot 414 } \\
   { 166 = 82 \cdot 2 + 2 } & { 2 = 166 + ( - 2 ) \cdot 82 } &{=( - 5 ) \cdot 662 + 8 \cdot 414 } \\
    { 82 = } & { 2 \cdot 41 } \end{array}$$
    
 The last non zero remainder is 2, so $gcd(414,662)=2=(-5)\cdot 662+8\cdot 414$
 \section{Relatively prime numbers}
\subsection{Definition}
Two numbers a and b are called relatively prime if gcd(a,b)=1
\subsection{Example}
The value $\phi(n) $ of Euler's $\phi$-function on a number n is the number of integers a with $1\leqslant a \leqslant n$ that are relatively prime with n
\end{document}