\documentclass{article}[18pt]
\ProvidesPackage{format}
%Page setup
\usepackage[utf8]{inputenc}
\usepackage[margin=0.7in]{geometry}
\usepackage{parselines} 
\usepackage[english]{babel}
\usepackage{fancyhdr}
\usepackage{titlesec}
\hyphenpenalty=10000

\pagestyle{fancy}
\fancyhf{}
\rhead{Sam Robbins}
\rfoot{Page \thepage}

%Characters
\usepackage{amsmath}
\usepackage{amssymb}
\usepackage{gensymb}
\newcommand{\R}{\mathbb{R}}

%Diagrams
\usepackage{pgfplots}
\usepackage{graphicx}
\usepackage{tabularx}
\usepackage{relsize}
\pgfplotsset{width=10cm,compat=1.9}
\usepackage{float}

%Length Setting
\titlespacing\section{0pt}{14pt plus 4pt minus 2pt}{0pt plus 2pt minus 2pt}
\newlength\tindent
\setlength{\tindent}{\parindent}
\setlength{\parindent}{0pt}
\renewcommand{\indent}{\hspace*{\tindent}}

%Programming Font
\usepackage{courier}
\usepackage{listings}
\usepackage{pxfonts}

%Lists
\usepackage{enumerate}
\usepackage{enumitem}

% Networks Macro
\usepackage{tikz}


% Commands for files converted using pandoc
\providecommand{\tightlist}{%
	\setlength{\itemsep}{0pt}\setlength{\parskip}{0pt}}
\usepackage{hyperref}

% Get nice commands for floor and ceil
\usepackage{mathtools}
\DeclarePairedDelimiter{\ceil}{\lceil}{\rceil}
\DeclarePairedDelimiter{\floor}{\lfloor}{\rfloor}

% Allow itemize to go up to 20 levels deep (just change the number if you need more you madman)
\usepackage{enumitem}
\setlistdepth{20}
\renewlist{itemize}{itemize}{20}

% initially, use dots for all levels
\setlist[itemize]{label=$\cdot$}

% customize the first 3 levels
\setlist[itemize,1]{label=\textbullet}
\setlist[itemize,2]{label=--}
\setlist[itemize,3]{label=*}

% Definition and Important Stuff
% Important stuff
\usepackage[framemethod=TikZ]{mdframed}

\newcounter{theo}[section]\setcounter{theo}{0}
\renewcommand{\thetheo}{\arabic{section}.\arabic{theo}}
\newenvironment{important}[1][]{%
	\refstepcounter{theo}%
	\ifstrempty{#1}%
	{\mdfsetup{%
			frametitle={%
				\tikz[baseline=(current bounding box.east),outer sep=0pt]
				\node[anchor=east,rectangle,fill=red!50]
				{\strut Important};}}
	}%
	{\mdfsetup{%
			frametitle={%
				\tikz[baseline=(current bounding box.east),outer sep=0pt]
				\node[anchor=east,rectangle,fill=red!50]
				{\strut Important:~#1};}}%
	}%
	\mdfsetup{innertopmargin=10pt,linecolor=red!50,%
		linewidth=2pt,topline=true,%
		frametitleaboveskip=\dimexpr-\ht\strutbox\relax
	}
	\begin{mdframed}[]\relax%
		\centering
		}{\end{mdframed}}



\newcounter{lem}[section]\setcounter{lem}{0}
\renewcommand{\thelem}{\arabic{section}.\arabic{lem}}
\newenvironment{defin}[1][]{%
	\refstepcounter{lem}%
	\ifstrempty{#1}%
	{\mdfsetup{%
			frametitle={%
				\tikz[baseline=(current bounding box.east),outer sep=0pt]
				\node[anchor=east,rectangle,fill=blue!20]
				{\strut Definition};}}
	}%
	{\mdfsetup{%
			frametitle={%
				\tikz[baseline=(current bounding box.east),outer sep=0pt]
				\node[anchor=east,rectangle,fill=blue!20]
				{\strut Definition:~#1};}}%
	}%
	\mdfsetup{innertopmargin=10pt,linecolor=blue!20,%
		linewidth=2pt,topline=true,%
		frametitleaboveskip=\dimexpr-\ht\strutbox\relax
	}
	\begin{mdframed}[]\relax%
		\centering
		}{\end{mdframed}}
\lhead{MCS - DMLA Term 2}


\begin{document}
\begin{center}
\underline{\huge Eigenvalues and Eigenvectors}
\end{center}
\section{Eigenvalues and Eigenvectors}
\subsection{Definition}
Let A be an $n\times n$ matrix. A non-zero vector $x\in \mathbb{R}^n$ is called an eigenvector of A if, for some scalar $\lambda$
$$Ax=\lambda x$$
In this case, $\lambda$ is called an eigenvalue of A and x is an eigenvector corresponding to $\lambda$
\begin{itemize}
	\item The assumption $x\neq 0$ is necessary to avoid the case $A0=\lambda0$ which always holds
	\item The meaning of the notion is that when x is multiplied by A it does not change direction (up to reversal)
\end{itemize}
\subsection{Example}
Vector $\mathbf{x}=\left( \begin{array}{l}{1} \\ {2}\end{array}\right)$ is an eigenvector of $A=\left( \begin{array}{rr}{3} & {0} \\ {8} & {-1}\end{array}\right)$ corresponding to the eigenvalue 3. Indeed,
$$A \mathbf{x}=\left( \begin{array}{rr}{3} & {0} \\ {8} & {-1}\end{array}\right) \left( \begin{array}{l}{1} \\ {2}\end{array}\right)=\left( \begin{array}{l}{3} \\ {6}\end{array}\right)=3 \mathbf{x}$$
\section{Characteristic equation of a matrix}
\subsection{Theorem}
If A is an $n\times n$ matrix then $\lambda$ is an eigenvalue of A iff it satisfies $\operatorname{det}(\lambda I-A)=0$\\
\\
The equation $\operatorname{det}(\lambda I-A)=0$ is called the characteristic equation of A
\subsection{Proof}
By definition, $\lambda$ is an eigenvalue of A iff $Ax=\lambda x$ for some $x\neq 0$\\
The equation $Ax=\lambda x$ can be re-written as $Ax=\lambda Ix$, and then as $(\lambda I-A)x=0$\\
By theorem about invertible matrices, the last equation has solution $x\neq 0$ iff $\operatorname{det}(\lambda I-A)=0$
\subsection{Example}
Find the eigenvalues of the matrix $A=\left( \begin{array}{cc}{2} & {-1} \\ {10} & {-9}\end{array}\right)$
\[ 
\operatorname{det}(\lambda I-A)=\left| \begin{array}{cc}{\lambda-2} & {1} \\ {-10} & {\lambda+9}\end{array}\right|=(\lambda-2) \cdot(\lambda+9)-1 \cdot(-10)=\lambda^{2}+7 \lambda-8
\]
So, the characteristic equation of A is $\lambda^2+7\lambda-8=0$\\
Its solutions are $\lambda_1=1$ and $\lambda_2=-8$ are the eigenvalues of A
\section{Characteristic polynomial of a matrix}
\begin{itemize}
	\item In general, the expression $\operatorname{det}(\lambda I-A)$ is a polynomial
	\[ 
	p(\lambda)=\lambda^{n}+c_{1} \lambda^{n-1}+\ldots+c_{n-1} \lambda+c_{n}
	\]
	\item Solving the equation $p(\lambda)=0$ is difficult in general (no closed formula). In practice, it is solved numerically.
\end{itemize}
\subsection{Example}
Find the eigenvalues of $A=\left( \begin{array}{rrr}{0} & {1} & {0} \\ {0} & {0} & {1} \\ {4} & {-17} & {8}\end{array}\right)$. We have
\[ 
\operatorname{det}(\lambda I-A)=\left( \begin{array}{ccc}{\lambda} & {-1} & {0} \\ {0} & {\lambda} & {-1} \\ {-4} & {17} & {\lambda-8}\end{array}\right)=\lambda^{3}-8 \lambda^{2}+17 \lambda-4=0
\]
The solutions are $\lambda=4, 2+\sqrt{3}$ and $2-\sqrt{3}$
\section{Eigenspaces and their bases}
\begin{itemize}
	\item Let $\lambda_0$ be an eigenvalue of A and consider the equation $(\lambda_0I-A)x=0$
	\item The solution set of the equation is a subspace of $\mathbb{R}^n$, it is the null space of the matrix $\lambda_0I-A$
	\item It is called the eigenspace of A corresponding to $\lambda_0$ because the non-zero vectors in this subspace are the eigenvectors of A corresponding to $\lambda_0$
	\item To find the basis in this subspace, use the algorithm for finding basis in null space of a matrix
\end{itemize}
\subsection{Example}
Find (a basis of) the eigenspace of $A=\left( \begin{array}{cc}{2} & {-1} \\ {10} & {-9}\end{array}\right)$ corresponding to $\lambda=8$\\
\\
Form the equation $(-8I-A)x=0$, or
\[ 
\left( \begin{array}{cc}{-10} & {1} \\ {-10} & {1}\end{array}\right) \left( \begin{array}{c}{x_{1}} \\ {x_{2}}\end{array}\right)=\left( \begin{array}{c}{0} \\ {0}\end{array}\right) \quad \text { or } \quad \begin{array}{c}{-10 x_{1}+x_{2}=0} \\ {-10 x_{1}+x_{2}=0}\end{array}
\]
The subspace consists of all vectors of the form $(x,10x)$. One basis is $\{(1,10)\}$
\section{Similarity of matrices}
\subsection{Definition}
Square matrices A and B are called \textbf{similar} if $A=P^{-1}BP$ for some invertible P\\
\\
Not that if $A=P^{-1}BP$ then $B=Q^{-1}AQ$ where $Q=P^{-1}$\\
Similar matrices have many features in common, including determinant, rank, nullity, characteristic polynomial, eigenvalues etc.
\subsection{Lemma}
If A and B are similar then $\det(A)=\det(B)$
\subsection{Proof}
$\operatorname{det}(A)=\operatorname{det}\left(P^{-1} B P\right)=\operatorname{det}\left(P^{-1}\right) \operatorname{det}(B) \operatorname{det}(P)=\frac{1}{\operatorname{det}(P)} \operatorname{det}(B) \operatorname{det}(P)=\operatorname{det}(B)$
\subsection{Definition}
A square matrix is called \textbf{diagonalisable} if it is similar to a diagonal matrix
\section{Diagonalisation}
\subsection{Theorem}
An $n\times n$ matrix is diagonalisable iff it has n linearly independent eigenvectors
\subsection{Proof}
We prove only $(\Rightarrow)$-direction. Assume that there is an invertible matrix P and a diagonal matrix $D=\operatorname{diag}(\lambda_1,...,\lambda_n)$ such that $D=P^{-1}AP$, or $AP=PD$.\\
Denote the column vectors of P by $\mathbf{p}_1,...,\mathbf{p}_n$, so that $P=[p_1|...|p_n]$. Then
\[ 
A P=A\left[\mathbf{p}_{1}|\ldots| \mathbf{p}_{n}\right]=\left[A \mathbf{p}_{1}|\ldots| A \mathbf{p}_{n}\right]
\]
On the other hand
\[ 
P D=\left[\lambda_{1} \mathbf{p}_{1}|\ldots| \lambda_{n} \mathbf{p}_{n}\right]
\]
Since $AP=PD$, we can conclude that $A\mathbf{p}_i=\lambda_i\mathbf{p}_i$ for all $1\leqslant i\leqslant n$.\\
Since P is invertible the vectors $\mathbf{p}_1,...,\mathbf{p}_n$ are linearly independent and in particular, none of $\mathbf{p}_1,...,\mathbf{p}_n$ is $\mathbf{0}$, so each of them is a linearly independent eigenvector.
\end{document}