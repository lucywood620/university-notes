\documentclass{article}[18pt]
\ProvidesPackage{format}
%Page setup
\usepackage[utf8]{inputenc}
\usepackage[margin=0.7in]{geometry}
\usepackage{parselines} 
\usepackage[english]{babel}
\usepackage{fancyhdr}
\usepackage{titlesec}
\hyphenpenalty=10000

\pagestyle{fancy}
\fancyhf{}
\rhead{Sam Robbins}
\rfoot{Page \thepage}

%Characters
\usepackage{amsmath}
\usepackage{amssymb}
\usepackage{gensymb}
\newcommand{\R}{\mathbb{R}}

%Diagrams
\usepackage{pgfplots}
\usepackage{graphicx}
\usepackage{tabularx}
\usepackage{relsize}
\pgfplotsset{width=10cm,compat=1.9}
\usepackage{float}

%Length Setting
\titlespacing\section{0pt}{14pt plus 4pt minus 2pt}{0pt plus 2pt minus 2pt}
\newlength\tindent
\setlength{\tindent}{\parindent}
\setlength{\parindent}{0pt}
\renewcommand{\indent}{\hspace*{\tindent}}

%Programming Font
\usepackage{courier}
\usepackage{listings}
\usepackage{pxfonts}

%Lists
\usepackage{enumerate}
\usepackage{enumitem}

% Networks Macro
\usepackage{tikz}


% Commands for files converted using pandoc
\providecommand{\tightlist}{%
	\setlength{\itemsep}{0pt}\setlength{\parskip}{0pt}}
\usepackage{hyperref}

% Get nice commands for floor and ceil
\usepackage{mathtools}
\DeclarePairedDelimiter{\ceil}{\lceil}{\rceil}
\DeclarePairedDelimiter{\floor}{\lfloor}{\rfloor}

% Allow itemize to go up to 20 levels deep (just change the number if you need more you madman)
\usepackage{enumitem}
\setlistdepth{20}
\renewlist{itemize}{itemize}{20}

% initially, use dots for all levels
\setlist[itemize]{label=$\cdot$}

% customize the first 3 levels
\setlist[itemize,1]{label=\textbullet}
\setlist[itemize,2]{label=--}
\setlist[itemize,3]{label=*}

% Definition and Important Stuff
% Important stuff
\usepackage[framemethod=TikZ]{mdframed}

\newcounter{theo}[section]\setcounter{theo}{0}
\renewcommand{\thetheo}{\arabic{section}.\arabic{theo}}
\newenvironment{important}[1][]{%
	\refstepcounter{theo}%
	\ifstrempty{#1}%
	{\mdfsetup{%
			frametitle={%
				\tikz[baseline=(current bounding box.east),outer sep=0pt]
				\node[anchor=east,rectangle,fill=red!50]
				{\strut Important};}}
	}%
	{\mdfsetup{%
			frametitle={%
				\tikz[baseline=(current bounding box.east),outer sep=0pt]
				\node[anchor=east,rectangle,fill=red!50]
				{\strut Important:~#1};}}%
	}%
	\mdfsetup{innertopmargin=10pt,linecolor=red!50,%
		linewidth=2pt,topline=true,%
		frametitleaboveskip=\dimexpr-\ht\strutbox\relax
	}
	\begin{mdframed}[]\relax%
		\centering
		}{\end{mdframed}}



\newcounter{lem}[section]\setcounter{lem}{0}
\renewcommand{\thelem}{\arabic{section}.\arabic{lem}}
\newenvironment{defin}[1][]{%
	\refstepcounter{lem}%
	\ifstrempty{#1}%
	{\mdfsetup{%
			frametitle={%
				\tikz[baseline=(current bounding box.east),outer sep=0pt]
				\node[anchor=east,rectangle,fill=blue!20]
				{\strut Definition};}}
	}%
	{\mdfsetup{%
			frametitle={%
				\tikz[baseline=(current bounding box.east),outer sep=0pt]
				\node[anchor=east,rectangle,fill=blue!20]
				{\strut Definition:~#1};}}%
	}%
	\mdfsetup{innertopmargin=10pt,linecolor=blue!20,%
		linewidth=2pt,topline=true,%
		frametitleaboveskip=\dimexpr-\ht\strutbox\relax
	}
	\begin{mdframed}[]\relax%
		\centering
		}{\end{mdframed}}
\lhead{MCS - DMLA}


\begin{document}
\begin{center}
\underline{\huge Linear Systems}
\end{center}
\section{Systems of linear equations}
\begin{itemize}
\item A \textbf{linear equation} in n variables $x_1,...,x_n$ is an equation of the form
$$a _ { 1 } x _ { 1 } + a _ { 2 } x _ { 2 } + \ldots + a _ { n } x _ { n } = b$$
where the $a_i$'s and b are co constant and not all $a_i$'s are equal to 0
\item A finite set of linear equations is called a \textbf{system of linear equations}, or simply a \textbf{linear system}
\item A general linear system can be written as
$$\begin{aligned} a _ { 11 } x _ { 1 } + a _ { 12 } x _ { 2 } + \ldots + a _ { 1 n } x _ { n } & = b _ { 1 } \\ a _ { 21 } x _ { 1 } + a _ { 22 } x _ { 2 } + \ldots + a _ { 2 n } x _ { n } & = b _ { 2 } \\ \vdots & = \vdots \\ a _ { m 1 } x _ { 1 } + a _ { m 2 } x _ { 2 } + \ldots + a _ { m n } x _ { n } & = b _ { m } \end{aligned}$$
\item A \textbf{solution} to such a system is a sequence of $s_1,..,s_n$ of numbers such that the assignment $x_1=s_1,...,x_n=s_n$ satisfies every equation
\item A linear system is called consistent if it has at least one solution and it is inconsistent otherwise
\end{itemize}
\section{A linear system with one solution}
Solve linear system
$$x-y=1$$
$$2x+y=6$$
Eliminate x from the 2nd equation by adding -2 times the 1st equation to the 2nd
$$x-y=1$$
$$2y=4$$
We have $y=4/3$, and from the 1st equation $x=7/3$\\
This system has \textbf{one solution}
\section{A linear system with no solutions}
Solve linear system
$$x+y=4$$
$$3x+3y=6$$
Eliminate x from the 2nd equation by adding -3 times the 1st equation to the 2nd
$$x+y=4$$
$$0=-6$$
The 2nd equation is contradictory. This system has \textbf{no solutions}
\section{A linear system with infinitely many solutions}
Solve linear system
$$2x+2y=1$$
$$8x-4y=2$$
Eliminate x from the 2nd equation by adding -3 times the 1st equation to the 2nd
$$2x-2y=1$$
$$0=0$$
The 2nd equation implies no restrictions on x and y, can be omitted\\
Any pair of values for x and y that satisfies $4x-2y=1$ is a solution\\
Solving it for x, we get $x=\frac{1}{4}+\frac{1}{2}y$\\
The solution set can be described as the set of all pairs of numbers of the form $x=\frac{1}{4}+\frac{1}{2}y$,y (y is a free variable here)\\
This system has \textbf{infinitely many solutions}
\section{Matrix form of a linear system}
A linear system
$$\begin{aligned} a _ { 11 } x _ { 1 } + a _ { 12 } x _ { 2 } + \ldots + a _ { 1 n } x _ { n } & = b _ { 1 } \\ a _ { 21 } x _ { 1 } + a _ { 22 } x _ { 2 } + \ldots + a _ { 2 n } x _ { n } & = b _ { 2 } \\ \vdots & = \vdots \\ a _ { m 1 } x _ { 1 } + a _ { m 2 } x _ { 2 } + \ldots + a _ { m n } x _ { n } & = b _ { m } \end{aligned}$$
can be written in a matrix form as $Ax=b$ where
$$A = \left( \begin{array} { c c c c } { a _ { 11 } } & { a _ { 12 } } & { \cdots } & { a _ { 1 n } } \\ { a _ { 21 } } & { a _ { 22 } } & { \cdots } & { a _ { 2 n } } \\ { \vdots } & { \vdots } & { } & { \vdots } \\ { a _ { m 1 } } & { a _ { m 2 } } & { \cdots } & { a _ { m n } } \end{array} \right) , \quad \mathbf { x } = \left( \begin{array} { c } { x _ { 1 } } \\ { x _ { 2 } } \\ { \vdots } \\ { x _ { n } } \end{array} \right) \quad \text { and } \mathbf { b } = \left( \begin{array} { c } { b _ { 1 } } \\ { b _ { 2 } } \\ { \vdots } \\ { b _ { m } } \end{array} \right)$$
The matrix A is called the coefficient matrix of the system\\
If A is (square and) invertible then the solution can be found as $x=A^{-1}b$
\section{The augmented matrix and elementary row operations}
The augmented matrix of a linear system is the matrix
$$( A | \mathbf { b } ) = \left( \begin{array} { c c c c | c } { a _ { 11 } } & { a _ { 12 } } & { \cdots } & { a _ { 1 n } } & { b _ { 1 } } \\ { a _ { 21 } } & { a _ { 22 } } & { \cdots } & { a _ { 2 n } } & { b _ { 2 } } \\ { \vdots } & { \vdots } & { } & { \vdots } & { \vdots } \\ { a _ { m 1 } } & { a _ { m 2 } } & { \cdots } & { a _ { m n } } & { b _ { m } } \end{array} \right)$$
The basic method for solving a linear system is to perform algebraic operations on the system that:
\begin{enumerate}[label=(\alph*)]
\item Do not alter the equation set
\item Produce increasingly simpler systems
\end{enumerate}
Typically the operations are
\begin{itemize}
\item Multiply an equation through by a non zero constant
\item Interchange two equations
\item Add a constant times one equation to another
\end{itemize}
This corresponds to the \textbf{elementary row operations} on the augmented matrix
\begin{itemize}
\item Multiply a row through by a non zero constant
\item Interchange two rows
\item Add a constant times one row to another
\end{itemize}
\section{Row echelon form}
Assume that we transform the augmented matrix of a linear system in variables x,y,z to the form
$$\left( \begin{array} { c c c | c } { 1 } & { 0 } & { 0 } & { 1 } \\ { 0 } & { 1 } & { 0 } & { 2 } \\ { 0 } & { 0 } & { 1 } & { 3 } \end{array} \right)$$
Then we know the solution: it's x=1, y=2, z=3\\
A matrix is in \textbf{row echelon form} if it has the following properties:
\begin{itemize}
\item If a row is not all 0s then the first non zero number in it is 1 (the leading 1)
\item The rows that are all 0s (if any) are grouped together at the bottom
\item If two successive rows are not all 0s then the leading 1 of the higher row occurs further to the left than the leading 1 of the lower row
\end{itemize}
A matrix is in \textbf{reduced row echelon form} if it has the above properties, plus
\begin{itemize}
\item Each column that contains a leading 1 has 0s everywhere else
\end{itemize}
Strategy for solving linear systems: use elementary row operations to transform the augmented matrix to (reduced) row echelon form
\section{Extracting solutions from row echelon form}
Assume that we have transformed the augmented matrix of a linear system to a (reduced) row echelon form\\
Examples:
$$\left( \begin{array} { l l l | l } { 1 } & { 0 } & { 0 } & { 2 } \\ { 0 } & { 1 } & { 0 } & { 5 } \\ { 0 } & { 0 } & { 0 } & { 1 } \\ { 0 } & { 0 } & { 0 } & { 0 } \end{array} \right) \left( \begin{array} { l l l | l } { 1 } & { 0 } & { 0 } & { 2 } \\ { 0 } & { 1 } & { 0 } & { 5 } \\ { 0 } & { 0 } & { 1 } & { 1 } \\ { 0 } & { 0 } & { 0 } & { 0 } \end{array} \right) \left( \begin{array} { c c c | c } { 1 } & { - 1 } & { 0 } & { 2 } \\ { 0 } & { 0 } & { 1 } & { 5 } \\ { 0 } & { 0 } & { 0 } & { 0 } \\ { 0 } & { 0 } & { 0 } & { 0 } \end{array} \right)$$
We have the following possibilities:
\begin{itemize}
\item Some row has a leading 1 in the last column.\\
Then the system includes equation $0\cdot x_1 +...+0\cdot x_n=1$\\
Then we know the system has no solutions.
\item The number of leading 1s is equal to the number of variables\\
(and there is no leading 1 in the last column)\\
Then the system has a unique solution
\item The number of leading 1s is smaller than the number of variables\\
(and there is no leading 1 in the last column)\\
Then the system has infinitely many solutions
\end{itemize}
\section{General solution (and an example)}
Assume the matrix in reduced row echelon form is as follows:
$$\left( \begin{array} { r r r r | r } { 1 } & { - 1 } & { 0 } & { 2 } & { 2 } \\ { 0 } & { 0 } & { 1 } & { - 1 } & { 5 } \\ { 0 } & { 0 } & { 0 } & { 0 } & { 0 } \\ { 0 } & { 0 } & { 0 } & { 0 } & { 0 } \end{array} \right)$$
In equations, this is
$$\begin{aligned} x _ { 1 } - x _ { 2 } & + 2 x _ { 4 } & = 2 \\ x _ { 3 } & - x _ { 4 } & = 5 \end{aligned}$$
\begin{itemize}
\item The variables corresponding to the leading 1s ($x_1$ and $x_3$ in this example) are the \textbf{leading variables}
\item The other variables are \textbf{free variables}
\item \textbf{General solution}: the leading variables expressed via free variables
\item For the above system $x_1=x_2-2x_4+2$, $x_3=x_4+5$ (where $x_2$ and $x_4$ are arbitrary numbers)
\end{itemize}
\section{Gaussian elimination procedure}
Goal: Transform a matrix to row echelon form by using row operations\\
\textbf{Step 1}: Locate the leftmost column that contains a non zero\\
$$\left( \begin{array} { c c c c c c } { \mathbf{0} } & { 0 } & { - 2 } & { 0 } & { 7 } & { 12 } \\ { \mathbf{2} } & { 4 } & { - 10 } & { 6 } & { 12 } & { 28 } \\ { \mathbf{2} } & { 4 } & { - 5 } & { 6 } & { - 5 } & { - 1 } \end{array} \right)$$
\textbf{Step 2}: Interchange the first row with another row (if necessary) to move a non zero to the top in this column
$$\left( \begin{array} { c c c c c c } { \textbf{2} } & { 4 } & { - 10 } & { 6 } & { 12 } & { 28 } \\ { \textbf{0} } & { 0 } & { - 2 } & { 0 } & { 7 } & { 12 } \\ { \textbf{2} } & { 4 } & { - 5 } & { 6 } & { - 5 } & { - 1 } \end{array} \right)$$
\textbf{Step 3}: If a is at the top, multiply the first row by 1/a (to get a leading 1)
$$\left( \begin{array} { c c c c c c } { \textbf{1} } & { 2 } & { - 5 } & { 3 } & { 6 } & { 14 } \\ { \textbf{0} } & { 0 } & { - 2 } & { 0 } & { 7 } & { 12 } \\ { \textbf{2} } & { 4 } & { - 5 } & { 6 } & { - 5 } & { - 1 } \end{array} \right)$$
\textbf{Step 4:} Add suitable multiples of the first to the rows below so that all numbers below the leading 1 are 0s
$$\left( \begin{array} { c c c c c c } { \textbf{1} } & { 2 } & { - 5 } & { 3 } & { 6 } & { 14 } \\ { \textbf{0} } & { 0 } & { - 2 } & { 0 } & { 7 } & { 12 } \\ { \textbf{0} } & { 0 } & { 5 } & { 0 } & { - 17 } & { - 29 } \end{array} \right)$$
\textbf{Step 5}: now separate the top row from the rest ("draw a line below it") and repeat steps 1-5 for the matrix below the line
$$\left( \begin{array} { c c c c c c } { \textbf{1} } & { 2 } & { - 5 } & { 3 } & { 6 } & { 14 } \\ \hline \textbf{0} & { 0 } & { - 2 } & { 0 } & { 7 } & { 12 } \\ { \textbf{0} } & { 0 } & { 5 } & { 0 } & { - 17 } & { - 29 } \end{array} \right)$$

$$\left( \begin{array} { c c c c c c } { 1 } & { 2 } & { - 5 } & { 3 } & { 6 } & { 14 } \\ { 0 } & { 0 } & { 1 } & { 0 } & { - 7 / 2 } & { - 6 } \\ { 0 } & { 0 } & { 5 } & { 0 } & { - 17 } & { - 29 } \end{array} \right)$$
$$\left( \begin{array} { c c c c c c } { 1 } & { 2 } & { - 5 } & { 3 } & { 6 } & { 14 } \\ { 0 } & { 0 } & { 1 } & { 0 } & { - 7 / 2 } & { - 6 } \\ \hline 0 & { 0 } & { 0 } & { 0 } & { 1 / 2 } & { 1 } \end{array} \right)$$
$$\left( \begin{array} { c c c c c c } { 1 } & { 2 } & { - 5 } & { 3 } & { 6 } & { 14 } \\ { 0 } & { 0 } & { 1 } & { 0 } & { - 7 / 2 } & { - 6 } \\ { 0 } & { 0 } & { 0 } & { 0 } & { 1 } & { 2 } \end{array} \right)$$
\section{Gauss-Jordan elimination}
$$\left( \begin{array} { c c c c c c } { 1 } & { 2 } & { - 5 } & { 3 } & { 6 } & { 14 } \\ { 0 } & { 0 } & { 1 } & { 0 } & { - 7 / 2 } & { - 6 } \\ { 0 } & { 0 } & { 0 } & { 0 } & { 1 } & { 2 } \end{array} \right)$$
To find the reduced row echelon form, we need one step on top of Gaussian.\\
\textbf{Step 6}: Beginning from the last non 0 row and working upward, add suitable multiples of each row to create 0s before the leading 1s
$$\left( \begin{array} { c c c c c c } { 1 } & { 2 } & { - 5 } & { 3 } & { 6 } & { 14 } \\ { 0 } & { 0 } & { 1 } & { 0 } & { 0 } & { 1 } \\ { 0 } & { 0 } & { 0 } & { 0 } & { 1 } & { 2 } \end{array} \right) \quad \text { added } ( 7 / 2 ) \times 3 \text {rd row to } 2 \text{nd row}$$
$$\left( \begin{array} { c c c c c c } { 1 } & { 2 } & { - 5 } & { 3 } & { 0 } & { 2 } \\ { 0 } & { 0 } & { 1 } & { 0 } & { 0 } & { 1 } \\ { 0 } & { 0 } & { 0 } & { 0 } & { 1 } & { 2 } \end{array} \right) \quad \text { added } ( - 6 ) \times 3 \text {rd row to } 1 \text{st roe}$$
$$\left( \begin{array} { l l l l l l } { 1 } & { 2 } & { 0 } & { 3 } & { 0 } & { 7 } \\ { 0 } & { 0 } & { 1 } & { 0 } & { 0 } & { 1 } \\ { 0 } & { 0 } & { 0 } & { 0 } & { 1 } & { 2 } \end{array} \right) \quad \text { added } 5 \times 2 n d \text { row to } 1 \text{st row}$$
\section{Example}
Solve linear system by Gauss-Jordan elimination:
$$\begin{aligned} &- 2 x _ { 3 } & &+ 7 x _ { 5 } & = 12 \\ 2 x _ { 1 } + 4 x _ { 2 } & - 10 x _ { 3 } & + 6 x _ { 4 } & + 12 x _ { 5 } & = 28 \\ 2 x _ { 1 } + 4 x _ { 2 } & - 5 x _ { 3 } & + 6 x _ { 4 } & - 5 x _ { 5 } & = - 1 \end{aligned}$$
The augmented matrix of system is
$$\left( \begin{array} { c c c c c | c } { 0 } & { 0 } & { - 2 } & { 0 } & { 7 } & { 12 } \\ { 2 } & { 4 } & { - 10 } & { 6 } & { 12 } & { 28 } \\ { 2 } & { 4 } & { - 5 } & { 6 } & { - 5 } & { - 1 } \end{array} \right)$$
We have already transformed the above matrix to reduced row echelon form (see four previous slides)
$$\left( \begin{array} { c c c c c | c } { 1 } & { 2 } & { 0 } & { 3 } & { 0 } & { 7 } \\ { 0 } & { 0 } & { 1 } & { 0 } & { 0 } & { 1 } \\ { 0 } & { 0 } & { 0 } & { 0 } & { 1 } & { 2 } \end{array} \right)$$
The general solution of the system is $x_1=-2x_2-3x_4+7, x_3=1, x_2$
\section{Homogeneous linear systems}
\begin{itemize}
\item A linear system $Ax=b$ is \textbf{homogeneous} if b is all 0s
\item Such a system has a \textbf{trivial solution}: x is all 0s. Any other solution is called \textbf{non-trivial}
\end{itemize}
\subsection{Theorem}
If a homogeneous linear system has n variables and the reduced row echelon form of its augmented matrix has r non-0 rows then the system has $n-r$ free variables.\\
\\
The above theorem follows immediately from the shape of the reduced row echelon form.\\
Being consistent and having free variables implies having infinitely many solutions.
\subsection{Corollary}
A homogeneous linear system with more variables than equations has infinitely many solutions


\end{document}