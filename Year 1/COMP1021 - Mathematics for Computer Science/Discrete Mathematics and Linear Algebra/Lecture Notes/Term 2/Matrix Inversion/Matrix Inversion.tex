\documentclass{article}[18pt]
\ProvidesPackage{format}
%Page setup
\usepackage[utf8]{inputenc}
\usepackage[margin=0.7in]{geometry}
\usepackage{parselines} 
\usepackage[english]{babel}
\usepackage{fancyhdr}
\usepackage{titlesec}
\hyphenpenalty=10000

\pagestyle{fancy}
\fancyhf{}
\rhead{Sam Robbins}
\rfoot{Page \thepage}

%Characters
\usepackage{amsmath}
\usepackage{amssymb}
\usepackage{gensymb}
\newcommand{\R}{\mathbb{R}}

%Diagrams
\usepackage{pgfplots}
\usepackage{graphicx}
\usepackage{tabularx}
\usepackage{relsize}
\pgfplotsset{width=10cm,compat=1.9}
\usepackage{float}

%Length Setting
\titlespacing\section{0pt}{14pt plus 4pt minus 2pt}{0pt plus 2pt minus 2pt}
\newlength\tindent
\setlength{\tindent}{\parindent}
\setlength{\parindent}{0pt}
\renewcommand{\indent}{\hspace*{\tindent}}

%Programming Font
\usepackage{courier}
\usepackage{listings}
\usepackage{pxfonts}

%Lists
\usepackage{enumerate}
\usepackage{enumitem}

% Networks Macro
\usepackage{tikz}


% Commands for files converted using pandoc
\providecommand{\tightlist}{%
	\setlength{\itemsep}{0pt}\setlength{\parskip}{0pt}}
\usepackage{hyperref}

% Get nice commands for floor and ceil
\usepackage{mathtools}
\DeclarePairedDelimiter{\ceil}{\lceil}{\rceil}
\DeclarePairedDelimiter{\floor}{\lfloor}{\rfloor}

% Allow itemize to go up to 20 levels deep (just change the number if you need more you madman)
\usepackage{enumitem}
\setlistdepth{20}
\renewlist{itemize}{itemize}{20}

% initially, use dots for all levels
\setlist[itemize]{label=$\cdot$}

% customize the first 3 levels
\setlist[itemize,1]{label=\textbullet}
\setlist[itemize,2]{label=--}
\setlist[itemize,3]{label=*}

% Definition and Important Stuff
% Important stuff
\usepackage[framemethod=TikZ]{mdframed}

\newcounter{theo}[section]\setcounter{theo}{0}
\renewcommand{\thetheo}{\arabic{section}.\arabic{theo}}
\newenvironment{important}[1][]{%
	\refstepcounter{theo}%
	\ifstrempty{#1}%
	{\mdfsetup{%
			frametitle={%
				\tikz[baseline=(current bounding box.east),outer sep=0pt]
				\node[anchor=east,rectangle,fill=red!50]
				{\strut Important};}}
	}%
	{\mdfsetup{%
			frametitle={%
				\tikz[baseline=(current bounding box.east),outer sep=0pt]
				\node[anchor=east,rectangle,fill=red!50]
				{\strut Important:~#1};}}%
	}%
	\mdfsetup{innertopmargin=10pt,linecolor=red!50,%
		linewidth=2pt,topline=true,%
		frametitleaboveskip=\dimexpr-\ht\strutbox\relax
	}
	\begin{mdframed}[]\relax%
		\centering
		}{\end{mdframed}}



\newcounter{lem}[section]\setcounter{lem}{0}
\renewcommand{\thelem}{\arabic{section}.\arabic{lem}}
\newenvironment{defin}[1][]{%
	\refstepcounter{lem}%
	\ifstrempty{#1}%
	{\mdfsetup{%
			frametitle={%
				\tikz[baseline=(current bounding box.east),outer sep=0pt]
				\node[anchor=east,rectangle,fill=blue!20]
				{\strut Definition};}}
	}%
	{\mdfsetup{%
			frametitle={%
				\tikz[baseline=(current bounding box.east),outer sep=0pt]
				\node[anchor=east,rectangle,fill=blue!20]
				{\strut Definition:~#1};}}%
	}%
	\mdfsetup{innertopmargin=10pt,linecolor=blue!20,%
		linewidth=2pt,topline=true,%
		frametitleaboveskip=\dimexpr-\ht\strutbox\relax
	}
	\begin{mdframed}[]\relax%
		\centering
		}{\end{mdframed}}
\lhead{MCS - DMLA}


\begin{document}
\begin{center}
\underline{\huge Matrix Inversion}
\end{center}

\section{Row equivalence of matrices}
Recall that the three \textbf{elementary row operations} on a matrix are:
\begin{itemize}
	\item Multiply a row my a non zero constant c
	\item Interchange two rows
	\item Add a constant c times one row $r_1$ to another row $r_2$
\end{itemize}
Observation: If B is obtained from A by using an elementary row operation then A can be obtained from B by using the \textbf{inverse elementary row operation}:
\begin{itemize}
	\item Multiply the same row by a non zero constant $1/c$
	\item Interchange the same two rows
	\item Add -c times row $r_1$ to row $r_2$
\end{itemize}
Matrices A and B are called \textbf{row equivalent} if either (hence each) can be obtained from the other by a sequence of elementary row operations
\section{Elementary matrices}
An $n\times n$ matrix is called an \textbf{elementary matrix} if it is obtained from the identity matrix $I_n$ by performing a single elementary row operation\\
Examples of elementary matrices:
$$\left( \begin{array} { r r } { 1 } & { 0 } \\ { 0 } & { - 3 } \end{array} \right) \left( \begin{array} { c c c c } { 1 } & { 0 } & { 0 } & { 0 } \\ { 0 } & { 0 } & { 1 } & { 0 } \\ { 0 } & { 1 } & { 0 } & { 0 } \\ { 0 } & { 0 } & { 0 } & { 1 } \end{array} \right) \left( \begin{array} { l l l } { 1 } & { 0 } & { 3 } \\ { 0 } & { 1 } & { 0 } \\ { 0 } & { 0 } & { 1 } \end{array} \right) \left( \begin{array} { l l l } { 1 } & { 0 } & { 0 } \\ { 0 } & { 1 } & { 0 } \\ { 0 } & { 0 } & { 1 } \end{array} \right)$$
\subsection{Lemma}
Suppose that an elementary matrix E is obtained from $I_m$ by performing an elementary row operation. If A is an $m\times n$ matrix then the product EA is the matrix obtained from A by performing the same row operation\\
\\
Thus, performing an elementary row operation has the same effect as multiplying by the corresponding elementary matrix (from the left)
\section{Elementary matrices are invertible}
\subsection{Lemma}
Every elementary matrix E is invertible, and the inverse is also elementary
\subsection{Proof}
By definition, E can be obtained from I by using some elementary row operation. Then I can be obtained from E by using the inverse elementary row operation. By the above, there is a matrix $E_0$ such that $E_0E=I$, hence E is invertible. We have $E_0=E^{-1}$, so $EE_0=I$, which implies that $E_0$ is also elementary
\section{Theorem about invertible matrices}
\subsection{Theorem}
If A is an $n\times n$ matrix, then the following are equivalent, i.e. all true or all false
\begin{enumerate}
	\item A is invertible
	\item The linear system $Ax=0$ has only the trivial solution $x=0$
	\item The reduced row echelon form of A is $I_n$
	\item A can be expressed as a product of elementary matrices
	\item $det(A)\neq 0$
\end{enumerate}
\subsection{Proof}
We will prove that $(1)\Rightarrow(2)\Rightarrow(3)\Rightarrow(4)\Rightarrow (1)$ and $1\Leftrightarrow (5)$ later today\\
$(1)\Rightarrow (2)$. Assume that A is invertible. If $Ax=0$ then $x=A^{-1}Ax=A^{-1}0=0$\\
\\
$(2)\Rightarrow (3)$. Assume that the system $Ax=0$ has only the trivial solution $x=0$. The augmented matrix of the system is $[A|0]$. If the reduced echelon form of this matrix is not $[I_n|0]$ then the system has a non trivial solution, which can't exist. Hence the reduced echelon form of $[A|0]$ is $[I_n|0]$, which immediately implies (3).\\
\\
$(3)\Rightarrow (4)$ If $I_n$ is obtained from A by a sequence of elementary row operations then there are elementary matrices $E_1,...E_k$ such that
$$E_k\cdots E_2E_1A=I_n$$
We proved today that each $E_i$ is invertible and each $E^{-1}_i$ is elementary. Hence
$$A = E _ { 1 } ^ { - 1 } E _ { 2 } ^ { - 1 } \cdots E _ { k } ^ { - 1 } l _ { n } = E _ { 1 } ^ { - 1 } E _ { 2 } ^ { - 1 } \cdots E _ { k } ^ { - 1 }$$
$(4)\Rightarrow (1)$. The product of invertible matrices is also invertible
\section{Inversion algorithm}
As an application of the above theorem, we give an algorithm which
\begin{enumerate}
	\item decides whether a given matrix A is invertible
	\item and, if so, finds the inverse $A^{-1}$
\end{enumerate}
Assume that $E_k\cdots E_2E_1A=I_n$. Multiplying by $A^{-1}$, get $E _ { k } \cdots E _ { 2 } E _ { 1 } l _ { n } = A ^ { - 1 }$\\
Therefore, if a sequence of elementary row operations transforms A to $I_n$ then the same sequence transforms $I_n$ to $A^{-1}$\\
Inversion algorithm:
\begin{enumerate}
	\item Write the matrix $[A|I_n]$
	\item Apply elementary row operations to the whole matrix to transform its left half  (i.e. A) to reduced row echelon form
	\item If this form (of the left half) is not $I_n$ then the matrix is not invertible
	\item Otherwise, the obtained matrix is $[I_n|A^{-1}]$
\end{enumerate}
\newpage
\subsection{Example}
Find the inverse (if it exists) of the matrix  $A = \left( \begin{array} { c c c } { 1 } & { 2 } & { 3 } \\ { 2 } & { 5 } & { 3 } \\ { 1 } & { 0 } & { 8 } \end{array} \right)$
$$\left( \begin{array} { l l l | l l l } { 1 } & { 2 } & { 3 } & { 1 } & { 0 } & { 0 } \\ { 2 } & { 5 } & { 3 } & { 0 } & { 1 } & { 0 } \\ { 1 } & { 0 } & { 8 } & { 0 } & { 0 } & { 1 } \end{array} \right) \rightarrow \left( \begin{array} { r r r | r r r } { 1 } & { 2 } & { 3 } & { 1 } & { 0 } & { 0 } \\ { 0 } & { 1 } & { - 3 } & { - 2 } & { 1 } & { 0 } \\ { 0 } & { - 2 } & { 5 } & { - 1 } & { 0 } & { 1 } \end{array} \right) \rightarrow \left( \begin{array} { c c c | c c c } { 1 } & { 2 } & { 3 } & { 1 } & { 0 } & { 0 } \\ { 0 } & { 1 } & { - 3 } & { - 2 } & { 1 } & { 0 } \\ { 0 } & { 0 } & { - 1 } & { - 5 } & { 2 } & { 1 } \end{array} \right) \rightarrow \left( \begin{array} { c c c | c c c } { 1 } & { 2 } & { 3 } & { 1 } & { 0 } & { 0 } \\ { 0 } & { 1 } & { - 3 } & { - 2 } & { 1 } & { 0 } \\ { 0 } & { 0 } & { 1 } & { 5 } & { - 2 } & { - 1 } \end{array} \right) \rightarrow$$
$$\left( \begin{array} { c c c | c c c } { 1 } & { 2 } & { 0 } & { - 14 } & { 6 } & { 3 } \\ { 0 } & { 1 } & { 0 } & { 13 } & { - 5 } & { - 3 } \\ { 0 } & { 0 } & { 1 } & { 5 } & { - 2 } & { - 1 } \end{array} \right) \rightarrow \left( \begin{array} { c c c | c c c } { 1 } & { 0 } & { 0 } & { - 40 } & { 16 } & { 9 } \\ { 0 } & { 1 } & { 0 } & { 13 } & { - 5 } & { - 3 } \\ { 0 } & { 0 } & { 1 } & { 5 } & { - 2 } & { - 1 } \end{array} \right)$$
We have $A ^ { - 1 } = \left( \begin{array} { r r r } { - 40 } & { 16 } & { 9 } \\ { 13 } & { - 5 } & { - 3 } \\ { 5 } & { - 2 } & { - 1 } \end{array} \right)$
\section{Determinants reminder}
\begin{itemize}
	\item The determinant of a $2\times 2$ matrix $A = \left( \begin{array} { l l } { a } & { b } \\ { c } & { d } \end{array} \right)$ is the number
	$$\operatorname { det } ( A ) = \left| \begin{array} { l l } { a } & { b } \\ { c } & { d } \end{array} \right| = a d - b c$$
	\item If A is a square matrix of order n, then the minor of the entry $a_{ij}$ denoted by $M_{ij}$, is the determinant of the matrix (of order n-1) obtained from A by removing its ith row and jth column
	\item The number $C _ { i j } = ( - 1 ) ^ { i + j } M _ { i j }$ is called the cofactor of $a_{ij}$
	\item If A is an $n\times n$ matrix then the determinant of A can be computed by any of the following \textbf{cofactor expansions} along the ith row and along the jth column, respectively 
	$$\operatorname { det } ( A ) = a _ { i 1 } C _ { i 1 } + a _ { i 2 } C _ { i 2 } + \ldots + a _ { i n } C _ { i n }$$
	$$\operatorname { det } ( A ) = a _ { 1 j } C _ { 1 j } + a _ { 2 j } C _ { 2 j } + \ldots + a _ { n j } C _ { n j }$$
	\item Easy to see: if A has a row of 0s or a column of 0s then det(A)=0
	\item Easy to see: it holds that $det(A)=det(A^T)$
\end{itemize}
\section{Determinants and elementary row operations}
How do elementary row operations affect the determinant of a square matrix?
\subsection{Theorem}
Let A be an $n\times n$ matrix
\begin{itemize}
	\item If B is obtained from A by multiplying a row by a constant k then $det(B)=k\cdot det(A)$
	\item If B is obtained from A by interchanging two rows then $det(B)=-det(A)$
	\item If B is obtained from A by adding a multiple of one row to another row then $det(B)=det(A)$
\end{itemize}
\subsection{Lemma}
If $A=(a_{ij})$ is an $n\times n$ upper triangular matrix, i.e. $a_{ij}=0$ whenever $i>j$ (all 0s under the diagonal), then $\operatorname { det } ( A ) = a _ { 11 } \cdot a _ { 22 } \cdots a _ { ( n - 1 ) ( n - 1 ) } \cdot a _ { n n }$
\section{Computing determinants}
The previous section suggests a strategy for computing the determinant of a matrix:
\begin{itemize}
	\item Use elementary row operations to transform the matrix into row echelon form
	\item Record how determinant changes during the transformation
	\item The row echelon form is upper triangular, its determinant is easy to find
\end{itemize}
Example:
$$\left| \begin{array} { r r r } { 0 } & { 1 } & { 5 } \\ { 3 } & { - 6 } & { 9 } \\ { 2 } & { 6 } & { 1 } \end{array} \right| = - \left| \begin{array} { r r r } { 3 } & { - 6 } & { 9 } \\ { 0 } & { 1 } & { 5 } \\ { 2 } & { 6 } & { 1 } \end{array} \right| = - 3 \left| \begin{array} { c c c } { 1 } & { - 2 } & { 3 } \\ { 0 } & { 1 } & { 5 } \\ { 2 } & { 6 } & { 1 } \end{array} \right| =- 3 \left| \begin{array} { r r r } { 1 } & { - 2 } & { 3 } \\ { 0 } & { 1 } & { 5 } \\ { 0 } & { 10 } & { - 5 } \end{array} \right| = - 3 \left| \begin{array} { r r r } { 1 } & { - 2 } & { 3 } \\ { 0 } & { 1 } & { 5 } \\ { 0 } & { 0 } & { - 55 } \end{array} \right| = ( - 3 ) \cdot ( - 55 ) = 165$$
We now have two ways of computing determinants:\\
by cofactor expansion (i.e. by definition) and by row reduction (as above)\\
They can be mixed: create many 0s by row reduction and use cofactor expansion
\section{Determinants of elementary matrices}
We have $det(I_n)=1$. The following is a special case of the previous theorem
\subsection{Corollary}
Let E be an $n\times n$ elementary matrix
\begin{itemize}
	\item If E is obtained from $I_n$ by multiplying a row by a constant k then $det(E)=k$
	\item If E is obtained from $I_n$ by interchanging two rows then $det(E)=-1$
	\item If E is obtained from $I_n$ by adding a multiple of one row to another row then $det(E)=1$
\end{itemize}
\subsection{Lemma}
If E and B are $n\times n$ matrices and E is elementary then $det(EB)=det(E)det(B)$
\subsection{Proof}
We consider only the 1st case from the above corollary, the other two are similar. If E is obtained from $I_n$ by multiplying a row by k, then EB is obtained from B by the same operation, so $det(EB)=k\cdot det (B)=det(E)det(B)$
\section{Inconvertibility criterion}
The following theorem is $(1)\Leftrightarrow (5)$ from the theorem about invertible matrices
\subsection{Theorem}
A square matrix A is invertible iff $det(A)\neq 0$
\subsection{Proof}
Let R be the reduced row echelon form of A. We have the following facts:
\begin{itemize}
	\item Either R=I (and $det(R)=1$) of R contains a row of 0s (and $det(R)=0$)
	\item A is invertible iff R=I, by the theorem about invertible matrices $(1)\Leftrightarrow (3)$
	\item We know that $R=E_r\cdots E_2E_1A$ for some elementary matrices $E_i$
	\item $\operatorname { det } ( R ) = \operatorname { det } \left( E _ { r } \right) \cdots \operatorname { det } \left( E _ { 2 } \right) \operatorname { det } \left( E _ { 1 } \right) \operatorname { det } ( A)$ by the previous lemma
	\item $det(E_i)\neq 0$ for all i, so $det(R)$ and $det(A)$ are either both 0 or both non 0
	\item Finally, A is invertible $\Leftrightarrow R = I \Leftrightarrow \operatorname { det } ( R ) \neq 0 \Leftrightarrow \operatorname { det } ( A ) \neq 0$
\end{itemize}
\section{Properties of determinants}
\subsection{Theorem}
If A and B are square matrices of the same size then $det(AB)=det(A)det(B)$
\subsection{Proof}
\begin{itemize}
	\item It can be shown that if A is invertible then neither is AB. In this case $det(A)=det(AB)=0$
	\item Assume that A is invertible, then $A=E_1E_2\cdots E_r$ for some elementary $E_i$
	\item Then $A B = E _ { 1 } E _ { 2 } \cdots E _ { r } B \text { and } \operatorname { det } ( A B ) = \operatorname { det } \left( E _ { 1 } \right) \operatorname { det } \left( E _ { 2 } \right) \cdots \operatorname { det } \left( E _ { r } \right) \operatorname { det } ( B )$
	\item Since $\operatorname { det } ( A ) = \operatorname { det } \left( E _ { 1 } \right) \operatorname { det } \left( E _ { 2 } \right) \cdots \operatorname { det } \left( E _ { r } \right)$, we have the required equality.
\end{itemize}
Applying the above theorem to the case when A is invertible and $B=A^{-1}$, we get
\subsection{Corollary}
If A is invertible then $det(A^{-1})=1/det(A)$\\
\\
Note that $det(A+B)\neq det(A)+det(B)$ in general. Try $A=I_2$ and $B=-I_2$
\section{Inverting a matrix via cofactors/adjoint}
\begin{itemize}
	\item If A is a square matrix of order n, then the \textbf{minor of the entry $a_{ij}$} denoted by $M_{ij}$, is the determinant of the matrix (of order n-1) obtained from A by removing its ith row and jth column
	\item The number $C _ { i j } = ( - 1 ) ^ { i + j } M _ { i j }$ is called the \textbf{cofactor of $a_{ij}$}
	\item The matrix
	$$\operatorname { cof } ( A ) = \left( \begin{array} { c c c c } { C _ { 11 } } & { C _ { 12 } } & { \dots } & { C _ { 1 n } } \\ { C _ { 21 } } & { C _ { 22 } } & { \cdots } & { C _ { 2 n } } \\ { \vdots } & { \vdots } & { } & { \vdots } \\ { c _ { n 1 } } & { C _ { n 2 } } & { \cdots } & { c _ { n n } } \end{array} \right)$$
	is called the \textbf{matrix of cofactors} of A
	\item The transpose of cof(A) is the \textbf{adjoint} of A, denoted by $adj(A)$
\end{itemize}
\subsection{Theorem}
If A is an invertible matrix then $A^{-1}=\dfrac{1}{det(A)}\cdot adj(A)$
\subsection{Example}
Find the inverse (if it exists) of the following matrix $A = \left( \begin{array} { c c c } { 0 } & { 1 } & { 5 } \\ { 3 } & { - 6 } & { 9 } \\ { 2 } & { 6 } & { 1 } \end{array} \right)$\\
\\
We have computed $det(A)=165$ earlier, so the inverse exists\\
We have:
$$\operatorname { cof } ( A ) = \left( \begin{array} { r r r } { - 60 } & { 15 } & { 30 } \\ { 29 } & { - 10 } & { 2 } \\ { 39 } & { 15 } & { - 3 } \end{array} \right) , \text { so adj } ( A ) = \left( \begin{array} { r r r } { - 60 } & { 29 } & { 39 } \\ { 15 } & { - 10 } & { 15 } \\ { 30 } & { 2 } & { - 3 } \end{array} \right)$$
Therefore,
$$A ^ { - 1 } = \frac { 1 } { 165 } \left( \begin{array} { r r r } { - 60 } & { 29 } & { 39 } \\ { 15 } & { - 10 } & { 15 } \\ { 30 } & { 2 } & { - 3 } \end{array} \right) = \left( \begin{array} { r r r } { - 60 / 165 } & { 29 / 165 } & { 39 / 165 } \\ { 15 / 165 } & { - 10 / 165 } & { 15 / 165 } \\ { 30 / 165 } & { 2 / 165 } & { - 3 / 165 } \end{array} \right)$$

\end{document}