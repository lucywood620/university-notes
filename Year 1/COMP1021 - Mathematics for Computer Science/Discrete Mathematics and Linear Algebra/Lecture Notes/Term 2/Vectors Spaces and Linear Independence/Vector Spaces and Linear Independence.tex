\documentclass{article}[18pt]
\ProvidesPackage{format}
%Page setup
\usepackage[utf8]{inputenc}
\usepackage[margin=0.7in]{geometry}
\usepackage{parselines} 
\usepackage[english]{babel}
\usepackage{fancyhdr}
\usepackage{titlesec}
\hyphenpenalty=10000

\pagestyle{fancy}
\fancyhf{}
\rhead{Sam Robbins}
\rfoot{Page \thepage}

%Characters
\usepackage{amsmath}
\usepackage{amssymb}
\usepackage{gensymb}
\newcommand{\R}{\mathbb{R}}

%Diagrams
\usepackage{pgfplots}
\usepackage{graphicx}
\usepackage{tabularx}
\usepackage{relsize}
\pgfplotsset{width=10cm,compat=1.9}
\usepackage{float}

%Length Setting
\titlespacing\section{0pt}{14pt plus 4pt minus 2pt}{0pt plus 2pt minus 2pt}
\newlength\tindent
\setlength{\tindent}{\parindent}
\setlength{\parindent}{0pt}
\renewcommand{\indent}{\hspace*{\tindent}}

%Programming Font
\usepackage{courier}
\usepackage{listings}
\usepackage{pxfonts}

%Lists
\usepackage{enumerate}
\usepackage{enumitem}

% Networks Macro
\usepackage{tikz}


% Commands for files converted using pandoc
\providecommand{\tightlist}{%
	\setlength{\itemsep}{0pt}\setlength{\parskip}{0pt}}
\usepackage{hyperref}

% Get nice commands for floor and ceil
\usepackage{mathtools}
\DeclarePairedDelimiter{\ceil}{\lceil}{\rceil}
\DeclarePairedDelimiter{\floor}{\lfloor}{\rfloor}

% Allow itemize to go up to 20 levels deep (just change the number if you need more you madman)
\usepackage{enumitem}
\setlistdepth{20}
\renewlist{itemize}{itemize}{20}

% initially, use dots for all levels
\setlist[itemize]{label=$\cdot$}

% customize the first 3 levels
\setlist[itemize,1]{label=\textbullet}
\setlist[itemize,2]{label=--}
\setlist[itemize,3]{label=*}

% Definition and Important Stuff
% Important stuff
\usepackage[framemethod=TikZ]{mdframed}

\newcounter{theo}[section]\setcounter{theo}{0}
\renewcommand{\thetheo}{\arabic{section}.\arabic{theo}}
\newenvironment{important}[1][]{%
	\refstepcounter{theo}%
	\ifstrempty{#1}%
	{\mdfsetup{%
			frametitle={%
				\tikz[baseline=(current bounding box.east),outer sep=0pt]
				\node[anchor=east,rectangle,fill=red!50]
				{\strut Important};}}
	}%
	{\mdfsetup{%
			frametitle={%
				\tikz[baseline=(current bounding box.east),outer sep=0pt]
				\node[anchor=east,rectangle,fill=red!50]
				{\strut Important:~#1};}}%
	}%
	\mdfsetup{innertopmargin=10pt,linecolor=red!50,%
		linewidth=2pt,topline=true,%
		frametitleaboveskip=\dimexpr-\ht\strutbox\relax
	}
	\begin{mdframed}[]\relax%
		\centering
		}{\end{mdframed}}



\newcounter{lem}[section]\setcounter{lem}{0}
\renewcommand{\thelem}{\arabic{section}.\arabic{lem}}
\newenvironment{defin}[1][]{%
	\refstepcounter{lem}%
	\ifstrempty{#1}%
	{\mdfsetup{%
			frametitle={%
				\tikz[baseline=(current bounding box.east),outer sep=0pt]
				\node[anchor=east,rectangle,fill=blue!20]
				{\strut Definition};}}
	}%
	{\mdfsetup{%
			frametitle={%
				\tikz[baseline=(current bounding box.east),outer sep=0pt]
				\node[anchor=east,rectangle,fill=blue!20]
				{\strut Definition:~#1};}}%
	}%
	\mdfsetup{innertopmargin=10pt,linecolor=blue!20,%
		linewidth=2pt,topline=true,%
		frametitleaboveskip=\dimexpr-\ht\strutbox\relax
	}
	\begin{mdframed}[]\relax%
		\centering
		}{\end{mdframed}}
\lhead{MCS - Ioannis}


\begin{document}
\begin{center}
\underline{\huge Vector Spaces and Linear Independence}
\end{center}
\section{Vectors in $\mathbb{R}^n$}
\begin{itemize}
	\item You are familiar with vectors in two and three dimensions
	\item Such a vector can be identified with an (ordered) tuple of real numbers $(a_1,a_2)$ or $(a_1,a_2,a_3)$ respectively
	\item The numbers in the tuple are the \textbf{components} of the vector
	\item The sets of all 2D and 3D vectors are denoted by $\mathbb{R}^2$ and $\mathbb{R}^3$ respectively
	\item Two vectors are equal iff all corresponding coordinates are equal 
	\item Main operations on vectors: addition and multiplication by a scalar
	\begin{itemize}
		\item If $a=(a_1,a_2)$, $b=(b_1,b_2)$ are vectors in $\mathbb{R}^2$ then $a+b=(a_1+b_1,a_2+b_2)$
		\item If k is a scalar (i.e. real number) and $a=(a_1,a_2)\in \mathbb{R}^2$ then $ka=(ka_1,ka_2)$
	\end{itemize}
	\item For example, if $a=(-1,3)$ and $b=(2,1)$ then $2a-5b=(-12,1)$
	\item All the above can be generalised to n-tuples of real numbers, for any fixed n
	\item Notation: $\mathbb { R } ^ { n } = \left\{ \left( a _ { 1 } , \dots , a _ { n } \right) | \text { all } a _ { i } \in \mathbb { R } \right\}$
	\item Note that one can view a vector in $\mathbb{R}^n$ as a $1\times n$ (or $n\times 1$) matrix
\end{itemize}
\section{Norm and dot product in $\mathbb{R}^n$}
\begin{itemize}
	\item The length (aka norm) of a vector $v=(v_1,v_2,...,v_n)\in \mathbb{R}^n$ is defined by the formula
	$$\| \mathbf { v } \| = \sqrt { v _ { 1 } ^ { 2 } + v _ { 2 } ^ { 2 } + \ldots + v _ { n } ^ { 2 } }$$
	\item It holds that
	\begin{itemize}
		\item $||v||\geqslant 0$, and $||v||=0$ iff $v=0$
		\item $||kv||=|k|\cdot ||v||$
	\end{itemize}
	\item A vector of length 1 is called a \textbf{unit vector}
	\item For any vector v, the vector $\dfrac{1}{||v||}v$ is a unit vector in the same direction as v. It is obtained by normalizing v
	\item The dot product (aka inner product) of vectors $u=(u_1,...,u_n)$ and $v=(v_1,...,v_n)$ in $\mathbb{ R }^n$ is defined as
	$$u \cdot v = u _ { 1 } v _ { 1 } + u _ { 2 } v _ { 2 } + \ldots + u _ { n } v _ { n }$$
	Note that $||v||=\sqrt{v\cdot v}$
	\item For example, if $u=(-1,3,5,7)$ and $v=(2,-1,3,-5)\in \mathbb{ R }^4$ then $\mathbf { u } \cdot \mathbf { v } = ( - 1 ) \cdot 2 + 3 \cdot ( - 1 ) + 5 \cdot 3 + 7 \cdot ( - 5 ) = - 25$
\end{itemize}
\section{Properties of dot product}
If u,v and w are vectors in $\mathbb{ R }^n$ then the following properties hold:
\begin{itemize}
	\item $u\cdot v=v\cdot u$ (symmetry)
	\item  $\mathbf { u } \cdot ( \mathbf { v } + \mathbf { w } ) = \mathbf { u } \cdot \mathbf { v } + \mathbf { u } \cdot \mathbf { w } ($ Distributivity $)$
	\item $k ( \mathbf { u } \cdot \mathbf { v } ) = ( k \mathbf { u } ) \cdot v ($ Homogeneity $)$
	\item $\mathbf { v } \cdot \mathbf { v } \geq 0$ and $\mathbf { v } \cdot \mathbf { v } = 0$ iff $\mathbf { v } = \mathbf { 0 } ($ Positivity $)$
\end{itemize}
\subsection{Theorem (Cauchy-Schwarz inequality, without proof)}
If u and $v$ are vectors in $\mathbb { R } ^ { n }$ then $u \cdot v \leq \| u \| \cdot \| v \|$
\subsection{Corollary (Triangle Inequality)}
If $u$ and $v$ are vectors in $\mathbb { R } ^ { n }$ then $\| \mathbf { u } + \mathbf { v } \| \leq \| \mathbf { u } \|$
\section{Orthogonality in $\mathbb{ R }^n$}
\begin{itemize}
	\item Two vectors u and v in $\mathbb{ R }^n$ are orthogonal (or perpendicular) if $u\cdot v=0$
	\item Example: vectors $u=(-2,3,1,4)$ and $v=(1,2,0.-1)$ in $\mathbb{ R }^4$ are orthogonal because $u\cdot v=(-2)\cdot 1+3\cdot 2+1\cdot 0+4\cdot (-1)=0$
\end{itemize}
\subsection{Theorem (projection theorem)}
If u and $a\neq 0$ are vectors in $\mathbb{ R }^n$ then u can be uniquely expressed as $u=w_1+w_2$ where $w_1=ka$ and a and $w_2$ are orthogonal
\subsubsection{Proof}
Let $k = ( \mathbf { u } \cdot \mathbf { a } ) / \| \mathbf { a } \| ^ { 2 } , \mathbf { w } _ { 1 } = k \mathbf { a } ,$ and $\mathbf { w } _ { 2 } = \mathbf { u } - \mathbf { w } _ { 1 }$\\
Check than $a\cdot w_2=0$\\
The vector $w_1$ is called the orthogonal projection of u on a
\subsection{Theorem (Pythagoras' theorem in $\mathbb{ R }^n$)}
If u and v are orthogonal vectors in $\mathbb{ R }^n$ then $\| \mathbf { u } + \mathbf { v } \| ^ { 2 } = \| \mathbf { u } \| ^ { 2 } + \| \mathbf { v } \| ^ { 2 }$
\subsubsection{Proof}
Since u and v are orthogonal, we have $u\cdot v=0$, hence 
$$\| \mathbf { u } + \mathbf { v } \| ^ { 2 } = ( \mathbf { u } + \mathbf { v } ) \cdot ( \mathbf { u } + \mathbf { v } ) = \| \mathbf { u } \| ^ { 2 } + 2 ( \mathbf { u } \cdot \mathbf { v } ) + \| \mathbf { v } \| ^ { 2 } = \| \mathbf { u } \| ^ { 2 } + \| \mathbf { v} \| ^ { 2 }$$
\section{General (real) vector spaces }
\subsection{Definition}
Let V be a set equipped with operations of "addition" and "multiplication my scalars", that is, for every $u,v\in V$ and every $k\in \mathbb{ R }$
\begin{itemize}
	\item the "sum" $u+v\in V$ is defined, and
	\item the "product" $ku\in V$ is defined
\end{itemize}
V is called a (real) vector space, or linear space, if the following axioms hold:
\begin{enumerate}
	\item $u+v=v+u$
	\item $u+(v+w)=(u+v)+w$
	\item there is an element $0\in V$ such that $u+0=0+u=u$ for all u
	\item For each $u\in V$, there is $-u\in V$ such that $u+(-u)=(-u)+u=0$
	\item $k(u+v)=ku+kv$
	\item $(k+m)u=ku+mu$
	\item $k(mu)=(km)u$
	\item $1u=u$
\end{enumerate}
The elements from V are called vectors
\subsection{Examples of vector spaces}
\begin{itemize}
	\item $\mathbb { R } ^ { n } = \left\{ \left( a _ { 1 } , \ldots , a _ { n } \right) | \text { all } a _ { i } \in \mathbb { R } \right\}$
	\item The set $\mathbb{ R }^\infty$ of all infinite sequences $u=(u_1,u_2,...,u_n,...)$ is a vector space with operations of point-wise addition and multiplication (just as in $\mathbb{ R }^n$)
	$$\left( u _ { 1 } , u _ { 2 } , \ldots , u _ { n } , \ldots \right) + \left( v _ { 1 } , v _ { 2 } , \ldots , v _ { n } , \ldots \right) = \left( u _ { 1 } + v _ { 1 } , u _ { 2 } + v _ { 2 } , \ldots , u _ { n } + v _ { n } , \ldots \right)$$
	$$k \left( u _ { 1 } , u _ { 2 } , \ldots , u _ { n } , \ldots \right) = \left( k u _ { 1 } , k u _ { 2 } , \ldots , k u _ { n } , \ldots \right)$$
	\item All matrices of size $m\times n$ form a vector space, denoted $\mathbb{M}_{mn}$, with the usual operations of matrix addition and multiplication by scalars
\end{itemize}
\section{Subspaces}
\subsection{Definition}
A subset W of a vector space V is called a \textbf{subspace} of V is W itself is a vector space, with operations inherited from V
\begin{itemize}
	\item To verify that W is a subspace of V, we don't need to check all 8 axioms
	\item We only need to check that W is closed under the operations of V, that is, if $u,v\in W$ and $k\in \mathbb{ R }$ then $u+1\in W$ and $ku\in W$
\end{itemize}
Examples:
\begin{itemize}
	\item $\{0\}$ is always a subspace (the zero subspace) of any vector space
	\item For any fixed $a\in \mathbb{ R }^n$, the set $\{ka|k\in \mathbb{ R }\}$ is a subspace of $\mathbb{ R }^n$. Indeed, if $u=k_1a$ and $v=k_2a$ then $u+v=(k_1+k_2)a$ and $ku=k(k_1a)=(kk_1)a$
	\item The solution set of a homogeneous linear system $Ax=0$ with n variables is a subspace of $\mathbb{ R }^n$. Indeed, if u and v are solutions, i.e. $Au=0$ and $Av=0$ then $A(u+v)=Au+Av=0$ and, for any k, $A(ku)=k(Au)=0$ 
\end{itemize}
Non example, Invertible $n\times n$ matrices do not form a subspace of $\mathbb{M}_{nm}$
\subsection{Lemma}
If $W_1,W_2,...,W_r$ are subspaces of V then so is $W_1\cap W_2 \cap ... \cap W_r$
\subsection{Proof}
If vectors $u,v$ are in $W_1\cap W_2 \cap ... \cap W_r$ then they belong to each $W_i$. Since each $W_i$ is a subspace, $u+v$ belongs to $W_i$. Hence $u+v\in W_1\cap W_2 \cap ... \cap W_r$.\\
The proof for multiplication by scalars is similar
\section{Linear Combinations}
Say that a vector $w\in V$ is a linear combination of vectors $v_1,...,v_r\in V$ if $w=k_1v_1+k_2v_2+...+k_rv_r$ for some scalars $k_1,...,k_r$
\subsection{Theorem}
If $S=\{v_1,...,v_r\}$ is a non empty subset of a vector space V then
\begin{itemize}
	\item The set $W = \left\{ \sum _ { i = 1 } ^ { r } k _ { i } \mathbf { v } _ { i } | k _ { i } \in \mathbb { R } \right\}$ of all linear combinations of the vectors in S is a subspace of V
	\item This set W is the (inclusion wise) smallest subspace of V that contains S
\end{itemize}
Inclusion wise minimal - The set in the collection that is not a superset of any other set in the collection
The set W is called a \textbf{span} of S, it is denoted by $\operatorname{span}(S)$ or $\operatorname{span}(v_1,...,v_r)$
\section{Spanning $\mathbb{ R }^n$}
\begin{itemize}
	\item The standard unit vectors in $\mathbb{ R }^n$ are
	$$\mathbf { e } _ { 1 } = ( 1,0,0 , \ldots , 0 ) , \mathbf { e } _ { 2 } = ( 0,1,0 , \ldots , 0 ) , \ldots , \mathbf { e } _ { n } = ( 0,0,0 , \ldots 1 )$$
	They span $\mathbb{ R }^n$ because any vector $(a_1,a_2,...,a_n)\in \mathbb{ R }^n$ can be represented as
	$$\left( a _ { 1 } , a _ { 2 } , \ldots , a _ { n } \right) = a _ { 1 } \mathbf { e } _ { 1 } + a _ { 2 } \mathbf { e } _ { 2 } + \ldots + a _ { n } \mathbf { e } _ { n }$$
	\item How do we test whether a given set of n vectors spans $\mathbb{ R }^n$? Let's take n=3
	\begin{itemize}
		\item Vectors $v_1,v_2,v_3$ span $\mathbb{ R }^3$ iff vectors $e_1,e_2,e_3$ can be expressed as linear combinations of the $v_i's$
		\item Let $A=[v_1|v_2|v_3]$ be the matrix whose columns are the vectors $v_1,v_2,v_3$
		\item The identity matrix $I_3$ can be written as $I_3=[e_1|e_2|e_3]$
		\item The vectors $e_1,e_2,e_3$ can be expressed as a linear combination of $v_i$'s iff there is a $3\times 3$ matrix B such that $AB=I_3$
		\item So, $v_1,v_2,v_3$ span $\mathbb{ R }^3$ iff the matrix $A=[v_1|v_2|v_3]$ is invertible
		\item Hence, we only need to check whether $det(A)\neq 0$
	\end{itemize}
\end{itemize}
\section{Linear in(dependence)}
\subsection{Definition}
Vectors $v_1,...,v_r$ are called linearly independent if
$$k _ { 1 } \mathbf { v } _ { 1 } + k _ { 2 } \mathbf { v } _ { 2 } + \ldots + k _ { r } \mathbf { v } _ { r } = \mathbf { 0 } \Rightarrow k _ { 1 } = k _ { 2 } = \ldots = k _ { r } = 0$$
Otherwise, they are linearly dependent
\subsection{Explanation}
\begin{itemize}
	\item Standard unit vectors in $\mathbb{ R }^n$ are linearly independent. Indeed, if $k _ { 1 } \mathbf { e } _ { 1 } + k _ { 2 } \mathbf { e } _ { 2 } + \ldots + k _ { n } \mathbf { e } _ { n } = \left( k _ { 1 } , k _ { 2 } , \ldots , k _ { n } \right) = 0$ then $k _ { 1 } = k _ { 2 } = \ldots = k _ { r } = 0$
	\item Determine whether vectors $v_1=(1,-2,3), v_2=(5,6,-1)$, and $v_3=(3,2,1)$ in $\mathbb{ R }^3$ are linearly independent\\
	Assume that $k_1v_1+k_2v_2+k_3v_3=0$ This can be written as the linear system
	$$\begin{aligned} k _ { 1 } &+ 5 k _ { 2 } & + 3 k _ { 3 } & = 0 \\ - 2 k _ { 1 } &+ 6 k _ { 2 } & + 2 k _ { 3 } & = 0 \\ 3 k _ { 1 } & - k _ { 2 } & + k _ { 3 } & = 0 \end{aligned}$$
	Let A be the matrix of this system. By Theorem bout invertible matrices, the system has only the trivial solution $k_1=k_2=k_3=0$ iff $det(A)\neq0$, hence, the vectors are linearly dependent
\end{itemize}
\subsection{Theorem}
A set S of two or more vectors is linearly dependent iff at least one of the vectors is expressible as a linear combination of the other vectors in S
\subsection{Proof}
Let $S = \left\{ \mathbf { v } _ { 1 } , \ldots , \mathbf { v } _ { r } \right\} . \text { Let } k _ { 1 } \mathbf { v } _ { 1 } + k _ { 2 } \mathbf { v } _ { 2 } + \ldots + k _ { r } \mathbf { v } _ { r } = \mathbf { 0 }$ and $k_i\neq 0$ for some i. Let $k_3$ be the first non zero coefficient. Then $\mathbf { v } _ { s } = - \frac { k _ { s + 1 } } { k _ { s } } \mathbf { v } _ { s + 1 } - \ldots - \frac { k _ { r } } { k _ { s } } \mathbf { v } _ { r }$ . The other direction of very easy
\subsection{Theorem}
Let $S=\{v_1,...,v_r\}$ be a subset of $\mathbb{ R }^n$. If $r>n$ then S is linearly dependent
\subsection{Proof}
Assume that $k_1v_1+k_2v_2+...+k_rv_r=0$\\
As in the example in the previous section, this can be written as a linear system. This is a homogeneous linear system with more variables than equations. Hence it has a non-trivial solution, so the vectors in S are linearly dependent.
\end{document}