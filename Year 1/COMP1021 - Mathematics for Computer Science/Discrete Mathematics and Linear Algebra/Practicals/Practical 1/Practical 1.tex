\documentclass{article}[18pt]
\ProvidesPackage{format}
%Page setup
\usepackage[utf8]{inputenc}
\usepackage[margin=0.7in]{geometry}
\usepackage{parselines} 
\usepackage[english]{babel}
\usepackage{fancyhdr}
\usepackage{titlesec}
\hyphenpenalty=10000

\pagestyle{fancy}
\fancyhf{}
\rhead{Sam Robbins}
\rfoot{Page \thepage}

%Characters
\usepackage{amsmath}
\usepackage{amssymb}
\usepackage{gensymb}
\newcommand{\R}{\mathbb{R}}

%Diagrams
\usepackage{pgfplots}
\usepackage{graphicx}
\usepackage{tabularx}
\usepackage{relsize}
\pgfplotsset{width=10cm,compat=1.9}
\usepackage{float}

%Length Setting
\titlespacing\section{0pt}{14pt plus 4pt minus 2pt}{0pt plus 2pt minus 2pt}
\newlength\tindent
\setlength{\tindent}{\parindent}
\setlength{\parindent}{0pt}
\renewcommand{\indent}{\hspace*{\tindent}}

%Programming Font
\usepackage{courier}
\usepackage{listings}
\usepackage{pxfonts}

%Lists
\usepackage{enumerate}
\usepackage{enumitem}

% Networks Macro
\usepackage{tikz}


% Commands for files converted using pandoc
\providecommand{\tightlist}{%
	\setlength{\itemsep}{0pt}\setlength{\parskip}{0pt}}
\usepackage{hyperref}

% Get nice commands for floor and ceil
\usepackage{mathtools}
\DeclarePairedDelimiter{\ceil}{\lceil}{\rceil}
\DeclarePairedDelimiter{\floor}{\lfloor}{\rfloor}

% Allow itemize to go up to 20 levels deep (just change the number if you need more you madman)
\usepackage{enumitem}
\setlistdepth{20}
\renewlist{itemize}{itemize}{20}

% initially, use dots for all levels
\setlist[itemize]{label=$\cdot$}

% customize the first 3 levels
\setlist[itemize,1]{label=\textbullet}
\setlist[itemize,2]{label=--}
\setlist[itemize,3]{label=*}

% Definition and Important Stuff
% Important stuff
\usepackage[framemethod=TikZ]{mdframed}

\newcounter{theo}[section]\setcounter{theo}{0}
\renewcommand{\thetheo}{\arabic{section}.\arabic{theo}}
\newenvironment{important}[1][]{%
	\refstepcounter{theo}%
	\ifstrempty{#1}%
	{\mdfsetup{%
			frametitle={%
				\tikz[baseline=(current bounding box.east),outer sep=0pt]
				\node[anchor=east,rectangle,fill=red!50]
				{\strut Important};}}
	}%
	{\mdfsetup{%
			frametitle={%
				\tikz[baseline=(current bounding box.east),outer sep=0pt]
				\node[anchor=east,rectangle,fill=red!50]
				{\strut Important:~#1};}}%
	}%
	\mdfsetup{innertopmargin=10pt,linecolor=red!50,%
		linewidth=2pt,topline=true,%
		frametitleaboveskip=\dimexpr-\ht\strutbox\relax
	}
	\begin{mdframed}[]\relax%
		\centering
		}{\end{mdframed}}



\newcounter{lem}[section]\setcounter{lem}{0}
\renewcommand{\thelem}{\arabic{section}.\arabic{lem}}
\newenvironment{defin}[1][]{%
	\refstepcounter{lem}%
	\ifstrempty{#1}%
	{\mdfsetup{%
			frametitle={%
				\tikz[baseline=(current bounding box.east),outer sep=0pt]
				\node[anchor=east,rectangle,fill=blue!20]
				{\strut Definition};}}
	}%
	{\mdfsetup{%
			frametitle={%
				\tikz[baseline=(current bounding box.east),outer sep=0pt]
				\node[anchor=east,rectangle,fill=blue!20]
				{\strut Definition:~#1};}}%
	}%
	\mdfsetup{innertopmargin=10pt,linecolor=blue!20,%
		linewidth=2pt,topline=true,%
		frametitleaboveskip=\dimexpr-\ht\strutbox\relax
	}
	\begin{mdframed}[]\relax%
		\centering
		}{\end{mdframed}}
\lhead{MCS - Discrete Mathematics and Linear Algebra}


\begin{document}
\begin{center}
\underline{\huge Practical 1}
\end{center}
\section{Question 1}
\textit{Prove that, for every positive integer n,}
$$1 \cdot 2 + 2 \cdot 3 + . . . + n(n + 1) = n(n + 1)(n + 2)/3$$
Basis Step: The statement is valid for n=1 as $1\cdot2=1(1+1)(1+2)/3$\\
\\
Assume the statement holds true for $n=k$:
$$1 \cdot 2 + 2 \cdot 3 + . . . + k(k + 1) = k(k + 1)(k + 2)/3$$
Is it true for $n=k+1$
$$1 \cdot 2 + 2 \cdot 3 + . . . + k(k + 1) + (k+1)(k+2) = ?$$
$$k(k+1)(k+2)/3+(k+1)(k+2)=?$$
$$[k(k+1)(k+2)+3(k+1)(k+2)]/3=?$$
$$(k^3+6k^2+11k+6)/3=?$$
$$[(k+1)(k^2+5k+6)]/3=?$$
$$(k+1)(k+2)(k+3)/3$$
Since the statement is \textbf{valid for n=1}, and that is \textbf{valid for n=k+1 if it is valid for n=k}, we conclude that it is \textbf{valid for all positive integers n}
\section{Question 2}
\textit{Prove that 3 divides $n^3 + 2n$ for every positive integer n}\\
Basis Step: The statement is valid for n=1 as $1^3+2\times1=3$\\
\\
Assume that the statement holds true for n=k\\
3 divides $k^3+2k$ for every positive integer k\\
\\
Is it true for $n=k+1$?\\
$$(k+1)^3+2(k+1)$$
$$k^3+3k^2+3k+1+(2k+2)$$
$$k^3+3k^2+5k+3$$
$$(k^3+2k)+3k^2+3k+3$$
$$(k^3+2k)+3(k^2+k+1)$$
As already assumed that $k^3+2k$ is divisible by 3, and $3(k^2+k+1)$ is divisible by 3, it must be true for $n=k+1$\\
\\
Since the statement is \textbf{valid for n=1}, and that is \textbf{valid for n=k+1 if it is valid for n=k}, we conclude that it is \textbf{valid for all positive integers n}







\end{document}