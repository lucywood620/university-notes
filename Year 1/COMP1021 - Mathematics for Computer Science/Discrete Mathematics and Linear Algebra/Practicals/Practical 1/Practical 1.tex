\documentclass{article}[18pt]
\input{../../../../format}
\lhead{MCS - Discrete Mathematics and Linear Algebra}


\begin{document}
\begin{center}
\underline{\huge Practical 1}
\end{center}
\section{Question 1}
\textit{Prove that, for every positive integer n,}
$$1 \cdot 2 + 2 \cdot 3 + . . . + n(n + 1) = n(n + 1)(n + 2)/3$$
Basis Step: The statement is valid for n=1 as $1\cdot2=1(1+1)(1+2)/3$\\
\\
Assume the statement holds true for $n=k$:
$$1 \cdot 2 + 2 \cdot 3 + . . . + k(k + 1) = k(k + 1)(k + 2)/3$$
Is it true for $n=k+1$
$$1 \cdot 2 + 2 \cdot 3 + . . . + k(k + 1) + (k+1)(k+2) = ?$$
$$k(k+1)(k+2)/3+(k+1)(k+2)=?$$
$$[k(k+1)(k+2)+3(k+1)(k+2)]/3=?$$
$$(k^3+6k^2+11k+6)/3=?$$
$$[(k+1)(k^2+5k+6)]/3=?$$
$$(k+1)(k+2)(k+3)/3$$
Since the statement is \textbf{valid for n=1}, and that is \textbf{valid for n=k+1 if it is valid for n=k}, we conclude that it is \textbf{valid for all positive integers n}
\section{Question 2}
\textit{Prove that 3 divides $n^3 + 2n$ for every positive integer n}\\
Basis Step: The statement is valid for n=1 as $1^3+2\times1=3$\\
\\
Assume that the statement holds true for n=k\\
3 divides $k^3+2k$ for every positive integer k\\
\\
Is it true for $n=k+1$?\\
$$(k+1)^3+2(k+1)$$
$$k^3+3k^2+3k+1+(2k+2)$$
$$k^3+3k^2+5k+3$$
$$(k^3+2k)+3k^2+3k+3$$
$$(k^3+2k)+3(k^2+k+1)$$
As already assumed that $k^3+2k$ is divisible by 3, and $3(k^2+k+1)$ is divisible by 3, it must be true for $n=k+1$\\
\\
Since the statement is \textbf{valid for n=1}, and that is \textbf{valid for n=k+1 if it is valid for n=k}, we conclude that it is \textbf{valid for all positive integers n}







\end{document}