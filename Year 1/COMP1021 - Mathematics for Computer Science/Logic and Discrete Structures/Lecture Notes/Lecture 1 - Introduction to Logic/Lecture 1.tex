\documentclass{article}[18pt]
\ProvidesPackage{format}
%Page setup
\usepackage[utf8]{inputenc}
\usepackage[margin=0.7in]{geometry}
\usepackage{parselines} 
\usepackage[english]{babel}
\usepackage{fancyhdr}
\usepackage{titlesec}
\hyphenpenalty=10000

\pagestyle{fancy}
\fancyhf{}
\rhead{Sam Robbins}
\rfoot{Page \thepage}

%Characters
\usepackage{amsmath}
\usepackage{amssymb}
\usepackage{gensymb}
\newcommand{\R}{\mathbb{R}}

%Diagrams
\usepackage{pgfplots}
\usepackage{graphicx}
\usepackage{tabularx}
\usepackage{relsize}
\pgfplotsset{width=10cm,compat=1.9}
\usepackage{float}

%Length Setting
\titlespacing\section{0pt}{14pt plus 4pt minus 2pt}{0pt plus 2pt minus 2pt}
\newlength\tindent
\setlength{\tindent}{\parindent}
\setlength{\parindent}{0pt}
\renewcommand{\indent}{\hspace*{\tindent}}

%Programming Font
\usepackage{courier}
\usepackage{listings}
\usepackage{pxfonts}

%Lists
\usepackage{enumerate}
\usepackage{enumitem}

% Networks Macro
\usepackage{tikz}


% Commands for files converted using pandoc
\providecommand{\tightlist}{%
	\setlength{\itemsep}{0pt}\setlength{\parskip}{0pt}}
\usepackage{hyperref}

% Get nice commands for floor and ceil
\usepackage{mathtools}
\DeclarePairedDelimiter{\ceil}{\lceil}{\rceil}
\DeclarePairedDelimiter{\floor}{\lfloor}{\rfloor}

% Allow itemize to go up to 20 levels deep (just change the number if you need more you madman)
\usepackage{enumitem}
\setlistdepth{20}
\renewlist{itemize}{itemize}{20}

% initially, use dots for all levels
\setlist[itemize]{label=$\cdot$}

% customize the first 3 levels
\setlist[itemize,1]{label=\textbullet}
\setlist[itemize,2]{label=--}
\setlist[itemize,3]{label=*}

% Definition and Important Stuff
% Important stuff
\usepackage[framemethod=TikZ]{mdframed}

\newcounter{theo}[section]\setcounter{theo}{0}
\renewcommand{\thetheo}{\arabic{section}.\arabic{theo}}
\newenvironment{important}[1][]{%
	\refstepcounter{theo}%
	\ifstrempty{#1}%
	{\mdfsetup{%
			frametitle={%
				\tikz[baseline=(current bounding box.east),outer sep=0pt]
				\node[anchor=east,rectangle,fill=red!50]
				{\strut Important};}}
	}%
	{\mdfsetup{%
			frametitle={%
				\tikz[baseline=(current bounding box.east),outer sep=0pt]
				\node[anchor=east,rectangle,fill=red!50]
				{\strut Important:~#1};}}%
	}%
	\mdfsetup{innertopmargin=10pt,linecolor=red!50,%
		linewidth=2pt,topline=true,%
		frametitleaboveskip=\dimexpr-\ht\strutbox\relax
	}
	\begin{mdframed}[]\relax%
		\centering
		}{\end{mdframed}}



\newcounter{lem}[section]\setcounter{lem}{0}
\renewcommand{\thelem}{\arabic{section}.\arabic{lem}}
\newenvironment{defin}[1][]{%
	\refstepcounter{lem}%
	\ifstrempty{#1}%
	{\mdfsetup{%
			frametitle={%
				\tikz[baseline=(current bounding box.east),outer sep=0pt]
				\node[anchor=east,rectangle,fill=blue!20]
				{\strut Definition};}}
	}%
	{\mdfsetup{%
			frametitle={%
				\tikz[baseline=(current bounding box.east),outer sep=0pt]
				\node[anchor=east,rectangle,fill=blue!20]
				{\strut Definition:~#1};}}%
	}%
	\mdfsetup{innertopmargin=10pt,linecolor=blue!20,%
		linewidth=2pt,topline=true,%
		frametitleaboveskip=\dimexpr-\ht\strutbox\relax
	}
	\begin{mdframed}[]\relax%
		\centering
		}{\end{mdframed}}
\lhead{Mathematics for Computer Science }


\begin{document}
\begin{center}
\underline{\huge Introduction to Logic}
\end{center}

\section{What is logic?}
There are two parts of logic
\begin{itemize}
\item Formal Language - Making statements about certain objects
\item Formal System - Reasoning about the properties of objects
\end{itemize}
Objective of logic:
\begin{itemize}
\item To carry out \textbf{precise} and \textbf{rigorous} arguments about \textbf{assertions} and \textbf{proofs} and to \textbf{implement} these arguments and \textbf{proofs}
\item We need a \textbf{language} whose structure (\textbf{syntax}) can be \textbf{precisely} described and whose meaning (\textbf{semantics}) can be \textbf{unambiguously} defined
\end{itemize}
\textbf{Proof System} - A system of \textbf{deduction} by which \textbf{proofs} can be constructed\\
\textbf{Semantics} - A notion of \textbf{meaning} by which the \textbf{truth} of some property of some object can be determined\\



\textbf{Propositional logic}- Joining propositions to create more complicated propositions\\
\textbf{First order logic} - Statement is broken down into a subject and a \textbf{predicate}, the predicate \textbf{modifies} the subject\\
\\
Logic comprises of 3 components:
\begin{itemize}
\item \textbf{Syntax} - The definition of the formulae of the logic
\item \textbf{Semantics} - The association of meaning and truth to the formulae of the logic
\item \textbf{Proof System} - The manipulation of formulae according to a system of rules
\end{itemize}
What is desired from these components is:
\begin{itemize}
\item \textbf{Completeness} - All the "true" (semantics) formulae should be "provable"(syntax, proof system)
\item \textbf{Soundness} - A formula that is "provable" (syntax, proof system) should be "true" (semantics)
\end{itemize}
\section{Logic in  action}
\subsection{Programming languages}
\textbf{Syntax} - Exactly which combinations of symbols constitute a legible program\\
***

\subsection{Circuits}
\textbf{Logic Gate} - Performs a Boolean operation on digital inputs and provides the result of this operation as an output\\
\textbf{Logic Circuit} - A collection of logic gates connected together\\
\textbf{Truth Table} - A model of the intended behaviour of a logic gate
\\
Propositional logic can be used to create a truth table\\
A specification of a logic circuit as a truth table may be incomplete due to a large circuit with multiple combinations that could be given, and not all need to be tested
\subsection{Databases}
\textbf{Database} - A structured collection of logical records\\
\textbf{Database Query Language} - A language for asking and answering questions of the structured data\\
The expressive power of SQL is very closely related to that of predicate logic
\subsection{Formal Methods}
\textbf{Formal Methods} - The use of mathematically based techniques for the specification and verification of computer systems - these prove that programs have certain property and don't just rely on testing\\
\textbf{Model Checking} - A branch of formal methods where a computer system is modelled as a mathematical structure then a specific property that this system might have is expressed by a formula of some logic\\
Model checking used for rapid prototyping systems\\
\\
Examples of use of formal methods:
\begin{itemize}
\item Microprocessor design
\item Design of data-communications protocol software
\item Critical Software
\item Operating Systems
\end{itemize}

\begin{center}
{\huge \textbf{Note: Non examinable from here on}}
\end{center}
\section{A history lesson}
Gottlob Frege attempted to show that all of mathematics grew out of logic, in doing so inventing first order predicate logic. His intention was to show that there was:
\begin{itemize}
\item A set of axioms (basic and obvious facts)
\item A set of logical rules (unambiguous)
\end{itemize}
This was so that all true mathematical statements could be expressed in this way
\section{Hilbert's Programme}
Hilbert believed that:
\begin{itemize}
\item All mathematical statements could be written in a formal language and manipulated according to formal rules
\item All true mathematical statements could be proved in the formalism
\item There would be an "Algorithm" to decide whether or not a mathematical statement is true or not
\end{itemize}
\section{Logic and Computation}
Part of Hilbert's Programme was to solve the Entscheidungsproblem which is an algorithm that takes the inputs:
\begin{itemize}
 \item A description of a formal language 
 \item A mathematical statement in the language
 \end{itemize} 
Then give the output true or false depending if the statement is true or false\\
\\
Later Alonzo Church and Alan Turing independently proved that a general solution to the problem is impossible, both using different methods
\section{Computational Complexity}
Graph colouring is NP Complete as the correct colouring cannot be easily determined for a graph, however given a coloured graph it can be checked quickly.

\end{document}