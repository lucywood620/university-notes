\documentclass{article}[18pt]
\ProvidesPackage{format}
%Page setup
\usepackage[utf8]{inputenc}
\usepackage[margin=0.7in]{geometry}
\usepackage{parselines} 
\usepackage[english]{babel}
\usepackage{fancyhdr}
\usepackage{titlesec}
\hyphenpenalty=10000

\pagestyle{fancy}
\fancyhf{}
\rhead{Sam Robbins}
\rfoot{Page \thepage}

%Characters
\usepackage{amsmath}
\usepackage{amssymb}
\usepackage{gensymb}
\newcommand{\R}{\mathbb{R}}

%Diagrams
\usepackage{pgfplots}
\usepackage{graphicx}
\usepackage{tabularx}
\usepackage{relsize}
\pgfplotsset{width=10cm,compat=1.9}
\usepackage{float}

%Length Setting
\titlespacing\section{0pt}{14pt plus 4pt minus 2pt}{0pt plus 2pt minus 2pt}
\newlength\tindent
\setlength{\tindent}{\parindent}
\setlength{\parindent}{0pt}
\renewcommand{\indent}{\hspace*{\tindent}}

%Programming Font
\usepackage{courier}
\usepackage{listings}
\usepackage{pxfonts}

%Lists
\usepackage{enumerate}
\usepackage{enumitem}

% Networks Macro
\usepackage{tikz}


% Commands for files converted using pandoc
\providecommand{\tightlist}{%
	\setlength{\itemsep}{0pt}\setlength{\parskip}{0pt}}
\usepackage{hyperref}

% Get nice commands for floor and ceil
\usepackage{mathtools}
\DeclarePairedDelimiter{\ceil}{\lceil}{\rceil}
\DeclarePairedDelimiter{\floor}{\lfloor}{\rfloor}

% Allow itemize to go up to 20 levels deep (just change the number if you need more you madman)
\usepackage{enumitem}
\setlistdepth{20}
\renewlist{itemize}{itemize}{20}

% initially, use dots for all levels
\setlist[itemize]{label=$\cdot$}

% customize the first 3 levels
\setlist[itemize,1]{label=\textbullet}
\setlist[itemize,2]{label=--}
\setlist[itemize,3]{label=*}

% Definition and Important Stuff
% Important stuff
\usepackage[framemethod=TikZ]{mdframed}

\newcounter{theo}[section]\setcounter{theo}{0}
\renewcommand{\thetheo}{\arabic{section}.\arabic{theo}}
\newenvironment{important}[1][]{%
	\refstepcounter{theo}%
	\ifstrempty{#1}%
	{\mdfsetup{%
			frametitle={%
				\tikz[baseline=(current bounding box.east),outer sep=0pt]
				\node[anchor=east,rectangle,fill=red!50]
				{\strut Important};}}
	}%
	{\mdfsetup{%
			frametitle={%
				\tikz[baseline=(current bounding box.east),outer sep=0pt]
				\node[anchor=east,rectangle,fill=red!50]
				{\strut Important:~#1};}}%
	}%
	\mdfsetup{innertopmargin=10pt,linecolor=red!50,%
		linewidth=2pt,topline=true,%
		frametitleaboveskip=\dimexpr-\ht\strutbox\relax
	}
	\begin{mdframed}[]\relax%
		\centering
		}{\end{mdframed}}



\newcounter{lem}[section]\setcounter{lem}{0}
\renewcommand{\thelem}{\arabic{section}.\arabic{lem}}
\newenvironment{defin}[1][]{%
	\refstepcounter{lem}%
	\ifstrempty{#1}%
	{\mdfsetup{%
			frametitle={%
				\tikz[baseline=(current bounding box.east),outer sep=0pt]
				\node[anchor=east,rectangle,fill=blue!20]
				{\strut Definition};}}
	}%
	{\mdfsetup{%
			frametitle={%
				\tikz[baseline=(current bounding box.east),outer sep=0pt]
				\node[anchor=east,rectangle,fill=blue!20]
				{\strut Definition:~#1};}}%
	}%
	\mdfsetup{innertopmargin=10pt,linecolor=blue!20,%
		linewidth=2pt,topline=true,%
		frametitleaboveskip=\dimexpr-\ht\strutbox\relax
	}
	\begin{mdframed}[]\relax%
		\centering
		}{\end{mdframed}}
\lhead{MCS - LDS}


\begin{document}
\begin{center}
\underline{\huge An overview of first order logic}
\end{center}
\section{Predicates and atomic formulae}
Whereas the fundamental building block in propositional logic is the propositional variable, with first order logic it is the \textbf{predicate} (we have already been introduced to predicates when we studied relations)\\
A \textbf{predicate symbol} (or \textbf{relation symbol}) is just a symbol with an associated arity e.g., P might be defined as a predicate symbol with arity r\\
Given a predicate symbol P of arity r and some variables $x_1,x_2,...,x_r$ (where it might be the case that some of these variables are the same), the formula
$$P(x_1,x_2,...,x_r)$$
is an \textbf{atomic formula} of first order logic\\
In order to know whether this atomic formula is \textbf{true} or \textbf{false}, we need to be given an r-ary relation P' over some domain D, say, and values $v_1,v_2,...,v_r$ from D for $x_1,x_2,...,x_r$
\section{Atomic formula: an example}
Suppose T is the ternary relation symbol. Then
$$T(x,y,x)$$
is an \textbf{atomic formula}\\
Now let T' be the following ternary relation on $\mathbb{N}$
$$\{(u,v,w): u,v,w\in \mathbb{N}, u=2v \  \text{and w is even}\}$$
and consider the interpretation (or model) of $T(x,y,x)$ in T' with x=6 and y=3. This is true\\
\\
In this case, we write $(T',x=6,y=3) \models T(x,y,x)$ or sometimes $\left( \mathbb { N } , T ^ { \prime } , x = 6 , y = 3 \right) = T ( x , y , x )$
\subsection{Lecture example}
$$(u,v,w)=(6,3,6)$$
This is in  T' as the last digit is even and the first digit is twice the second digit
\section{Building formula}
Given some atomic formulae, we can build more complicated formulae from these atomic formulae by using the usual connectives of propositional logic, namely $\lnot,\land, \lor, \Rightarrow$ and $\Leftrightarrow$. For example,
$$E \left( x _ { 1 } , x _ { 2 } \right) \vee \left( T \left( x _ { 1 } , x _ { 1 } , x _ { 3 } \right) \Rightarrow \neg E \left( x _ { 2 } , x _ { 3 } \right) \right)$$
is a formula of first-order logic, where E is a predicate symbol of arity 3, and $x_1,x_2$ and $x_3$ are variables.\\
In order to interpret this formula, we need a binary relation for E, a ternary relation for T and values for $x_1,x_2$ and $x_3$. The domains of the relations for E and T must be the same.\\
\\
Is the following interpretation true?\\
$E = \left\{ \left( u _ { 1 } , u _ { 2 } \right) \in \mathbb { N } ^ { 2 } : u _ { 1 } \leq u _ { 2 } \right\} , T = \left\{ \left( u _ { 1 } , u _ { 2 } , u _ { 3 } \right) \in \mathbb { N } ^ { 3 } : u _ { 1 } \cdot u _ { 2 } = u _ { 3 } \right\}$\\
and $x_1=3,x_2=2$ and $x_3=9$\\
\\
This is not true as $F\lor T\Rightarrow F$, which is false.\\
\\
Not only do we allow formulae such at $P(x_1,x_2,...,x_r)$ as atomic formulae, but we are also allowed formulae of the form $x=y$, where x and y are variables (this constitutes all atomic formulae)\\
The semantics of x=y is that this atomic formula is true only if the value of x is equal to the value of y (in an interpretation)
\subsection{Example}
Let E be a binary predicate symbol. Consider the formula
$$( E ( x , y ) \wedge E ( y , z ) ) \Rightarrow \neg ( x = z )$$
(We sometimes abbreviate $\lnot(x=z)$ by $x\neq z$)\\
If E is interpreted as:
$$E = \left\{ ( x , y ) \in \mathbb { N } ^ { 2 } : x < y \right\}$$
and x=5, y=7 and z=11 then is the formula true in this interpretation?\\
\\
$E(x,y)$ is true as $5<7$\\
$E(y,z)$ is true as $7<11$\\
$\lnot(x=z)$ is true as $5\neq 11$\\
So the whole formula is true
\subsection{More on building formulae}
Formulae  built from atomic formulae are called quantifier-free formula and the free variables are those variables appearing in a formula
\section{Quantifiers}
Given a formula with \textbf{free} variables, we can now "quantify" over these variables using the universal quantifier (or the for-all quantifier) $\forall$ and the existential quantifier (or the exists quantifier) $\exists$\\
Suppose that $\phi(x)$ is a quantifier-free formula with one free variable x. Then $\forall x \phi (x)$ is a formula of first order logic and has no free variables. The variable x is a \textbf{bound} variable in $\forall x \phi (x)$
\subsection{Example}
Suppose that Q is a unary relation symbol. Consider the formula $\forall x Q(x)$. Is it true for the following interpretations?
\begin{itemize}
	\item Interpret Q as the relation $Q=\{u \in \mathbb{ N }: \text{u is even}\}$\\
	This is not true as there are odd natural numbers
	\item Interpret Q as the relation $Q\{u\in \mathbb{ N }: \text{u has a square root}\}$\\
	This is true as every natural number is a square root
\end{itemize}
Then thinking about the formula $\exists x Q(x)$ and the relation $Q=\{u \in \mathbb{ N }: \text{u is even}\}$, it is then true as 2 is even.\\
\\
A formula e.g. $\lnot(\forall x Q(x))$ is the same as $\exists x \lnot Q(x)$ 
\section{More complicated formulae}
We can apply quantifiers to quantifier-free formula even when there is more than one free variable in the formula.\\
Let $\phi(x_1,x_2,...,x_r)$ be a quantifier free formula with free variables $x_1,x_2,...,x_r$. Then the following are two examples of formulae of first order logic.
$$\forall x _ { 1 } \phi \left( x _ { 1 } , x _ { 2 } , \ldots , x _ { r } \right) \quad \exists x _ { 3 } \phi \left( x _ { 1 } , x _ { 2 } , \ldots , x _ { r } \right)$$
The first has free variables $x_2,...,x_r$ and bound variable $x_1$ (as it is outside the $\phi$); and the second has free variables $x_1,x_2,x_4,...,x_r$ and bound variable $x_3$\\
An \textbf{interpretation} of such formulae are as before except that relations and values for the free variables have to be supplied in order for any interpretation to make sense.
\subsection{Examples}
\begin{itemize}
\item If $\phi(x)$ is the formula $\forall y(x=y \lor E(x,y))$ and $E = \left\{ ( u , v ) \in \mathbb { N } ^ { 2 } : u < v \right\}$ then
$$( E , x = 0 ) \models \phi ( x )$$
$$\forall y (0=y \lor E(0,y))$$
$$\forall y (0=y \lor 0<y) \text{True as only natural numbers is either} \ 0 \ or \ >0$$
but
$$(E,x=v)\models \lnot\phi (x) \ \text{wherever} \ v\neq 0$$
\item If $\phi(x)$ is the formula $\exists y E(y,x)$ and $E = \left\{ ( u , v ) \in \mathbb { N } ^ { 2 } : u < v \right\}$ then we have
$$(E,x=0)\models \lnot \phi (x)$$
but
$$(E,x=v)\models \phi (x) \ \text{wherever} \  v\neq 0$$
\end{itemize}
\section{More complicated formulae}
We can also apply quantifiers to formulae already involving quantifiers.\\
Consider the formula $\forall y(x=y \lor E(x,y))$. There is one free variable and we can quantify over this free variable; like this
$$\exists x \forall y ( x = y \vee E ( x , y ) )$$
Let the binary relation $E = \left\{ ( u , v ) \in \mathbb { N } ^ { 2 } : u < v \right\}$\\
For formula above to be \textbf{true} in this interpretation, we need that there exists some value $u\in \mathbb{ N }$ for x such that for any value $v\in \mathbb{ N }$ for y, we have that $u=v\lor E(u,v)$; that is, either $u=v$ or $u<v$\\
There clearly does exist such a value u, namely $u=0$. However, if $E = \left\{ ( u , v ) \in \mathbb { Z } ^ { 2 } : u < v \right\}$ then the formula is \textbf{false} as given any value for x, there is always some integer that is strictly less than this value for x\\
\\
We can also build new formula, using the usual \textbf{propositional connectives}, from existing formulae that involve \textbf{quantifiers}. Consider the formula
$$\exists x \forall y ( x = y \vee E ( x , y ) ) , \text { and } \exists x \forall w ( x = w \vee E ( w , x ) )$$
If we \textbf{interpret} E as $\left\{ ( u , v ) \in \mathbb { N } ^ { 2 } : u < v \right\}$ then is the following formula true?
$$\exists x \forall y ( x = y \vee E ( x , y ) ) \wedge \exists x \forall w ( x = w \vee E ( w , x ) )$$
What if we interpret E as
$$\{ ( u , v ) \in \{ 0,1 , \ldots , 9 \} \times \{ 0,1 , \ldots , 9 \} : u < v \}$$
Notice how the same variable, x, is quantified twice in the same formula yet the two quantifications are entirely separate!




\end{document}