\documentclass{article}[18pt]
\ProvidesPackage{format}
%Page setup
\usepackage[utf8]{inputenc}
\usepackage[margin=0.7in]{geometry}
\usepackage{parselines} 
\usepackage[english]{babel}
\usepackage{fancyhdr}
\usepackage{titlesec}
\hyphenpenalty=10000

\pagestyle{fancy}
\fancyhf{}
\rhead{Sam Robbins}
\rfoot{Page \thepage}

%Characters
\usepackage{amsmath}
\usepackage{amssymb}
\usepackage{gensymb}
\newcommand{\R}{\mathbb{R}}

%Diagrams
\usepackage{pgfplots}
\usepackage{graphicx}
\usepackage{tabularx}
\usepackage{relsize}
\pgfplotsset{width=10cm,compat=1.9}
\usepackage{float}

%Length Setting
\titlespacing\section{0pt}{14pt plus 4pt minus 2pt}{0pt plus 2pt minus 2pt}
\newlength\tindent
\setlength{\tindent}{\parindent}
\setlength{\parindent}{0pt}
\renewcommand{\indent}{\hspace*{\tindent}}

%Programming Font
\usepackage{courier}
\usepackage{listings}
\usepackage{pxfonts}

%Lists
\usepackage{enumerate}
\usepackage{enumitem}

% Networks Macro
\usepackage{tikz}


% Commands for files converted using pandoc
\providecommand{\tightlist}{%
	\setlength{\itemsep}{0pt}\setlength{\parskip}{0pt}}
\usepackage{hyperref}

% Get nice commands for floor and ceil
\usepackage{mathtools}
\DeclarePairedDelimiter{\ceil}{\lceil}{\rceil}
\DeclarePairedDelimiter{\floor}{\lfloor}{\rfloor}

% Allow itemize to go up to 20 levels deep (just change the number if you need more you madman)
\usepackage{enumitem}
\setlistdepth{20}
\renewlist{itemize}{itemize}{20}

% initially, use dots for all levels
\setlist[itemize]{label=$\cdot$}

% customize the first 3 levels
\setlist[itemize,1]{label=\textbullet}
\setlist[itemize,2]{label=--}
\setlist[itemize,3]{label=*}

% Definition and Important Stuff
% Important stuff
\usepackage[framemethod=TikZ]{mdframed}

\newcounter{theo}[section]\setcounter{theo}{0}
\renewcommand{\thetheo}{\arabic{section}.\arabic{theo}}
\newenvironment{important}[1][]{%
	\refstepcounter{theo}%
	\ifstrempty{#1}%
	{\mdfsetup{%
			frametitle={%
				\tikz[baseline=(current bounding box.east),outer sep=0pt]
				\node[anchor=east,rectangle,fill=red!50]
				{\strut Important};}}
	}%
	{\mdfsetup{%
			frametitle={%
				\tikz[baseline=(current bounding box.east),outer sep=0pt]
				\node[anchor=east,rectangle,fill=red!50]
				{\strut Important:~#1};}}%
	}%
	\mdfsetup{innertopmargin=10pt,linecolor=red!50,%
		linewidth=2pt,topline=true,%
		frametitleaboveskip=\dimexpr-\ht\strutbox\relax
	}
	\begin{mdframed}[]\relax%
		\centering
		}{\end{mdframed}}



\newcounter{lem}[section]\setcounter{lem}{0}
\renewcommand{\thelem}{\arabic{section}.\arabic{lem}}
\newenvironment{defin}[1][]{%
	\refstepcounter{lem}%
	\ifstrempty{#1}%
	{\mdfsetup{%
			frametitle={%
				\tikz[baseline=(current bounding box.east),outer sep=0pt]
				\node[anchor=east,rectangle,fill=blue!20]
				{\strut Definition};}}
	}%
	{\mdfsetup{%
			frametitle={%
				\tikz[baseline=(current bounding box.east),outer sep=0pt]
				\node[anchor=east,rectangle,fill=blue!20]
				{\strut Definition:~#1};}}%
	}%
	\mdfsetup{innertopmargin=10pt,linecolor=blue!20,%
		linewidth=2pt,topline=true,%
		frametitleaboveskip=\dimexpr-\ht\strutbox\relax
	}
	\begin{mdframed}[]\relax%
		\centering
		}{\end{mdframed}}
\lhead{MCS - LDS}


\begin{document}
\begin{center}
\underline{\huge First order Logic - Logical Equivalence}
\end{center}
\section{Logical Equivalence}
Two formulae $\phi$ and $\psi$ are logically equivalent of they are true for the same set of models, in which case we write $\phi\equiv\psi$\\
All logical equivalences from propositional logic give rise to equivalences in first-order logic: for example, as
\begin{center}
	$p\Rightarrow q \equiv \lnot p \lor q$ for any propositional variables p and q
\end{center}
We must have that
\begin{center}
	$\phi\Rightarrow\psi \equiv \lnot \phi \lor \psi$, for any first-order formulae $\phi$ and $\psi$
\end{center}
Note, however, that care must be taken as to exactly what an interpretation is when we "plug in" formulae as in the previous example: if
\begin{itemize}
	\item $\phi$ is over the signature consisting of the binary relation symbol E and the constant symbol C
	\item $\psi$ is over the signature consisting of the binary relation symbol E and the ternary relation symbol M
\end{itemize}
Then an interpretation for $\lnot \phi \lor \psi$ is over the signature consisting of the symbols E, C and M
\section{Some tricks}
\subsection{Renaming variables}
Consider some first-order formula of the form $\forall x \phi(x)$ where $y$ does not appear in $\phi(x)$\\
If we replace every occurrence of the variable $x$ in $\phi$ with the variable $y$, we claim that $\forall x \phi (x) \equiv \forall y \phi (y)$:
\begin{itemize}
	\item Let I be some interpretation for $\forall x \phi (x)$ in which $\forall x \phi (x)$ is true
	\item For every value u in the domain of I, we have that $(I,x=u)\models \phi (x)$
	\item So, for every value u in the domain of I, we have that $(I,y=u)\models \phi (y)$
	\item Hence, I is an interpretation in which $\forall y \phi (y)$ is true.\\
	Similarly, if I is an interpretation in which $\forall y \phi (y)$ is true then I is an interpretation in which $\forall x \phi (x)$ is true
\end{itemize}
In general, and by the same reasoning, if we ever have some formula $\phi$ in which there is a quantification, $\forall x$, say, then we can replace
\begin{itemize}
	\item Every occurrence of x in the scope of this quantification with the variable $y$
	\item The quantification $\forall x$ by $\forall y$
\end{itemize}
So long as $y$ does not appear in $\phi$, without changing the semantics\\
Of course, the same can be said of $\exists x \phi (x)$ and, more generally, any formula containing a quantification $\exists x$\\
\\
But, consider the formula $\exists x E(x,y)$\\
If we simply replace $x$ with $y$ and $\exists x$ with $\exists y$ then we get $\exists y E(y,y)$ which is semantically very different from $\exists x E(x,y)$
\subsection{Substitution}
Consider the formula $\phi$ in which there is contained a sub formula $\psi$\\
Suppose further that $\psi$ has free variables $x_1,x_2,...,x_k$\\
If $\psi$ is logically equivalent to a formula $\chi (x_1,x_2,...,x_k)$ then we can replace $\psi$ in $\phi$ with the formula $\chi$ and not change the semantics
\section{Some common equivalences}
More interesting re the interactions between the quantifiers $\forall$ and $\exists$ and the logical connectives $\lnot, \lor$ and $\land$\\
Consider the formula $\lnot \forall x \phi$, where $\phi(x)$ is a first-order formula with free variable x\\
Let I be some interpretation fr $\lnot \forall x \phi$. We have that:
\begin{itemize}
	\item $I\models \lnot \forall x \phi$\\
	Iff it is not the case that $I\models \forall x \phi$\\
	Iff it is not the case that for every value $u$ in the domain of I, we have that $\phi(u)$ holds in I\\
	Iff there exists some value $u$ in the domain of I such that $\lnot \phi (u)$ holds in I\\
	Iff $I\models \exists x\lnot \phi$
\end{itemize}
($\phi(u)$ is shorthand for saying that $x$ is to be interpreted as $u$).\\
So for evert first-order formula $\phi(x)$
$$\lnot \forall x \phi \equiv \exists x \lnot \phi$$
Consider the formula $\lnot \exists x \phi$, where $\phi (x)$ is a first-order formula with free variable x\\
Let I be some interpretation for $\lnot \exists x \phi$. We have that:
\begin{itemize}
	\item $I\models \lnot \exists x \phi$\\
	Iff it is not the case that $I\models \exists x \phi$\\
	Iff it is not the case that there exists some value $u$ in the domain of I such that $\phi(u)$ holds in I\\
	Iff for every value $u$ in the domain of I, we have that $\lnot \phi(u)$ holds in I\\
	Iff $I\models \forall x \lnot \phi$
\end{itemize}
So, for every first order formula $\phi(x)$:
$$\lnot \exists x \phi\equiv \forall x \lnot \phi$$
\textbf{General rule}: negations can be "pushed through" universal quantifiers if we change the universal quantifier to an existential quantifier\\
\textbf{Another general rule}: negations can be "pushed through" existential quantifiers if we change the existential quantifier to a universal quantifier
\subsection{Example}
Consider the formula $\neg \exists x \forall y ( \neg E ( x , y ) \vee M ( y , y , z , x ) )$. We have
$$\neg \exists x \forall y ( \neg E ( x , y ) \vee M ( y , y , z , x ) )$$
$$\equiv \forall x \neg \forall y ( \neg E ( x , y ) \vee M ( y , y , z , x ) )$$
$$\equiv \forall x \exists y \neg ( \neg E ( x , y ) \vee M ( y , y , z , x ) )$$
\section{More complicated equivalences}
Consider the formula $\forall x \phi \land \exists y \psi$ where $\phi(x)$ and $\psi (y)$ are first order formulae with free variables x and y, respectively.\\
By renaming bound variables (if necessary), we may assume that x does not appear in $\psi$ and y does not appear in $\phi$\\
Let I be some interpretation for $\forall x \phi \land \exists y \psi$\\
We have $I\models \forall x \phi\land \exists y\psi$ iff $I\models \forall x \phi$ and $I\models \exists y\psi$:
\begin{itemize}
	\item $I\models \forall x \phi$ iff no matter which value from the domain of I we give to the variable x, we have that $\phi(x)$ holds in I
	\item $I\models \exists y\psi$ iff there exists some value from the domain of I for the variables y such that $\psi(y)$ holds in I
\end{itemize} 
Thus: $I\models \forall x \phi \land \exists y \psi$ iff:\\
No matter which value we give to x, we have that $\phi(x)$ holds in I, and there exists some value for y such that $\psi (y)$ holds in I\\
\\
Consider $\forall x \exists y(\phi \land \psi)$\\
Suppose that $I\models \forall x \exists y (\phi \land \psi)$\\
Choose any $u$ for $x$. There exists a v for y such that $\phi(u)\land \psi(v)$ holds.\\
So, $I\models \forall x \phi \land \exists y \psi$\\
Hence, $\forall x \phi \land \exists y \psi\equiv \forall x \exists y (\phi \land \psi)$.\\
Indeed, by the same token, $I\models \forall x \phi \land \exists y \psi$ iff $I\models \exists y\forall x (\phi \land \psi)$.\\
\textbf{General rule}: Quantifications can be "pulled out" from inside logical connectives and the order of quantifiers doesn't matter, so long as the names of the quantified variables are not used elsewhere
\section{Some more complicated equivalences}
\subsection{Example 1}
If we assume that
\begin{itemize}
	\item $x$ does not appear in $\psi$ and $\chi$
	\item $y$ does not appear in $\phi$ and $\chi$
	\item $z$ does not appear in $\phi$ and $\psi$
\end{itemize}
Applying this general rule yields:
$$\begin{aligned} ( \forall x \phi \wedge \exists y \psi ) \vee \forall z \chi & \equiv \forall x \exists y ( \phi \wedge \psi ) \vee \forall Z \chi \\ & \equiv \forall x \exists y \forall z ( ( \phi \wedge \psi ) \vee \chi ) \end{aligned}$$
\subsection{Example 2}
Consider the formula $( \forall x \phi \vee \forall x \psi ) \wedge \exists x \chi$\\
We can rename two of the bound occurrences of x to get
$$( \forall x \phi ( x ) \vee \forall y \psi ( y ) ) \wedge \exists z\chi ( z )$$
(assuming y and z do not appear in $\phi$ and $\chi$, respectively).\\
Now we get the equivalent formulae
$$\begin{aligned} ( \forall x \phi ( x ) \vee \forall y \psi ( y ) ) & \wedge \exists \chi ( z ) \\ & \equiv \forall x \forall y ( \phi ( x ) \vee \psi ( y ) ) \wedge \exists \chi \chi ( z ) \\ & \equiv \forall x \forall y \exists z ( \phi ( x ) \vee \psi ( y ) \wedge \chi ( z ) ) \end{aligned}$$
\section{Be careful when applying general rules}
Great care has to be taken when manipulating quantifiers:
\begin{itemize}
	\item The order of the quantification matters
	\item Consider other occurrences of a quantified variable \textbf{outside the scope}
\end{itemize}
\subsection{Example}
Consider the first-order sentence $\forall x \exists y E(x,y)$\\
Let I be the interpretation with domain $\{1,2,3,4\}$ where $E = \{ ( 1,2 ) , ( 2,3 ) , ( 3,4 ) , ( 4,1 ) \}$\\
Clearly, $I\models \forall x \exists y E(x,y)$ but $I\not \models \exists x \forall y E(x,y)$\\
Consider the first order sentence $\forall x \exists y E(x,y)\land \forall z\lnot E(z,z)$\\
Whilst $I\models \forall x \exists y E(x,y)\land \forall z \lnot E(z,z)$\\
$I = \forall z \forall x \exists y ( E ( x , y ) \wedge \neg E ( z , z ) )$\\
$I = \forall x \forall z \exists y ( E ( x , y ) \wedge \neg E ( z , z ) )$\\
It is not the case that $I = \forall z \exists y \forall x ( E ( x , y ) \wedge \neg E ( z , z ) )$
\section{More on bound occurrences}
Consider the first order formula $\forall x \exists y E ( x , y ) \wedge \exists x U ( x )$\\
It does not make sense to pull the quantifiers out, as we could get $\forall x \exists y \exists x ( E ( x , y ) \wedge U ( x ) )$\\
Actually, semantically this second sentence is logically equivalent to
$$\exists y \exists x ( E ( x , y ) \wedge U ( x ) )$$
(as existentially quantified x "overwrites" the universally quantified x) which is certainly not equivalent to the sentence we started with. To see this, consider the interpretation where the domain is $\{ 1,2 \} , E = \{ ( 1,2 ) \}$ and $U = \{ 1 \}$\\
We need to ensure that the two original bound occurrences of x have "nothing to do with each other". In order to ensure this, we need to rename one of them:
$$\begin{aligned} \forall x \exists y E ( x , y ) \wedge \exists x U ( x ) & \equiv \forall x \exists y E ( x , y ) \wedge \exists z \mathrm { U } ( z ) \\ & \equiv \forall x \exists y \exists z ( E ( x , y ) \wedge U ( z ) ) \end{aligned}$$
\end{document}