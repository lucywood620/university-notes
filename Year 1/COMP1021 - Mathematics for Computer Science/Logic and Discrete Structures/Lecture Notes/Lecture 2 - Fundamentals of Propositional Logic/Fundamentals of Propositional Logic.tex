\documentclass{article}[18pt]
\ProvidesPackage{format}
%Page setup
\usepackage[utf8]{inputenc}
\usepackage[margin=0.7in]{geometry}
\usepackage{parselines} 
\usepackage[english]{babel}
\usepackage{fancyhdr}
\usepackage{titlesec}
\hyphenpenalty=10000

\pagestyle{fancy}
\fancyhf{}
\rhead{Sam Robbins}
\rfoot{Page \thepage}

%Characters
\usepackage{amsmath}
\usepackage{amssymb}
\usepackage{gensymb}
\newcommand{\R}{\mathbb{R}}

%Diagrams
\usepackage{pgfplots}
\usepackage{graphicx}
\usepackage{tabularx}
\usepackage{relsize}
\pgfplotsset{width=10cm,compat=1.9}
\usepackage{float}

%Length Setting
\titlespacing\section{0pt}{14pt plus 4pt minus 2pt}{0pt plus 2pt minus 2pt}
\newlength\tindent
\setlength{\tindent}{\parindent}
\setlength{\parindent}{0pt}
\renewcommand{\indent}{\hspace*{\tindent}}

%Programming Font
\usepackage{courier}
\usepackage{listings}
\usepackage{pxfonts}

%Lists
\usepackage{enumerate}
\usepackage{enumitem}

% Networks Macro
\usepackage{tikz}


% Commands for files converted using pandoc
\providecommand{\tightlist}{%
	\setlength{\itemsep}{0pt}\setlength{\parskip}{0pt}}
\usepackage{hyperref}

% Get nice commands for floor and ceil
\usepackage{mathtools}
\DeclarePairedDelimiter{\ceil}{\lceil}{\rceil}
\DeclarePairedDelimiter{\floor}{\lfloor}{\rfloor}

% Allow itemize to go up to 20 levels deep (just change the number if you need more you madman)
\usepackage{enumitem}
\setlistdepth{20}
\renewlist{itemize}{itemize}{20}

% initially, use dots for all levels
\setlist[itemize]{label=$\cdot$}

% customize the first 3 levels
\setlist[itemize,1]{label=\textbullet}
\setlist[itemize,2]{label=--}
\setlist[itemize,3]{label=*}

% Definition and Important Stuff
% Important stuff
\usepackage[framemethod=TikZ]{mdframed}

\newcounter{theo}[section]\setcounter{theo}{0}
\renewcommand{\thetheo}{\arabic{section}.\arabic{theo}}
\newenvironment{important}[1][]{%
	\refstepcounter{theo}%
	\ifstrempty{#1}%
	{\mdfsetup{%
			frametitle={%
				\tikz[baseline=(current bounding box.east),outer sep=0pt]
				\node[anchor=east,rectangle,fill=red!50]
				{\strut Important};}}
	}%
	{\mdfsetup{%
			frametitle={%
				\tikz[baseline=(current bounding box.east),outer sep=0pt]
				\node[anchor=east,rectangle,fill=red!50]
				{\strut Important:~#1};}}%
	}%
	\mdfsetup{innertopmargin=10pt,linecolor=red!50,%
		linewidth=2pt,topline=true,%
		frametitleaboveskip=\dimexpr-\ht\strutbox\relax
	}
	\begin{mdframed}[]\relax%
		\centering
		}{\end{mdframed}}



\newcounter{lem}[section]\setcounter{lem}{0}
\renewcommand{\thelem}{\arabic{section}.\arabic{lem}}
\newenvironment{defin}[1][]{%
	\refstepcounter{lem}%
	\ifstrempty{#1}%
	{\mdfsetup{%
			frametitle={%
				\tikz[baseline=(current bounding box.east),outer sep=0pt]
				\node[anchor=east,rectangle,fill=blue!20]
				{\strut Definition};}}
	}%
	{\mdfsetup{%
			frametitle={%
				\tikz[baseline=(current bounding box.east),outer sep=0pt]
				\node[anchor=east,rectangle,fill=blue!20]
				{\strut Definition:~#1};}}%
	}%
	\mdfsetup{innertopmargin=10pt,linecolor=blue!20,%
		linewidth=2pt,topline=true,%
		frametitleaboveskip=\dimexpr-\ht\strutbox\relax
	}
	\begin{mdframed}[]\relax%
		\centering
		}{\end{mdframed}}
\lhead{MCS - Logic and Discrete Structures}


\begin{document}
\begin{center}
\underline{\huge Fundamentals of Propositional Logic}
\end{center}
\section{The Rudiments of Propositional Logic}
Propositional Logic:
\begin{itemize}
\item The most fundamental logic, lying at the heart of many other things
\item Formalises day-to-day, common sense reasoning
\end{itemize}
Key to propositional logic are \textbf{propositions}:
\begin{itemize}
\item Declarative sentences can be either \textbf{true} or \textbf{false}
\end{itemize}
Propositions are represented by \textbf{propositional variables} (\textbf{Boolean variables, atoms})
\begin{itemize}
\item Usually letters such as \textbf{x,Y,a} or subscripted letters such as $\mathbf{x_2,Y_0,a_1}$
\item Which can take a truth value T (true) or F(false)
\end{itemize}
Syntax\\
New propositions called \textbf{formulae} or \textbf{Boolean formulae} or \textbf{propositional formulae} or \textbf{compound propositions} are formed from propositional variables and formulae by the use of logical operators
\begin{itemize}
\item $\land$ - conjunction(and)
\item $\lor$ - disjunction(or)
\item $\lnot$ - negation(not)
\item $\Rightarrow$ - implies (if left statement true, then right statement must be true, if LHS false, whole statement becomes true)
\item $\Leftrightarrow$ - if and only if (iff)
\end{itemize}
\section{Some formulae}
\subsection{Construction}
The operators $\land, \lor , \Rightarrow$ and $\Leftrightarrow$ take two propositional formulae $\varphi$ and $\psi$\\
\\
The operator $\lnot$ takes one propositional formula $\varphi$ and yields a new one
\begin{itemize}
\item $\lnot\varphi$
\end{itemize}
\subsection{Use of parentheses}
$(\varphi \land \psi)\lor\chi$ means first build $\varphi\land\psi$ and then build $(\varphi \land \psi)\lor\chi$\\
$\varphi \land (\psi\lor\chi)$ means first build $\psi\lor\chi$ and then build $\varphi \land (\psi\lor\chi)$
\subsection{Some typical well formed formulae}
$$\lnot((\lnot b\land a)\Rightarrow(c\lor\lnot d))$$
$$((a\land\lnot a)\lor((b\lor c)\lor d))\Leftrightarrow d$$
$$(((a\Rightarrow b)\Rightarrow c)\Rightarrow d)$$
\section{Semantics of propositional logic}
Semantics: all propositional variables take the value \textbf{T(True)} or \textbf{F(false)}
\begin{itemize}
\item The value of a formula under some \textbf{truth assignment} is ascertained by using the \textbf{truth tables} for the above logical connectives
\end{itemize}
The truth tables for our logical connectives are as follows:\\
\begin{tabular}{|l|l|l|l|l|l|l|}
\hline
 p&q&$p\land q$&$p\lor q$&$\lnot p$&$p\Rightarrow q$&$p\Leftrightarrow q$  \\ \hline
 T&T&T&T&F&T&T  \\ \hline
 T&F&F&T&F&F&F  \\ \hline
 F&T&F&T&T&T&F  \\ \hline
 F&F&F&F&T&T&T  \\ \hline
\end{tabular}\\ \\
In order to build the truth table of a formula we decompose the formula into sub formulae e.g.:\\
\\
\begin{tabular}{|l|l|llllllllllll}
\cline{1-2}
p & q & ((p                    & $\land$                & $\lnot$                & q)                     & $\lor$                 & p)                     & $\land$                & $\lnot$                 & (p                     & $\lor$                 & $\lnot$                & q                     ) \\ \hline
T & T & \multicolumn{1}{l|}{T} & \multicolumn{1}{l|}{F} & \multicolumn{1}{l|}{F} & \multicolumn{1}{l|}{T} & \multicolumn{1}{l|}{T} & \multicolumn{1}{l|}{T} & \multicolumn{1}{l|}{F} & \multicolumn{1}{l|}{F} & \multicolumn{1}{l|}{T} & \multicolumn{1}{l|}{T} & \multicolumn{1}{l|}{F} & \multicolumn{1}{l|}{T} \\ \hline
T & F & \multicolumn{1}{l|}{T} & \multicolumn{1}{l|}{T} & \multicolumn{1}{l|}{T} & \multicolumn{1}{l|}{F} & \multicolumn{1}{l|}{T} & \multicolumn{1}{l|}{T} & \multicolumn{1}{l|}{F} & \multicolumn{1}{l|}{F} & \multicolumn{1}{l|}{T} & \multicolumn{1}{l|}{T} & \multicolumn{1}{l|}{T} & \multicolumn{1}{l|}{F} \\ \hline
F & T & \multicolumn{1}{l|}{F} & \multicolumn{1}{l|}{F} & \multicolumn{1}{l|}{F} & \multicolumn{1}{l|}{T} & \multicolumn{1}{l|}{F} & \multicolumn{1}{l|}{F} & \multicolumn{1}{l|}{F} & \multicolumn{1}{l|}{T} & \multicolumn{1}{l|}{F} & \multicolumn{1}{l|}{F} & \multicolumn{1}{l|}{F} & \multicolumn{1}{l|}{T} \\ \hline
F & F & \multicolumn{1}{l|}{F} & \multicolumn{1}{l|}{F} & \multicolumn{1}{l|}{T} & \multicolumn{1}{l|}{F} & \multicolumn{1}{l|}{F} & \multicolumn{1}{l|}{F} & \multicolumn{1}{l|}{F} & \multicolumn{1}{l|}{F} & \multicolumn{1}{l|}{F} & \multicolumn{1}{l|}{T} & \multicolumn{1}{l|}{T} & \multicolumn{1}{l|}{F} \\ \hline
\end{tabular}
\\
\\
While this looks very complicated, just follow the logic through and it is simple to do.
\section{Some basic notation}
If we have a propositional formula $\varphi(x_1,x_2,...,x_n)$ then we can call an assignment f of either \textbf{T} or \textbf{F} to each $x_1,x_2,...,x_n$ i.e. a function
$$f:\{x_1,x_2,...,x_n\}\rightarrow \{T,F\}$$
a \textbf{truth assignment (interpretation, valuation)} for $\varphi$\\
\\
We say that $\varphi$ \textbf{evaluates} to \textbf{T}(with respective to \textbf{F} under f. If the row of the truth table for $\varphi$ corresponding to \textbf{f} evaluates to \textbf{T}(with respect to \textbf{F}.\\
\\
for a set of inputs f, if a formula with these inputs comes to true, then f is a satisfying truth assignment of $\varphi$\\
\\
$\varphi$ is the formula under which the inputs f are put under, for example $p\land q$\\
\\
If \textbf{f} evaluates $\varphi$ to \textbf{T} then f \textbf{satisfies} $\varphi$ or is a \textbf{satisfying truth assignment of} $\varphi$ or a \textbf{model} of $\varphi$\\
\\
If $\varphi$ evaluates to \textbf{T} for every f then $\varphi$ is a \textbf{tautology}\\
If $\varphi$ evaluates to \textbf{F} for every f then $\varphi$ is a \textbf{contradiction}\\
\\
A \textbf{literal} is either a propositional variable, say \textbf{x}, or the negation of a propositional variable, say $\lnot$x
\section{Logical equivalence}
Steps in a mathematical proof are often just the replacement of one statement by another (equivalent) statement which says the same thing e.g.\\
"If I don't explain this clearly then the students won't understand" is the same thing as\\
"Either I explain this clearly or the students won't understand"\\
\\
To see this, denote the sub-statement "I don't explain this clearly" as \textbf{X} and\\
denote the sub statement "the students won't understand" as \textbf{Y}\\
\\
The former statement is thus $X\Rightarrow Y$ and the latter:\\
\\
\begin{tabular}{|l|l|lll|llll}
\cline{1-2}
X & Y & X & $\Rightarrow$ & Y & $\lnot$ & X & $\lor$     & Y \\ \hline
T & T & T & \textbf{T}    & T & F       & T & \textbf{T} & T \\ \cline{1-2}
T & F & T & \textbf{F}    & F & F       & T & \textbf{F} & F \\ \cline{1-2}
F & T & F & \textbf{T}    & T & T       & F & \textbf{T} & T \\ \cline{1-2}
F & F & F & \textbf{T}    & F & T       & F & \textbf{T} & F \\ \cline{1-2}
\end{tabular}\\
\\
We say that the two propositional formulae are \textbf{(logically)} equivalent if they have identical truth tables\\
If $\varphi$ and $\psi$ are equivalent then we write $\varphi\equiv\psi$
\section{A spot of practice}
The \textbf{exclusive-OR} is written $X\oplus Y$ and if \textbf{true} iff exactly one of \textbf{X} and \textbf{Y} is \textbf{true}\\
\textit{Prove that $X\oplus Y$ is logically equivalent to both $(X\land\lnot Y)\lor(\lnot X\land Y)$ and $\lnot(X\Leftrightarrow Y)$}\\
\\
\begin{tabular}{|l|l|lll|lllllllll|llll}
\cline{1-2}
X & Y & X & $\oplus$   & Y & (X & $\land$ & $\lnot$ & Y) & $\lor$     & $(\lnot$ & X & $\land$ & Y & $\lnot$    & (X & $\Leftrightarrow$ & Y) \\ \hline
T & T & T & \textbf{F} & T & T  & F       & F       & T  & \textbf{F} & F        & T & F       & T & \textbf{F} & T  & T                 & T  \\ \cline{1-2}
T & F & T & \textbf{T} & F & T  & T       & T       & F  & \textbf{T} & F        & T & F       & F & \textbf{T} & T  & F                 & F  \\ \cline{1-2}
F & T & F & \textbf{T} & T & F  & F       & F       & T  & \textbf{T} & T        & F & T       & T & \textbf{T} & F  & F                 & T  \\ \cline{1-2}
F & F & F & \textbf{F} & F & F  & F       & T       & F  & \textbf{F} & T        & F & F       & F & \textbf{F} & F  & T                 & F  \\ \cline{1-2}
\end{tabular}
\section{De Morgan's Laws}
These are two very useful logical equivalences known as De Morgan's Laws\\
\textbf{De Morgan's Laws} are:
\begin{itemize}
\item $\lnot(X\land Y)\equiv \lnot X\lor\lnot Y$
\item $\lnot(X\lor Y)\equiv \lnot X\land\lnot Y$
\end{itemize}
These formulae are indeed equivalences:\\
\\

\begin{tabular}{|l|l|llll|lllll||llll|lllll}
\cline{1-2}
X & Y & $\lnot$    & (X & $\land$ & Y) & $\lnot$ & X & $\lor$     & $\lnot$ & Y & $\lnot$    & (X & $\lor$ & Y) & $\lnot$ & X & $\land$    & $\lnot$ & y \\ \hline
T & T & \textbf{F} & T  & T       & T  & F       & T & \textbf{F} & F       & T & \textbf{F} & T  & T      & T  & F       & T & \textbf{F} & F       & T \\ \cline{1-2}
T & F & \textbf{T} & T  & F       & F  & F       & T & \textbf{T} & T       & F & \textbf{F} & T  & T      & F  & F       & T & \textbf{F} & T       & F \\ \cline{1-2}
F & T & \textbf{T} & F  & F       & T  & T       & F & \textbf{T} & F       & T & \textbf{F} & F  & T      & T  & T       & F & \textbf{F} & F       & T \\ \cline{1-2}
F & F & \textbf{T} & F  & F       & F  & T       & F & \textbf{T} & T       & F & \textbf{T} & F  & F      & F  & T       & F & \textbf{T} & T       & F \\ \cline{1-2}
\end{tabular}
\begin{itemize}
\item De Morgan's Laws can be applied not just to variables, but to formulae $\varphi$ and $\psi$
\item De Morgan's Laws are often used to simplify formulae with regard to negations
\end{itemize}
\section{Applying De Morgan's Laws}
In fact, not only can De Morgan's Laws be applied to formula, they can be applied to \textbf{sub-formulae} within a formula\\
\\
Take the propositional formula:
$$\lnot(p\lor\lnot(q\land\lnot p))\land\lnot(p\Rightarrow q)$$
and the sub formula
$$\lnot(q\land\lnot p)$$
\\
By De Morgan's Laws
$$\lnot(q\land\lnot p)\equiv \lnot q \lor \lnot \lnot p\equiv \lnot q \lor p$$
\\
So:
$$\lnot(p\lor \lnot(q\land\lnot p))\land \lnot(p\Rightarrow q)\equiv \lnot(p\lor(\lnot q \lor p))\land \lnot (p\Rightarrow q)$$
\\
Indeed, we can always replace any sub-formula of some propositional formula with an \textbf{equivalent formula} without affecting the truth(table) of the original.
\section{A spot of practice}
Consider $\lnot(p\lor\lnot(q\land\lnot p))\land \lnot(p\Rightarrow q)$\\
Can we manipulate it so as to simplify it?\\
\begin{tabular}{l l}
$\lnot(p\lor\lnot(q\land\lnot p))\land \lnot (p\Rightarrow q)$& Apply De Morgan's laws\\
$\lnot(p\lor(\lnot q\lor \lnot\lnot p))\land \lnot(p\Rightarrow q)$&Remove double negation\\
$\lnot(p\lor(\lnot q \lor p))\land \lnot(p\Rightarrow q)$&apply De Morgan's laws\\
$(\lnot p \land \lnot(\lnot q \land p)) \land \lnot (p\Rightarrow q)$&apply De Morgan's laws\\
$(\lnot p \land (\lnot\lnot q \land \lnot p)) \land \lnot(p\Rightarrow q)$&Remove double negation\\
$(\lnot p \land (q \land \lnot p)) \land \lnot (p\Rightarrow q)$& $\Rightarrow$ using $\lor, \lnot$\\
$(\lnot p \land (q \land \lnot p)) \land \lnot(\lnot p \lor q)$& apply De Morgan's Laws\\
$(\lnot p \land (q \land \lnot p)) \land (\lnot\lnot p \land \lnot q)$&remove double negation\\
$(\lnot p \land (q \land \lnot p)) \land p(\land \lnot q)$& Associativity of $\land$\\
$(\lnot p \land q \land \lnot p) \land (p \land \lnot q$& Associativity of $\land$\\
$\lnot p \land q \land \lnot p \land p \land \lnot q$ & Commutativity of $\land$\\
$\lnot p \land \lnot p \land p \land q \land \lnot q$& $X\land \lnot X\equiv F$\\
$\lnot p \land F \land q \land \lnot q$& $F\land \varphi \equiv F$\\
$F$&
\end{tabular}
\section{Generalised De Morgan's Laws}
We can actually generalise De Morgan's laws so that negations can be "pushed inside" conjunction/disjunctions of more than two literals\\
\\
To do this we apply De Morgan's laws to sub formulae of a formula\\
\\
Consider $\lnot(X\lor Y\lor Z)$\\
Rewrite this formula as $\lnot(X\lor(Y\lor Z))$ and denote $Y\lor Z$ by $\varphi$\\
Applying De Morgan's laws to $\lnot(X\lor \varphi)$

\section{Some rules}
$$(p\land q)\land r \equiv p\land (q \land r)$$
$$(p \lor q)\lor r \equiv p \lor (q \lor r)$$
$$p \land \lnot p \equiv F$$
\end{document}