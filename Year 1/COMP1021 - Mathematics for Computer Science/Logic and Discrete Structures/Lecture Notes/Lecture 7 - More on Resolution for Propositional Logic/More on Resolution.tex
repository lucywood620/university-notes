\documentclass{article}[18pt]
\usepackage{../../../../../format}
\lhead{MCS - Logic and Discrete Structures}


\begin{document}
\begin{center}
\underline{\huge More on resolution for Propositional Logic}
\end{center}
\section{Example 1}
\begin{itemize}
	\item Let $\varphi$ be the formula $\neg ( ( p \lor q ) \land ( \neg p \lor q ) \land ( p \lor \neg q ) \land ( \neg p \lor \neg q ) )$
	\item Is $\varphi$ a theorem?
	\item In order to prove this using resolution we negate $\varphi$ and put it in conjunctive normal form if necessary
	\item So $\lnot \varphi$ is the formula $( p \lor q ) \land ( \neg p \lor q ) \land ( p \lor \neg q ) \land ( \neg p \lor \neg q )$ which is in conjunctive normal form already
	\item We will now try to apply resolution on $\lnot\varphi$ until
	\begin{itemize}
		\item Either we infer the empty clause, which means that $\lnot \varphi$ is a contradiction, and hence $\varphi$ is a theorem or,
		\item We do not infer the empty clause but at some point we do not find any new clauses either; in that case we can find a truth assignment that makes $\lnot\varphi$ true, and hence, $\varphi$ false, which means that $\varphi$ is not a theorem
	\end{itemize}
\end{itemize}
Applying resolution:\\
$p\lor q$\\
$\lnot p \lor q$\\
$p\lor \lnot q$\\
$\lnot p\lor \lnot q$\\
\\
$q\lor q$ (1 and 2)\\
$q\lor \lnot q$ (2 and 3)\\
$\lnot q \lor \lnot q$ (3 and 4)\\
$\varnothing$\\
This implies that $\varphi$ is a theorem



\section{Example 2}
\begin{itemize}
	\item Use resolution to prove that if
	\begin{itemize}
		\item "It is not raining or I have my umbrella" $\lnot r \lor u$
		\item "I do not have my umbrella or I do not get wet" $\lnot u \lor \lnot w$
		\item "It is raining or I do not get wet" $r \lor \lnot w$
	\end{itemize}
	then
	\begin{itemize}
		\item I do not get wet
	\end{itemize}
	\item A formula $\varphi\Rightarrow\psi$ is logically equivalent to $\lnot\varphi\lor\psi$
	\begin{itemize}
		\item So that the negation of our formula is $\varphi\land\lnot\psi$
		\begin{itemize}
			\item That is
			$$(\lnot R\lor U)\land(\lnot U\lor \lnot W)\land (R\lor\lnot W)\land W$$
		\end{itemize}
	\end{itemize}
\end{itemize}
So we must apply resolution on clauses - it is already in cnf so is easy to do
$$\lnot R\lor U, \lnot U\lor \lnot W,R\lor\lnot W, W$$
$ $\\
R - It is raining\\
U - I have my umbrella\\
W - I get Wet\\
\\
$\varphi \Rightarrow \psi$\\
$\lnot \varphi \lor \psi$\\
$\lnot(\lnot \varphi \lor \psi)$\\
$\lnot\lnot \varphi \land \lnot \psi$\\
$\varphi \land \lnot \psi$\\
\\
\\
$W, R\lor\lnot W \Rightarrow R$\\
$W, \lnot U\lor \lnot W \Rightarrow \lnot U$\\
R is true, so not R is false, not U is true, so U is false. Is a theorem\\
\\
$u\lor \lnot W$ (1 and 3)\\
$\lnot W \lor \lnot W \Rightarrow \lnot W$ (new 1 and 2) 



\section{Example 3}
Applying resolution to the following set of clauses
$$a\lor b\lor c \qquad a\lor\lnot c\lor d \qquad \lnot a\lor e\lor f \qquad c\lor \lnot e\lor f \qquad c\lor d\lor \lnot f$$
\begin{enumerate}
	\item (1,3) $b\lor c\lor e\lor f$
	\item (2,3) $\lnot c\lor d\lor e \lor f$
	\item (4,n2) $\lnot d \lor f \lor \lnot f$
\end{enumerate}

\end{document}