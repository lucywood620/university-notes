\documentclass{article}[18pt]
\input{../../../../../format}
\lhead{CT - ECC}
\usepackage{tabularx}
\begin{document}
\begin{center}
\underline{\huge Hamming Codes}
\end{center}
\section{Linear Codes}
\subsection{Summary on linear codes}
An $(n,k,d_{min})$-linear code C is a linear subspace of dimension k of $F^n$ with minimum distance $d_{min}$\\
\\
We can represent it by
\begin{itemize}
	\item Generator matrix $\mathbf{G}(k\times n)$ used for encoding 
	$$C = \left\{ \mathbf { m } \mathbf { G } : \mathbf { m } \in F ^ { k } \right\}$$
	mG means the message multiplied by the generator matrix
	\item Parity-check matrix $\mathbf{H}(n-k\times n)$ used for detecting errors
	$$C = \left\{ \mathbf { c } \in F ^ { n } : \mathbf { c H } ^ { \top } = \mathbf { 0 } \right\}$$
\end{itemize}
$$cH^T\Leftrightarrow c_1=c_2=c_3=c_4=...$$
The repetition code and Parity-Check codes are dual to each other, meaning that they have symmetry. The generator matrix of a repetition code is H of a parity-check code and vice versa.
\subsection{Parameters of a linear code}
A \textbf{linear code} is simply any subspace of $F^n$\\
\\
Parameters:
\begin{itemize}
	\item Length n
	\item Dimension k
	\item Redundancy $r=n-k$
	\item Rate $R=k/n$
	\item Minimum distance $d_{min}$ - this shows how many errors can be corrected
	\item Error-correction capability $t=\lfloor (d_{min}-1)/2\rfloor$
\end{itemize}
We usually write $(n,k,d_{min})$-"name of code"
\subsection{Parameters of parity-check and repetition codes}
\begin{tabularx}{\textwidth}{|X|X|X|}
	\hline 
	Parameter & Parity-check & Repetition \\ 
	\hline 
	Length n & n & n \\ 
	\hline 
	Dimension k & n-1 & 1 \\ 
	\hline 
	Rendundancy r & 1 & n-1 \\ 
	\hline 
	Rate R & 1-1/n & 1/n \\ 
	\hline 
	Minimum distance $d_{min}$ & 2 & n \\ 
	\hline 
	Error-correction capability t & 0 & $\lfloor (n-1)/2\rfloor$ \\ 
	\hline 
\end{tabularx} 
\subsection{Minimum distance of a linear code}
\textbf{Theorem:} Viewing the columns of H as vectors in $F^{n-k}$, $d_{min}$ is the minimum number of linearly dependent columns of H\\
\\
E.g. for the $(5,1,5)$-repetition code
$$H_ { \text { repetition } } = \left( \begin{array} { c c c c c } { 1 } & { 0 } & { 0 } & { 0 } & { 1 } \\ { 0 } & { 1 } & { 0 } & { 0 } & { 1 } \\ { 0 } & { 0 } & { 1 } & { 0 } & { 1 } \\ { 0 } & { 0 } & { 0 } & { 1 } & { 1 } \end{array} \right)$$
Any set of four columns is linearly independent, but the set of all five columns is linearly dependent, therefore $d_{min}=5$\\
\\
This is true for all repetition codes, you can see this adding all the columns together, remembering modulo 2, the linear combination is 0.
\section{Hamming codes}
\subsection{The Hamming code of redundancy 3}
\textbf{Definition:} This is the linear code with the parity check matrix
$$H = \left( \begin{array} { l l l l l l l } { 0 } & { 0 } & { 0 } & { 1 } & { 1 } & { 1 } & { 1 } \\ { 0 } & { 1 } & { 1 } & { 0 } & { 0 } & { 1 } & { 1 } \\ { 1 } & { 0 } & { 1 } & { 0 } & { 1 } & { 0 } & { 1 } \end{array} \right)$$
The columns are the integers 1 to 7 written in binary\\
\\
Any two columns are linearly independent; the first 3 are linearly dependent.\\
\\
Therefore $\mathbf{d_{min}=3}$ this is the (7,4,3)-Hamming code\\
\\
So for any matrix, the number of linearly dependent rows is equal to $d_{min}$
\subsection{Generalisation}
For any $r\geqslant 2$ we use the parity-check matrix whose columns are all the integers from 1 to $2^r-1$ in binary
\begin{itemize}
	\item For r=2: the (3,1,3)-repetition code
	\item For r=3: the (7,4,3)-Hamming code
	\item For r=4: the (15,11,4)-Hamming code with parity-check matrix
\end{itemize}
$$\mathbf { H } = \left( \begin{array} { l l l l l l l l l l l l l l l } { 0 } & { 0 } & { 0 } & { 0 } & { 0 } & { 0 } & { 0 } & { 1 } & { 1 } & { 1 } & { 1 } & { 1 } & { 1 } & { 1 } & { 1 } \\ { 0 } & { 0 } & { 0 } & { 1 } & { 1 } & { 1 } & { 1 } & { 0 } & { 0 } & { 0 } & { 0 } & { 1 } & { 1 } & { 1 } & { 1 } \\ { 0 } & { 1 } & { 1 } & { 0 } & { 0 } & { 1 } & { 1 } & { 0 } & { 0 } & { 1 } & { 1 } & { 0 } & { 0 } & { 1 } & { 1 } \\ { 1 } & { 0 } & { 1 } & { 0 } & { 1 } & { 0 } & { 1 } & { 0 } & { 1 } & { 0 } & { 1 } & { 0 } & { 1 } & { 0 } & { 1 } \end{array} \right)$$
Note that $d_{min}=3$ for all r
\subsection{Parameters}
In general a Hamming code has parameters
\begin{itemize}
	\item Length $n=2^r-1$
	\item Dimension $k=2^r-r-1$
	\item Redundancy $r=n-k$
	\item Rate $R=1-r/(2^r-1)$
	\item Minimum distance $d_{min}=3$
	\item Error correction capability $t=1$
\end{itemize}
\subsection{Optimality of Hamming codes}
The \textbf{sphere-packing bound}: If C is a code of length n which can correct at least one error (and hence $d_{min}=3$), then
$$| C | \leq \frac { 2 ^ { n } } { n + 1 }$$
Consider again the spheres showing the errors that can be corrected.\\
The volume of a sphere is $\frac{4}{3}\pi r^3$\\
There are $|C|$ spheres which do not intersect.\\
There are $|C|(n+1)$ vectors in all the spheres\\
$|C|(n+1)\leqslant 2^n$ (can't be greater than the number of vectors)\\
\\
\section{CW}
The hamming encoder is mG\\
\\
Message decoder is just doing the opposite of creating the message in the first question.\\
\\

\end{document}