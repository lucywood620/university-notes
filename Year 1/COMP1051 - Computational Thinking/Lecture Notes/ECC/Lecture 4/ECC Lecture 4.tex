\documentclass{article}[18pt]
\ProvidesPackage{format}
%Page setup
\usepackage[utf8]{inputenc}
\usepackage[margin=0.7in]{geometry}
\usepackage{parselines} 
\usepackage[english]{babel}
\usepackage{fancyhdr}
\usepackage{titlesec}
\hyphenpenalty=10000

\pagestyle{fancy}
\fancyhf{}
\rhead{Sam Robbins}
\rfoot{Page \thepage}

%Characters
\usepackage{amsmath}
\usepackage{amssymb}
\usepackage{gensymb}
\newcommand{\R}{\mathbb{R}}

%Diagrams
\usepackage{pgfplots}
\usepackage{graphicx}
\usepackage{tabularx}
\usepackage{relsize}
\pgfplotsset{width=10cm,compat=1.9}
\usepackage{float}

%Length Setting
\titlespacing\section{0pt}{14pt plus 4pt minus 2pt}{0pt plus 2pt minus 2pt}
\newlength\tindent
\setlength{\tindent}{\parindent}
\setlength{\parindent}{0pt}
\renewcommand{\indent}{\hspace*{\tindent}}

%Programming Font
\usepackage{courier}
\usepackage{listings}
\usepackage{pxfonts}

%Lists
\usepackage{enumerate}
\usepackage{enumitem}

% Networks Macro
\usepackage{tikz}


% Commands for files converted using pandoc
\providecommand{\tightlist}{%
	\setlength{\itemsep}{0pt}\setlength{\parskip}{0pt}}
\usepackage{hyperref}

% Get nice commands for floor and ceil
\usepackage{mathtools}
\DeclarePairedDelimiter{\ceil}{\lceil}{\rceil}
\DeclarePairedDelimiter{\floor}{\lfloor}{\rfloor}

% Allow itemize to go up to 20 levels deep (just change the number if you need more you madman)
\usepackage{enumitem}
\setlistdepth{20}
\renewlist{itemize}{itemize}{20}

% initially, use dots for all levels
\setlist[itemize]{label=$\cdot$}

% customize the first 3 levels
\setlist[itemize,1]{label=\textbullet}
\setlist[itemize,2]{label=--}
\setlist[itemize,3]{label=*}

% Definition and Important Stuff
% Important stuff
\usepackage[framemethod=TikZ]{mdframed}

\newcounter{theo}[section]\setcounter{theo}{0}
\renewcommand{\thetheo}{\arabic{section}.\arabic{theo}}
\newenvironment{important}[1][]{%
	\refstepcounter{theo}%
	\ifstrempty{#1}%
	{\mdfsetup{%
			frametitle={%
				\tikz[baseline=(current bounding box.east),outer sep=0pt]
				\node[anchor=east,rectangle,fill=red!50]
				{\strut Important};}}
	}%
	{\mdfsetup{%
			frametitle={%
				\tikz[baseline=(current bounding box.east),outer sep=0pt]
				\node[anchor=east,rectangle,fill=red!50]
				{\strut Important:~#1};}}%
	}%
	\mdfsetup{innertopmargin=10pt,linecolor=red!50,%
		linewidth=2pt,topline=true,%
		frametitleaboveskip=\dimexpr-\ht\strutbox\relax
	}
	\begin{mdframed}[]\relax%
		\centering
		}{\end{mdframed}}



\newcounter{lem}[section]\setcounter{lem}{0}
\renewcommand{\thelem}{\arabic{section}.\arabic{lem}}
\newenvironment{defin}[1][]{%
	\refstepcounter{lem}%
	\ifstrempty{#1}%
	{\mdfsetup{%
			frametitle={%
				\tikz[baseline=(current bounding box.east),outer sep=0pt]
				\node[anchor=east,rectangle,fill=blue!20]
				{\strut Definition};}}
	}%
	{\mdfsetup{%
			frametitle={%
				\tikz[baseline=(current bounding box.east),outer sep=0pt]
				\node[anchor=east,rectangle,fill=blue!20]
				{\strut Definition:~#1};}}%
	}%
	\mdfsetup{innertopmargin=10pt,linecolor=blue!20,%
		linewidth=2pt,topline=true,%
		frametitleaboveskip=\dimexpr-\ht\strutbox\relax
	}
	\begin{mdframed}[]\relax%
		\centering
		}{\end{mdframed}}
\lhead{CT - ECC}


\begin{document}
\begin{center}
\underline{\huge Decoding Hamming Codes}
\end{center}
\section{Decoding}
Encoding is easy: use the generator matrix G\\
\\
Decoder problem:
\begin{itemize}
	\item Input: a vector $v\in F^n$
	\item Output: The unique codeword c at Hamming distance $\leqslant 1$ from v
\end{itemize}
Remarkable property of the Hamming code: a vector $v\in F^n$ either is a codeword, or is at Hamming distance 1 from a unique codeword
\subsection{Example}
The source and destination use the (7,4,3)-Hamming code\\
\\
The source wants to transmit the four bit message
$$m=(0,0,1,1)$$
The source encodes the message
$$c=mG=(1,0,0,0,0,1,1)$$
During transmission on the channel, the sixth bit is flipped, the reciever then obtains
$$v=(1,0,0,0,0,0,1)$$
$$\begin{aligned} \mathbf { m } = ( 0,0,1,1 ) & \stackrel { \text { encoding } } { \longrightarrow } \mathbf { c } = ( 1,0,0,0,0,1,1 ) \\ & \stackrel { \text { channel } } { \longrightarrow } \mathbf { v } = ( 1,0,0,0,0,0,1 ) \end{aligned}$$
\section{Decoding}
\subsection{Brute force}
First method: Brute force\\
\\
Denote the vectors of $F^k$ as:
$$\mathbf { m } _ { 0 } = ( 0,0 , \ldots , 0 ) , \mathbf { m } _ { 1 } = ( 0,0 , \ldots , 1 ) , \ldots , \mathbf { m } _ { 2 ^ { k } - 1 } = ( 1,1 , \ldots , 1 )$$
\textbf{Description}: Compute the Hamming distance between the received vector v and the ith codeword $m_iG$ until it is no more than 1.\\
\textbf{Remark}: For the brute force algorithm, we need G. It will be given in your practicals.
\subsubsection{Example}
For the (7,4,3)-Hamming code. Recieve $v=(1,0,0,0,0,0,1)$
$$\begin{array} { l } { \mathbf { m } _ { 0 } \mathbf { G } = ( 0,0,0,0,0,0,0 ) : d _ { H } \left( \mathbf { m } _ { 0 } \mathbf { G } , \mathbf { v } \right) = 2 } \\ { \mathbf { m } _ { 1 } \mathbf { G } = ( 1,1,0,1,0,0,1 ) : d _ { H } \left( \mathbf { m } _ { 1 } \mathbf { G } , \mathbf { v } \right) = 2 } \\ { \mathbf { m } _ { 2 } \mathbf { G } = ( 0,1,0,1,0,1,0 ) : d _ { H } \left( \mathbf { m } _ { 2 } \mathbf { G } , \mathbf { v } \right) = 5 } \\ { \mathbf { m } _ { 3 } \mathbf { G } = ( 1,0,0,0,0,1,1 ) : d _ { H } \left( \mathbf { m } _ { 3 } \mathbf { G } , \mathbf { v } \right) = 1 } \end{array}$$
Then the codeword is $c=m_3G=(1,0,0,0,0,1,1)$
\subsection{Local Search}
We know that the codeword c must be either v or of the form v with one bit flipped.\\
\\
For $1\leqslant i\leqslant n$ define $e_i=(0,...,0,1,0,...,0)$ were 1 is in the position i, then either c=v or $c=v+e_i$ for some i\\
\\
\textbf{Description:} Check whether v is a codeword. If not, then flip each bit until we obtain a codeword.\\
\\
This decoding algorithm does not require you to compute G\\
\\
To check if a vector is a codeword, multiply by $H^T$, if the result is zero, then it is a codeword
\subsubsection{Example}
For the (7,4,3) - Hamming code. Recieve v=(1,0,0,0,0,0,1)
$$\begin{aligned} \mathbf { v H } ^ { \top } & = ( 1,1,0 ) \\ \left( \mathbf { v } + \mathbf { e } _ { 1 } \right) \mathbf { H } ^ { \top } & = ( 1,1,1 ) \\ \left( \mathbf { v } + \mathbf { e } _ { 2 } \right) \mathbf { H } ^ { \top } & = ( 1,0,0 ) \\ \left( \mathbf { v } + \mathbf { e } _ { 3 } \right) \mathbf { H } ^ { \top } & = ( 1,0,1 ) \\ \left( \mathbf { v } + \mathbf { e } _ { 4 } \right) \mathbf { H } ^ { \top } & = ( 0,1,0 ) \\ \left( \mathbf { v } + \mathbf { e } _ { 5 } \right) \mathbf { H } ^ { \top } & = ( 0,1,1 ) \\ \left( \mathbf { v } + \mathbf { e } _ { 6 } \right) \mathbf { H } ^ { \top } & = ( 0,0,0 ) \end{aligned}$$
Then the codeword is $c=v+e_6=(1,0,0,0,0,1,1)$
\subsection{Syndrome - The best one to use and implement}
The received word v is either a codeword or of the form $c+e_i$ for some i.\\
\\
If v is a codeword, we have $vH^T=0$. Otherwise
$$\begin{aligned} \mathbf { v H } ^ { \top } & = \left( \mathbf { c } + \mathbf { e } _ { i } \right) \mathbf { H } ^ { \top } \\ & = \mathbf { c H } ^ { \top } + \mathbf { e } _ { i } \mathbf { H } ^ { \top } \\ & = \mathbf { e } _ { i } \mathbf { H } ^ { \top } \\ & = i ^ { t h } \text { column of } \mathbf { H } \end{aligned}$$
\textbf{Description:} Compute the \textbf{syndrome} $vH^T$ to obtain i, and hence the correct codeword $v+e_i$
\subsubsection{Example}
For the $(7,4,3)$-Hamming code. Receive v=(1,0,0,0,0,0,1)\\
Compute
$$vH^T=(1,1,0)$$
Then $i=1\times 4+ 1\times 2+0\times1=6$\\
The codeword is $c=v+e_6$=(1,0,0,0,0,1,1)
\subsubsection{Example}
Case 1: $vH^T=0\Rightarrow v=c$\\
Case 2 $vH^T\neq 0\Rightarrow v=c+e_i$\\
$vH^T=(c+e_i)H^T=cH^T(0)+e_iH^T=e_iH^T=$ ith column of H=The number i in binary\\
\\
$vH^T$ gives 6 in binary, so the 6th bit is an error
\section{Recovering the original message}
Once we get the codeword c, we still need to get the original message m.\\
\\
This is very easy with our choice of generator matrix: remove the positions $1,2,4,...,2^{r-1}$ from c.\\
\\
E.g.: C=(1,0,\textbf{0},0,\textbf{0,1,1}) means that the original message is $m=(0,0,1,1)$
\end{document}