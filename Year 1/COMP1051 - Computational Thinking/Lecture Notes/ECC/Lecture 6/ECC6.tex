\documentclass{article}[18pt]
\ProvidesPackage{format}
%Page setup
\usepackage[utf8]{inputenc}
\usepackage[margin=0.7in]{geometry}
\usepackage{parselines} 
\usepackage[english]{babel}
\usepackage{fancyhdr}
\usepackage{titlesec}
\hyphenpenalty=10000

\pagestyle{fancy}
\fancyhf{}
\rhead{Sam Robbins}
\rfoot{Page \thepage}

%Characters
\usepackage{amsmath}
\usepackage{amssymb}
\usepackage{gensymb}
\newcommand{\R}{\mathbb{R}}

%Diagrams
\usepackage{pgfplots}
\usepackage{graphicx}
\usepackage{tabularx}
\usepackage{relsize}
\pgfplotsset{width=10cm,compat=1.9}
\usepackage{float}

%Length Setting
\titlespacing\section{0pt}{14pt plus 4pt minus 2pt}{0pt plus 2pt minus 2pt}
\newlength\tindent
\setlength{\tindent}{\parindent}
\setlength{\parindent}{0pt}
\renewcommand{\indent}{\hspace*{\tindent}}

%Programming Font
\usepackage{courier}
\usepackage{listings}
\usepackage{pxfonts}

%Lists
\usepackage{enumerate}
\usepackage{enumitem}

% Networks Macro
\usepackage{tikz}


% Commands for files converted using pandoc
\providecommand{\tightlist}{%
	\setlength{\itemsep}{0pt}\setlength{\parskip}{0pt}}
\usepackage{hyperref}

% Get nice commands for floor and ceil
\usepackage{mathtools}
\DeclarePairedDelimiter{\ceil}{\lceil}{\rceil}
\DeclarePairedDelimiter{\floor}{\lfloor}{\rfloor}

% Allow itemize to go up to 20 levels deep (just change the number if you need more you madman)
\usepackage{enumitem}
\setlistdepth{20}
\renewlist{itemize}{itemize}{20}

% initially, use dots for all levels
\setlist[itemize]{label=$\cdot$}

% customize the first 3 levels
\setlist[itemize,1]{label=\textbullet}
\setlist[itemize,2]{label=--}
\setlist[itemize,3]{label=*}

% Definition and Important Stuff
% Important stuff
\usepackage[framemethod=TikZ]{mdframed}

\newcounter{theo}[section]\setcounter{theo}{0}
\renewcommand{\thetheo}{\arabic{section}.\arabic{theo}}
\newenvironment{important}[1][]{%
	\refstepcounter{theo}%
	\ifstrempty{#1}%
	{\mdfsetup{%
			frametitle={%
				\tikz[baseline=(current bounding box.east),outer sep=0pt]
				\node[anchor=east,rectangle,fill=red!50]
				{\strut Important};}}
	}%
	{\mdfsetup{%
			frametitle={%
				\tikz[baseline=(current bounding box.east),outer sep=0pt]
				\node[anchor=east,rectangle,fill=red!50]
				{\strut Important:~#1};}}%
	}%
	\mdfsetup{innertopmargin=10pt,linecolor=red!50,%
		linewidth=2pt,topline=true,%
		frametitleaboveskip=\dimexpr-\ht\strutbox\relax
	}
	\begin{mdframed}[]\relax%
		\centering
		}{\end{mdframed}}



\newcounter{lem}[section]\setcounter{lem}{0}
\renewcommand{\thelem}{\arabic{section}.\arabic{lem}}
\newenvironment{defin}[1][]{%
	\refstepcounter{lem}%
	\ifstrempty{#1}%
	{\mdfsetup{%
			frametitle={%
				\tikz[baseline=(current bounding box.east),outer sep=0pt]
				\node[anchor=east,rectangle,fill=blue!20]
				{\strut Definition};}}
	}%
	{\mdfsetup{%
			frametitle={%
				\tikz[baseline=(current bounding box.east),outer sep=0pt]
				\node[anchor=east,rectangle,fill=blue!20]
				{\strut Definition:~#1};}}%
	}%
	\mdfsetup{innertopmargin=10pt,linecolor=blue!20,%
		linewidth=2pt,topline=true,%
		frametitleaboveskip=\dimexpr-\ht\strutbox\relax
	}
	\begin{mdframed}[]\relax%
		\centering
		}{\end{mdframed}}
\lhead{CT - ECC}


\begin{document}
\begin{center}
\underline{\huge Design of codes}
\end{center}
\section{Introduction}
\subsection{Design of codes}
Recall our general problem: design a code:
\begin{itemize}
	\item With high rate
	\item Which can detect many errors
	\item Which is easy to encode and decode
\end{itemize}
\subsection{How do we design good codes?}
Start with a good code, and modify it:
\begin{itemize}
	\item cut it
	\item add a parity check bit
	\item Take a subset of the codewords
	\item Take the dual
\end{itemize}
\section{EAN and ISBN}
\subsection{EAN}
This uses a variant of the parity check code $c=(c_1,...,c_{13})$ where:
$$c _ { 13 } = - \sum _ { i = 0 } ^ { 5 } \left( c _ { 2 i + 1 } + 3 c _ { 2 i + 2 } \right) \quad \bmod 10$$
Ex: 5-045092-36551\textbf{?}
$$\begin{aligned} c _ { 13 } & = - [ 5 + ( 3 \times 0 ) + 4 + ( 3 \times 5 ) + 0 + ( 3 \times 9 ) + 2 + ( 3 \times 3 ) \\ & + 6 + ( 3 \times 5 ) + 5 + ( 3 \times 1 ) ] \\ & = - ( 5 + 4 + 5 + 7 + 2 + 9 + 6 + 5 + 5 + 3 ) \\ & = - 1 = 9 \end{aligned}$$
\subsection{ISBN}
This is another variant of the parity check code where 
$$c _ { 10 } = \sum _ { i = 1 } ^ { 9 } i c _ { i } \quad \bmod 11$$
Example: ISBN-10 number 0-262-06141-?
$$\begin{aligned} c _ { 10 } & = [ ( 1 \times 0 ) + ( 2 \times 2 ) + ( 3 \times 6 ) + ( 4 \times 2 ) + ( 5 \times 0 ) \\ & + ( 6 \times 6 ) + ( 7 \times 1 ) + ( 8 \times 4 ) + ( 9 \times 1 ) ] \\ & = 4 + 7 + 8 + 3 + 7 + 10 + 9 \\ & = 4 \end{aligned}$$
\section{Introduction to algebraic codes}
\subsection{More structure}
We can use polynomials for more complicated codes using sequences of digits
\subsection{GF(4)}
Let $\alpha$ be a root of $x^2+x+1$, i.e.
$$\alpha ^ { 2 } + \alpha + 1 = 0 , \quad \text { or equivalently, } \quad \alpha ^ { 2 } = \alpha + 1$$


$$\begin{array} { | c | c | c | c | c | } \hline + & { 0 } & { 1 } & { \alpha } & { \alpha + 1 } \\ \hline 0 & { 0 } & { 1 } & { \alpha } & { \alpha + 1 } \\ \hline 1 & { 1 } & { 0 } & { \alpha + 1 } & { \alpha } \\ \hline \alpha & { \alpha } & { 0 } & { \alpha + 1 } & { \alpha } \\ \hline \alpha & { \alpha } & { \alpha + 1 } & { 0 } & { 1 } \\ \hline \alpha + 1 & { \alpha + 1 } & { \alpha } & { 1 } & { 0 } \\ \hline \end{array}$$

$$\begin{array} { | c | c | c | c | c | } \hline \times & { 0 } & { 1 } & { \alpha } & { \alpha ^ { 2 } } \\ \hline 0 & { 0 } & { 0 } & { 0 } & { 0 } \\ \hline 1 & { 0 } & { 1 } & { \alpha } & { \alpha ^ { 2 } } \\ \hline \alpha & { 0 } & { \alpha } & { \alpha ^ { 2 } } & { 1 } \\ \hline \alpha ^ { 2 } & { 0 } & { \alpha ^ { 2 } } & { 1 } & { \alpha } \\ \hline \end{array}$$
\subsection{GF(8)}
The construction can be extended for any $GF(2^m)$\\
\\
E.g. GF(8). Let $\beta$ be a root of $x^3+x+1$ i.e.
$$\beta^3+\beta+1=0$$
Then GF(8)=$\{ 0,1 , \beta , \beta ^ { 2 } , \beta ^ { 3 } = \beta + 1 , \beta ^ { 4 } = \beta ^ { 2 } + \beta , \beta ^ { 5 } =\beta ^ { 2 } + \beta + 1 , \beta ^ { 6 } = \beta ^ { 2 } + 1 \}$
\subsection{Reed-Solomon codes}
The code $RS(k,k)$ is the set of all evaluations of polynomials of degree at most $k-1$ over all nonzero elements of GF(q) where n=q-1\\
\\
Let $q=2^m$ and $\gamma$ generate GF(q) i.e.
$$\mathrm { GF } ( q ) = \left\{ 0,1 , \gamma , \ldots , \gamma ^ { q - 2 } \right\}$$
For any polynomial c(x) with coefficients in GF(q), let
$$\mathbf { c } = \left( c ( 1 ) , c ( \gamma ) , \ldots , c \left( \gamma ^ { q - 2 } \right) \right) \in \mathrm { GF } ( q ) ^ { n }$$
Then
$$R S ( n , k ) = \{ \mathbf { c } : \operatorname { deg } c ( x ) \leq k - 1 \}$$
\subsection{Generator matrix of Reed-Solomon Codes}
E.g. for RS(7,2)
$$\mathbf { G } _ { R S ( 7,2 ) } = \left( \begin{array} { c c c c c c c } { 1 } & { 1 } & { 1 } & { 1 } & { 1 } & { 1 } & { 1 } \\ { 1 } & { \beta } & { \beta ^ { 2 } } & { \beta ^ { 3 } } & { \beta ^ { 4 } } & { \beta ^ { 5 } } & { \beta ^ { 6 } } \end{array} \right)$$
E.g. for encoding $\left( \beta ^ { 2 } , \beta \right) ,$ i.e. $c ( x ) = \beta ^ { 2 } + \beta x$
$$\begin{aligned} \mathbf { c } & = \left( c ( 1 ) , c ( \beta ) , c \left( \beta ^ { 2 } \right) , c \left( \beta ^ { 3 } \right) , c \left( \beta ^ { 4 } \right) , c \left( \beta ^ { 5 } \right) , c \left( \beta ^ { 6 } \right) \right) \\ & = \left( \beta ^ { 4 } , 0 , \beta ^ { 5 } , \beta , \beta ^ { 3 } , 1 , \beta ^ { 6 } \right) \end{aligned}$$
\subsection{Bound on the minimum distance}
The \textbf{Singleton bound}: if C is an $(n,k,d_{min})$-code, then
$$d_{min}\leqslant n-k+1$$
Proof: look at the parity check matrix: the columns have size n-k
\begin{itemize}
	\item At most n-k linearly independent columns
	\item Any set of n-k-1 columns is linearly dependent
\end{itemize}
\subsection{Minimum distance of RS codes}
For any two polynomials $c(x)\neq d(x)$ of degrees $\leqslant k-1$
\begin{itemize}
	\item $c(x)-d(x)\neq 0$ has degree $\leqslant k-1$
	\item $c(x)-d(x)$ has at most $k-1$ roots
	\item c and d agree on at most k-1 positions
	\item $d_H(c,d)\geqslant n-k+1$
\end{itemize}
By the singleton bound, we obtain:
$$d_{min}=n-k+1$$
\subsection{RS Decoding}
Due to their structure RS are easy to decode, but we won't go into that
\end{document}