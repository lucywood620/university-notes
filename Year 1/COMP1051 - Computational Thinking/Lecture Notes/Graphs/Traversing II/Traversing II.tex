\documentclass{article}[18pt]
\usepackage{../../../../../format}
\lhead{CT - Graphs}
\lstset{language=C,
	basicstyle=\ttfamily,
	keywordstyle=\bfseries,
	showstringspaces=false,
	morekeywords={if, else, then, print, end, for, do, while},
	tabsize=4,
	mathescape=true,
	escapechar=£,
	numbers=left,
	stepnumber=1,
}

\usepackage{caption}
\DeclareCaptionFont{white}{\color{white}}
\DeclareCaptionFormat{listing}{\colorbox{gray}{\parbox{\textwidth}{#1#2#3}}}
\captionsetup[lstlisting]{format=listing,labelfont=white,textfont=white}

\begin{document}
\begin{center}
\underline{\huge Graph Traversing II}
\end{center}
\section{Graph Traversing}
Two main approaches for graph exploration:
\begin{itemize}
	\item Breadth-First Search (BFS)
	\begin{itemize}
		\item Search in \textbf{breadth}
		\item layer by layer
	\end{itemize}
	\item Depth-First Search
	\begin{itemize}
		\item Search in \textbf{depth}
		\item dig deeper, until not possible any more
	\end{itemize}
\end{itemize}
\begin{itemize}
	\item DFS is \textbf{not} appropriate for shortest paths
	\begin{itemize}
		\item We may reach the target vertex b via a \textbf{very long} path, as we just "dig deeper"
	\end{itemize}
	\item Both BFS and DFS
	\begin{itemize}
		\item Appropriate for graph exploration
		\item Can list all reachable vertices from a start vertex a
		\item very fast (linear time)
	\end{itemize}
\end{itemize}
\section{A generic search algorithm}
\begin{lstlisting}[caption=Generic Graph Search]
reachable[a]=1	£\hfill£ //Initialisation
S={a}£\hfill£	//initialisation; set of vertices, from which we continue exploration
for i=1 to n£\hfill£	//Iterate until no more reachable vertices
	choose a vertex $u\in S$£\hfill£	//the crucial choice of the search
	print u£\hfill£	// u is the next vertex in the output list
	for each vertex $v\in Adj[u]$
		if reachable[v]==0 then £\hfill£	//We found a new vertex v to reach
			reachable[v]=1£\hfill£ 	//"mark" v as reached
			$S=S\cup \{v\}$£\hfill£	//add v to the list of reachable vertices
\end{lstlisting}
\begin{itemize}
	\item The set S changed dynamically
	\item BFS and DFS have different "policy" for the choice at line 4
	\item BFS prefers vertices "closer to a"
	\item DFS prefers vertices that are always "one step further"
\end{itemize}
\section{The policy of BFS}
\begin{itemize}
	\item The policy of BFS
	\begin{itemize}
		\item Remove the element that has been \textbf{longer in S}
		\item a First-In-First-Out (FIFO) policy
	\end{itemize}
	\item This data structure is called a queue
	\item In other words
	\begin{itemize}
		\item Add new vertices at the end of the queue
		\item remove vertices from the beginning of the queue
		\item first process vertices that are closer to the start vertex
	\end{itemize}
\end{itemize}
\section{The policy of DFS}
\begin{itemize}
	\item The policy of DFS:
	\begin{itemize}
		\item Remove the element that has been shorter S
		\item a Last-In-First-Out (LIFO) policy
	\end{itemize}
	\item This data structure is called a "stack"
	\item In other words:
	\begin{itemize}
		\item add new vertices at the end (top) of the stack
		\item remove vertices from the end(top) of the stack
		\item first process vertices that always "one step further"
	\end{itemize}
\end{itemize}
\section{Longest paths}
Computing a Longest path is NP complete\\
\\
Intuitively:
\begin{itemize}
	\item vertices are balls
	\item edges are strings tight on the balls
\end{itemize}
Shortest path problem:
\begin{itemize}
	\item Pull firmly two specific balls away from each
	\item the length of the string between then is their distance in the graph
\end{itemize}
Longest path problem:
\begin{itemize}
	\item You need to investigate all (possibly "strange") paths between the two balls through the net of strings
	\item which can be very complex
\end{itemize}
However:
\begin{itemize}
	\item If the graph has no cycles, then it is easy
	\item such graphs are called "trees"
\end{itemize}

\end{document}