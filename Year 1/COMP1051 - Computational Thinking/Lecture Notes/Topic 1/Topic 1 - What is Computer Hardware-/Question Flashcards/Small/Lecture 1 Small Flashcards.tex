\documentclass[grid,avery5371]{flashcards}
\usepackage{amssymb}
\usepackage[utf8]{inputenc}
\usepackage[T1]{fontenc}
\usepackage{verse}
\usepackage[version=3]{mhchem}
\usepackage{graphicx}
\settowidth{\versewidth}{It lies behind stars and under hills,}
\addtolength{\versewidth}{2em}
\usepackage{pgfplots}
\geometry{headheight=12pt}
\usepackage{fancyhdr}
\pagestyle{fancy}
\fancyhf{}
\renewcommand{\headrulewidth}{0pt}


\title{Conversion flashcards}
\author{Sam Robbins}

\cardbackstyle[\large]{plain}
\cardfrontstyle[\large]{headings}
%\cardbackstyle[\large]{headings}

\begin{document}

\begin{flashcard}[]{In a PC, there tend to be two chip-sets: the northbridge; and the
southbridge. What is the essential difference between the two? [2]}

The northbridge handles fast communications amongst the CPU, the
memory and the graphics processing unit [1], whereas the southbridge
handles (slower) communications involving external hard-disks, the
mouse, the keyboard, the Internet, the printer and other such devices
[1].
\end{flashcard}

\begin{flashcard}[]{ What are the two main components of an integrated circuit? (Both of
these components usually appear in their millions.) [1]}

Transistors [$\frac{1}{2}$] interconnected by microscopic wires [$\frac{1}{2}$].
\end{flashcard}

\begin{flashcard}[]{What is the current trend in microprocessor design so as to overcome
difficulties with power dissipation as we build faster and faster single
processors? [2]}
Multi-core processors [1] where one CPU with a high clock-speed is
replaced with a number of CPUs with lower clock-speeds but which,
when working together, can give better computational power [1].
\end{flashcard}

\begin{flashcard}[]{What is Moore’s Law? [1]}
A rough description that long-term transistor capacity doubles every
18–24 months (coined by Gordon Moore in 1965) [1].
\end{flashcard}

\begin{flashcard}[]{What is a Boolean function? In relation to Boolean functions, what is
so special about NOT-, AND- and OR-gates? [2]}

A Boolean function is a function $f:\{0, 1\}^n \rightarrow \{0, 1\}$ [1]. We can build
a circuit to compute any Boolean function using only NOT-, AND- and
OR-gates [1].

\end{flashcard}

\begin{flashcard}[]{
Explain the basic principle behind using half-adders and full-adders to
build a circuit that computes the sum of two n-bit strings of 0s and 1s.
(You need not describe the resulting circuit in full detail; just give an
overview of its construction.) [6]
}
{\small Let the two n-bit strings be $X_1X_2 ... X_n$ and $Y_1Y_2 . . . Y_n$. A half-adder
takes $X_1$ and $Y_1$ as input and outputs the sum, with the carry fed into
a full-adder [2]. This full-adder also has inputs $X_2$ and $Y_2$ and its sum
is output, with its carry fed into a full-adder [2]. This full-adder also
has $X_3$ and $Y_3$ as inputs and its sum is output, with its carry fed into
a full-adder, and so on [2]. Both the sum and the carry of the final
full-adder are output.}

\end{flashcard}

\begin{flashcard}[]{What is the ultimate aim of research in formal methods? [1]}
The ultimate aim of formal methods is to enable us to mathematically
prove properties of designs, programs and so on, so as not just to rely
on empirical testing [1].
\end{flashcard}

\begin{flashcard}[]{A hardware description language is used so as to better enable the
design of integrated circuits. Name two distinctive attributes of computer
hardware that are normally expressible in a hardware description
language. [2]}
Unlike a normal programming language, an HDL includes explicit notations
for expressing \textbf{time} [1] and \textbf{concurrency} [1], which are primary
attributes of computer hardware.
\end{flashcard}

\begin{flashcard}[]{What is the von Neumann bottleneck? Which component of a modern
CPU did the von Neumann bottleneck give rise to? [3]}
The von Neumann bottleneck is a limitation of the rate of data transfer
between the CPU and memory (data and instructions have to be
fetched in sequential order and idle time is wasted whilst waiting for
data items or instructions to be fetched from memory) [2]. The von
Neumann bottleneck gave rise to the use of caches [1].

\end{flashcard}

\begin{flashcard}[]{How does the Harvard architecture differ from the von Neumann architecture?
[2]}
The Harvard architecture has memory that is partitioned into data
memory and instruction memory with dedicated buses for each of them
[2].
\end{flashcard}

\begin{flashcard}[]{ Explain the difference between a byte and a word in relation to computer architecture. How is the size of a word related to the width of a
bus? [2]}
1 byte of storage consists of 8 bits [1]. A word, being a number of bytes,
is the amount of data that can be handled by a processor as one unit
[1].
\end{flashcard}

\begin{flashcard}[]{What is a basic principle as regards different types of memory and its physical distance from the CPU? [1]}
The cost and performance of memory is generally proportional to its
physical distance from the CPU [1].
\end{flashcard}

\begin{flashcard}[]{What is the difference between static RAM and dynamic RAM? [3]}
Dynamic RAM (DRAM) is where a bit of data is stored using a transistor/capacitor
combination [1]. Static RAM (SRAM) is where a bit
of data is stored by a flip-flop, which incorporates 4-6 transistors [1].
Static RAM is stable and fast but takes up more memory $[\frac{1}{2}]$ whereas
dynamic RAM is cheap, slow and needs to be refreshed because of
‘leaky’ capacitors $[\frac{1}{2}]$.
\end{flashcard}

\begin{flashcard}[]{What is the purpose of cache memory in a CPU? [1]}
Caches are expensive memory that are used to store rapidly accessed
items [1].
\end{flashcard}

\begin{flashcard}[]{What are registers in the CPU? [1]}
Registers are on-chip memory locations that are limited in number.
They provide the fastest way to access data [1].
\end{flashcard}

\begin{flashcard}[]{Modern parallel computing can come in many shapes and forms. Briefly explain the fundamental difference between multi-core computing and
GPGPU computing. [2]}
A multi-core computer is a computer where more than one processor
is integrated on one IC [1]. GPGPU computing uses the processors
within the graphics processing unit to compute with rather than deal
with the screen pixels [1].
\end{flashcard}

\begin{flashcard}[]{Give two illustrations of principles of Computational Thinking in the context of hardware. [2]}
‘Processing in parallel’ in multi-core or GPGPU computing. ‘Using
abstraction and decomposition in tackling a large complex task’ in IC
design and computer system design. ‘Prefetching and caching in anticipation
of future use’ in caching within CPUs. ‘Interpreting code as data
and data as code’ in processor architectures. [1] for each illustration.
\end{flashcard}


\end{document}