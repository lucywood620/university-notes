\documentclass[grid,avery5371]{flashcards}
\usepackage{amssymb}
\usepackage[utf8]{inputenc}
\usepackage[T1]{fontenc}
\usepackage{verse}
\usepackage[version=3]{mhchem}
\usepackage{graphicx}
\settowidth{\versewidth}{It lies behind stars and under hills,}
\addtolength{\versewidth}{2em}
\usepackage{pgfplots}
\geometry{headheight=12pt}
\usepackage{fancyhdr}
\pagestyle{fancy}
\fancyhf{}
\renewcommand{\headrulewidth}{0pt}
\hyphenpenalty=100000

\title{Conversion flashcards}
\author{Sam Robbins}

\cardbackstyle[\large]{plain}
\cardfrontstyle[\large]{headings}
%\cardbackstyle[\large]{headings}

\begin{document}

\begin{flashcard}[]{Briefly (and, necessarily, without being too precise) explain the difference between an algorithm and a program. Suppose we have an
algorithm and wish to implement it in Python. How many different
implementations are there of this algorithm? [5]}

An algorithm is a sequence of precise instructions that can be applied to specific data items. A program is the implantation of the algorithm in a form that can be executed by a computer, or at least compiled to a form that can be executed by a computer. There are many different implementations of an algorithm as a program.
\end{flashcard}

\begin{flashcard}[]{Give 3 general properties any programming language should have. [3]}
\normalsize
\begin{itemize}
\itemsep0em 

\item Be easy to use, with its programs easy to read, write and understand
\item Support abstraction so that adding new features and concepts should be possible
\item Support testing, debugging and program verification
\item Be inexpensive to use, in terms of execution time, memory usage and maintenance costs
\end{itemize}
\end{flashcard}

\begin{flashcard}[]{Explain very briefly how a Prolog program computes. [4]}
A Prolog program consists of a list of facts (atoms) and rules that can be applied to the facts. It then takes queries about the facts which it can answer using the atoms and the rules.

\end{flashcard}

\begin{flashcard}[]{Give an advantage of compilation over interpretation, and of interpretation over compilation. [2]}
{\normalsize
\begin{itemize}
\itemsep-0.5em 

\item Compiled programs have faster execution.
\item When a program is compiled it can be optimised at the same time, this can’t be done through interpretation.
\item Interpreted programs use memory better as only a few instructions need to be stored at once.

\end{itemize}}
\end{flashcard}

\begin{flashcard}[]{In compilation, over 50\% of the time taken can be spent on lexical
analysis; that is, character handling. In moving from a program as a
string of symbols to a token stream, which algebraic construction is
usually used to define tokens? [2]}
Regular Expressions
\end{flashcard}


\begin{flashcard}[]{Give a regular expression that denotes the set of strings over the alphabet
\{a, b, c\} consisting of strings with the property that there is always at most one c. [5]}
$((a|b)*|c(a|b)*)|(a|b)* $
\end{flashcard}

\begin{flashcard}[]{Give a regular expression that denotes the set of strings over the alphabet
\{a, b\} consisting of strings with the property that any a must always be followed by bb. [7]}
$(b|abb)*$
\end{flashcard}

\begin{flashcard}[]{Define carefully a finite state machine and explain how one is used to
accept a set of strings. [8]}
$M=(\Sigma, Q,\delta:Q \times \Sigma \rightarrow Q, q_0 \in Q, F)$\\
Where
\begin{itemize}
\item $\Sigma$ is some finite alphabet
\item Q is some finite set of \textbf{states} with \textbf{initial state} $q_0$ and a set of \textbf{final states} F $\subseteq Q$
\item $\delta: Q\times \Sigma \rightarrow Q$ is the \textbf{transition function}
\end{itemize}


\end{flashcard}


\begin{flashcard}[]{What is the fundamental principle of the research area known as ubiquitous computing? [2]}
The integration of computers and software into everyday objects and activities so that we can control remote aspects of our lives, mostly through RFID.
\end{flashcard}

\begin{flashcard}[]{Give two illustrations of principles of Computational Thinking in the context of software. [2]}
{\normalsize \textbf{Green Computing}:An area of computer science involving energy conservation within the world of information technology. This involves writing the main unit of resource is energy expended.\\
\textbf{Parallel processing}: Certain kinds of problems can be conveniently represented as multiple communicating threads which help to structure code in a more modular manner, e.g., by modelling user interface components as separate threads.}

\end{flashcard}

\begin{flashcard}[]{What is the relationship between regular expressions and finite state machines? [2]}
A set of strings is represented by a regular expression if, and
only if, it is accepted by a finite state machine

\end{flashcard}



\end{document}