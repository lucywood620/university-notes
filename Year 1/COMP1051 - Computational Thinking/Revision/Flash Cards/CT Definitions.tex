\documentclass[grid,avery5371]{flashcards}
\usepackage{amssymb}
\usepackage[utf8]{inputenc}
\usepackage[T1]{fontenc}
\usepackage{verse}
\usepackage[version=3]{mhchem}
\usepackage{graphicx}
\settowidth{\versewidth}{It lies behind stars and under hills,}
\addtolength{\versewidth}{2em}
\usepackage{pgfplots}
\geometry{headheight=12pt}
\usepackage{fancyhdr}
\pagestyle{fancy}
\fancyhf{}
\renewcommand{\headrulewidth}{0pt}
\hyphenpenalty=100000

\title{Conversion flashcards}
\author{Sam Robbins}

\cardbackstyle[\large]{plain}
\cardfrontstyle[\large]{headings}
\cardfrontfoot{CT}
%\cardbackstyle[\large]{headings}
\setlength{\cardheight}{1.0in}
\setlength{\oddevenshift}{1.0in}

%\renewcommand{\cardrows}{8}
\begin{document}

\begin{flashcard}[]{Three Phases of IC Design}
Functional Specification\\
Register Transfer Level\\
Map to physical layout
\end{flashcard}


\begin{flashcard}[]{Three key components in CPU Microarchitecture}
	Datapath\\
	Control\\
	Cache
\end{flashcard}

\begin{flashcard}[]{Fetch decode fetch execute cycle}
	Instruction Fetch\\
	Instruction Decode\\
	Operand Fetch\\
	Execute Instruction
\end{flashcard}

\begin{flashcard}[]{Two main components of an IC}
	Transistors interconnected by microscopic wires
\end{flashcard}

\begin{flashcard}[]{von Neumann bottleneck}
	A limitation of the rate of data transfer between the CPU and memory
\end{flashcard}

\begin{flashcard}[]{Registers}
	On chip memory locations providing fast access to data
\end{flashcard}

\begin{flashcard}[]{4 Programming Paradigms}
	Imperative\\
	Declarative\\
	Data-Oriented\\
	Scripting
\end{flashcard}

\begin{flashcard}[]{4 Drivers of Programming Languages}
	Productivity\\
	Reliability\\
	Security\\
	Execution
\end{flashcard}

\begin{flashcard}[]{Syntax}
	Rules that govern what make a program 'legitimately written'
\end{flashcard}

\begin{flashcard}[]{Semantics}
	The rules which govern what a program 'means'
\end{flashcard}


\begin{flashcard}[]{ISA}
	Interface between hardware and software
\end{flashcard}


\begin{flashcard}[]{Process Control Block}
	A data structure that the kernel uses in order to manage a process
\end{flashcard}

\begin{flashcard}[]{Types of MIPS Instructions}
	I Type - Involve data transfer\\
	R Type - Work on registers\\
	J Type - Involve jumps
\end{flashcard}

\begin{flashcard}[]{Data Security (in the context of OS)}
	Ensuring that the memory allocated to each program is kept separate and secure from other programs
\end{flashcard}

\begin{flashcard}[]{Virtualisation (in the context of OS)}
	Providing abstractions that present clean interfaces to make the computer easier to use
\end{flashcard}


\begin{flashcard}[]{Mutual exclusion}
	Ensuring that two threads are not in the critical selection at the same time
\end{flashcard}

\begin{flashcard}[]{Critical Selection}
	Exclusive access to some shared resource such as memory location
\end{flashcard}

\begin{flashcard}[]{Life cycle of processes within the operating system}
	new\\
	ready\\
	running\\
	blocked\\
	exit
\end{flashcard}

\begin{flashcard}[]{State transitions in a CPU}
	{\small
	admit\\
	dispatch\\
	timeout/yield\\
	event-wait\\
	event\\
	release\\}
\end{flashcard}

\begin{flashcard}[]{Context switching}
	Where the operating system pauses one process and resumes another
\end{flashcard}

\begin{flashcard}[]{Notions relating to problem solving}
	Computation\\
	Resource\\
	Correctness
\end{flashcard}

\begin{flashcard}[]{Planar graph}
	A graph that can be drawn in a plane without any graph edges crossing
\end{flashcard}

\end{document}