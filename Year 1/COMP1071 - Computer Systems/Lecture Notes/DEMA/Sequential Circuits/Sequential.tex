\documentclass{article}[18pt]
\usepackage{../../../../format}
\lhead{CSys}


\begin{document}
\begin{center}
\underline{\huge Sequential Circuits}
\end{center}
\section{Circuits}
\begin{center}
	\includegraphics[scale=0.7]{figure1}
\end{center}
\section{SR Latch}
\begin{center}
	\includegraphics[scale=0.7]{figure2}
\end{center}
\begin{center}
	\includegraphics[scale=0.7]{figure3}
\end{center}
\begin{center}
	\includegraphics[scale=0.7]{figure4}
\end{center}
\section{D Latch}
\begin{center}
	\includegraphics[scale=0.7]{figure5}
\end{center}
\begin{itemize}
	\item The output will be updated to whatever the data is whenever the clock gives a pulse
\end{itemize}
\section{D Flip Flop}
\begin{center}
	\includegraphics[scale=0.7]{figure6}
\end{center}
\begin{itemize}
	\item Whenever the clock is low, the data will transmit through the master, when the clock is high, the data will not transmit through the master
\end{itemize}
\begin{center}
	\includegraphics[scale=0.7]{figure7}
\end{center}
\section{Enabled Flip Flop}
\begin{center}
	\includegraphics[scale=0.7]{figure8}
\end{center}
\begin{itemize}
	\item Therefore the 1st diagram is the best one to use as by using the end gate, timing errors can be included
\end{itemize}
\section{Flip Flop Design}
\begin{center}
	\includegraphics[scale=0.7]{figure9}
\end{center}
\section{Example}
\begin{center}
	\includegraphics[scale=0.7]{figure10}
\end{center}
\section{Problem Circuits}
\begin{center}
	\includegraphics[scale=0.7]{figure11}
\end{center}
\begin{center}
	\includegraphics[scale=0.7]{figure12}
\end{center}
\begin{center}
	\includegraphics[scale=0.7]{figure13}
\end{center}
\section{Synchronous Circuits}
\begin{center}
	\includegraphics[scale=0.7]{figure14}
\end{center}
\section{Examples}
\begin{center}
	\includegraphics[scale=0.7]{figure15}
\end{center}
b,d,e,g,
\section{Timing}
\begin{center}
	\includegraphics[scale=0.7]{figure16}
\end{center}
\begin{itemize}
	\item ccq - contamination clock to q
	\item pcq - propagation clock to q
\end{itemize}
\section{Setting Time}
\begin{center}
	\includegraphics[scale=0.7]{figure17}
\end{center}
\begin{center}
	\includegraphics[scale=0.7]{figure18}
\end{center}
\begin{center}
	\includegraphics[scale=0.7]{figure19}
\end{center}
\section{Example}
\begin{center}
	\includegraphics[scale=0.7]{figure20}
\end{center}
\begin{itemize}
	\item Critical path goes through all 3 gates
	\begin{itemize}
		\item $CL_{pd}=120ps$
	\end{itemize}
	\item $T_c\geqslant80+120+50=250ps$
	\item $1/250ps$ = 4GHz
\end{itemize}
There would be a hold time violation


\section{Fixing the hold time violation}
\begin{center}
	\includegraphics[scale=0.7]{figure21}
\end{center}
\section{Metastable States}
\begin{center}
	\includegraphics[scale=0.7]{figure22}
\end{center}
\section{Synchronisers}
\begin{center}
	\includegraphics[scale=0.7]{figure23}
\end{center}
\section{Pipelining}
\begin{center}
	\includegraphics[scale=0.7]{figure24}
\end{center}
\begin{center}
	\includegraphics[scale=0.7]{figure25}
\end{center}
\begin{center}
	\includegraphics[scale=0.7]{figure26}
\end{center}
\begin{itemize}
	\item Doing this gives higher latency, but greater throughput
\end{itemize}



\end{document}