\documentclass{article}[18pt]
\usepackage{../../../../format}
\lhead{Computer Systems - Dr Magnus Bordewich}


\begin{document}
\begin{center}
\underline{\huge Number Systems}
\end{center}
\section{Decimal}
The "decimal point" in general is called the \textbf{radix point}, this indicates the position of the "units", immediately to the left of the radix point
\section{Positional number systems}
The \textbf{base}(or \textbf{radix}) of the number system is the number of symbols (including 0). \\
The subscript after the number indicates the base\\
\\
The contribution of symbol x, which is the $i^{th}$ symbol in the order, is the
$$(i-1)\times base^{position}$$
Where position is the number of places to the \textbf{left} of the units\\
\\
\begin{tabular}{|c|c|c|c|c|c|c|}
\hline
Position&2&1&0&.&-1&-2\\
\hline
$Base^{Position}$&$2^2$&$2^1$&$2^0$&.&$2^{-1}$&$2^{-2}$\\
\hline
Decimal Value&4&2&1&.&.5&.25\\
\hline
Example&1&1&0&.&1&1\\
\hline
\end{tabular}\\
\\
For this example it is equivalent to the decimal 6.75
\section{Binary}
Each digit in a binary number system is known as a bit
\begin{itemize}
\item \textbf{B}inary dig\textbf{IT}
\end{itemize}
A bit can have only one of two possible values
\begin{itemize}
\item 0 or 1 (false/true, off/on, LOW/HIGH)
\end{itemize}
Groups of bits are known as:
\begin{itemize}
\item \textbf{Nibble} - 4 bits
\item \textbf{Byte} - 8 bits
\item \textbf{Half Word} - 16 bits
\item \textbf{Word} - 32 bits
\item \textbf{Double Word} - 64 bits
\end{itemize}
Note that the value of a "word" is CPU dependent, as for a 64 bit CPU, a word is 64 bits, rather than the 32 stated above
\section{Hexadecimal}
This has 16 distinct symbols: \textbf{0,1,2,3,4,5,6,7,8,9,A,B,C,D,E,F}\\
\\
Why do we need Hexadecimal?
\begin{itemize}
\item Reading and writing binary values is difficult for humans
\end{itemize}
Advantages to using Hexadecimal
\begin{itemize}
\item \textbf{More Compact} than other number systems
\item \textbf{Easy to convert} between binary and decimal
\end{itemize}
Programmers must be aware of what they are writing
\begin{itemize}
\item BEEF and BEEF$_{16}$ have very different meanings
\item In Java use a prefix to denote a hexadecimal value: 0xBEEF=BEEF$_{16}$
\end{itemize}
\subsection{Hexadecimal conversion}
\begin{tabular}{|c|c|c|c|c|c|c|}
\hline
Position&2&1&0&.&-1&-2\\
\hline
$Base^{Position}$&$16^2$&$16^1$&$16^0$&.&$16^{-1}$&$16^{-2}$\\
\hline
Decimal Value&256&16&1&.&.0625&.00390625\\
\hline
Example&C&2&D&.&1&0\\
\hline
\end{tabular}\\
\\
For this example it is equal to 3117.0625 in decimal
\section{Conversion}
\subsection{Binary to Hex}
\begin{enumerate}
\item Starting from the \textbf{radix point}, separate the binary number into groups of \textbf{four} binary digits (nibbles)
\item Then \textbf{translate each group} (nibble) into its hexadecimal equivalent, group by group, maintaining left to right order
\end{enumerate}
Remember to add zeroes on the start and end to ensure that you get full nibbles if needed\\
\\
\begin{tabular}{c c c c c c}
\textcolor{blue}{00}11&0101&1101&1000&.&001\textcolor{blue}{0}\\
3&5&D&8&.&2
\end{tabular}
\subsection{Hex to binary}
\begin{enumerate}
\item Starting from the \textbf{radix point}, separate the hexadecimal number into digits
\item Then translate each digit into a \textbf{4-digit binary nibble}, maintaining right to left order
\end{enumerate}











\end{document}