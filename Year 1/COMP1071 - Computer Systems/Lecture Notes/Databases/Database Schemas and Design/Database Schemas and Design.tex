\documentclass{article}[18pt]
\ProvidesPackage{format}
%Page setup
\usepackage[utf8]{inputenc}
\usepackage[margin=0.7in]{geometry}
\usepackage{parselines} 
\usepackage[english]{babel}
\usepackage{fancyhdr}
\usepackage{titlesec}
\hyphenpenalty=10000

\pagestyle{fancy}
\fancyhf{}
\rhead{Sam Robbins}
\rfoot{Page \thepage}

%Characters
\usepackage{amsmath}
\usepackage{amssymb}
\usepackage{gensymb}
\newcommand{\R}{\mathbb{R}}

%Diagrams
\usepackage{pgfplots}
\usepackage{graphicx}
\usepackage{tabularx}
\usepackage{relsize}
\pgfplotsset{width=10cm,compat=1.9}
\usepackage{float}

%Length Setting
\titlespacing\section{0pt}{14pt plus 4pt minus 2pt}{0pt plus 2pt minus 2pt}
\newlength\tindent
\setlength{\tindent}{\parindent}
\setlength{\parindent}{0pt}
\renewcommand{\indent}{\hspace*{\tindent}}

%Programming Font
\usepackage{courier}
\usepackage{listings}
\usepackage{pxfonts}

%Lists
\usepackage{enumerate}
\usepackage{enumitem}

% Networks Macro
\usepackage{tikz}


% Commands for files converted using pandoc
\providecommand{\tightlist}{%
	\setlength{\itemsep}{0pt}\setlength{\parskip}{0pt}}
\usepackage{hyperref}

% Get nice commands for floor and ceil
\usepackage{mathtools}
\DeclarePairedDelimiter{\ceil}{\lceil}{\rceil}
\DeclarePairedDelimiter{\floor}{\lfloor}{\rfloor}

% Allow itemize to go up to 20 levels deep (just change the number if you need more you madman)
\usepackage{enumitem}
\setlistdepth{20}
\renewlist{itemize}{itemize}{20}

% initially, use dots for all levels
\setlist[itemize]{label=$\cdot$}

% customize the first 3 levels
\setlist[itemize,1]{label=\textbullet}
\setlist[itemize,2]{label=--}
\setlist[itemize,3]{label=*}

% Definition and Important Stuff
% Important stuff
\usepackage[framemethod=TikZ]{mdframed}

\newcounter{theo}[section]\setcounter{theo}{0}
\renewcommand{\thetheo}{\arabic{section}.\arabic{theo}}
\newenvironment{important}[1][]{%
	\refstepcounter{theo}%
	\ifstrempty{#1}%
	{\mdfsetup{%
			frametitle={%
				\tikz[baseline=(current bounding box.east),outer sep=0pt]
				\node[anchor=east,rectangle,fill=red!50]
				{\strut Important};}}
	}%
	{\mdfsetup{%
			frametitle={%
				\tikz[baseline=(current bounding box.east),outer sep=0pt]
				\node[anchor=east,rectangle,fill=red!50]
				{\strut Important:~#1};}}%
	}%
	\mdfsetup{innertopmargin=10pt,linecolor=red!50,%
		linewidth=2pt,topline=true,%
		frametitleaboveskip=\dimexpr-\ht\strutbox\relax
	}
	\begin{mdframed}[]\relax%
		\centering
		}{\end{mdframed}}



\newcounter{lem}[section]\setcounter{lem}{0}
\renewcommand{\thelem}{\arabic{section}.\arabic{lem}}
\newenvironment{defin}[1][]{%
	\refstepcounter{lem}%
	\ifstrempty{#1}%
	{\mdfsetup{%
			frametitle={%
				\tikz[baseline=(current bounding box.east),outer sep=0pt]
				\node[anchor=east,rectangle,fill=blue!20]
				{\strut Definition};}}
	}%
	{\mdfsetup{%
			frametitle={%
				\tikz[baseline=(current bounding box.east),outer sep=0pt]
				\node[anchor=east,rectangle,fill=blue!20]
				{\strut Definition:~#1};}}%
	}%
	\mdfsetup{innertopmargin=10pt,linecolor=blue!20,%
		linewidth=2pt,topline=true,%
		frametitleaboveskip=\dimexpr-\ht\strutbox\relax
	}
	\begin{mdframed}[]\relax%
		\centering
		}{\end{mdframed}}
\lhead{CSys - Databases}


\begin{document}
\begin{center}
\underline{\huge Database Schemas }
\end{center}
\section{Transactions and concurrency control}
\begin{itemize}
	\item We need to trust a DBMS
	\item We need mechanisms to ensure that the database:
	\begin{itemize}
		\item is reliable
		\item remains in a consistent state
	\end{itemize}
	\item Especially when
	\begin{itemize}
		\item software/hardware failures
		\item multiple users access the database simultaneously
	\end{itemize}
	\item Database Recovery - The process of restoring a database to a correct state after a failure
	\item Concurrency control protocols: prevent database accesses to interfere with each other
	\item Central notion in a DBMS:
	\begin{itemize}
		\item Transaction: an action carried out by a single user/program which reads/updates the database
	\end{itemize}
	\item At the end of a transaction:
	\begin{itemize}
		\item database again in consistent state
		\item valid integrity/referential constraints
	\end{itemize}
	\item During the execution of a transaction
	\begin{itemize}
		\item maybe in an inconsistent state
	\end{itemize}
	\item A transaction can have two outcomes:
	\begin{itemize}
		\item Committed - completes successfully
		\item Rolled back - does not complete successfully
	\end{itemize}
	\item Concurrency control: the process of managing simultaneous operations on the DB
	\item Two transactions may be:
	\begin{itemize}
		\item both correct by themselves
		\item but when executed simultaneously they may cause inconsistency in the database
	\end{itemize}
\end{itemize}
\section{Abstract data models}
\begin{itemize}
	\item Data definition model - specifies entities/attributes/relationships/constraints for the stored data
	\item However, DDL is too low level to describe the data organisation in a simple way, understandable by most users
	\item We need a data model - a collection of intuitive concepts describing data, their relationships and constraints
\end{itemize}
\section{Types of data organisation}
\begin{itemize}
	\item Three characterisations of data:
	\begin{itemize}
		\item Structured data
		\begin{itemize}
			\item Data represented in strict format
			\item The DBMS checks to ensure that the data follows:
			\begin{itemize}
				\item the structures
				\item the integrity and referential constraints
			\end{itemize}
		\end{itemize}
		\item Semi structured data
		\begin{itemize}
			\item Self describing data
			\item the "schema" information is mixed with the data values
			\item How do we end up with such data?
			\begin{itemize}
				\item sometimes ad-hoc
				\item not known in advance how it will be stored/managed
			\end{itemize}
			\item This data may have some structure, but:
			\begin{itemize}
				\item not all the parts of the data have the same structure
				\item each data object may have different attributes that are not known in advance
			\end{itemize}
		\end{itemize}
		\item Unstructured data
		\begin{itemize}
			\item Very limited indication of the type/structure of the data
		\end{itemize}
	\end{itemize}
\end{itemize}
\section{Database design}
\begin{itemize}
	\item Conceptual design
	\begin{itemize}
		\item Construct a first, high level model of the data: ER model
		\item using the users' requirements specification
		\item independently of any physical considerations
		\item it serves as the fundamental understanding of the system
	\end{itemize}
	\item Logical design
	\begin{itemize}
		\item construct the relational data model of the data
		\item Using the conceptual design map entities and relationships to tables
		\item use normalization techniques to eliminate data redundancy and anomalies
	\end{itemize}
	\item Physical design
	\begin{itemize}
		\item describe the database implementation of the logical design
		\item specific storage structures/ access methods/ security protection
		\item aim is optimum performance
	\end{itemize}
\end{itemize}
\end{document}