\documentclass{article}[18pt]
\usepackage{../../../../../format}
\lhead{CSys - Databases}


\begin{document}
\begin{center}
\underline{\huge The relational data model}
\end{center}
\section{The relational data model}
\begin{itemize}
	\item DDL is too low level and not easily understandable by most users
	\item We need a data model: A collection of intuitive concepts describing data, their relationships and constraints
	\item Relational data model
	\begin{itemize}
		\item Relations between data are stored in tables
		\item Based on the concept of mathematical relations
		\item The most widely used data model (for structured data)
	\end{itemize}
	\item Intuitively in our data, every entity combines various attributes together
	\item The schema of a relation - the description of a particular collection of data in the model
	\item Let $A_1,A_2,...,A_n$ be a set of attributes that can be related
	\item The $R(A_1,A_2,...,A_n)$ is the scheme of the relation R
	\item In a relation schema the ordering of the attributes does not matter
	\item A database has many entities, each with its own attributes
	\item This information is decomposed into smaller pieces where every relation stores only one piece of the information
	\item However there is data duplication where two customers have the same account, and null values are needed where data is not complete
\end{itemize}
\section{Relation Model Terminology}
\begin{itemize}
	\item \textbf{Relation} - A table
	\item \textbf{Attribute} - A named column of a relation. Every attribute has a unique name
	\item \textbf{Domain} - The set of allowable values of an attribute
	\item \textbf{Tuple} - A row of a relation - every tuple has a concrete value for every attribute (not left empty, use NULL if no data)
	\item \textbf{Cell} - The intersection of a row and a column
	\item \textbf{Degree} - The number of attributes (every row stores as many values as the degree of the relation)
	\item \textbf{Cardinality} - The number of tuples
	\item \textbf{Normalized} - Appropriately structured (every cell has exactly one value, no repetitions of two identical rows)
	\item \textbf{Relational Database} - A collection of normalised relations
\end{itemize}
\section{Instances of branch and staff relations}
\begin{itemize}
	\item NULL Value
	\begin{itemize}
		\item a special case of a cell entry
		\item It represents an attribute value that is either currently unknown or not applicable
		\item Not the same as 0
		\item May or may not belong to the domain of the attribute
	\end{itemize}
\end{itemize}
\section{Properties of relations}
\begin{itemize}
	\item The relation name is distinct from all other relation names in the relational schema
	\item Each attribute within a relation has a distinct name
	\item Values of an attribute are all from the same domain
	\item Each cell of relation contains exactly one atomic value
	\item Each tuple is distinct among the tuples of the relation
	\item The ordering of the attributes has no significance
	\item The ordering of tuples has no significance
\end{itemize}
\section{Keys}
\begin{itemize}
	\item How do we uniquely identify a tuple in a normalized table?
	\begin{itemize}
		\item attribute names are unique within a table
		\item but two tuples may share attribute values
	\end{itemize}
	\item Every table must have some attributes, such as:
	\begin{itemize}
		\item Their value uniquely determines a tuple of the table
		\item These attribute are the primary key of the table
	\end{itemize}
\end{itemize}
\begin{itemize}
	\item \textbf{Candidate key:} a minimal set of attributes whose values uniquely identify the tuples
	\item \textbf{Primary key:} The candidate key selected to identify rows uniquely with the table
	\item \textbf{Alternate key:} Those candidate keys not selected as primary key
	\item \textbf{Simple key:} The key consists of only one attribute
	\item \textbf{Composite key:} The key consists of only one several attributes
\end{itemize}
\section{Integrity constraints}
\begin{itemize}
\item So far we have seen \textbf{domain constraints} for the attributes
\item Entity integrity - every attribute of a \textbf{primary key} can \textbf{not} be \textbf{NULL}
\item Purpose of entity integrity
\begin{itemize}
	\item guarantees that each entity has a unique identifier
	\item ensures that foreign key values can reference primary key values
\end{itemize}
\end{itemize}
\section{Integrity constraints}
\begin{itemize}
\item Referential integrity
\begin{itemize}
\item a foreign key either matches the primary key it refers to or it is null
\end{itemize}
\item Purpose of referential integrity
\begin{itemize}
\item any reference between tables is valid (or it has not been set yet)
\item Prevents deleting a row in a table B, if the primary key of B has a matching foreign key in another table A
\end{itemize}
\end{itemize}
\section{Summary: Characteristics of a relational table}
\begin{itemize}
\item A relation is represented by a two dimensional table
\item Each row (tuple) signifies a entity occurence
\item No two rows can be identical (each row of the table is unique)
\item Each column represents an attribute and has a distinct name
\item The intersection of a row and column has a single value (atomic)
\item All values in a column must be of the same type
\item One (or more) attributes uniquely identify each row (primary key)
\item Two tables can be dependent (the primary key is the foreign key of another table)
\item The ordering of rows and columns does not matter
\end{itemize}
\section{Views}
\begin{itemize}
\item So far all relations we have seen
\begin{itemize}
\item Base relations
\item Its tuples are \textbf{physically stored} in the database
\end{itemize}
\item A different type of relation: a view
\begin{itemize}
\item a virtual relation
\item it does not exist physically in the database
\end{itemize}
\item The content of a view
\begin{itemize}
\item is derived from one (or more) base relations
\item is computer upon request by a user, at the time of request
\item changes when the underlying base relations change
\end{itemize}
\item Main use
\begin{itemize}
\item show customised information to every user
\item computer dynamic quantities
\end{itemize}
\end{itemize}
\section{Alternatives to the relational data model}
\begin{itemize}
\item Network data model
\begin{itemize}
\item records appear as nodes
\item relationships appear as edges
\end{itemize}

\item Hierarchical data model
\begin{itemize}
\item Special case of the network data model, where the graph is a tree graph
\item its structure mirrors parent child relationship
\item limitations of the model e.g.
\begin{itemize}
\item deleting a parent
\item adding a record without a parent
\end{itemize}
\end{itemize}
\end{itemize}
\end{document}