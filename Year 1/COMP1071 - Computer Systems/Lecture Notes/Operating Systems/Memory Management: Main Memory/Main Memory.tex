\documentclass{article}[18pt]
\ProvidesPackage{format}
%Page setup
\usepackage[utf8]{inputenc}
\usepackage[margin=0.7in]{geometry}
\usepackage{parselines} 
\usepackage[english]{babel}
\usepackage{fancyhdr}
\usepackage{titlesec}
\hyphenpenalty=10000

\pagestyle{fancy}
\fancyhf{}
\rhead{Sam Robbins}
\rfoot{Page \thepage}

%Characters
\usepackage{amsmath}
\usepackage{amssymb}
\usepackage{gensymb}
\newcommand{\R}{\mathbb{R}}

%Diagrams
\usepackage{pgfplots}
\usepackage{graphicx}
\usepackage{tabularx}
\usepackage{relsize}
\pgfplotsset{width=10cm,compat=1.9}
\usepackage{float}

%Length Setting
\titlespacing\section{0pt}{14pt plus 4pt minus 2pt}{0pt plus 2pt minus 2pt}
\newlength\tindent
\setlength{\tindent}{\parindent}
\setlength{\parindent}{0pt}
\renewcommand{\indent}{\hspace*{\tindent}}

%Programming Font
\usepackage{courier}
\usepackage{listings}
\usepackage{pxfonts}

%Lists
\usepackage{enumerate}
\usepackage{enumitem}

% Networks Macro
\usepackage{tikz}


% Commands for files converted using pandoc
\providecommand{\tightlist}{%
	\setlength{\itemsep}{0pt}\setlength{\parskip}{0pt}}
\usepackage{hyperref}

% Get nice commands for floor and ceil
\usepackage{mathtools}
\DeclarePairedDelimiter{\ceil}{\lceil}{\rceil}
\DeclarePairedDelimiter{\floor}{\lfloor}{\rfloor}

% Allow itemize to go up to 20 levels deep (just change the number if you need more you madman)
\usepackage{enumitem}
\setlistdepth{20}
\renewlist{itemize}{itemize}{20}

% initially, use dots for all levels
\setlist[itemize]{label=$\cdot$}

% customize the first 3 levels
\setlist[itemize,1]{label=\textbullet}
\setlist[itemize,2]{label=--}
\setlist[itemize,3]{label=*}

% Definition and Important Stuff
% Important stuff
\usepackage[framemethod=TikZ]{mdframed}

\newcounter{theo}[section]\setcounter{theo}{0}
\renewcommand{\thetheo}{\arabic{section}.\arabic{theo}}
\newenvironment{important}[1][]{%
	\refstepcounter{theo}%
	\ifstrempty{#1}%
	{\mdfsetup{%
			frametitle={%
				\tikz[baseline=(current bounding box.east),outer sep=0pt]
				\node[anchor=east,rectangle,fill=red!50]
				{\strut Important};}}
	}%
	{\mdfsetup{%
			frametitle={%
				\tikz[baseline=(current bounding box.east),outer sep=0pt]
				\node[anchor=east,rectangle,fill=red!50]
				{\strut Important:~#1};}}%
	}%
	\mdfsetup{innertopmargin=10pt,linecolor=red!50,%
		linewidth=2pt,topline=true,%
		frametitleaboveskip=\dimexpr-\ht\strutbox\relax
	}
	\begin{mdframed}[]\relax%
		\centering
		}{\end{mdframed}}



\newcounter{lem}[section]\setcounter{lem}{0}
\renewcommand{\thelem}{\arabic{section}.\arabic{lem}}
\newenvironment{defin}[1][]{%
	\refstepcounter{lem}%
	\ifstrempty{#1}%
	{\mdfsetup{%
			frametitle={%
				\tikz[baseline=(current bounding box.east),outer sep=0pt]
				\node[anchor=east,rectangle,fill=blue!20]
				{\strut Definition};}}
	}%
	{\mdfsetup{%
			frametitle={%
				\tikz[baseline=(current bounding box.east),outer sep=0pt]
				\node[anchor=east,rectangle,fill=blue!20]
				{\strut Definition:~#1};}}%
	}%
	\mdfsetup{innertopmargin=10pt,linecolor=blue!20,%
		linewidth=2pt,topline=true,%
		frametitleaboveskip=\dimexpr-\ht\strutbox\relax
	}
	\begin{mdframed}[]\relax%
		\centering
		}{\end{mdframed}}
\lhead{CSys - OS}


\begin{document}
\begin{center}
\underline{\huge Memory Management: Main Memory}
\end{center}
\section{Usage of Memory}
\subsection{Memory Sharing}
As a result of CPU scheduling, we can improve:
\begin{itemize}
	\item the utilisation of the CPU
	\item the speed of the computer's response to its users
\end{itemize}
To realise this increase in performance, however, we must be able to keep several processes in memory at once
\subsection{Types of Memory}
Memory comes in many types:
\begin{itemize}
	\item Cache
	\item Main Memory
	\item Storage Memory
	\item Virtual Memory
\end{itemize}
The cache and main memory are referred to as primary storage. The cache primarily supports, and it managed by, the CPU. It is invisible to the operating system. It uses the same memory management approaches as main memory, therefore we are going to focus on main memory
\subsection{Logical/Physical Addressing}
\begin{itemize}
	\item \textbf{Logical address}: The address generated by the CPU
	\item \textbf{Physical address}: The address seen by the memory management unit
	\item \textbf{Memory Management Unit (MMU) scheme}: the value in the relocation register is added to every address generated by a user process at the time it is sent to memory
	\item The user (application) program deals with logical addresses; it never sees the real physical address
\end{itemize}
\subsection{Memory Partitioning}
Memory is a finite resource and as with the CPU, needs managed to be used as efficiently as possible\\
\\
The main memory is usually split into two partitions:
\begin{itemize}
	\item Kernel processes are usually held in one partition
	\item User processes are held in another partition
\end{itemize}
Each partition is contained in a single contiguous section of memory
\subsection{Contiguous allocation}
The main memory is usually split into two partitions:
\begin{itemize}
	\item The resident operating system is usually held in low memory with the interrupt vector
	\item User processes are then held in high memory
	\item Each process is contained in a single contiguous section of memory
\end{itemize}
Basic hardware
\begin{itemize}
	\item Relocation register scheme is used to protect user processes from each other, and from changing operating-system code and data
	\item The relocation register contains the value of the smallest physical address
	\item The limit register contains the range of logical addresses and each logical address must be less than the limit register
\end{itemize}
Multiple-partition allocation:
\begin{itemize}
	\item Hole: block of available memory
	\item Holes of various sizes are scattered throughout memory
	\item When a process arrives, it is allocated memory from a hole large enough to accommodate it
	\item The OS maintains information about allocated partitions and free partitions (holes) 
\end{itemize}
\subsection{Fragmentation}
\textbf{External fragmentation}\\
External fragmentation is memory space that exists to satisfy a request but it is not contiguous\\
\\
We can reduce external fragmentation by \textbf{compaction}:
\begin{itemize}
	\item Shuffle memory contents to place all free memory in one block
	\item Only possible if relocation is dynamic, and is done at execution time
\end{itemize}
\textbf{Internal fragmentation}\\
Internal fragmentation is allocated memory that may be slightly larger than requested memory.\\
This size difference is memory internal to a partition, but not being used
\section{Paging}
The logical address space of a process can be non contiguous: the process is allocated physical memory whenever it is available\\
\\
The physical memory is divided into fixed size blocks called \textbf{frames}\\
Typically, the size is a power of 2, between 512 and 8192 bytes\\
\\
The logical memory is divided into blocks of the same size called \textbf{pages}\\
\\
The OS sets up a page table to translate logical to physical addresses\\
\\
Paging then leads to internal fragmentation
\subsection{Address translation scheme}
The address generated by the CPU is divided into:
\begin{itemize}
	\item Page number (p): used as an index into a page table which contains the base address of each page in physical memory
	\item Page offset (d): combined with base address to define the physical memory address that is sent to the memory unit
\end{itemize}
\subsection{Protection}
Memory implementation implemented by associating protection bit with each frame.\\
\\
A \textbf{Valid-invalid bit} is attached to each entry in the page table:
\begin{itemize}
	\item "vaid" indicates that the associated page is in the process: logical address space, and thus a legal page
	\item "invalid" indicates that the page is not in the process: logical address space
\end{itemize}
\section{Virtual memory}
Definition:\\
\textbf{Virtual memory} is the capability of the operating system that enables programs to address more memory locations than are actually provided in main memory\\
\\
Virtual memory systems help remove much of the burden of memory management from the programmers, freeing them to concentrate on application development.
\begin{itemize}
	\item The logical address space is much larger than the physical address space
	\item Pages are swapped in and out of main memory
	\item The physical address space is shared by several processes
	\item Only part of the process needs to be in the physical address space for execution
	\item A free frame list of the physical address space is maintained by the operating system
\end{itemize}
\subsection{Virtual address space}
Each process views the address space as a contiguous block of memory holding the objects it needs to execute
\begin{itemize}
	\item Code
	\item Data
	\item Heap
	\item Stack
\end{itemize}


\subsection{Instruction Execute Cycle}
A typical instruction execute cycle
\begin{itemize}
	\item Fetch an instruction from memory
	\item Decode the instruction
	\item Execute the instruction\\
	This may cause operands to be fetched into memory
	\item After the instruction acts over the operands the result may need to be stored in memory
\end{itemize}
\subsection{Binding of Instructions}
Program must be brought into memory and placed within a process for it to be executed from the ready queue\\
\\
Address binding of the instructions and data to memory addresses can happen at three different stages
\begin{itemize}
	\item \textbf{Compile time}: If memory location known, absolute code can be generated
	\item \textbf{Load time} Must generate relocatable code if memory location is not known at compile time
	\item \textbf{Execution time}: Binding delayed until run-time if the process can be moved during its execution from one memory segment to another\\
	\textit{Need hardware support for address maps e.g. base and limit registers} 
\end{itemize}
\subsection{Dynamic Loading}
\begin{itemize}
	\item Better memory space utilization (unused routine is never loaded)\\
	\item Useful when large amounts of code are needed to handle infrequently occurring cases
	\item No special support from the operating system is required (implemented through program design)
\end{itemize}
\subsection{Dynamic Linking}
\begin{itemize}
	\item Dynamic linking for system libraries, postpones until execution time
	\item A small piece of code, called the stub, is used to locate the appropriate memory-resident library routine
	\item The Stub replaces itself with the address of the routine, and executes the routine
	\item The Operating system is needed to check if a called routine is in some process: memory allocation
	\item All processes that use a library execute the same copy of the library
\end{itemize}

\subsection{Swapping}
A process can be \textbf{swapped} temporarily out of memory to a backing store, and brought back for continued execution\\
\\
A backing store is a fast disk large enough to accommodate copies of all memory images for all uses and must provide direct access to these memory images\\
\\
Roll out, roll in: swapping variant used for priority bases scheduling algorithms: lower priority process is swapped ot so higher priority processes can be loaded and executed\\\
\\
Major part of swap time is transfer time: total transfer time is directly proportional to the amount of memory swapped
\section{Memory Allocation}

\subsection{Dynamic storage problem}
How to satisfy a request of size n from a list of free holes?
\begin{itemize}
	\item \textbf{First-fit} Allocate the first hole that is big enough
	\item \textbf{Best-fit} Allocate the smallest hole that is big enough
	\begin{itemize}
		\item Must search the entire list, unless ordered by size
		\item Produces the smallest leftover hole
	\end{itemize}
	\item \textbf{Worst fit}: Allocate the largest hole:
	\begin{itemize}
		\item Must search the entire list
		\item Produces the largest leftover hole
	\end{itemize}
\end{itemize}
First fit and best fir are usually better than worst fir in terms of speed and storage utilisation


\subsection{Implementation of Page Table}
\begin{itemize}
	\item Page table it kept in main memory
	\item Page table base register (PTBR) points to the page table
	\item Page table length register (PTLR) indicates the size of the page table
	\item In this scheme every data/instruction access requires two memory accesses
	\begin{itemize}
		\item One for the page table and one for the data/instruction
		\item The two memory access problem can be solved with the use of a special fast-lookup hardware cache called the associative registers or translation look-aside buffers (TLBs)
	\end{itemize} 
\end{itemize}

\subsection{Shared spaces}
Private code and data:
\begin{itemize}
	\item Each process keeps a separate copy of the code and data
	\item The pages for the private code and data can appear anywhere in the logical address space
\end{itemize}
Shared code:
\begin{itemize}
	\item One copy of read only (re-entrant) code shared among processes (text editors, compilers, window systems)
	\item Shared code must appear in the same location in the logical address space of all processes
\end{itemize}
\section{Segmentation}
\textbf{Segmentation}: Memory-management scheme that supports the user view of memory.\\
\\
A program is a collection of segments. A segment is a logical unit such as:
\begin{itemize}
	\item Main program
	\item Procedure
	\item Function
	\item Local variables, global variables
	\item Common block
	\item Stack
	\item Symbol table, arrays...
\end{itemize}
\subsection{Benefits and drawbacks of Segmentation}
\textbf{Benefits}:
\begin{itemize}
	\item Protection is improved because segments represent semantically defined portions of the program, therefore instructions can be read only whereas data can be written to
	\item Sharing of code segment level is easier due to the read only properties
\end{itemize}
\textbf{Issues}:
\begin{itemize}
	\item Like with contiguous allocation: \textit{fragmentation}
	\item Like with paging: \textit{segment table} 
\end{itemize}
\end{document}