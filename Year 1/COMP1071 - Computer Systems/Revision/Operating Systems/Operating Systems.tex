\documentclass{article}[18pt]
\usepackage{../../../../format}
\lhead{CSys}


\begin{document}
\begin{center}
\underline{\huge Operating Systems}
\end{center}
\section{Introduction to Operating Systems}
\textbf{Operating Systems} - A program that acts as an intermediary between a user and the hardware\\
\\
Goals of an operating system:
\begin{itemize}
	\item Execute user programs
	\item Make solving user problems easier
	\item Make the computer system convenient to use
	\item Use the resources of the system fairly and efficiently
\end{itemize}
An operating system is a:
\begin{itemize}
	\item Resource allocator - Responsible for the management of the computer system resources
	\item Control program - Controls the execution of user programs and operation of I/O resources
	\item Kernel - The one program that runs all the time
\end{itemize}
\subsection{Processes}
\textbf{Process} - A unit of execution\\
\\
The operating system is responsible for process management including:
\begin{itemize}
	\item Process creation and deletion
	\item Process holding and resuming
	\item Mechanisms for process synchronization
\end{itemize}
A process includes:
\begin{itemize}
	\item Code: text selection
	\item Current activity, represented by the program counter and the contents of the CPU's registers
	\item Data stack: temporary data such as local variables
	\item Data section: Global variables
	\item Heap: Memory allocated while the process is running
\end{itemize}
Information about the process is represented by a PCB including:
\begin{itemize}
	\item Unique identifier
	\item State
	\item CPU utilization
	\item CPU scheduling
	\item Memory usage
	\item Other information
\end{itemize}
Process states:
\begin{itemize}
	\item New - The process being created
	\item Running - Instructions being executed
	\item Waiting - Process waiting for some event to occur
	\item Ready - Process ready to be dispatched
	\item Terminated - Process completed execution
\end{itemize}
\begin{center}
	\includegraphics[scale=0.7]{"Process State"}
\end{center}
\subsection{Process Creation}
A new process, as a parent process, can create a number of child processes, which, in turn create other processes, forming a tree of processes\\
\\
Resource sharing: three possible cases
\begin{itemize}
	\item The parent and child processes share all resources
	\item The child process shares a subset of parent's resources
	\item The parent and child processes share no resources
\end{itemize}
Execution: two possible cases:
\begin{itemize}
	\item The parent and child execute concurrently
	\item The parent waits until the child terminates
\end{itemize}
\subsection{Process Termination}
The process executes its last statement and asks the operating system to delete it:
\begin{itemize}
	\item Outputs data from the child's process to parent
	\item The child process's resources are de-allocated by operating system
\end{itemize}
The parent process may terminate execution of child processes if:
\begin{itemize}
	\item The child process has exceeded its allocated resources
	\item The task assigned to child is no longer required
	\item The parent is itself terminating (cascade termination)
\end{itemize}
\subsection{The Kernel}
Aims to provide an environment in which processes can exist.\\
\\
Four essential components:
\begin{itemize}
	\item Privileged instruction set
	\item Interrupt mechanism
	\item Memory protection
	\item Real-time clock
\end{itemize}
The kernel consists of:
\begin{itemize}
	\item The first-level interrupt handler: to manage interrupts
	\item The dispatcher: to switch the CPU between processes
	\item Intra operating system communications
\end{itemize}
\subsection{Interrupts}
Interrupt - A signal from either hardware or software of an event that will cause a change of process, for example
\begin{itemize}
	\item Hardware - Triggers an interrupt by sending a signal to the CPU 
	\item Software - Triggers an interrupt by executing a system call for some action by the operating system
\end{itemize}
Interrupt Routines - OS routines that execute whenever an interrupt occurs
\subsection{First Level Interrupt handler}
The function of the FLIH is to:
\begin{itemize}
	\item Determine the source of the interrupt (prioritise)
	\item Initiate servicing of the interrupt (selection of suitable process of the dispatcher)
\end{itemize}
\subsection{Privileged instructions}
Some instructions must be accessible only to the operating system: privileged instruction set\\
\\
Privileged instructions include functions as:
\begin{itemize}
	\item Managing interrupts
	\item Performing I/O
	\item Halting a process
\end{itemize}
\subsection{Dual Mode}
Aim: To distinguish between execution of an operating system code and user-defined code. We do not let the user execute instructions that would cause harm.\\
Two modes:
\begin{itemize}
	\item User mode
	\item Kernel mode
\end{itemize}
Switching from user mode to kernel mode occurs when:
\begin{itemize}
	\item A user process calls on the operating system to execute a function needing a privileged instruction (system call)
	\item An interrupt occurs (hardware)
	\item An error condition occurs in a user process (software)
	\item An attempt is made to execute a privileged instruction while in user mode
\end{itemize} 
The dispatcher
\begin{itemize}
	\item Assigns processing resource for processes
	\item Is later initiated when
	\begin{itemize}
		\item A current process cannot continue
		\item The CPU may be better used elsewhere, for instance:
		\begin{itemize}
			\item After an interrupt changes a process state
			\item After a system call which results in the current process not being able to continue
			\item After an error which causes a process to suspend
		\end{itemize}
	\end{itemize}
\end{itemize}
\section{Process Management}
\textbf{Independent process} - Cannot affect or be affected by the execution of other processes\\
\textbf{Co-operating processes} - Can affect or be affected by the execution of another process
\subsection{Multiprogramming}
Advantages of multiprogramming:
\begin{itemize}
	\item Computation speed-up - if there are multiple CPUs and/or cores
	\item Convenience  - Single user wants to run many tasks
	\item Information sharing - For example shared files
	\item Modularity - Programming
\end{itemize}
The aim of multiprogramming is to maximise CPU utilisation\\
With multiprogramming several memory are kept in memory concurrently\\
Use CPU scheduling to fit them so the CPU is always in use
\subsection{Schedulers}
\textbf{Long term scheduler} - Selects processes to be brought into the ready queue\\
\textbf{Medium term scheduler} - Removes process from active contention for the CPU by swapping processes in and out of the ready queue\\
\textbf{Short term scheduler} - Selects the process to be executed next and allocated to the CPU
\subsection{Scheduling Algorithms}
There are a number of different algorithms including:
\begin{itemize}
	\item First Come, First Served
	\begin{itemize}
		\item Just put them in a queue as they turn up
	\end{itemize}
	\item Shortest Job First
	\begin{itemize}
		\item A priority queue with the length of the job as the priority
	\end{itemize}
	\item Shortest Remaining Time First
	\begin{itemize}
		\item A pre-emptive version of SJF, if a new process arrives with time less than the remaining time of the process currently running, then pre-empt it
	\end{itemize}
	\item Priority (with or without pre-empting)
	\item Priority (with or without ageing)
	\item Round-Robin
	\begin{itemize}
		\item Give each a time slice unless they finish early, rotate FCFS
		\item If using priority, if two processes compete for the back of the queue, the higher priority goes first
	\end{itemize}
	\item Multilevel queue/feedback queue
\end{itemize}
\subsection{Priority Scheduling}
Some algorithms use priority for scheduling\\
\\
The problem with this is starvation - The low priority process may never execute\\
The solution to this is ageing - As time progresses increase the priority of the process\\
\\
\textbf{Multilevel queue scheduling} - Processes are partitioned into different queues, which have different (performance) requirements\\
\\
This has a problem that a process stays in the same queue regardless of adapting requirements\\
Feedback queues are used to solve this problem, allowing processes to proceed between queues, this is difficult to implement.
\begin{itemize}
	\item The ready queue is partitioned into separate queues (for example foreground and background)
	\item Each queue has its own scheduling algorithm
	\item Scheduling must then be undertaken between queues:
	\begin{itemize}
		\item Fixed priority scheduling
		\item Time slice - each queue gets a certain amount of CPU time which it can schedule amongst its processes
	\end{itemize}
\end{itemize}
\subsection{Threads}
\textbf{Thread} - A basic unit of CPU utilization\\
A thread shares some attributes with its peer threads (the same process) but may execute different code\\
\\
The benefit of using threading includes the following:
\begin{itemize}
	\item Responsiveness: The process continues running even if part of it is blocked or performing a lengthy operation
	\item Resource sharing: threads share the resources of their common process
	\item Economy: Allocating memory and resources for process creation is costly
	\item Multiprocessor architectures: a single threaded process can only run on one processor
	\item Performance: Cooperation of multiple threads in same job delivers higher throughput and improved performance
\end{itemize}
\subsubsection{Multithreading models}
Many to one:
\begin{itemize}
	\item Many user application threads to one kernel thread
	\item Programmer controls the number of application threads
\end{itemize}
One to one:
\begin{itemize}
	\item One user application to one kernel thread
\end{itemize}
Many-to-many
\begin{itemize}
	\item N user application threads to $\leqslant N$ kernel threads
	\item Defined by the capabilities of the operating system
\end{itemize}
\section{Memory Management: Main Memory and Virtual Memory}
\subsection{Types of Memory}
Memory comes in many types:
\begin{itemize}
	\item Cache memory/ CPU cache
	\item Main memory (e.g. RAM)
	\item Storage memory
	\item Virtual memory
\end{itemize}
\subsection{Logical/Physical Addressing}
\textbf{Logical address} - An address created by the CPU\\
\textbf{Physical address} - An address on the physical memory\\
\textbf{Memory Management Unit (MMU)} - Automatically translates virtual addresses to physical addresses
A user (application) program deals with logical addresses and never knows the real physical addresses
\subsection{Memory Partitioning}
Main memory is usually split into two partitions:
\begin{itemize}
	\item Kernel processes
	\item User processes
\end{itemize}
Each partition is contained in a single contiguous section of memory
\subsection{Contiguous Allocation}
\textbf{Hole} - Block of available memory\\
When a process arrives, it is allocated memory from a hole large enough to accommodate it
\subsection{Memory Hole Allocation}
There are a variety of strategies to satisfy a request from a list of free holes
\begin{itemize}
	\item First fit - first hole big enough
	\item Best fit - smallest hole big enough
	\item Worst fit - largest hole
\end{itemize}
First fit is fast and best fit is efficient
\subsection{Fragmentation}
\textbf{External fragmentation} - Memory space able to satisfy a request but not contiguous\\
\textbf{Compaction} - Shuffle memory contents to place all free memory in one block\\
\textbf{Internal fragmentation} - Memory allocated successfully but may be slightly larger than the requested amount of memory (due to blocks being powers of 2)
\subsection{Paging}
\textbf{Frames} - Fixed size blocks of physical memory\\
\textbf{Pages} - Blocks of logical memory\\
\textbf{Paging} - Finding the number of free frames needed to load the program\\
\textbf{Page table} - Maps logical and physical addresses
\subsection{Address translation scheme}
The logical address generated by the CPU is divided into:
\begin{itemize}
	\item Page number (p): used as an index into a page table which contains the base address of each page in physical memory
	\item Page offset (d): combined with base address to define the physical memory address that is sent to the memory unit
\end{itemize}
\subsection{Protection}
Memory protection implemented by associating protection bit with each frame\\
\\
A valid-invalid bit is attached to each entry in the page table:
\begin{itemize}
	\item Valid indicates that the associated page is in the process' logical address space, and is thus a legal page
\end{itemize}
\subsection{Virtual memory}
\textbf{Virtual memory} - The capability of the operating system that enable programs to address more memory locations than are actually provided in main memory
\subsection{Demand paging}
Bring a page into memory only when it is required by the process during its execution
\subsection{A Process's Page Table}
\begin{itemize}
	\item When reference is made to a page's address
	\begin{itemize}
		\item Invalid reference $\rightarrow$ abort
		\item Not in memory $\rightarrow$ bring into memory
	\end{itemize}
	\item A valid/invalid bit is associated with each process's page table entry to indicate if the page is already in memory
	\begin{itemize}
		\item 1 $\rightarrow$ in memory
		\item $0 \rightarrow$ not in memory
	\end{itemize}
	\item Address translation
	\begin{itemize}
		\item When valid/invalid bit in page table entry is 0 $\rightarrow$ page fault trap
	\end{itemize}
	\item Initially the valid/invalid bit is set to 0 on all entries
\end{itemize}
\subsection{Handling Page Fault Traps}
If there is a page fault trap
\begin{itemize}
	\item Identify a free frame from the free frame list
	\item Read the page into the identified frame
	\item Update the process's page table
	\item Restart the instruction interrupted by the page fault trap
\end{itemize}
If there is no free frame then page replacement needs to be used\\
\textbf{Page replacement} - Find some page in memory, but not really in use, and swap it out
\subsection{Page replacement algorithms}
\subsubsection{Belady's Anomaly}
This is that for some page fault algorithms, the page fault rate my increase as the number of allocated frames increases
\subsubsection{Optimal Algorithm}
Replace the page that will not be needed for the longest time\\
This is impossible to implement as it would need you to know the future
\subsubsection{Least Recently used}
\begin{itemize}
	\item Associate with each page the time of its last use
	\item Page to replace: Page that has not been used for the longest period of time
	\item This will never exhibit Belady's Anomaly as the pages of smaller frames are always a subset of pages for larger frames
\end{itemize}
This has an approximation with an additional reference bit:
\begin{itemize}
	\item When a page is referenced set to 1
	\item Replace one of the pages with the bit set to 0
	\item Replace a random page when none are set to 0
\end{itemize}
Another approximation is the second chance algorithm:
\begin{itemize}
	\item A circular FIFO queue
	\item When a page enters main memory it joins the tail of the queue and has reference bit set to 1
	\item When a victim page needs selecting
	\begin{itemize}
		\item Start at the head of the queue
		\item If the reference page is 0 select it as the victim page
		\item If the page reference is 1, change it to 0 and move on to the next page
	\end{itemize}
\end{itemize}
\subsection{Allocation of frames}
Global replacement - Select a replacement frame from the set of all frames, one process can take a frame from another\\
Local replacement  - Select from only the process' own set of allocated frames\\
Fixed allocation - Give a set of frames at the start of the process\\
Priority allocation - Get the replacement frame from a process with a lower priority
\subsection{Thrashing}
This occurs when a process spends more time paging than doing actual work
\subsection{Prepaging}
\textbf{Prepaging} - Page into memory at one time all the pages that will be needed\\
This should prevent thrashing, but we need to evaluate the cost of prepaging
\subsection{Page-Fault Frequency Scheme}
\begin{itemize}
	\item Establish an acceptable page fault rate
	\item If too low then take frames away
	\item If too high give frames 
\end{itemize}
\section{Mass-Storage Systems}
\textbf{Transfer rate} - Rate of data flow between drive and computer\\
\textbf{Positioning time} - Time to move disk arm to desired cylinder (\textbf{seek time}) and time for desired sector to rotate under the disk head (\textbf{rotational latency})\\
\textbf{Head crash} - Head makes contact with disk surface\\
\\
Access Latency=Average access time=Average seek time + average latency\\
Average I/O time=average access time +(amount to transfer/transfer rate)+controller overhead 
\subsection{Disk scheduling algorithms}
\subsubsection{First come first served}
Just move through the list
\subsubsection{Shortest Seek Time First}
Selects the request with the minimum seek time from the current head position
May cause starvation of some requests
\subsubsection{SCAN}
Start at one end of the disk and move to the end of the disk servicing requests, then reverse and continue servicing
\subsubsection{C-SCAN}
Like SCAN, but doesn't service requests on the way back
\subsubsection{C-LOOK}
Arm only does as far as the last request in each direction
\subsection{Disk management}
\textbf{Low level formatting} - Dividing a disk into sectors that the disk controller can read and write\\
\textbf{Partition} - Dividing a disk into one or more groups of cylinders, each treated as a logical disk\\
\textbf{Logical Formatting} - Making a file system\\
\textbf{Cluster} - A group of blocks\\
\textbf{Bootstrap loader} - A program stored in boot blocks of boot partition
\subsection{Swap-space management}
\textbf{Swap-space} - Virtual memory uses disk space as an extension of main memory\\
Swap space can be carved out of the normal file system, but is usually in a separate partition\\
\subsection{RAID}
RAID - Using multiple disk drives to provide reliability via redundancy \\
Increases mean time to failure\\
Striping - Using a group of disks as one storage unit
\section{File-System Interface}
File - Contiguous logical address space
\subsection{File locking}
File locks can exist either shared for many processes (used when reading) or exclusive lock (used when writing)
File locks can be:
\begin{itemize}
	\item Mandatory - Access denied depending on locks held
	\item Advisory - Process find status of locks and decide what to do
\end{itemize}
\section{File System Implementation}
\textbf{File control block} - Storage structure consisting of information about a file\\
\textbf{Device driver} - Controls the physical device\\
\textbf{Boot control block} - Contains info needed by system to boot OS from that volume\\
\textbf{Virtual File system} - A way to provide an object oriented way of implementing file systems\\
\textbf{Linear List} - List of file names with pointer to the data blocks\\
\textbf{Hash Table} - Linear list with file data structure\\
\textbf{Collisions} - Situations where two file names hash to the same location\\
\textbf{Contiguous Allocation} - Each file occupies a set of contiguous blocks\\
\textbf{Extent} - Contiguous chunk of blocks\\
\textbf{Indexed allocation} - Each file has its own index block(s) of pointers to its data blocks
\end{document}