\documentclass{article}[18pt]
\usepackage{../../../../format}
\lhead{Algorithms and Data Structures - Matthew Johnson}
\usepackage{listings}
\usepackage{pxfonts}

\lstset{language=C,
    basicstyle=\ttfamily,
    keywordstyle=\bfseries,
    showstringspaces=false,
    morekeywords={if, else, then, print, end, for, do, while},
    tabsize=4
}


\begin{document}
\begin{center}
\underline{\huge Introduction and Pseudocode}
\end{center}
\section{Algorithms}
\textbf{Algorithm} - A method or a process followed to solve a problem\\
Properties an algorithm must have
\begin{itemize}
\item Correctness
\item Composed of concrete unambiguous steps
\item The number of steps must be finite
\item Must terminate
\end{itemize}
\section{Data Structures}
\textbf{Data Structure} - A particular way of storing and organizing data in a computer so it can be used efficiently
\section{Machine Model}
Random Access Machine 
\begin{itemize}
\item Memory consists of an infinite array
\item Instructions executed sequentially one at a time
\item All instructions take unit time. Running time is the number of executions executed 
\end{itemize}
\section{Pseudocode}
To describe algorithms we will use generic pseudocode, not any one programming language.\\
There are no very strict rules on how it should be written.\\
\\
Will often use variables: int, float, char, str\\
Declare variable type before using.\\
\\
Different formats for setting equal.\\
\\
Some conventions declare type separate to value, some do not\\
\\
Logical operators included.\\
\\
Keyword for output is print, to concatenate strings use a comma, for example "The value of z is",z\\
\\
\subsection{If then else}
\textbf{if} condition \textbf{then}\\
$\textrm{} \qquad$ statement\\
\textbf{end if}\\
To make one thing happen if a statement is true, and one if it is false, two if statements can be used, but using an else statement is better, for example\\
\begin{lstlisting}[mathescape=true]
if x $\neq$ 0 then
	z=y/x
else
	print "division by zero error"
end if
\end{lstlisting}



end if and indentation is used to make things clearer
\subsection{For loop}
If you want to iterate some (numeric) variable through some range\\
\\
Great many variations in how languages do this, simplest is probably
\begin{lstlisting}[mathescape=true]
for variable=lower to upper do
	body
end for
\end{lstlisting}
Body is simply a sequence of statements\\
Example:
\begin{lstlisting}[mathescape=true]
s=0
integer L
integer U
for i=L to U do
	s=s+i
end for
print s
\end{lstlisting}
Things this for loop does without being written down:
\begin{itemize}
\item Set I=L
\item Iterate i by 1
\item Check if $i\leqslant U$
\end{itemize}


In this for loop, it can only iterate consecutive integers\\
A more generic one can iterate over a given bases set:
\begin{lstlisting}[mathescape=true]
for value in {value1, value2 ...} do
	body
end for
\end{lstlisting}
To iterate in more than 1, for example in 2, this can also use negative numbers to reduce:
\begin{lstlisting}[mathescape=true]
for i=0 to 9;i+=2 do
	print i
end for
\end{lstlisting}

\subsection{While loop}
Do something while a condition is true:
\begin{lstlisting}[mathescape=true]
while condition do
	body
end while
\end{lstlisting}
Important difference between \textbf{for} and \textbf{while} over numerical values:\\
\textbf{for} increments loop-variable automatically; \textbf{while} doesn't\\
\\
Everything that happens in a \textbf{while} loop is explicit, unlike a for loop\\

\begin{lstlisting}[mathescape=true]
for i=1 to 10 do
	print i
end for
\end{lstlisting}
\begin{lstlisting}[mathescape=true]
i=1
while $i\leqslant10$ do
	print i
	i=i+1
end while
\end{lstlisting}






\end{document}