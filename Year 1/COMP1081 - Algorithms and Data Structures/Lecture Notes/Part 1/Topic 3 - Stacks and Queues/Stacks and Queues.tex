\documentclass{article}[18pt]
\ProvidesPackage{format}
%Page setup
\usepackage[utf8]{inputenc}
\usepackage[margin=0.7in]{geometry}
\usepackage{parselines} 
\usepackage[english]{babel}
\usepackage{fancyhdr}
\usepackage{titlesec}
\hyphenpenalty=10000

\pagestyle{fancy}
\fancyhf{}
\rhead{Sam Robbins}
\rfoot{Page \thepage}

%Characters
\usepackage{amsmath}
\usepackage{amssymb}
\usepackage{gensymb}
\newcommand{\R}{\mathbb{R}}

%Diagrams
\usepackage{pgfplots}
\usepackage{graphicx}
\usepackage{tabularx}
\usepackage{relsize}
\pgfplotsset{width=10cm,compat=1.9}
\usepackage{float}

%Length Setting
\titlespacing\section{0pt}{14pt plus 4pt minus 2pt}{0pt plus 2pt minus 2pt}
\newlength\tindent
\setlength{\tindent}{\parindent}
\setlength{\parindent}{0pt}
\renewcommand{\indent}{\hspace*{\tindent}}

%Programming Font
\usepackage{courier}
\usepackage{listings}
\usepackage{pxfonts}

%Lists
\usepackage{enumerate}
\usepackage{enumitem}

% Networks Macro
\usepackage{tikz}


% Commands for files converted using pandoc
\providecommand{\tightlist}{%
	\setlength{\itemsep}{0pt}\setlength{\parskip}{0pt}}
\usepackage{hyperref}

% Get nice commands for floor and ceil
\usepackage{mathtools}
\DeclarePairedDelimiter{\ceil}{\lceil}{\rceil}
\DeclarePairedDelimiter{\floor}{\lfloor}{\rfloor}

% Allow itemize to go up to 20 levels deep (just change the number if you need more you madman)
\usepackage{enumitem}
\setlistdepth{20}
\renewlist{itemize}{itemize}{20}

% initially, use dots for all levels
\setlist[itemize]{label=$\cdot$}

% customize the first 3 levels
\setlist[itemize,1]{label=\textbullet}
\setlist[itemize,2]{label=--}
\setlist[itemize,3]{label=*}

% Definition and Important Stuff
% Important stuff
\usepackage[framemethod=TikZ]{mdframed}

\newcounter{theo}[section]\setcounter{theo}{0}
\renewcommand{\thetheo}{\arabic{section}.\arabic{theo}}
\newenvironment{important}[1][]{%
	\refstepcounter{theo}%
	\ifstrempty{#1}%
	{\mdfsetup{%
			frametitle={%
				\tikz[baseline=(current bounding box.east),outer sep=0pt]
				\node[anchor=east,rectangle,fill=red!50]
				{\strut Important};}}
	}%
	{\mdfsetup{%
			frametitle={%
				\tikz[baseline=(current bounding box.east),outer sep=0pt]
				\node[anchor=east,rectangle,fill=red!50]
				{\strut Important:~#1};}}%
	}%
	\mdfsetup{innertopmargin=10pt,linecolor=red!50,%
		linewidth=2pt,topline=true,%
		frametitleaboveskip=\dimexpr-\ht\strutbox\relax
	}
	\begin{mdframed}[]\relax%
		\centering
		}{\end{mdframed}}



\newcounter{lem}[section]\setcounter{lem}{0}
\renewcommand{\thelem}{\arabic{section}.\arabic{lem}}
\newenvironment{defin}[1][]{%
	\refstepcounter{lem}%
	\ifstrempty{#1}%
	{\mdfsetup{%
			frametitle={%
				\tikz[baseline=(current bounding box.east),outer sep=0pt]
				\node[anchor=east,rectangle,fill=blue!20]
				{\strut Definition};}}
	}%
	{\mdfsetup{%
			frametitle={%
				\tikz[baseline=(current bounding box.east),outer sep=0pt]
				\node[anchor=east,rectangle,fill=blue!20]
				{\strut Definition:~#1};}}%
	}%
	\mdfsetup{innertopmargin=10pt,linecolor=blue!20,%
		linewidth=2pt,topline=true,%
		frametitleaboveskip=\dimexpr-\ht\strutbox\relax
	}
	\begin{mdframed}[]\relax%
		\centering
		}{\end{mdframed}}
\lhead{Algorithms and Data Structures}
\lstset{language=Python,
    basicstyle=\ttfamily,
    keywordstyle=\bfseries,
    showstringspaces=false,
    morekeywords={if, else, then, print, end, for, do, while},
    tabsize=4
}


\begin{document}
\begin{center}
\underline{\huge Stacks and Queues}
\end{center}
\section{Syntax Checking}
How can we check whether the syntax is correct?\\
\texttt{public void add( int idx, AnyType x)
\{ if( theItems.length == size( ) )
ensureCapacity( size ( ) * 2 + 1); for( int
i=theSize; i > idx; i- ) theItems[ i ] =
theItems[ i - 1 ]; theItems[ idx ] = x;
theSize++; \}  }\\
\\
Check that all opened brackets are closed as part of checking that the syntax is correct\\
\\
A stack is good for the problem of bracket checking as the bracket closed should match with the last opened bracket.
\begin{lstlisting}[mathescape=true]
input B, a sequence of brackets
output accept or reject

stack S
for b in B
	if b is an open bracket then
		S.push(b)
	end if	
	
	if b is a close bracket then
		b'=S.pop()
		if b$\neq$b' don't match then
			reject
if S.isEmpty is false then
	reject
accept
\end{lstlisting}
Requirements
\begin{itemize}
\item Every open bracket must be closed
\item Each close bracket matches the last open bracket that has not been closed
\end{itemize}
\section{Stacks}
\begin{itemize}
\item A stack is a collection of objects that are inserted and removed according to the \textbf{last-in-first-out} (LIFO) principle
\item Objects can be inserted into a stack at any time, but only the most recently inserted object (the \textbf{last}) can be removed at any time
\end{itemize}
\subsection{Methods}
A stack supports the following methods
\begin{itemize}
\item \textbf{push(e)}: Insert element e at the top of the stack
\item \textbf{pop}: Remove and return the top element of the stack; an error occurs if the stack is empty
\end{itemize}
And possibly also:
\begin{itemize}
\item \textbf{size}: Return the number of elements in the stack
\item \textbf{isEmpty}: Return a Boolean indicating if the stack is empty
\item \textbf{top}: Return the top element in the stack, without removing it; an error occurs if the stack is empty
\end{itemize}
\subsection{Example}
\textit{What are the effects of the following on an initially empty stack? What is the output of each and what are the contents of the stack?}
\begin{itemize}
\item push(5) - 5
\item push(3) - 3,5
\item pop - 5
\item push(7) - 7,5
\item pop - 5
\item top -5
\item pop - 
\item pop - ERROR
\item isEmpty - True
\end{itemize}
\subsection{Implementation using arrays}
\begin{itemize}
\item In an array based implementation, the stack consists of an N-element array S, and an integer variable t that gives the top element of the stack
\item We initialise t to -1, and we use this value for t to identify an empty stack. The size of the stack is t+1
\end{itemize}
\subsection{Methods}
\begin{lstlisting}[mathescape=true, language=Python]
# How do you find the find the size of the stack
size
	return t+1
\end{lstlisting}

\begin{lstlisting}[mathescape=true, language=Python]
# How do you determine if the stack is empty
isEmpty
	if t<0
		return True
	else
		return False
\end{lstlisting}
\begin{lstlisting}[mathescape=true, language=Python]
# Return the top element of the stack
top
	if isEmpty then
		throw a EmptyStackException
	end if
	return S[i]
\end{lstlisting}
\begin{lstlisting}[mathescape=true, language=Python]
# Add an element to the stack
push(e)
	# If size = N then the stack is full
	if size = N then
		throw a FullStackException
	end if
	t=t+1
	S[t]=e
\end{lstlisting}
\begin{lstlisting}[mathescape=true]
# Remove an element from the stack
pop()
if isEmpty then
	throw a EmptyStackException
end if
e = S[t]
S[t] = NULL
t = t - 1
return e
\end{lstlisting}
\subsection{Implementation using arrays}
\begin{itemize}
\item The array based stack implementation is time efficient. The time taken by all methods does not depend on the size of the stack
\item However, the fixed size N of the array can be a serious limitation:
\begin{itemize}
\item If the size of the stack is much less than the size of the array, we waste memory
\item If the size of the stack exceeds the size of the array, the implementation will generate an exception
\end{itemize}
\item The array based implementation of the stack has fixed capacity
\end{itemize}
\textit{How could you implement the stack using a linked list?}\\
The easiest way is for the top of the stack to be at the head of the list, as it is better to insert and remove data from here




\section{Queues}
\begin{itemize}
\item A \textbf{queue} is a collection of objects that are inserted and removed according to the \textbf{first-in-first-out} principle
\item Element access and deletion are restricted to the first element in the sequence, which is called the \textbf{front} of the queue
\item Element insertion is restricted to the end of the sequence, which is called the \textbf{rear} of the queue
\end{itemize}
\subsection{Methods}
A queue supports the following methods
\begin{itemize}
\item \textbf{enqueue(e)}: Insert element e at the rear of the queue
\item \textbf{dequeue}: Remove and return from the queue the element at the front; an error occurs if the queue is empty
\end{itemize}
And possibly also:
\begin{itemize}
\item \textbf{size}: Return the number of elements in the queue
\item \textbf{isEmpty}: Return a boolean indicating if the queue is empty
\item \textbf{front}: Return the front element of the queue, without removing it; an error occurs if the queue is empty
\end{itemize}
\subsection{Example}
\textit{What are the effects of the following on an initially empty queue? What is the output of each and what are the contents of the queue?}
\begin{itemize}
\item enqueue(5) - [5]
\item enqueue(3) - [3,5]
\item dequeue - [3]
\item enqueue(7) - [7,3]
\item dequeue - [7]
\item front - Output 7
\item dequeue - []
\item dequeue - ERROR
\item isEmpty - True
\end{itemize}
\subsection{Implementation using arrays}
\textit{How can we implement a queue using an array Q of size N?}
\begin{itemize}
\item We could put the front of the queue at Q[0] and let the queue grow from there
\item This is not efficient. It requires moving all the elements forward one array cell each time we perform a dequeue operation .
\item Instead, we use two variables f and r which have the following meaning:
\begin{itemize}
\item f is an index to the cell of Q storing the front of the queue, unless the queue is empty, in which case f=r
\item r is an index to the next available cell in Q, that is, the cell after the rear of Q, if Q is not empty
\end{itemize}
\item Initially we assign f=r=0, indicating that the queue is empty
\item After each enqueue operation we increment r. After each dequeue operation we increment f
\item r is incremented after each enqueue operation and never decremented. After N enqueue operations we would get an array-out-of-bounds error. So something is done when r=0
\item To avoid this problem, we let r and f \textbf{wrap around} the end of Q, by using modulo N  arithmetic on them
\end{itemize}
\subsection{Methods}
\subsubsection{Size}
\begin{lstlisting}[mathescape=true]
return (N-f+r) mod N
# This allows for loop around as when the number gets bigger than N 
# it will return the number into the LHS the array goes
\end{lstlisting}
\subsubsection{isEmpty}
\begin{lstlisting}[mathescape=true]
return (f=r)
\end{lstlisting}
\subsubsection{Front}
\begin{lstlisting}[mathescape=true]
if isEmpty then
	throw a EmptyQueueException
end if
\end{lstlisting}
\subsubsection{enqueue(e)}
\begin{lstlisting}[mathescape=true]
if size=N-1 then
	throw a FullQueueException
end if
Q[r]=e
r=r+1 mod N
\end{lstlisting}
Plus 1 so that don't use last cell as f=r and the queue would be empty, even though it is full
\subsubsection{Dequeue}
\begin{lstlisting}[mathescape=true]
if isEmpty then
	throw a EmptyQueueException
end if
temp=Q[f]
Q[f]=NULL
f=f+1 mod N
return temp
\end{lstlisting}



\subsection{Implementation using arrays}
\begin{itemize}
\item If the size of the queue is N, then f=r and the isEmpty method returns true, even though the queue is not empty
\item We avoid this problem by keeping the maximum number of elements that can be stored in the queue to N-1. See the FullQueueException in the enqueue algorithm
\item The array based implementation of the queue is time efficient. All methods run in constant time
\item Similarly to the array based implementation of the stack, the capacity of array based implementation of array based implementation of the queue is fixed
\item What if we used linked list implementation instead? 
\begin{itemize}
\item Put in data with each element linked together, front of queue at head of list, rear of queue at tail of list
\item To take something from the front of the queue, update the head to the next element in the queue
\item To add new data to the queue, create a new node, and point the tail at that, and point the new node to the previously added node
\end{itemize}

\section{More}
\begin{lstlisting}[mathescape=true]
input integer k
output binary representation of k
stack S
while k>0
	S.push (remainder of k/2)
	k=k/2
while S isEmpty=False
	S.pop()

\end{lstlisting}
\end{itemize}





\end{document}