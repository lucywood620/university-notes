\documentclass{article}[18pt]
\ProvidesPackage{format}
%Page setup
\usepackage[utf8]{inputenc}
\usepackage[margin=0.7in]{geometry}
\usepackage{parselines} 
\usepackage[english]{babel}
\usepackage{fancyhdr}
\usepackage{titlesec}
\hyphenpenalty=10000

\pagestyle{fancy}
\fancyhf{}
\rhead{Sam Robbins}
\rfoot{Page \thepage}

%Characters
\usepackage{amsmath}
\usepackage{amssymb}
\usepackage{gensymb}
\newcommand{\R}{\mathbb{R}}

%Diagrams
\usepackage{pgfplots}
\usepackage{graphicx}
\usepackage{tabularx}
\usepackage{relsize}
\pgfplotsset{width=10cm,compat=1.9}
\usepackage{float}

%Length Setting
\titlespacing\section{0pt}{14pt plus 4pt minus 2pt}{0pt plus 2pt minus 2pt}
\newlength\tindent
\setlength{\tindent}{\parindent}
\setlength{\parindent}{0pt}
\renewcommand{\indent}{\hspace*{\tindent}}

%Programming Font
\usepackage{courier}
\usepackage{listings}
\usepackage{pxfonts}

%Lists
\usepackage{enumerate}
\usepackage{enumitem}

% Networks Macro
\usepackage{tikz}


% Commands for files converted using pandoc
\providecommand{\tightlist}{%
	\setlength{\itemsep}{0pt}\setlength{\parskip}{0pt}}
\usepackage{hyperref}

% Get nice commands for floor and ceil
\usepackage{mathtools}
\DeclarePairedDelimiter{\ceil}{\lceil}{\rceil}
\DeclarePairedDelimiter{\floor}{\lfloor}{\rfloor}

% Allow itemize to go up to 20 levels deep (just change the number if you need more you madman)
\usepackage{enumitem}
\setlistdepth{20}
\renewlist{itemize}{itemize}{20}

% initially, use dots for all levels
\setlist[itemize]{label=$\cdot$}

% customize the first 3 levels
\setlist[itemize,1]{label=\textbullet}
\setlist[itemize,2]{label=--}
\setlist[itemize,3]{label=*}

% Definition and Important Stuff
% Important stuff
\usepackage[framemethod=TikZ]{mdframed}

\newcounter{theo}[section]\setcounter{theo}{0}
\renewcommand{\thetheo}{\arabic{section}.\arabic{theo}}
\newenvironment{important}[1][]{%
	\refstepcounter{theo}%
	\ifstrempty{#1}%
	{\mdfsetup{%
			frametitle={%
				\tikz[baseline=(current bounding box.east),outer sep=0pt]
				\node[anchor=east,rectangle,fill=red!50]
				{\strut Important};}}
	}%
	{\mdfsetup{%
			frametitle={%
				\tikz[baseline=(current bounding box.east),outer sep=0pt]
				\node[anchor=east,rectangle,fill=red!50]
				{\strut Important:~#1};}}%
	}%
	\mdfsetup{innertopmargin=10pt,linecolor=red!50,%
		linewidth=2pt,topline=true,%
		frametitleaboveskip=\dimexpr-\ht\strutbox\relax
	}
	\begin{mdframed}[]\relax%
		\centering
		}{\end{mdframed}}



\newcounter{lem}[section]\setcounter{lem}{0}
\renewcommand{\thelem}{\arabic{section}.\arabic{lem}}
\newenvironment{defin}[1][]{%
	\refstepcounter{lem}%
	\ifstrempty{#1}%
	{\mdfsetup{%
			frametitle={%
				\tikz[baseline=(current bounding box.east),outer sep=0pt]
				\node[anchor=east,rectangle,fill=blue!20]
				{\strut Definition};}}
	}%
	{\mdfsetup{%
			frametitle={%
				\tikz[baseline=(current bounding box.east),outer sep=0pt]
				\node[anchor=east,rectangle,fill=blue!20]
				{\strut Definition:~#1};}}%
	}%
	\mdfsetup{innertopmargin=10pt,linecolor=blue!20,%
		linewidth=2pt,topline=true,%
		frametitleaboveskip=\dimexpr-\ht\strutbox\relax
	}
	\begin{mdframed}[]\relax%
		\centering
		}{\end{mdframed}}
\lhead{Algorithms and Data Structures}
\lstset{language=Python,
    basicstyle=\ttfamily,
    keywordstyle=\bfseries,
    showstringspaces=false,
    morekeywords={if, else, then, print, end, for, do, while},
    tabsize=4
}

\begin{document}
\begin{center}
\underline{\huge Recursion}
\end{center}
\section{Factorials}
The factorial of a positive integer n is the product of the integers 1 to n, hence we can implement the calculation of factorials using recursion
\begin{lstlisting}[mathescape=true]
#Iterative Factorial
total=1
for i=1 to n do
	total=total$\times i$
end for
return total
\end{lstlisting}
\begin{lstlisting}[mathescape=true]
#Recursive Factorial factorial(n)
if n=0 then
	return 1
else
	return n$\times$factorial(n-1)
end if
\end{lstlisting}
\section{Recursive Algorithms}
\begin{itemize}
\item A recursive algorithm must have a \textbf{base case}
\item A recursive algorithm must \textbf{change} its state and move towards the base case
\item A recursive algorithm must \textbf{call itself}, recursively 
\end{itemize}
\section{Lists}
\begin{lstlisting}[mathescape=true]
# Iterative sum of a list L
sum=0
for i in L do
	sum=sum+i
end for
return sum
\end{lstlisting}
\begin{lstlisting}[mathescape=true]
#Recursive sum of a list L:listsum(L)
if len(L)=1 then
	return L(0)
else
	return L[0]+listsum(L[1:])
end if
return sum
\end{lstlisting}
\section{Examples on the whiteboard}
\subsection{Fibonacci}
\begin{lstlisting}[mathescape=true]
#Iterative Fibonacci sequence
input n
fibs=[0,1]
for i=2 to n
	fibs=fibs+sum of last 2 entries
return fibs[n]
\end{lstlisting}
\begin{lstlisting}[mathescape=true]
# Recursive Fibonacci sequence
Input n
for n $\leqslant$ 1
	return n
fibs
	return fibs(n-1)+fibs(n-2) 
\end{lstlisting}
\begin{itemize}
\item This is exceptionally slow as the whole previous Fibonacci sequence is calculated in order to calculate each number, whereas with iteration the values are calculated then stored.
\item The number of calls will be on the order of $2^{N}$ with recursion, whereas iteration will use N calls
\item A way to speed up the recursive step is to store the values calculated, as values are calculated multiple times in the previous method
\end{itemize}
\subsection{Palindromes}
\begin{lstlisting}[mathescape=true]
input string s
output true or false
isPalindrome(S)
	if length S$\leqslant$1 then
		return true
	end if
	if the first and last characters don't match then
		return false
	else
		isPalindrome(S with first and last characters removed)
\end{lstlisting}
\subsection{Flood fill a bounded area}
\begin{lstlisting}[mathescape=true]
floodfill(x,y)
	input screen,pixel(x,y)
	output screen with the contiguous region that contains that pixel filled with the colour
	if (x,y) is filled then
		return
	fill(x,y) #change (x,y) to blue
	floodfill(x+1,y),floodfill(x-1,y),floodfill(x,y+1),floodfill(x,y-1)
\end{lstlisting}


\section{Memoization}
Store the obtained results.\\
\\
For example a memoized recursive algorithm for fibonacci numbers would be:
\begin{lstlisting}[mathescape=true]
fib_memo[0]=0
fib_memo[1]=1
fib(n) input n, output nth Fibonacci Number
if not fib_memo[n]
	fib_memo[n]=fib(n-1)+fib(n-2)
return fib_memo[n]
\end{lstlisting}

\section{Backtracking}
\begin{itemize}
\item A technique for problems with many \textbf{candidate} solutions but too many to try
\item For example there are a huge number of ways to fill a sudoku grid
\item General idea: Build up the solution one step at a time, \textbf{backtracking} when unable to continue
\end{itemize}
\subsection{Informal generic algorithm}
\begin{enumerate}
\item Do I have a solution yet?
\item No. Can I extend my solution by one "step"?
\item If yes, do that
\item Do I have a solution now? If yes, I'm done
\item If not, try and extend again
\item When I can't extend, take one step back and try a different way
\item If no other extension available, then give up - no solution can be found
\end{enumerate}
\subsection{Pseudocode generic algorithm}
\begin{lstlisting}[mathescape=true]
#extend_solution(current solution)
if current solution is valid then
	if current solution is complete then
		return current solution
	else
		for each extension of the current solution do
			extend_solution(extension)
		end for
	end if
end if
\end{lstlisting}
\section{Knights tour}
A knight's tour: To move a knight around a chessboard such that each square is visited exactly \textbf{once} before the knight returns to its \textbf{starting position}
\begin{lstlisting}[mathescape=true]
#extend_solution(current solution)
if cnew move is to unvisited square on board then
	if every square is visited then
		return current solution
	else
		for each of eight possible moves do
			extend_solution(with move added)
		end for
	end if
end if
\end{lstlisting}
\subsection{Implementing Knights Tour}
\begin{itemize}
\item Rather than having a list of moves made, it is easier to maintain an $8\times8$ array recording when each square was visited (initially all values are zero)
\item Use a counter to record how many squares have been visited
\item If every square has been visited, we must also check that we can return to the start in one more move
\end{itemize}
\section{Queens Problem}
Queen's Problem: Find all possible ways of placing 8 queens on a chessboard such that none attacks any other
\begin{lstlisting}[mathescape=true]
#extend_solution(current solution)
if new queen is not under attack then
	if 8 queens on board then
		return current solution
	else
		for each of eight squares in the next row do
			extend_solution(with extra queen)
		end for
	end if
end if
\end{lstlisting}



\end{document}