\documentclass{article}[18pt]
\usepackage{../../../../format}
\usepackage{listings}
\lhead{Algorithms and Data Structures}
\lstset{language=python,
	basicstyle=\ttfamily,
	keywordstyle=\bfseries,
	showstringspaces=false,
	tabsize=4,
	mathescape=true
}


\begin{document}
\begin{center}
\underline{\huge Practical 5}
\end{center}
\section{Question 1}
\begin{tabular}{|c|c|c|c|c|c|c|c|c|c|}
	\hline 
	0 & 1 & 2 & 3 & 4 & 5 & 6 & 7 & 8 & 9 \\ 
	\hline 
	&  &  & Sandy &  &  &  & George & Alan & Amy \\ 
	\hline 
\end{tabular} 
\section{Question 2}
A function indexing at 0 is not as good as indexing as one because there would be a disproportionate number of names at index 0 because any name including an a would be at that index.
\section{Question 3}
\begin{lstlisting}
def hash(d):
	#initialize table
	table = ["-"]*13
	#now you do the rest
	for item in d:
		itemsto=item
		if item in table:
			continue
		if '-' not in table:
			break
		while(True):
			index=itemsto%13
				if table[index]=='-':
					table[index]=item
					break
				else:
					itemsto+=1
	return(table)	
\end{lstlisting}

\section{Question 4}
\lstset{morekeywords={if, else, then, print, end, for, do, while}}
\begin{lstlisting}
modulus(m,n)
	if m<n then
		return m
	else
		return modulus(m-n,n)
	end if
\end{lstlisting}
\section{Question 5}
\begin{lstlisting}
DigitSum(n)
if length(n)=1 then
	return n
else
	return n mod 10 + digitsum(n/10)
end if
\end{lstlisting}
For any number n, doing n mod 10 will provide the last number in the number. Then by calling digitsum(n/10) as pseudocode floors all decimal numbers, n/10 will be the number, excuding the last digit.
\end{document}