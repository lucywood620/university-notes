\documentclass{article}[18pt]
\ProvidesPackage{format}
%Page setup
\usepackage[utf8]{inputenc}
\usepackage[margin=0.7in]{geometry}
\usepackage{parselines} 
\usepackage[english]{babel}
\usepackage{fancyhdr}
\usepackage{titlesec}
\hyphenpenalty=10000

\pagestyle{fancy}
\fancyhf{}
\rhead{Sam Robbins}
\rfoot{Page \thepage}

%Characters
\usepackage{amsmath}
\usepackage{amssymb}
\usepackage{gensymb}
\newcommand{\R}{\mathbb{R}}

%Diagrams
\usepackage{pgfplots}
\usepackage{graphicx}
\usepackage{tabularx}
\usepackage{relsize}
\pgfplotsset{width=10cm,compat=1.9}
\usepackage{float}

%Length Setting
\titlespacing\section{0pt}{14pt plus 4pt minus 2pt}{0pt plus 2pt minus 2pt}
\newlength\tindent
\setlength{\tindent}{\parindent}
\setlength{\parindent}{0pt}
\renewcommand{\indent}{\hspace*{\tindent}}

%Programming Font
\usepackage{courier}
\usepackage{listings}
\usepackage{pxfonts}

%Lists
\usepackage{enumerate}
\usepackage{enumitem}

% Networks Macro
\usepackage{tikz}


% Commands for files converted using pandoc
\providecommand{\tightlist}{%
	\setlength{\itemsep}{0pt}\setlength{\parskip}{0pt}}
\usepackage{hyperref}

% Get nice commands for floor and ceil
\usepackage{mathtools}
\DeclarePairedDelimiter{\ceil}{\lceil}{\rceil}
\DeclarePairedDelimiter{\floor}{\lfloor}{\rfloor}

% Allow itemize to go up to 20 levels deep (just change the number if you need more you madman)
\usepackage{enumitem}
\setlistdepth{20}
\renewlist{itemize}{itemize}{20}

% initially, use dots for all levels
\setlist[itemize]{label=$\cdot$}

% customize the first 3 levels
\setlist[itemize,1]{label=\textbullet}
\setlist[itemize,2]{label=--}
\setlist[itemize,3]{label=*}

% Definition and Important Stuff
% Important stuff
\usepackage[framemethod=TikZ]{mdframed}

\newcounter{theo}[section]\setcounter{theo}{0}
\renewcommand{\thetheo}{\arabic{section}.\arabic{theo}}
\newenvironment{important}[1][]{%
	\refstepcounter{theo}%
	\ifstrempty{#1}%
	{\mdfsetup{%
			frametitle={%
				\tikz[baseline=(current bounding box.east),outer sep=0pt]
				\node[anchor=east,rectangle,fill=red!50]
				{\strut Important};}}
	}%
	{\mdfsetup{%
			frametitle={%
				\tikz[baseline=(current bounding box.east),outer sep=0pt]
				\node[anchor=east,rectangle,fill=red!50]
				{\strut Important:~#1};}}%
	}%
	\mdfsetup{innertopmargin=10pt,linecolor=red!50,%
		linewidth=2pt,topline=true,%
		frametitleaboveskip=\dimexpr-\ht\strutbox\relax
	}
	\begin{mdframed}[]\relax%
		\centering
		}{\end{mdframed}}



\newcounter{lem}[section]\setcounter{lem}{0}
\renewcommand{\thelem}{\arabic{section}.\arabic{lem}}
\newenvironment{defin}[1][]{%
	\refstepcounter{lem}%
	\ifstrempty{#1}%
	{\mdfsetup{%
			frametitle={%
				\tikz[baseline=(current bounding box.east),outer sep=0pt]
				\node[anchor=east,rectangle,fill=blue!20]
				{\strut Definition};}}
	}%
	{\mdfsetup{%
			frametitle={%
				\tikz[baseline=(current bounding box.east),outer sep=0pt]
				\node[anchor=east,rectangle,fill=blue!20]
				{\strut Definition:~#1};}}%
	}%
	\mdfsetup{innertopmargin=10pt,linecolor=blue!20,%
		linewidth=2pt,topline=true,%
		frametitleaboveskip=\dimexpr-\ht\strutbox\relax
	}
	\begin{mdframed}[]\relax%
		\centering
		}{\end{mdframed}}
\lhead{ADS}
\usepackage{enumitem}
\usepackage{listings}
\lstset{language=C,
	basicstyle=\ttfamily,
	keywordstyle=\bfseries,
	showstringspaces=false,
	morekeywords={if, else, then, print, end, for, do, while},
	tabsize=4,
	mathescape=true
}
\begin{document}
\begin{center}
\underline{\huge Practical 6}
\end{center}
\section{Question 1}
Consider the numbers $K_n$ defined by
$$K _ { n } = \left\{ \begin{array} { l l } { n } & { \text { for } n \leq 3 } \\ { K _ { n - 1 } + 2 K _ { n - 2 } + 3 K _ { n - 3 } } & { \text { for } n \geq 4 } \end{array} \right.$$
\begin{enumerate}[label=(\alph*)]
\item Calculate $K_8$
\begin{itemize}
	\item $K_8=K_7+2k_6+3k_5=293$
	\item $K_7=k_6+k_5+2k_4=125$
	\item $k_6=k_5+2k_4+3k_3=22+20+9=51$
	\item $k_5=k_4+6+6=22$
	\item $k_4=3+4+3=10$
\end{itemize}
\item Write pseudocode for a recursive function that returns $K_n$ for an integer $n>0$
\begin{lstlisting}
def K (n)
if n $\leqslant$ 3 then
	return n
else
	return K(n-1)+2K(n-2)+3K(n-3)
\end{lstlisting}
\item Write pseudocode for a non recursive function that returns $K_n$ for an integer $n>0$
\begin{lstlisting}
list L[1,2,3]
input n
for i in range 3,n do
	L.append(i[n-1]+2i[n-2]+3i[n-3])
end for
print L[n+3]	
\end{lstlisting}
\item Which of the two functions you have defined would it be better to implement
\begin{itemize}
	\item Both would take a similar time and storage capacity as they both have to store the data in a list/stack for the results from 1 up to n, and so are performing a very similar operation
\end{itemize}
\end{enumerate}
\section{Question 2}
Consider the problem of finding increasing subsequences in a sequence of
integers. For example, in the sequence:
$$2, 6, 1, 9, 4, 8, 5, 10, 7$$
2,6,9 is such a subsequence, 1,4,8,10 is another and 1,4,5,7 a third (since the
numbers appear within the sequence in that order, not necessarily consecutively).
Use recursion to describe a procedure for finding the longest increasing
subsequence. (Is using recursion the best approach?)
\begin{lstlisting}
LIS(s)
result R
if len(S) =1
	return s
else
	x=s[-1]
	L=LIS(s-x)
	t=s with x and larger integers removed
	M=LIS(t)+x
	return whichever of L and M is longer
end if
\end{lstlisting}
\section{Question 4}
\begin{lstlisting}
Input: coordinates of the midpoint(mp), the length of the line (l) and recursion depth d.
Output: drawing of a H Tree.
	calculate the points of the H h1,h2,h3 and h4
if d = 0 then
	draw a H with midpoint mp and line length l
else
	draw a H on each of the points of the h1...h4 with line length l/2
end if
\end{lstlisting}


\end{document}