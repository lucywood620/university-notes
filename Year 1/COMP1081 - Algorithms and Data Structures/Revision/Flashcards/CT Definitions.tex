\documentclass[grid,avery5371]{flashcards}
\usepackage{amssymb}
\usepackage[utf8]{inputenc}
\usepackage[T1]{fontenc}
\usepackage{verse}
\usepackage[version=3]{mhchem}
\usepackage{graphicx}
\settowidth{\versewidth}{It lies behind stars and under hills,}
\addtolength{\versewidth}{2em}
\usepackage{pgfplots}
\geometry{headheight=12pt}
\usepackage{fancyhdr}
\pagestyle{fancy}
\fancyhf{}
\renewcommand{\headrulewidth}{0pt}
\hyphenpenalty=100000

\title{Conversion flashcards}
\author{Sam Robbins}

\cardbackstyle[\large]{plain}
\cardfrontstyle[\large]{headings}
\cardfrontfoot{ADS}
%\cardbackstyle[\large]{headings}
\setlength{\cardheight}{1.0in}
\setlength{\oddevenshift}{1.0in}

%\renewcommand{\cardrows}{8}
\begin{document}

\begin{flashcard}[]{Algorithm}
A method or process followed to solve a problem
\end{flashcard}

\begin{flashcard}[]{Data Structure}
	A particular way of storing and organising data in a computer so that it can be used efficiently 
\end{flashcard}

\begin{flashcard}[]{Stack}
	A collection of elements that are inserted and removed according to the last-in-first-out (LIFO) principle
\end{flashcard}

\begin{flashcard}[]{Queue}
	A collection of objects that are inserted and removed according to the first-in-first-out (FIFO) principle
\end{flashcard}

\begin{flashcard}[]{Hash table}
	Consists of a bucket array and a hash function
\end{flashcard} 

\begin{flashcard}[]{Bucket Array}
	An array A of size N where each cell A is thought of as a bucket storing a collection of key-value pairs
\end{flashcard}

\begin{flashcard}[]{Capacity}
	The size of the hash table
\end{flashcard}

\begin{flashcard}[]{Open addressing schemes}
	Store at most one entry in each bucket
\end{flashcard}

\begin{flashcard}[]{Separate Chaining}
	Each bucket A[i] stores a list holding the entries (k,v) such that h(k)=i
\end{flashcard}

\begin{flashcard}[]{Second-Choice Hashing}
	Compute two hash functions and store the key value pair in the bucket containing the fewest items
\end{flashcard}

\begin{flashcard}[]{Linear Probing}
	Try to insert into A[i], then A[(i+1) mod N] and so on until we find and empty bucket
\end{flashcard}

\begin{flashcard}[]{Robin Hood Hashing}
	{\small
	Variation of linear probing. If, during probing with a new key, an existing key is found that is "closer to home" than the new key, then the existing key is displaced and replaced by the new key}
\end{flashcard}

\begin{flashcard}[]{Quadratic Probing}
{\small
This iteratively tries the buckets:
$$A[(i+f(j))mod N], \text{for} j=0,1,2,..., \text{where} f(j)=j^2$$
}
\end{flashcard}

\begin{flashcard}[]{Double hashing}
{\small
Choose a secondary hash function h' to choose a bucket if the initial hash function leads to a full bucket
\[
A[(i+f(j)) \bmod N], \text { for } j=0,1,2, \ldots, \text { where } f(j)=j \dot{h'}(k)
\]
}
\end{flashcard}

\begin{flashcard}[]{Cuckoo Hashing}
There are two tables with two corresponding hash functions
\end{flashcard}

\begin{flashcard}[]{Tombstone}
A marker left in a bucket after something has been deleted
\end{flashcard}

\begin{flashcard}[]{Backtracking}
	Build up the solution one step at a time, backtracking when unable to continue
\end{flashcard}

\begin{flashcard}[]{Degree of a polynomial}
	The highest power in the polynomial
\end{flashcard}

\begin{flashcard}[]{Monotonic}
	Always going in one direction (either increasing or decreasing)
\end{flashcard}


\begin{flashcard}[]{Time complexity}
	Expressed in terms of the number of basic operations used by the algorithm when the input has a particular size
\end{flashcard}

\begin{flashcard}[]{Big-O}
	$$|f(x)|\leqslant C\cdot |g(x)|$$
	whenever $x\geqslant k$
\end{flashcard}

\begin{flashcard}[]{Sum Rule}
The $\mathcal{O}$ of the sum of two functions is the maximum of the $\mathcal{O}$ of the two
\end{flashcard}

\begin{flashcard}[]{Product Rule}
	The $\mathcal{O}$ of the product of two functions is the product of their $\mathcal{O}$
\end{flashcard}

\begin{flashcard}[]{Big-Omega}
	$$|f(x)|\geqslant C\cdot |g(x)|$$
\end{flashcard}

\begin{flashcard}[]{Theta}
	$f(x)$ is $\mathcal{O}(g(x))$ and $g(x)$ is $\mathcal{O}(f(x))$
\end{flashcard}

\begin{flashcard}[]{Little-o}
	\[
	\lim _{x \rightarrow \infty} \frac{f(x)}{g(x)}=0
	\]
\end{flashcard}

\begin{flashcard}[]{Little-Omega}
	\[
	f=\omega(g) \quad \Leftrightarrow \quad g=o(f)
	\]
\end{flashcard}

\begin{flashcard}[]{Comparisons needed for a comparison based sorting algorithm}
	For any comparison based sorting algorithm $\mathcal{A}$ and any $n\in \mathbb{N}$ large enough there exists an input of length n that requires $\mathcal{A}$ to perform $\Omega(n\log n)$ comparisons
\end{flashcard}

\begin{flashcard}[]{Bucket Sort}
Puts elements with key i into the ith bucket, then empties one bucket after another
\end{flashcard}

\begin{flashcard}[]{Radix sort}
	Like bucket sort but keeps sorting by different levels
\end{flashcard}

\begin{flashcard}[]{Binary Search}
	Look at the middle of the list, use that to determine which half it is in, recursively call on the sublists
\end{flashcard}

\begin{flashcard}[]{Tree}
	A connected graph without cycles
\end{flashcard}

\begin{flashcard}[]{Binary Search Tree}
	A tree in which no node as more than two children. All elements in the left subtree are smaller than v and all in the right are bigger
\end{flashcard}

\begin{flashcard}[]{Types of edges for DFS}	{\small
	\textbf{Tree} - Edges in the DFS-Forest\\
	\textbf{Back} - Join a vertex to an ancestor\\
	\textbf{Forward} - Not in the tree but join a vertex to its descendant\\
	\textbf{Cross} - All others}
\end{flashcard}

\begin{flashcard}[]{Kruskal's Algorithm}
	{\small
Sort edges by weight\\
Let A=$\varnothing$\\
Consider the edges in increasing order of weight. For each edge e, add e to A unless it would create a cycle}
\end{flashcard}

\begin{flashcard}[]{Prim's Algorithm}
{\small
Let $U=\{u\}$ where u is some vertex chosen arbitrarily\\
Let $A=\varnothing$\\
Until U contains all vertices: find the least-weight edge e that joins a vertex w not in U and add e to A and w to U


}
\end{flashcard}



\end{document}