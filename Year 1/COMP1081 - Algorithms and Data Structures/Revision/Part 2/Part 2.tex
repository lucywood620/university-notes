\documentclass{article}[18pt]
\ProvidesPackage{format}
%Page setup
\usepackage[utf8]{inputenc}
\usepackage[margin=0.7in]{geometry}
\usepackage{parselines} 
\usepackage[english]{babel}
\usepackage{fancyhdr}
\usepackage{titlesec}
\hyphenpenalty=10000

\pagestyle{fancy}
\fancyhf{}
\rhead{Sam Robbins}
\rfoot{Page \thepage}

%Characters
\usepackage{amsmath}
\usepackage{amssymb}
\usepackage{gensymb}
\newcommand{\R}{\mathbb{R}}

%Diagrams
\usepackage{pgfplots}
\usepackage{graphicx}
\usepackage{tabularx}
\usepackage{relsize}
\pgfplotsset{width=10cm,compat=1.9}
\usepackage{float}

%Length Setting
\titlespacing\section{0pt}{14pt plus 4pt minus 2pt}{0pt plus 2pt minus 2pt}
\newlength\tindent
\setlength{\tindent}{\parindent}
\setlength{\parindent}{0pt}
\renewcommand{\indent}{\hspace*{\tindent}}

%Programming Font
\usepackage{courier}
\usepackage{listings}
\usepackage{pxfonts}

%Lists
\usepackage{enumerate}
\usepackage{enumitem}

% Networks Macro
\usepackage{tikz}


% Commands for files converted using pandoc
\providecommand{\tightlist}{%
	\setlength{\itemsep}{0pt}\setlength{\parskip}{0pt}}
\usepackage{hyperref}

% Get nice commands for floor and ceil
\usepackage{mathtools}
\DeclarePairedDelimiter{\ceil}{\lceil}{\rceil}
\DeclarePairedDelimiter{\floor}{\lfloor}{\rfloor}

% Allow itemize to go up to 20 levels deep (just change the number if you need more you madman)
\usepackage{enumitem}
\setlistdepth{20}
\renewlist{itemize}{itemize}{20}

% initially, use dots for all levels
\setlist[itemize]{label=$\cdot$}

% customize the first 3 levels
\setlist[itemize,1]{label=\textbullet}
\setlist[itemize,2]{label=--}
\setlist[itemize,3]{label=*}

% Definition and Important Stuff
% Important stuff
\usepackage[framemethod=TikZ]{mdframed}

\newcounter{theo}[section]\setcounter{theo}{0}
\renewcommand{\thetheo}{\arabic{section}.\arabic{theo}}
\newenvironment{important}[1][]{%
	\refstepcounter{theo}%
	\ifstrempty{#1}%
	{\mdfsetup{%
			frametitle={%
				\tikz[baseline=(current bounding box.east),outer sep=0pt]
				\node[anchor=east,rectangle,fill=red!50]
				{\strut Important};}}
	}%
	{\mdfsetup{%
			frametitle={%
				\tikz[baseline=(current bounding box.east),outer sep=0pt]
				\node[anchor=east,rectangle,fill=red!50]
				{\strut Important:~#1};}}%
	}%
	\mdfsetup{innertopmargin=10pt,linecolor=red!50,%
		linewidth=2pt,topline=true,%
		frametitleaboveskip=\dimexpr-\ht\strutbox\relax
	}
	\begin{mdframed}[]\relax%
		\centering
		}{\end{mdframed}}



\newcounter{lem}[section]\setcounter{lem}{0}
\renewcommand{\thelem}{\arabic{section}.\arabic{lem}}
\newenvironment{defin}[1][]{%
	\refstepcounter{lem}%
	\ifstrempty{#1}%
	{\mdfsetup{%
			frametitle={%
				\tikz[baseline=(current bounding box.east),outer sep=0pt]
				\node[anchor=east,rectangle,fill=blue!20]
				{\strut Definition};}}
	}%
	{\mdfsetup{%
			frametitle={%
				\tikz[baseline=(current bounding box.east),outer sep=0pt]
				\node[anchor=east,rectangle,fill=blue!20]
				{\strut Definition:~#1};}}%
	}%
	\mdfsetup{innertopmargin=10pt,linecolor=blue!20,%
		linewidth=2pt,topline=true,%
		frametitleaboveskip=\dimexpr-\ht\strutbox\relax
	}
	\begin{mdframed}[]\relax%
		\centering
		}{\end{mdframed}}
\lhead{ADS}
\lstset{language=Python,
	basicstyle=\ttfamily,
	keywordstyle=\bfseries,
	showstringspaces=false,
	morekeywords={if, else, then, print, end, for, do, while},
	tabsize=4,
	mathescape=true,
	numbers=left,
	stepnumber=1,
}
\usepackage{caption}
\DeclareCaptionFont{white}{\color{white}}
\DeclareCaptionFormat{listing}{\colorbox{gray}{\parbox{\textwidth}{#1#2#3}}}
\captionsetup[lstlisting]{format=listing,labelfont=white,textfont=white}
\begin{document}
\begin{center}
\underline{\huge Part 2}
\end{center}
\section{Recap}
\textbf{Degree of a polynomial} - The highest power in the polynomial\\
\textbf{Monotonic} - Always going in one direction (either increasing or decreasing)
\subsection{Logarithms}
\[
\log _{a} x y=\log _{a} x+\log _{a} y
\]
\[
\log _{a} \frac{x}{y}=\log _{a} x-\log _{a} y
\]
\[
\log _{a} x^{s}=s \cdot \log _{a} x
\]
\[
\log _{a} x=\frac{\log _{b} x}{\log _{b} a}
\]
\section{Asymptotics}
\textbf{Time complexity} - Expressed in terms of the number of basic operations used by the algorithm when the input has a particular size\\
\textbf{Worst-Case time complexity} - Expressed in terms of the largest number of basic operations used by the algorithm when the input has a particular size
\subsection{Big-O}
Let f(x) and g(x) be functions from the set of integers or the set of real numbers to the set of real numbers. We say that f(x) is $\mathcal{O}(g(x))$ if there are constants C and k such that
\[
|f(x)| \leq C \cdot|g(x)|
\]
whenever $x\geqslant k$\\
\\
The constants C and k in the definition of big-$\mathcal{O}$ notation are called witnesses to the relationship $f(x)$ is $\mathcal{O}(g(x))$
\subsubsection{Example}
\textit{Let $f(x)=x^2+2x+1$. Then $f(x)=\mathcal{O}(x^2)$}\\
\\
For $x\geqslant 1$, we have $1\leqslant x \leqslant x^2$. That gives
\[
f(x)=x^{2}+2 x+1 \leq x^{2}+2 x^{2}+x^{2}=4 x^{2}
\]
for $x\geqslant 1$. Because the above inequality holds for every positive $x\geqslant 1$, using $k=1$ and $C=4$ as witnesses, we get
\[
f(x) \leq C \cdot x^{2}
\]
for every $x\geqslant k$
\subsubsection{Example 2}
\textit{Let $f(x)=3^x$. Then $f(x)$ is not $\mathcal{O}(2^x)$ }\\
Assume that there are constants k and C such that $3^x\leqslant C\cdot 2^x$ when $x\geqslant k$. Then
\[
\left(\frac{3}{2}\right)^{x} \leq C
\]
when $x\geqslant k$\\
\\
But any exponential function $a^x$ grows monotonically whenever $a\geqslant 1$; a contradiction
\subsubsection{Sum and Product Rules}
\textbf{The sum rule}
\[
\text { If } f_{1}(x) \text { is } O\left(g_{1}(x)\right) \text { and } f_{2}(x) \text { is } O\left(g_{2}(x)\right), \text { then } f_{1}(x)+f_{2}(x) \text { is } O\left(\max \left\{\left|g_{1}(x)\right|,\left|g_{2}(x)\right|\right\}\right)
\]
\textbf{The product rule}
\[
\text { If } f_{1}(x) \text { is } O\left(g_{1}(x)\right) \text { and } f_{2}(x) \text { is } O\left(g_{2}(x)\right), \text { then } f_{1}(x) \cdot f_{2}(x) \text { is } O\left(g_{1}(x) \cdot g_{2}(x)\right)
\]
\subsection{Big-Omega}
Let $f(x)$ and $g(x)$ be functions from the set of real numbers to the set of real numbers. We say that $f(x)$ is $\Omega(g(x))$ if there are positive constants C and k such that
\[
|f(x)| \geq C \cdot|g(x)|
\]
whenever $x>k$. Note that this implies that $f(x)$ is $\Omega(g(x))$ iff $g(x)$ is $\mathcal{O}(f(x))$
\subsection{Theta}
Let $f(x)$ and $g(x)$ be functions from the set of real numbers to the set of real numbers. We say that $f(x)$ is $\Theta(g(x))$ if $f(x)$ is $\mathcal{O}(g(x))$ and $f(x)$ is $\Omega(g(x))$\\
\\
This is the equivalent to saying that $f(x)$ is $\Theta(g(x))$ if $f(x)$ is $\mathcal{O}(g(x))$ and $g(x)$ is $\mathcal{O}(f(x))$
\subsection{Little-o}
Let $f(x)$ and $g(x)$ be functions from the set of real numbers to the set of real numbers. We say that $f(x)$ is $o(g(x))$ when
\[
\lim _{x \rightarrow \infty} \frac{f(x)}{g(x)}=0
\]
The definition without the limit is
\[
o(g)=\{f : \mathbb{N} \rightarrow \mathbb{N} | \forall C>0 \exists k>0 : C \cdot f(n)<g(n) \forall n \geq k\}
\]
This shows that $f(x)$ is $o(g(x))$ implies $f(x)$ is $\mathcal{O}(g(x))$
\subsubsection{Sublinear functions}
A function is called sublinear if it grows slower than a linear function. With little-o notation, we can express this as\\
\\
A function $f(x)$ is called sublinear if $f(x)$ is $o(x)$, so if
\[
\lim _{x \rightarrow \infty} \frac{f(x)}{x}=0
\]
\subsection{Little-omega}
$\omega$ is to o what $\Omega$ is to $\mathcal{O}$
\[
f=\omega(g) \quad \Leftrightarrow \quad g=o(f)
\]
or
\[
\omega(g)=\{f : \mathbb{N} \rightarrow \mathbb{N} | \forall C>0 \exists k>0 : f(n)>C \cdot g(n) \forall n \geq k\}
\]
\subsection{General Rules}
$$ \text { If } f _ { 1 } ( x ) \text { is } o ( g ( x ) ) \text { and } f _ { 2 } ( x ) \text { is o } ( g ( x ) ) , \text { then } f _ { 1 } ( x ) + f _ { 2 } ( x ) \text { is }  { o ( g ( x ) ) . }$$
$$ { \text { If } f _ { 1 } ( x ) \text { is } O ( g ( x ) ) \text { and } f _ { 2 } ( x ) \text { is o } ( g ( x ) ) , \text { then } f _ { 1 } ( x ) + f _ { 2 } ( x ) \text { is } } { O ( g ( x ) ) . } $$
\begin{center}
	If $f _ { 1 } ( x ) \) is \( \Theta ( g ( x ) ) \) and \( f _ { 2 } ( x ) \) is o \( ( g ( x ) ) , \) then \( f _ { 1 } ( x ) + f _ { 2 } ( x ) \) is
	\( \Theta ( g ( x ) ) .$
\end{center}

Equivalent to $\leqslant$
$$\mathcal { O } ( g ) = \{ f : \mathbb{ N } \rightarrow \mathbb{ N } | \textcolor{blue}{\exists} C , k > 0 : \mathbf{f ( n ) \leq C \cdot g ( n )} \forall n \geq k \}$$
Equivalent to $\geqslant$
$$\Omega ( g ) = \{ f :  \mathbb{ N } \rightarrow \mathbb{ N } | \textcolor{blue}{\exists} C , k > 0 :\mathbf{ f ( n ) \geq C \cdot g ( n )} \forall n \geq k \}$$
Equivalent to =
$$\Theta ( g ) = \left\{ f : \mathbb{ N } \rightarrow \mathbb{ N } | \textcolor{blue}{\exists} C _ { \mathbf { 1 } } , C _ { 2 } , k > 0 : \mathbf{\quad C _ { \mathbf { 1 } } \cdot g ( n ) \leq f ( n ) \leq C _ { 2 } \cdot g ( n )} \forall n \geq k \right\}$$
Equivalent to $<$
$$o ( g ) = \{ f : \mathbb{ N } \rightarrow \mathbb{ N } | \textcolor{red}{\forall} C > 0 \exists k > 0 : \mathbf{C \cdot f ( n ) < g ( n )} \forall n \geq k \}$$
Equivalent to $>$
$$\omega ( g ) = \{ f : \mathbb{ N } \rightarrow \mathbb{ N } | \textcolor{red}{\forall} C > 0 \exists k > 0 : \mathbf{f ( n ) > C \cdot g ( n )} \forall n \geq k \}$$
\section{Sorting}
\subsection{Insertion Sort}
\begin{lstlisting}[caption={InsertionSort $(a_1\ldots,a_n\in \mathbb{R}, n\geqslant 2)$}]
for j=2 to n do
	$x=a_j$
	i=j-1
	while i>0 and $a_i>x$ do
		$a_{i+1}=a_i$
		i=i-1
	end while
	$a_{i+1}=x$
end for
\end{lstlisting}
We know:
\begin{itemize}
	\item When j has a certain value, it inserts the j-th element into already sorted sequence $a_1,\ldots, a_{j-1}$
	\item Can be proved  by using invariant "after jth iteration first j+1 elements are in order"
	\item Running time between $n-1$ and $\dfrac{n(n-1)}{2}$ - worst case $\mathcal{O}(n^2)$
\end{itemize}
\subsection{Selection sort}
\begin{lstlisting}[caption={SelectionSort $(a_1,...,a_n\in \mathbb{R}, n\geqslant 2)$}]
for i=1 to n-1 do
	elem = $a_i$
	pos = i
	for j=i+1 to n do
		if $a_j$<elem then
			elem=$a_j$
			pos=j
		end if
	end for
	swap $a_i$ and $a_{pos}$
end for			
\end{lstlisting}
How does it work?\\
Invariant: after ith iteration positions 1,...,i contain the overall i many smallest elements in order\\
Not necessarily the first i elements (as it was in InsertionSort)\\
In the ith iteration of outer loop, we search the ith smallest element in remainder (positions $i_1,...,n$) of input and swap it into position i
\begin{itemize}
	\item elem keeps track of current idea of \textbf{value} ith smalllest element
	\item pos keeps track of the current idea of \textbf{position} of ith smallest element
\end{itemize}
Time complexity:
\[
\begin{aligned} \sum_{i=1}^{n-1} \sum_{j=i+1}^{n} 1 &=\sum_{i=1}^{n-1}(n-i) \\ &=\left(\sum_{i=1}^{n-1} n\right)-\left(\sum_{i=1}^{n-1} i\right) \\ &= (n-1)\cdot n-\dfrac{n(n-1)}{2}\\ &=\frac{n(n-1)}{2} \\ &=O\left(n^{2}\right) \end{aligned}
\]
\subsection{Bubble sort}
\begin{lstlisting}[caption={BubbleSort-1 $(a_1,...,a_n\in \mathbb{R}, n\geqslant 2)$}]
for i=1 to n-2 do
	for j=1 to n-1 do
		if $a_j>a_{j+1}$ then
			swap $a_j$ and $a_{j+1}$
		end if
	end for
end for		
\end{lstlisting}
This can be improved by keeping track of whether or not an element was swapped
\begin{lstlisting}[caption={BubbleSort-1 $(a_1,...,a_n\in \mathbb{R}, n\geqslant 2)$}]
for i=1 to n-1 do
	swaps=0
	for j=1 to n-1 do
		if $a_j>a_{j+1}$ then
			swap $a_j$ and $a_{j+1}$
			swaps=swaps+1
		end if
	end for
	if swaps ==0 then
		break
	end if
end for		
\end{lstlisting}
\textbf{Proof of correctness}\\
A sequence $(a_1,...,a_n)$ is sorted if for every adjacent pair $a_i,a_{i+1}$ we have $a_i\leqslant a_{i+1}$\\
Bubble sort achieves just that\\
\\
\textbf{Time complexity}
\[
\begin{aligned} \sum_{i=1}^{n-1} \sum_{j=1}^{n-1} 1 &=\sum_{i=1}^{n-1}(n-1) \\ &=(n-1)^{2}=O\left(n^{2}\right) \end{aligned}
\]
\subsection{Mergesort}
\begin{lstlisting}[caption={list MergeSort (list m)}]
if length(m) $\leqslant$ 1 then
	return m
end if
int middle = length(m) / 2
list left, right, leftsorted, rightsorted
left = m[1..middle]
right = m[middle+1..length(m)]
leftsorted = MergeSort(left)
rightsorted = MergeSort(right)
return Merge(leftsorted, rightsorted)
\end{lstlisting}
There is then the merge function:
\begin{lstlisting}[caption={list MergeSort (list left, list right)}]
list result
while length(left)>0 or length(right)>0 do
	if length(left)>0 and length(right)>0 then
		if first(left) $\leqslant$ first(right) then
			append first(left) to result
			left = rest(left)
		else
		#Keeping extra copies of the data in the result array
			append first(right) to result
			right = rest(right)
		end if
	else if length(left)>0 then
		append left to result
		left = empty list
	else 	# Length(right) $>$ 0
		append right to result
		right = empty list
	end if
end while
return result
\end{lstlisting}
\subsection{Quicksort}
\begin{lstlisting}[caption={QuickSort(int {$A[1..n]$}, int left, int right)}]
if (left<right) then
	#rearrange/partition in place
	#return value "pivot" is index of pivot element in A[] after partitioning
	pivot=Partition(A,left,right)
	#Now:
	#Everything in A[left...pivot-1] is smaller than pivot
	#Everything in A[pivot+1..right] is bigger than pivot
	QuickSort (A,left,pivot-1)
	QuickSort(A,pivot_1,right)
end if
\end{lstlisting}
An example of the partition function is
\begin{lstlisting}[caption={int Partition({$A[1...n]$}, int left, int right)}]
int x =A[right]
int i=left-1
for j=left to right-1 do
	if A[j]<x then
		i=i+1
		swap A[i] and A[j]
	end if
end for
swap A[i+1] and A[right]
return i+1
\end{lstlisting}
\section{Recurrences}
\subsection{Induction}
Basically:
\begin{itemize}
	\item "guess" correct solution (good job all sorting algorithms are n log n)
	\item verify base case(s) and step
\end{itemize}
Consider the recurrence for merge sort:
\[
T(n) \leq \left\{\begin{array}{ll}{d} & {\text { if } n \leq c, \text { for constants } c, d>0} \\ {2 \cdot T(n / 2)+a \cdot n} & {\text { otherwise }}\end{array}\right.
\]
To get the base case, do as follows:
\[
d \leq \alpha n \log _{2} n \leq \alpha c \log _{2} c \quad \Leftrightarrow \quad \alpha \geq \frac{d}{c \log _{2} c}
\]
As for the inductive step, plug the guess in
\[
\begin{array}{l}{T(n) \leq 2 T(n / 2)+a n} \\ {T(n) \leq 2 \alpha \frac{n}{2} \log _{2} \frac{n}{2}+a n} \\ {T(n) \leq 2 \alpha \frac{n}{2}\left(\log _{2}(n)-\log _{2}(2)\right)+a n} \\ {T(n) \leq 2 \alpha \frac{n}{2}\left(\log _{2}(n)-1\right)+a n} \\ {T(n) \leq \alpha n\left(\log _{2}(n)-1\right)+a n} \\ {T(n) \leq \alpha n \log _{2}(n)-\alpha n+a n} \\ {T(n) \leq \alpha n \log _{2} n \quad \text { if } \alpha n \geq a n \Leftrightarrow \alpha \geq a}\end{array}
\]
The requirement that $\alpha\geqslant a$ suggests there isn't much room for (asymptotic) improvement, as both are constants
\subsection{Iterative substitution}
Expand the recurrence
\[
\begin{aligned} T(n) & \leq 2 T(n / 2)+a n \\ & \leq 2(2 T(n / 4)+a n / 2)+a n=4 T(n / 4)+a n+a n \\ &=4 T(n / 4)+2 a n \leq 4(2 T(n / 8)+a n / 4)+2 a n \\ &=8 T(n / 8)+a n+2 a n=8 T(n / 8)+3 a n \\ & \leq 8(2 T(n / 16)+a n / 8)+3 a n=16 T(n / 16)+a n+3 a n \\ &=16 T(n / 16)+4 a n \end{aligned}
\]
\subsection{Master Theorem}
This can be used to solve recurrences of the form:
\[
T(n)=a T(n / b)+f(n)
\]
for $a\geqslant 1$ and $b\geq 1$\\
\\
There are 3 cases, but rob says he'll give you them
\end{document}