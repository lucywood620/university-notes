\documentclass{article}[18pt]
\ProvidesPackage{format}
%Page setup
\usepackage[utf8]{inputenc}
\usepackage[margin=0.7in]{geometry}
\usepackage{parselines} 
\usepackage[english]{babel}
\usepackage{fancyhdr}
\usepackage{titlesec}
\hyphenpenalty=10000

\pagestyle{fancy}
\fancyhf{}
\rhead{Sam Robbins}
\rfoot{Page \thepage}

%Characters
\usepackage{amsmath}
\usepackage{amssymb}
\usepackage{gensymb}
\newcommand{\R}{\mathbb{R}}

%Diagrams
\usepackage{pgfplots}
\usepackage{graphicx}
\usepackage{tabularx}
\usepackage{relsize}
\pgfplotsset{width=10cm,compat=1.9}
\usepackage{float}

%Length Setting
\titlespacing\section{0pt}{14pt plus 4pt minus 2pt}{0pt plus 2pt minus 2pt}
\newlength\tindent
\setlength{\tindent}{\parindent}
\setlength{\parindent}{0pt}
\renewcommand{\indent}{\hspace*{\tindent}}

%Programming Font
\usepackage{courier}
\usepackage{listings}
\usepackage{pxfonts}

%Lists
\usepackage{enumerate}
\usepackage{enumitem}

% Networks Macro
\usepackage{tikz}


% Commands for files converted using pandoc
\providecommand{\tightlist}{%
	\setlength{\itemsep}{0pt}\setlength{\parskip}{0pt}}
\usepackage{hyperref}

% Get nice commands for floor and ceil
\usepackage{mathtools}
\DeclarePairedDelimiter{\ceil}{\lceil}{\rceil}
\DeclarePairedDelimiter{\floor}{\lfloor}{\rfloor}

% Allow itemize to go up to 20 levels deep (just change the number if you need more you madman)
\usepackage{enumitem}
\setlistdepth{20}
\renewlist{itemize}{itemize}{20}

% initially, use dots for all levels
\setlist[itemize]{label=$\cdot$}

% customize the first 3 levels
\setlist[itemize,1]{label=\textbullet}
\setlist[itemize,2]{label=--}
\setlist[itemize,3]{label=*}

% Definition and Important Stuff
% Important stuff
\usepackage[framemethod=TikZ]{mdframed}

\newcounter{theo}[section]\setcounter{theo}{0}
\renewcommand{\thetheo}{\arabic{section}.\arabic{theo}}
\newenvironment{important}[1][]{%
	\refstepcounter{theo}%
	\ifstrempty{#1}%
	{\mdfsetup{%
			frametitle={%
				\tikz[baseline=(current bounding box.east),outer sep=0pt]
				\node[anchor=east,rectangle,fill=red!50]
				{\strut Important};}}
	}%
	{\mdfsetup{%
			frametitle={%
				\tikz[baseline=(current bounding box.east),outer sep=0pt]
				\node[anchor=east,rectangle,fill=red!50]
				{\strut Important:~#1};}}%
	}%
	\mdfsetup{innertopmargin=10pt,linecolor=red!50,%
		linewidth=2pt,topline=true,%
		frametitleaboveskip=\dimexpr-\ht\strutbox\relax
	}
	\begin{mdframed}[]\relax%
		\centering
		}{\end{mdframed}}



\newcounter{lem}[section]\setcounter{lem}{0}
\renewcommand{\thelem}{\arabic{section}.\arabic{lem}}
\newenvironment{defin}[1][]{%
	\refstepcounter{lem}%
	\ifstrempty{#1}%
	{\mdfsetup{%
			frametitle={%
				\tikz[baseline=(current bounding box.east),outer sep=0pt]
				\node[anchor=east,rectangle,fill=blue!20]
				{\strut Definition};}}
	}%
	{\mdfsetup{%
			frametitle={%
				\tikz[baseline=(current bounding box.east),outer sep=0pt]
				\node[anchor=east,rectangle,fill=blue!20]
				{\strut Definition:~#1};}}%
	}%
	\mdfsetup{innertopmargin=10pt,linecolor=blue!20,%
		linewidth=2pt,topline=true,%
		frametitleaboveskip=\dimexpr-\ht\strutbox\relax
	}
	\begin{mdframed}[]\relax%
		\centering
		}{\end{mdframed}}
\lhead{ADS}
\lstset{language=C,
	basicstyle=\ttfamily,
	keywordstyle=\bfseries,
	showstringspaces=false,
	morekeywords={if, else, then, print, end, for, do, while},
	tabsize=4,
	mathescape=true,
	numbers=left,
	stepnumber=1,
	escapechar=£,
}
\usepackage{caption}
\DeclareCaptionFont{white}{\color{white}}
\DeclareCaptionFormat{listing}{\colorbox{gray}{\parbox{\textwidth}{#1#2#3}}}
\captionsetup[lstlisting]{format=listing,labelfont=white,textfont=white}

\begin{document}
\begin{center}
\underline{\huge Part 4}
\end{center}
\section{Breadth-First Search}
\subsection{Graphs}
\begin{itemize}
	\item A graph G=(V,E) is a pair of sets: vertices V and edges E
	\item To give an adjacency list representation of a graph, for each vertex v list all the vertices adjacent to v
	\item To give an adjacency matrix representation of a graph create a square matrix A and label the rows and columns of the vertices: the entry in row i column j is 1 if vertex j is adjacent to vertex i and 0 if it is not
\end{itemize}
\subsection{Breadth-First Search}
\begin{itemize}
	\item BFS maintains a queue that contains vertices that have been discovered but are waiting to be processed
	\item BFS colours the vertices:
	\begin{itemize}
		\item \textbf{White} indicates that a vertex is undiscovered
		\item \textbf{Grey} indicates that a vertex is discovered but unprocessed
		\item \textbf{Black} indicates that a vertex has been processed
	\end{itemize}
	\item The algorithm maintains an array d (distance)
	\begin{itemize}
		\item d[s]=0 where s is the source vertex
		\item If we discover a new vertex v while processing u, we set d[v]=d[u]+1
	\end{itemize}
\end{itemize}
\begin{lstlisting}[caption={BFS(G,s)}]
for each vertex u$\in$ V[G]-{s}
	do colour[u]$\leftarrow$ WHITE
		d[u]$\leftarrow \infty$
			$\pi[u]\leftarrow$ NIL
colour[s]=GREY
d[s]$\leftarrow$ 0
$\pi[s]\leftarrow$ NIL
Q$\leftarrow \varnothing$
ENQUEUE(Q,s)
while $Q\neq \varnothing$
	do u$\leftarrow$DEQUEUE(Q)
		for each v$\in$ Adj[u]
			do if colour[v]=WHITE
				then colour[v]=GREY
					d[v]$\leftarrow$d[u]+1
					$\pi[v]\leftarrow$ u
					ENQUEUE(Q,v)
		colour[u]$\leftarrow$ BLACK			
\end{lstlisting}
\subsection{Analysis of running time}
\begin{itemize}
	\item We want an upper bound on the worst-case running time
	\item Assume that it takes constant time for each operation such as to test and update colours, to make changes to distance and to enqueue and dequeue
	\item Initialisation takes time $\mathcal{O}(V)$
	\item Each vertex enters (and leaves) the queue exactly once. So queueing operations take $\mathcal{O}(V)$
	\item In the loop the adjacency lists of each vertex are scanned. Each list is read once, and the combined lengths of the lists is $\mathcal{O}(E)$
	\item This the total running time is $\mathcal{O}(V+E)$
\end{itemize}
\section{Depth-First Search}
\begin{itemize}
	\item Initialize: source vertex grey, others white; source discovered at time 1
	\item Repeat:
	\begin{itemize}
		\item Increment the time
		\item If there is a white neighbour of the current vertex, then it is coloured grey and its discovery time noted and it becomes current
		\item Else colour the current vertex black, not its finish time and return to its predecessor or jump to an undiscovered vertex
	\end{itemize}
\end{itemize}
\begin{lstlisting}[caption=DFS(G)]
for each vertex u$\in$ V[G]
	do colour[u]$\leftarrow$ WHITE
		$\pi[u] \leftarrow$ NIL
time $\leftarrow$ 0
for each vertex u $\in$ V[G]
	do if colour[u] = WHITE
		then DFS-VISIT(u)
\end{lstlisting}
\begin{lstlisting}[caption=DFS-VISIT(u)]
colour[u]$\leftarrow$ GREY £\hfill£ [vertex u has just been discovered]
time$\leftarrow$time+1
d[u]$\leftarrow$time
for each vertex $v\in$ Adj[u] £\hfill£ [explore edge (u,v)]
	do if colour[v]=WHITE
		then $\pi[v]\leftarrow u$
			DFS-VISIT(v)
colour[u]$\leftarrow$BLACK £\hfill£ [u has been processed]
f[u]$\leftarrow$time$\leftarrow$time+1
\end{lstlisting}
\begin{itemize}
	\item Initialisation takes time $\mathcal{O}(V)$
	\item Time $\mathcal{O}(V)$ spent on incrementing time, colouring vertices and updating d and f
	\item Each vertex in each adjacency list is considered at most once. This takes time $\mathcal{O}(E)$
	\item Total time is $\mathcal{O}(V+E)$
\end{itemize}
\subsection{Classification of the edges}
\begin{itemize}
	\item \textbf{Tree} edges are those edges in the DFS-forest
	\item \textbf{Back} edges are edges that join a vertex to an ancestor
	\item \textbf{Forward} edges are edges not in the tree that join a vertex to its descendant
	\item \textbf{Cross} edges: all other edges
\end{itemize}
The classification is ambiguous for undirected graphs (back edges and forward edges are the same thing)\\
\\
Let us redefine the definition: suppose that e is an edge that joins a vertex u to its descendant v
\begin{itemize}
	\item e is a forward edge if DFS first considers e from u
	\item e is a back edge if DFS first considers e from v
\end{itemize}
\begin{center}
	\textit{In an undirected graph, every edge is a tree edge or a back edge}
\end{center}
\subsection{Using depth first search}
\begin{itemize}
	\item Every edge in an undirected graph is either a tree edge or a back edge
	\item A graph is connected if each pair of vertices is joined by a path
	\item A cycle is a sequence of edges that start and end at the same vertex
	\item An articulation point is a vertex whose removal disconnects the graph
\end{itemize}
\section{Minimum Spanning Trees}
The minimum spanning tree problem
\begin{center}
	\textit{Find a tree that spans the vertices and has minimum cost}
\end{center}
Basic properties of MSTs
\begin{itemize}
	\item Have $|V|-1$ edges
	\item Have no cycles
	\item Might not be unique
\end{itemize}
\subsection{Kruskal's algorithm}
\begin{enumerate}
	\item Sort the edges by weight
	\item Let $A=\varnothing$
	\item Consider edges in increasing order of weight. For each edge e, add e to A unless this would create a cycle
\end{enumerate}
Running time is $\mathcal{O}(E\log V)$
\subsection{Prim's algorithm}
\begin{enumerate}
	\item Let $U=\{u\}$ where u is some vertex chosen arbitrarily
	\item Let $A=\varnothing$
	\item Until U contains all vertices: find the least-weight edge e that joins a vertex v in U to a vertex w not in U and add e to A and w to U
\end{enumerate}
Running time is $\mathcal{O}(V\log V+E)$

\end{document}