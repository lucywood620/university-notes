\documentclass{article}[18pt]
\ProvidesPackage{format}
%Page setup
\usepackage[utf8]{inputenc}
\usepackage[margin=0.7in]{geometry}
\usepackage{parselines} 
\usepackage[english]{babel}
\usepackage{fancyhdr}
\usepackage{titlesec}
\hyphenpenalty=10000

\pagestyle{fancy}
\fancyhf{}
\rhead{Sam Robbins}
\rfoot{Page \thepage}

%Characters
\usepackage{amsmath}
\usepackage{amssymb}
\usepackage{gensymb}
\newcommand{\R}{\mathbb{R}}

%Diagrams
\usepackage{pgfplots}
\usepackage{graphicx}
\usepackage{tabularx}
\usepackage{relsize}
\pgfplotsset{width=10cm,compat=1.9}
\usepackage{float}

%Length Setting
\titlespacing\section{0pt}{14pt plus 4pt minus 2pt}{0pt plus 2pt minus 2pt}
\newlength\tindent
\setlength{\tindent}{\parindent}
\setlength{\parindent}{0pt}
\renewcommand{\indent}{\hspace*{\tindent}}

%Programming Font
\usepackage{courier}
\usepackage{listings}
\usepackage{pxfonts}

%Lists
\usepackage{enumerate}
\usepackage{enumitem}

% Networks Macro
\usepackage{tikz}


% Commands for files converted using pandoc
\providecommand{\tightlist}{%
	\setlength{\itemsep}{0pt}\setlength{\parskip}{0pt}}
\usepackage{hyperref}

% Get nice commands for floor and ceil
\usepackage{mathtools}
\DeclarePairedDelimiter{\ceil}{\lceil}{\rceil}
\DeclarePairedDelimiter{\floor}{\lfloor}{\rfloor}

% Allow itemize to go up to 20 levels deep (just change the number if you need more you madman)
\usepackage{enumitem}
\setlistdepth{20}
\renewlist{itemize}{itemize}{20}

% initially, use dots for all levels
\setlist[itemize]{label=$\cdot$}

% customize the first 3 levels
\setlist[itemize,1]{label=\textbullet}
\setlist[itemize,2]{label=--}
\setlist[itemize,3]{label=*}

% Definition and Important Stuff
% Important stuff
\usepackage[framemethod=TikZ]{mdframed}

\newcounter{theo}[section]\setcounter{theo}{0}
\renewcommand{\thetheo}{\arabic{section}.\arabic{theo}}
\newenvironment{important}[1][]{%
	\refstepcounter{theo}%
	\ifstrempty{#1}%
	{\mdfsetup{%
			frametitle={%
				\tikz[baseline=(current bounding box.east),outer sep=0pt]
				\node[anchor=east,rectangle,fill=red!50]
				{\strut Important};}}
	}%
	{\mdfsetup{%
			frametitle={%
				\tikz[baseline=(current bounding box.east),outer sep=0pt]
				\node[anchor=east,rectangle,fill=red!50]
				{\strut Important:~#1};}}%
	}%
	\mdfsetup{innertopmargin=10pt,linecolor=red!50,%
		linewidth=2pt,topline=true,%
		frametitleaboveskip=\dimexpr-\ht\strutbox\relax
	}
	\begin{mdframed}[]\relax%
		\centering
		}{\end{mdframed}}



\newcounter{lem}[section]\setcounter{lem}{0}
\renewcommand{\thelem}{\arabic{section}.\arabic{lem}}
\newenvironment{defin}[1][]{%
	\refstepcounter{lem}%
	\ifstrempty{#1}%
	{\mdfsetup{%
			frametitle={%
				\tikz[baseline=(current bounding box.east),outer sep=0pt]
				\node[anchor=east,rectangle,fill=blue!20]
				{\strut Definition};}}
	}%
	{\mdfsetup{%
			frametitle={%
				\tikz[baseline=(current bounding box.east),outer sep=0pt]
				\node[anchor=east,rectangle,fill=blue!20]
				{\strut Definition:~#1};}}%
	}%
	\mdfsetup{innertopmargin=10pt,linecolor=blue!20,%
		linewidth=2pt,topline=true,%
		frametitleaboveskip=\dimexpr-\ht\strutbox\relax
	}
	\begin{mdframed}[]\relax%
		\centering
		}{\end{mdframed}}
\lhead{MODULE}


\usepackage{hyperref}


\providecommand{\tightlist}{%
	\setlength{\itemsep}{0pt}\setlength{\parskip}{0pt}}
\begin{document}
\begin{center}
\underline{\huge TITLE}
\end{center}

\hypertarget{comp1101-programming-summative-assessment-1-draft}{%
	\section{COMP1101 Programming Summative Assessment 1
		(DRAFT)}\label{comp1101-programming-summative-assessment-1-draft}}

\begin{center}\rule{0.5\linewidth}{\linethickness}\end{center}

\hypertarget{term-1-programming-exercise-outline}{%
	\subsection{Term 1 Programming Exercise
		Outline}\label{term-1-programming-exercise-outline}}

\begin{itemize}
	\tightlist
	\item
	Submission by 14:00 Thursday 17/1/2019
	\item
	Return by 14/2/2019
	\item
	Contributes 35\% of module marks
	\item
	Includes peer review feedback which you will be allocated
	\item
	Peer reviews need to be submitted by 14:00 31/1/2019
	\item
	Quality of your peer reviews contribute 5\% to your module mark
\end{itemize}

\begin{center}\rule{0.5\linewidth}{\linethickness}\end{center}

\hypertarget{subject-specific-knowledge}{%
	\subsection{Subject-specific
		Knowledge}\label{subject-specific-knowledge}}

\begin{itemize}
	\tightlist
	\item
	Interaction between JavaScript progams and the Document Object Model
	(DOM)
	\item
	Using control statements to loop and make decisions.
	\item
	An understanding of the nature of imperative programming in the
	object-oriented style.
	\item
	A knowledge and understanding of good programming practice (for
	example, reuse, documentation and style)
\end{itemize}

\begin{center}\rule{0.5\linewidth}{\linethickness}\end{center}

\hypertarget{key-skills}{%
	\subsection{Key Skills}\label{key-skills}}

\begin{itemize}
	\tightlist
	\item
	an ability to recognise and apply the principles of abstraction and
	modelling
\end{itemize}

\begin{center}\rule{0.5\linewidth}{\linethickness}\end{center}

\hypertarget{tasks}{%
	\subsection{Tasks}\label{tasks}}

\begin{itemize}
	\tightlist
	\item
	Fork https://github.com/stevenaeola/Durham-p5-lib
	\item
	Choose a sketch from
	\href{https://www.openprocessing.org/}{openprocessing.org}
	\item
	Put the original sketch code into a subdirectory of the repository
	\item
	Adapt it into a reusable component using JavaScript classes
	
	\begin{itemize}
		\tightlist
		\item
		Appropriate constructor
		\item
		Get and set methods for properties
		\item
		\texttt{draw} method with optional p5.Renderer as parameter
	\end{itemize}
\end{itemize}

\begin{itemize}
	\tightlist
	\item
	Build an example page with properties controlled by form controls
	\item
	Write documentation of your code using
	\href{https://guides.github.com/features/mastering-markdown/}{Markdown}
\end{itemize}

\begin{center}\rule{0.5\linewidth}{\linethickness}\end{center}

\hypertarget{submission}{%
	\subsection{Submission}\label{submission}}

\begin{itemize}
	\tightlist
	\item
	Submit via duo a link to a github (or other git) repository containing
	your code and documentation
	\item
	Make repository public on submission
	\item
	Make a pull request to https://github.com/stevenaeola/Durham-p5-lib
	with your new component
\end{itemize}

\begin{center}\rule{0.5\linewidth}{\linethickness}\end{center}

\hypertarget{marking-criteria}{%
	\subsection{Marking Criteria}\label{marking-criteria}}

Weighted equally

\begin{itemize}
	\tightlist
	\item
	Usability of code
	\item
	Development of original
	\item
	Quality of example
	\item
	Quality of documentation
	\item
	Code quality and management
\end{itemize}

\begin{center}\rule{0.5\linewidth}{\linethickness}\end{center}

\hypertarget{usability-of-code}{%
	\subsection{Usability of code}\label{usability-of-code}}

\begin{itemize}
	\tightlist
	\item
	Appropriate parameterisation including defaults
	\item
	Encapsulation (private fields where appropriate)
	\item
	Useful methods including \texttt{draw}
\end{itemize}

\begin{center}\rule{0.5\linewidth}{\linethickness}\end{center}

\hypertarget{development-of-original}{%
	\subsection{Development of original}\label{development-of-original}}

\begin{itemize}
	\tightlist
	\item
	Original code included in initial commit
	\item
	Work done in refactoring code to class
	\item
	Work done in useful parameterisation
	\item
	Work done in extending scope
\end{itemize}

\begin{center}\rule{0.5\linewidth}{\linethickness}\end{center}

\hypertarget{quality-of-example}{%
	\subsection{Quality of example}\label{quality-of-example}}

\begin{itemize}
	\tightlist
	\item Need to make an example of your package being used
	\item
	HTML page is valid
	\item
	Appropriate on-page instructions
	\item
	Appropriate on-page controls (form)
\end{itemize}

\begin{center}\rule{0.5\linewidth}{\linethickness}\end{center}

\hypertarget{quality-of-documentation}{%
	\subsection{Quality of documentation}\label{quality-of-documentation}}

\begin{itemize}
	\tightlist
	\item Good "template" is the documentation for the p5 library itself
	\item
	All methods and parameters explained (including constructor)
	\item
	Explanation of example
	\item
	Source of initial code acknowledged (including licence)
\end{itemize}

\begin{center}\rule{0.5\linewidth}{\linethickness}\end{center}

\hypertarget{code-quality-eslint}{%
	\subsection{Code quality: ESLint}\label{code-quality-eslint}}

Apply rules from
\href{https://eslint.org/docs/rules/}{eslint.org/docs/rules/}:

\begin{itemize}
	\tightlist
	\item
	Possible Errors
	\item
	Best Practices
	\item
	Variables
	\item
	Stylistic Issues
	\item
	ECMAScript 6
\end{itemize}

\begin{center}\rule{0.5\linewidth}{\linethickness}\end{center}

\hypertarget{code-management-git}{%
	\subsection{Code management: git}\label{code-management-git}}

\begin{itemize}
	\tightlist
	\item
	Appropriate commits including comments
	\item
	Consistent development trajectory
\end{itemize}




\end{document}