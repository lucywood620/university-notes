\PassOptionsToPackage{unicode=true}{hyperref} % options for packages loaded elsewhere
\PassOptionsToPackage{hyphens}{url}
%
\documentclass[]{article}
\usepackage{lmodern}
\usepackage{amssymb,amsmath}
\usepackage{ifxetex,ifluatex}
\usepackage{fixltx2e} % provides \textsubscript
\ifnum 0\ifxetex 1\fi\ifluatex 1\fi=0 % if pdftex
  \usepackage[T1]{fontenc}
  \usepackage[utf8]{inputenc}
  \usepackage{textcomp} % provides euro and other symbols
\else % if luatex or xelatex
  \usepackage{unicode-math}
  \defaultfontfeatures{Ligatures=TeX,Scale=MatchLowercase}
\fi
% use upquote if available, for straight quotes in verbatim environments
\IfFileExists{upquote.sty}{\usepackage{upquote}}{}
% use microtype if available
\IfFileExists{microtype.sty}{%
\usepackage[]{microtype}
\UseMicrotypeSet[protrusion]{basicmath} % disable protrusion for tt fonts
}{}
\IfFileExists{parskip.sty}{%
\usepackage{parskip}
}{% else
\setlength{\parindent}{0pt}
\setlength{\parskip}{6pt plus 2pt minus 1pt}
}
\usepackage{hyperref}
\hypersetup{
            pdfborder={0 0 0},
            breaklinks=true}
\urlstyle{same}  % don't use monospace font for urls
\usepackage{color}
\usepackage{fancyvrb}
\newcommand{\VerbBar}{|}
\newcommand{\VERB}{\Verb[commandchars=\\\{\}]}
\DefineVerbatimEnvironment{Highlighting}{Verbatim}{commandchars=\\\{\}}
% Add ',fontsize=\small' for more characters per line
\newenvironment{Shaded}{}{}
\newcommand{\AlertTok}[1]{\textcolor[rgb]{1.00,0.00,0.00}{\textbf{#1}}}
\newcommand{\AnnotationTok}[1]{\textcolor[rgb]{0.38,0.63,0.69}{\textbf{\textit{#1}}}}
\newcommand{\AttributeTok}[1]{\textcolor[rgb]{0.49,0.56,0.16}{#1}}
\newcommand{\BaseNTok}[1]{\textcolor[rgb]{0.25,0.63,0.44}{#1}}
\newcommand{\BuiltInTok}[1]{#1}
\newcommand{\CharTok}[1]{\textcolor[rgb]{0.25,0.44,0.63}{#1}}
\newcommand{\CommentTok}[1]{\textcolor[rgb]{0.38,0.63,0.69}{\textit{#1}}}
\newcommand{\CommentVarTok}[1]{\textcolor[rgb]{0.38,0.63,0.69}{\textbf{\textit{#1}}}}
\newcommand{\ConstantTok}[1]{\textcolor[rgb]{0.53,0.00,0.00}{#1}}
\newcommand{\ControlFlowTok}[1]{\textcolor[rgb]{0.00,0.44,0.13}{\textbf{#1}}}
\newcommand{\DataTypeTok}[1]{\textcolor[rgb]{0.56,0.13,0.00}{#1}}
\newcommand{\DecValTok}[1]{\textcolor[rgb]{0.25,0.63,0.44}{#1}}
\newcommand{\DocumentationTok}[1]{\textcolor[rgb]{0.73,0.13,0.13}{\textit{#1}}}
\newcommand{\ErrorTok}[1]{\textcolor[rgb]{1.00,0.00,0.00}{\textbf{#1}}}
\newcommand{\ExtensionTok}[1]{#1}
\newcommand{\FloatTok}[1]{\textcolor[rgb]{0.25,0.63,0.44}{#1}}
\newcommand{\FunctionTok}[1]{\textcolor[rgb]{0.02,0.16,0.49}{#1}}
\newcommand{\ImportTok}[1]{#1}
\newcommand{\InformationTok}[1]{\textcolor[rgb]{0.38,0.63,0.69}{\textbf{\textit{#1}}}}
\newcommand{\KeywordTok}[1]{\textcolor[rgb]{0.00,0.44,0.13}{\textbf{#1}}}
\newcommand{\NormalTok}[1]{#1}
\newcommand{\OperatorTok}[1]{\textcolor[rgb]{0.40,0.40,0.40}{#1}}
\newcommand{\OtherTok}[1]{\textcolor[rgb]{0.00,0.44,0.13}{#1}}
\newcommand{\PreprocessorTok}[1]{\textcolor[rgb]{0.74,0.48,0.00}{#1}}
\newcommand{\RegionMarkerTok}[1]{#1}
\newcommand{\SpecialCharTok}[1]{\textcolor[rgb]{0.25,0.44,0.63}{#1}}
\newcommand{\SpecialStringTok}[1]{\textcolor[rgb]{0.73,0.40,0.53}{#1}}
\newcommand{\StringTok}[1]{\textcolor[rgb]{0.25,0.44,0.63}{#1}}
\newcommand{\VariableTok}[1]{\textcolor[rgb]{0.10,0.09,0.49}{#1}}
\newcommand{\VerbatimStringTok}[1]{\textcolor[rgb]{0.25,0.44,0.63}{#1}}
\newcommand{\WarningTok}[1]{\textcolor[rgb]{0.38,0.63,0.69}{\textbf{\textit{#1}}}}
\setlength{\emergencystretch}{3em}  % prevent overfull lines
\providecommand{\tightlist}{%
  \setlength{\itemsep}{0pt}\setlength{\parskip}{0pt}}
\setcounter{secnumdepth}{0}
% Redefines (sub)paragraphs to behave more like sections
\ifx\paragraph\undefined\else
\let\oldparagraph\paragraph
\renewcommand{\paragraph}[1]{\oldparagraph{#1}\mbox{}}
\fi
\ifx\subparagraph\undefined\else
\let\oldsubparagraph\subparagraph
\renewcommand{\subparagraph}[1]{\oldsubparagraph{#1}\mbox{}}
\fi

% set default figure placement to htbp
\makeatletter
\def\fps@figure{htbp}
\makeatother


\date{}

\begin{document}

\hypertarget{assignment-questions}{%
\section{Assignment questions}\label{assignment-questions}}

\hypertarget{assignment-questions-1}{%
\subsection{Assignment questions}\label{assignment-questions-1}}

\begin{itemize}
\tightlist
\item
  What does a reusable component look like?
\item
  What is a \texttt{draw} method with optional p5.Renderer as parameter?
\item
  When are private fields appropriate?
\item
  How to use ESLint?
\item
  Can I convert a processing sketch?
\item
  Others?
\end{itemize}

\hypertarget{what-does-a-reusable-component-look-like}{%
\subsection{What does a reusable component look
like?}\label{what-does-a-reusable-component-look-like}}

\begin{itemize}
\tightlist
\item
  Variables in sketch become properties of the object
\item
  Setup code in sketch becomes the constructor
\item
  Draw method in sketch becomes draw method of the object
\item
  Other methods for changing parameters and handling events
\end{itemize}

\hypertarget{minimal-usage-in-sketch-index.js}{%
\subsection{Minimal usage in sketch
(index.js)}\label{minimal-usage-in-sketch-index.js}}

Include p5 library your sketch class definition

\begin{Shaded}
\begin{Highlighting}[]
\KeywordTok{var}\NormalTok{ c}\OperatorTok{;}

\KeywordTok{function} \AttributeTok{setup}\NormalTok{()}\OperatorTok{\{}
\NormalTok{   c }\OperatorTok{=} \KeywordTok{new} \AttributeTok{Component}\NormalTok{()}\OperatorTok{;}
\OperatorTok{\}}

\KeywordTok{function} \AttributeTok{draw}\NormalTok{()}\OperatorTok{\{}
   \VariableTok{c}\NormalTok{.}\AttributeTok{draw}\NormalTok{()}\OperatorTok{;}
\OperatorTok{\}}
\end{Highlighting}
\end{Shaded}

\hypertarget{what-is-a-draw-method-with-optional-p5.renderer-as-parameter}{%
\subsection{\texorpdfstring{What is a \texttt{draw} method with optional
p5.Renderer as
parameter?}{What is a draw method with optional p5.Renderer as parameter?}}\label{what-is-a-draw-method-with-optional-p5.renderer-as-parameter}}

\begin{itemize}
\tightlist
\item
  p5.Renderer objects are returned by \texttt{createCanvas} and
  \texttt{createGraphics}
\item
  Can be used e.g.~as texture for another object
\item
  All calls to graphics e.g.~\texttt{rect}, \texttt{ellipse},
  \texttt{fill} can be applied
\end{itemize}

\begin{Shaded}
\begin{Highlighting}[]
\KeywordTok{function} \AttributeTok{draw}\NormalTok{(g)}\OperatorTok{\{}
  \ControlFlowTok{if}\NormalTok{(g)}\OperatorTok{\{}
    \VariableTok{g}\NormalTok{.}\AttributeTok{ellipse}\NormalTok{(}\KeywordTok{this}\NormalTok{.}\AttributeTok{x}\OperatorTok{,} \KeywordTok{this}\NormalTok{.}\AttributeTok{y}\OperatorTok{,} \KeywordTok{this}\NormalTok{.}\AttributeTok{d}\OperatorTok{,} \KeywordTok{this}\NormalTok{.}\AttributeTok{d}\NormalTok{)}\OperatorTok{;}
  \OperatorTok{\}}
  \ControlFlowTok{else}\OperatorTok{\{}
    \AttributeTok{ellipse}\NormalTok{(}\KeywordTok{this}\NormalTok{.}\AttributeTok{x}\OperatorTok{,} \KeywordTok{this}\NormalTok{.}\AttributeTok{y}\OperatorTok{,} \KeywordTok{this}\NormalTok{.}\AttributeTok{d}\OperatorTok{,} \KeywordTok{this}\NormalTok{.}\AttributeTok{d}\NormalTok{)}\OperatorTok{;}
  \OperatorTok{\}}

\OperatorTok{\}}
\end{Highlighting}
\end{Shaded}

To avoid duplication you could

\begin{itemize}
\tightlist
\item
  (Optionally) create the canvas in the constructor and save the return
  value
\item
  OR create delegation functions inside your draw function e.g.
\end{itemize}

\begin{Shaded}
\begin{Highlighting}[]
   \AttributeTok{rect}\NormalTok{(x}\OperatorTok{,}\NormalTok{y}\OperatorTok{,}\NormalTok{w}\OperatorTok{,}\NormalTok{h)}\OperatorTok{\{}
     \ControlFlowTok{if}\NormalTok{(g)}\OperatorTok{\{}
       \VariableTok{g}\NormalTok{.}\AttributeTok{rect}\NormalTok{(x}\OperatorTok{,}\NormalTok{y}\OperatorTok{,}\NormalTok{w}\OperatorTok{,}\NormalTok{h)}\OperatorTok{;}
     \OperatorTok{\}}
     \ControlFlowTok{else}\OperatorTok{\{}
       \AttributeTok{rect}\NormalTok{(x}\OperatorTok{,}\NormalTok{y}\OperatorTok{,}\NormalTok{w}\OperatorTok{,}\NormalTok{h)}\OperatorTok{;}
     \OperatorTok{\}}
   \OperatorTok{\}}
\end{Highlighting}
\end{Shaded}

\hypertarget{when-are-private-fields-appropriate}{%
\subsection{When are private fields
appropriate?}\label{when-are-private-fields-appropriate}}

\begin{itemize}
\tightlist
\item
  They are a a post-ES6 experimental feature
\item
  On reflection, probably not appropriate to use
\end{itemize}

\hypertarget{how-to-use-eslint}{%
\subsection{How to use ESLint?}\label{how-to-use-eslint}}

\begin{itemize}
\tightlist
\item
  Want to use \href{https://eslint.org/docs/rules/}{eslint:recommended}
  rules
\item
  \href{https://eslint.org/docs/user-guide/getting-started}{Set up on
  your machine}
\item
  When running \texttt{eslint\ -\/-init} choose
  \texttt{Answer\ questions\ ...}
\end{itemize}

\begin{verbatim}
DA-ECS-01:ball dcs0spb$ eslint --init
? How would you like to configure ESLint? Answer questions about your style
? Which version of ECMAScript do you use? ES2015
? Are you using ES6 modules? No
? Where will your code run? Browser
? Do you use CommonJS? No
\end{verbatim}

\begin{verbatim}
? Do you use JSX? No
? What style of indentation do you use? Spaces
? What quotes do you use for strings? Single
? What line endings do you use? Unix
? Do you require semicolons? Yes
? What format do you want your config file to be in? JavaScript
\end{verbatim}

You might want to use Windows line endings

\hypertarget{eslint-output}{%
\subsection{ESLint output}\label{eslint-output}}

\begin{verbatim}
DA-ECS-01:ball dcs0spb$ eslint ball.js

/Users/dcs0spb/Documents/Prog18potatoes/ball/ball.js
   1:7   error  'Ball' is defined but never used                    no-unused-vars
   3:1   error  Expected indentation of 8 spaces but found 1 tab    indent
   4:1   error  Expected indentation of 8 spaces but found 1 tab    indent
   5:1   error  Expected indentation of 8 spaces but found 1 tab    indent
   5:24  error  Unnecessary semicolon                               no-extra-semi
\end{verbatim}

\hypertarget{can-i-convert-a-processing-sketch}{%
\subsection{Can I convert a processing
sketch?}\label{can-i-convert-a-processing-sketch}}

Yes

\begin{itemize}
\tightlist
\item
  Replace variable types with \texttt{var} (or \texttt{let} or
  \texttt{const})
\item
  Replace method return types e.g.~\texttt{void} with \texttt{function}
\item
  Find equivalent objects e.g.~\texttt{new\ PVector} becomes
  \texttt{createVector}
\end{itemize}

\hypertarget{others}{%
\subsection{Others?}\label{others}}

\end{document}
