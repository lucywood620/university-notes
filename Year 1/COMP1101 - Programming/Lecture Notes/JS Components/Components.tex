\hypertarget{javascript-objects}{%
\section{Javascript Objects}\label{javascript-objects}}

\hypertarget{collection-of-properties}{%
\subsection{Collection of properties}\label{collection-of-properties}}

Each property is named (with a key) and has a value

In Javascript Object Notation (JSON) we can write

\begin{verbatim}
let ball = {x: 200, y: 300, radius: 50};
\end{verbatim}

\hypertarget{obj.prop}{%
\subsection{obj.prop}\label{obj.prop}}

Access and properties like this

\begin{verbatim}
ellipse(ball.x, ball.y, ball.radius*2, ball.radius*2);
ball.x += 5;
ball.z = 8;
ball["colour"] = "red";
\end{verbatim}

\hypertarget{function-valued-properties}{%
\subsection{Function-valued
properties}\label{function-valued-properties}}

Object properties can be any type, including functions

\begin{verbatim}
ball.draw = function(){ alert("I am a ball");}
ball.draw();
\end{verbatim}

\hypertarget{this}{%
\subsection{this}\label{this}}

\texttt{this} refers to the object it was called on

\begin{verbatim}
ball.draw = function(){
      ellipse(this.x, this.y, this.radius*2, this.radius*2);
      }
ball.draw();
\end{verbatim}

\hypertarget{prototypal-inheritance}{%
\subsection{Prototypal Inheritance}\label{prototypal-inheritance}}

\begin{itemize}
\tightlist
\item
  Every object has a property \texttt{\_\_proto\_\_} which refers to
  another object
\item
  If a property isn't found in an object's own properties, then
  \texttt{\_\_proto\_\_} is checked
\item
  Every function has a property \texttt{prototype} which can be used
  when creating an object
\item
  The \texttt{new} keyword is used with a constructor function to create
  an object and set its \texttt{\_\_proto\_\_}
\item
  Read more at
  \href{https://developer.mozilla.org/en-US/docs/Web/JavaScript/Inheritance_and_the_prototype_chain}{MDN}
\end{itemize}

\hypertarget{inheriting-behaviour}{%
\subsection{Inheriting behaviour}\label{inheriting-behaviour}}

\begin{itemize}
\item
  In other languages (e.g Java, C\#) every object belongs to a
  \emph{class}

  \begin{itemize}
  \tightlist
  \item
    Data values (fields) are associated with objects
  \item
    Behaviour (methods) are associated with classes
  \end{itemize}
\item
  Things of the same type (class) can do the same things
\item
  JS is more flexible: each object can define its own behaviour
\item
  JS allows inheritance (common behaviour) through prototypes
\item
  Java uses \emph{class-based inheritance}
\item
  JS use \emph{prototypal inheritance}
\end{itemize}

\hypertarget{emulating-classes-in-js}{%
\subsection{Emulating classes in JS}\label{emulating-classes-in-js}}

\href{https://developer.mozilla.org/en-US/docs/Web/JavaScript/Reference/Classes}{Simple
syntax for constructors and prototype functions}

\begin{verbatim}
class Ball{
   constructor(x, y, r){
      this.x = x;
      this.y = y;
      this.radius = r;
   }
   draw(){
      ellipse(this.x, this.y,
         this.radius*2, this.radius*2);
      }
}
let b = new Ball(400,300,20);
b.draw();
\end{verbatim}

\hypertarget{why-classes}{%
\subsection{Why classes?}\label{why-classes}}

\begin{itemize}
\tightlist
\item
  Reduces cut-and-paste: eases maintenance
\item
  Encourages \emph{encapsulation}: hide the details so they can be
  changed easily
\item
  Make reusable compoments with classes
\item
  Reuse in the same project (multiple balls) or in different projects
\end{itemize}
