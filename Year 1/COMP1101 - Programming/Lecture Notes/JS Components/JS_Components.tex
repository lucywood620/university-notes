\documentclass{article}[18pt]
\ProvidesPackage{format}
%Page setup
\usepackage[utf8]{inputenc}
\usepackage[margin=0.7in]{geometry}
\usepackage{parselines} 
\usepackage[english]{babel}
\usepackage{fancyhdr}
\usepackage{titlesec}
\hyphenpenalty=10000

\pagestyle{fancy}
\fancyhf{}
\rhead{Sam Robbins}
\rfoot{Page \thepage}

%Characters
\usepackage{amsmath}
\usepackage{amssymb}
\usepackage{gensymb}
\newcommand{\R}{\mathbb{R}}

%Diagrams
\usepackage{pgfplots}
\usepackage{graphicx}
\usepackage{tabularx}
\usepackage{relsize}
\pgfplotsset{width=10cm,compat=1.9}
\usepackage{float}

%Length Setting
\titlespacing\section{0pt}{14pt plus 4pt minus 2pt}{0pt plus 2pt minus 2pt}
\newlength\tindent
\setlength{\tindent}{\parindent}
\setlength{\parindent}{0pt}
\renewcommand{\indent}{\hspace*{\tindent}}

%Programming Font
\usepackage{courier}
\usepackage{listings}
\usepackage{pxfonts}

%Lists
\usepackage{enumerate}
\usepackage{enumitem}

% Networks Macro
\usepackage{tikz}


% Commands for files converted using pandoc
\providecommand{\tightlist}{%
	\setlength{\itemsep}{0pt}\setlength{\parskip}{0pt}}
\usepackage{hyperref}

% Get nice commands for floor and ceil
\usepackage{mathtools}
\DeclarePairedDelimiter{\ceil}{\lceil}{\rceil}
\DeclarePairedDelimiter{\floor}{\lfloor}{\rfloor}

% Allow itemize to go up to 20 levels deep (just change the number if you need more you madman)
\usepackage{enumitem}
\setlistdepth{20}
\renewlist{itemize}{itemize}{20}

% initially, use dots for all levels
\setlist[itemize]{label=$\cdot$}

% customize the first 3 levels
\setlist[itemize,1]{label=\textbullet}
\setlist[itemize,2]{label=--}
\setlist[itemize,3]{label=*}

% Definition and Important Stuff
% Important stuff
\usepackage[framemethod=TikZ]{mdframed}

\newcounter{theo}[section]\setcounter{theo}{0}
\renewcommand{\thetheo}{\arabic{section}.\arabic{theo}}
\newenvironment{important}[1][]{%
	\refstepcounter{theo}%
	\ifstrempty{#1}%
	{\mdfsetup{%
			frametitle={%
				\tikz[baseline=(current bounding box.east),outer sep=0pt]
				\node[anchor=east,rectangle,fill=red!50]
				{\strut Important};}}
	}%
	{\mdfsetup{%
			frametitle={%
				\tikz[baseline=(current bounding box.east),outer sep=0pt]
				\node[anchor=east,rectangle,fill=red!50]
				{\strut Important:~#1};}}%
	}%
	\mdfsetup{innertopmargin=10pt,linecolor=red!50,%
		linewidth=2pt,topline=true,%
		frametitleaboveskip=\dimexpr-\ht\strutbox\relax
	}
	\begin{mdframed}[]\relax%
		\centering
		}{\end{mdframed}}



\newcounter{lem}[section]\setcounter{lem}{0}
\renewcommand{\thelem}{\arabic{section}.\arabic{lem}}
\newenvironment{defin}[1][]{%
	\refstepcounter{lem}%
	\ifstrempty{#1}%
	{\mdfsetup{%
			frametitle={%
				\tikz[baseline=(current bounding box.east),outer sep=0pt]
				\node[anchor=east,rectangle,fill=blue!20]
				{\strut Definition};}}
	}%
	{\mdfsetup{%
			frametitle={%
				\tikz[baseline=(current bounding box.east),outer sep=0pt]
				\node[anchor=east,rectangle,fill=blue!20]
				{\strut Definition:~#1};}}%
	}%
	\mdfsetup{innertopmargin=10pt,linecolor=blue!20,%
		linewidth=2pt,topline=true,%
		frametitleaboveskip=\dimexpr-\ht\strutbox\relax
	}
	\begin{mdframed}[]\relax%
		\centering
		}{\end{mdframed}}
\lhead{Programming}

\usepackage{minted}
\usepackage{hyperref}


\providecommand{\tightlist}{%
	\setlength{\itemsep}{0pt}\setlength{\parskip}{0pt}}
\begin{document}
\begin{center}
\underline{\huge JS Components}
\end{center}


\hypertarget{javascript-objects}{%
	\section{Javascript Objects}\label{javascript-objects}}

\hypertarget{collection-of-properties}{%
	\subsection{Collection of properties}\label{collection-of-properties}}

Each property is named (with a key) and has a value

In Javascript Object Notation (JSON) we can write

\begin{minted}{js}
let ball = {x: 200, y: 300, radius: 50};
\end{minted}

\hypertarget{obj.prop}{%
	\subsection{obj.prop}\label{obj.prop}}

Access and properties like this

\begin{minted}{js}
ellipse(ball.x, ball.y, ball.radius*2, ball.radius*2);
ball.x += 5;
ball.z = 8;
ball["colour"] = "red";
\end{minted}
The bottom line has the same effect as ball.colour="red";

\hypertarget{function-valued-properties}{%
	\subsection{Function-valued
		properties}\label{function-valued-properties}}

Object properties can be any type, including functions

\begin{minted}{js}
ball.draw = function(){ alert("I am a ball");}
ball.draw();
\end{minted}

\hypertarget{this}{%
	\subsection{this}\label{this}}

\texttt{this} refers to the object it was called on

\begin{minted}{js}
ball.draw = function(){
ellipse(this.x, this.y, this.radius*2, this.radius*2);
}
ball.draw();
\end{minted}

\hypertarget{prototypal-inheritance}{%
	\subsection{Prototypal Inheritance}\label{prototypal-inheritance}}

\begin{itemize}
	\tightlist
	\item
	Every object has a property \texttt{\_\_proto\_\_} which refers to
	another object
	\item
	If a property isn't found in an object's own properties, then
	\texttt{\_\_proto\_\_} is checked
	\item
	Every function has a property \texttt{prototype} which can be used
	when creating an object
	\item
	The \texttt{new} keyword is used with a constructor function to create
	an object and set its \texttt{\_\_proto\_\_}
	\item
	Read more at
	\href{https://developer.mozilla.org/en-US/docs/Web/JavaScript/Inheritance_and_the_prototype_chain}{MDN}
\end{itemize}

\hypertarget{inheriting-behaviour}{%
	\subsection{Inheriting behaviour}\label{inheriting-behaviour}}

\begin{itemize}
	\item
	In other languages (e.g Java, C\#) every object belongs to a
	\emph{class}
	
	\begin{itemize}
		\tightlist
		\item
		Data values (fields) are associated with objects
		\item
		Behaviour (methods) are associated with classes
	\end{itemize}
	\item
	Things of the same type (class) can do the same things
	\item
	JS is more flexible: each object can define its own behaviour
	\item
	JS allows inheritance (common behaviour) through prototypes
	\item
	Java uses \emph{class-based inheritance} (object to class)
	\item
	JS use \emph{prototypal inheritance} (object to object)
\end{itemize}

\hypertarget{emulating-classes-in-js}{%
	\subsection{Emulating classes in JS}\label{emulating-classes-in-js}}

\href{https://developer.mozilla.org/en-US/docs/Web/JavaScript/Reference/Classes}{Simple
	syntax for constructors and prototype functions}

\begin{minted}{js}
class Ball{
constructor(x, y, r){
this.x = x;
this.y = y;
this.radius = r;
}
draw(){
ellipse(this.x, this.y,
this.radius*2, this.radius*2);
}
}
let b = new Ball(400,300,20);
b.draw();
\end{minted}

\hypertarget{why-classes}{%
	\subsection{Why classes?}\label{why-classes}}

\begin{itemize}
	\tightlist
	\item
	Reduces cut-and-paste: eases maintenance
	\item
	Encourages \emph{encapsulation}: hide the details so they can be
	changed easily
	\item
	Make reusable compoments with classes
	\item
	Reuse in the same project (multiple balls) or in different projects
\end{itemize}




\end{document}