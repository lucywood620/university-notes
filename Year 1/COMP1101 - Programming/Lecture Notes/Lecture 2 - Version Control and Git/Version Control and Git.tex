\documentclass{article}[18pt]
\usepackage{../../../../format}
\lhead{Programming}
\usepackage{listings}
\usepackage{courier}


\begin{document}
\begin{center}
\underline{\huge Version Control and Git}
\end{center}
Linear history - Sequential history of saving on top of a file, when working alone\\
Multiple authors - Document sent to multiple people, each of who make edits, forming a tree (no cycles, directed edges)\\
Merging changes - Merging two different versions of an original document into one document
\section{Version control software}
\begin{itemize}
\item RCS (Revision control system) - in order to save space, only stored difference between files, rather than the whole files. Stores latest versions, and stores diffs between previous versions
\item CVS (Concurrent Versions System) - Managed under RCS, manages collections of files
\item Microsoft word track changes
\item Subversion 
\item git - Doesn't use diff and patch
\end{itemize}
\section{All about git}
RCS was designed to be compact, git is designed to be fast
\begin{itemize}
\item Distributed version control system - no central repository of anything
\item Developed by Linus Torvalds and others to manage the linux kernel
\item Designed to be fast
\item Very widely used in academia and industry
\item Stores every version of every file produced
\end{itemize}
\subsection{Git under the hood}
\begin{itemize}
\item Different from earlier systems such as RCS - no diffs
\item Originally developed under Linux, but available elsewhere
\item No central repository, but can synchronise with remotes
\item Cloud hosted repository servers: github etc
\end{itemize}
\subsection{Key concepts in git}
\begin{itemize}
\item A \textbf{file} (in a path)
\item A \textbf{commit}: a snapshot of a collection of files at a particular time
\item A \textbf{branch}: A linear sequence of commits - always a previous version on a branch, and relation known
\item A \textbf{repository}: (possibly) many branches of a project
\item A \textbf{remote}: another place where a repository is stored
\end{itemize}
\subsection{Key commands in git}
\begin{lstlisting}[mathescape=true]
git init
git add
git status
git commit
git push
\end{lstlisting}
\subsubsection{git init}
Creates a directory \texttt{.git} where everything is stored. You may also want to do \texttt{git config} at this stage. Think about adding a \texttt{.gitignore} file
\subsubsection{git add}
Puts current working version of a file into the staging area (area for all files to be committed, copy not cut)\\
Preparing for a commit\\
Check what will be committed with git status
\subsubsection{git commit}
Creates a new commit in the git branch\\
Makes a commit based on currently staged files\\
Will start an editor (\texttt{git config})\\
Consider \texttt{git commit -m "message"} to avoid editor
\subsubsection{git push}
Pushes a branch to a remote repository\\
\texttt{git push origin master}\\
origin defined by \texttt{git remote add origin} or \texttt{git clone}
\section{Using git}
To make a git folder, either use git init to start a new repo, or git clone to copy a repo from a remote repository.\\
\\
Git add adds the files you want to commit to staging\\
\\
Git commit will move the files from the staging area to the head of one of the branches in the repository.\\
\\
Git push moves the files into the remote repository\\
\\
Git add does not also commit as there are some changed files that you do not want to commit to the repository.\\
\\
To move a repo to github, create an empty repository and set the remote.


\end{document}