\documentclass{article}[18pt]
\ProvidesPackage{format}
%Page setup
\usepackage[utf8]{inputenc}
\usepackage[margin=0.7in]{geometry}
\usepackage{parselines} 
\usepackage[english]{babel}
\usepackage{fancyhdr}
\usepackage{titlesec}
\hyphenpenalty=10000

\pagestyle{fancy}
\fancyhf{}
\rhead{Sam Robbins}
\rfoot{Page \thepage}

%Characters
\usepackage{amsmath}
\usepackage{amssymb}
\usepackage{gensymb}
\newcommand{\R}{\mathbb{R}}

%Diagrams
\usepackage{pgfplots}
\usepackage{graphicx}
\usepackage{tabularx}
\usepackage{relsize}
\pgfplotsset{width=10cm,compat=1.9}
\usepackage{float}

%Length Setting
\titlespacing\section{0pt}{14pt plus 4pt minus 2pt}{0pt plus 2pt minus 2pt}
\newlength\tindent
\setlength{\tindent}{\parindent}
\setlength{\parindent}{0pt}
\renewcommand{\indent}{\hspace*{\tindent}}

%Programming Font
\usepackage{courier}
\usepackage{listings}
\usepackage{pxfonts}

%Lists
\usepackage{enumerate}
\usepackage{enumitem}

% Networks Macro
\usepackage{tikz}


% Commands for files converted using pandoc
\providecommand{\tightlist}{%
	\setlength{\itemsep}{0pt}\setlength{\parskip}{0pt}}
\usepackage{hyperref}

% Get nice commands for floor and ceil
\usepackage{mathtools}
\DeclarePairedDelimiter{\ceil}{\lceil}{\rceil}
\DeclarePairedDelimiter{\floor}{\lfloor}{\rfloor}

% Allow itemize to go up to 20 levels deep (just change the number if you need more you madman)
\usepackage{enumitem}
\setlistdepth{20}
\renewlist{itemize}{itemize}{20}

% initially, use dots for all levels
\setlist[itemize]{label=$\cdot$}

% customize the first 3 levels
\setlist[itemize,1]{label=\textbullet}
\setlist[itemize,2]{label=--}
\setlist[itemize,3]{label=*}

% Definition and Important Stuff
% Important stuff
\usepackage[framemethod=TikZ]{mdframed}

\newcounter{theo}[section]\setcounter{theo}{0}
\renewcommand{\thetheo}{\arabic{section}.\arabic{theo}}
\newenvironment{important}[1][]{%
	\refstepcounter{theo}%
	\ifstrempty{#1}%
	{\mdfsetup{%
			frametitle={%
				\tikz[baseline=(current bounding box.east),outer sep=0pt]
				\node[anchor=east,rectangle,fill=red!50]
				{\strut Important};}}
	}%
	{\mdfsetup{%
			frametitle={%
				\tikz[baseline=(current bounding box.east),outer sep=0pt]
				\node[anchor=east,rectangle,fill=red!50]
				{\strut Important:~#1};}}%
	}%
	\mdfsetup{innertopmargin=10pt,linecolor=red!50,%
		linewidth=2pt,topline=true,%
		frametitleaboveskip=\dimexpr-\ht\strutbox\relax
	}
	\begin{mdframed}[]\relax%
		\centering
		}{\end{mdframed}}



\newcounter{lem}[section]\setcounter{lem}{0}
\renewcommand{\thelem}{\arabic{section}.\arabic{lem}}
\newenvironment{defin}[1][]{%
	\refstepcounter{lem}%
	\ifstrempty{#1}%
	{\mdfsetup{%
			frametitle={%
				\tikz[baseline=(current bounding box.east),outer sep=0pt]
				\node[anchor=east,rectangle,fill=blue!20]
				{\strut Definition};}}
	}%
	{\mdfsetup{%
			frametitle={%
				\tikz[baseline=(current bounding box.east),outer sep=0pt]
				\node[anchor=east,rectangle,fill=blue!20]
				{\strut Definition:~#1};}}%
	}%
	\mdfsetup{innertopmargin=10pt,linecolor=blue!20,%
		linewidth=2pt,topline=true,%
		frametitleaboveskip=\dimexpr-\ht\strutbox\relax
	}
	\begin{mdframed}[]\relax%
		\centering
		}{\end{mdframed}}
\lhead{Programming}
\lstset{basicstyle=\ttfamily,
  showstringspaces=false,
  commentstyle=\color{red},
  keywordstyle=\color{blue},
  basicstyle=\footnotesize\ttfamily,
  keywordsprefix=--,
  keywordsprefix=-,
  keywordstyle=\bfseries\color{green!40!black},
  commentstyle=\itshape\color{purple!40!black},
  identifierstyle=\color{black},
  stringstyle=\color{red},
  language=bash,
  tabsize=4
}

\begin{document}
\begin{center}
\underline{\huge Programming Practical 2 - git}
\end{center}
\section{Setup}
\begin{itemize}
\item To set your username and password:
\begin{lstlisting}[mathescape=true]
git config --global user.name "Username"
git config --global user.name "email@domain.com"
\end{lstlisting}
\item To set the editor
\begin{lstlisting}[mathescape=true]
git config --global core.editor "nano -w"
\end{lstlisting}
\item To check settings
\begin{lstlisting}[mathescape=true]
git config --list
\end{lstlisting}
\end{itemize}
\section{Creating a repository}
\begin{itemize}
\item To initialise a repository
\begin{lstlisting}[mathescape=true]
git init
\end{lstlisting}
\item To check the status of the project
\begin{lstlisting}[mathescape=true]
git status
\end{lstlisting}
\end{itemize}
\section{Tracking changes}
\begin{itemize}
\item To add a directory to staging
\begin{lstlisting}[mathescape=true]
git add <file>
\end{lstlisting}
\item To commit the staging area to the repository
\begin{lstlisting}[mathescape=true]
git commit -m "Commit comment"
\end{lstlisting}
\item To see what has been done recently
\begin{lstlisting}[mathescape=true]
git log
\end{lstlisting}
\item To see what has been changed since the last commit
\begin{lstlisting}[mathescape=true]
git diff
\end{lstlisting}
\item To see what has been changed and added to the staging area since the last commit
\begin{lstlisting}[mathescape=true]
git diff --staged
\end{lstlisting}
\item To limit the log size:
\begin{lstlisting}[mathescape=true]
git log -1
\end{lstlisting}
To display a more compressed version of the log
\begin{lstlisting}[mathescape=true]
git log --oneline
\end{lstlisting}
\end{itemize}
\section{Exploring history}
\begin{itemize}
\item HEAD can be used to refer to the most recent commit, so to get the changes in the most recent commit:
\begin{lstlisting}[mathescape=true]
git diff HEAD <file>
\end{lstlisting}
\item To refer to commits before the head, use a ~ after the head, so to refer to the penultimate commit:
\begin{lstlisting}[mathescape=true]
git diff HEAD~1 <file>
\end{lstlisting}
\item To see the changes, as well as the commit message use git show, for example
\begin{lstlisting}[mathescape=true]
git show HEAD~2 <file>
\end{lstlisting}
\item To refer to specific commits, use the hex string provided when using git log --oneline
\begin{lstlisting}[mathescape=true]
git diff f22b25e <file>
\end{lstlisting}
\item To change things back to how they were use git checkout
\begin{lstlisting}[mathescape=true]
git checkout f22b25e <file>
\end{lstlisting}

\end{itemize}



\end{document}