\documentclass{article}[18pt]
\usepackage{../../../../format}
\lhead{Networks and Systems - Compiler Design}


\begin{document}
\begin{center}
\underline{\huge Lexical Analysis}
\end{center}
The role of a lexical analyser
\begin{itemize}
	\item The first phase of a compiler
	\item Reads input characters
	\item Groups them into lexemes
	\item Produces as output a sequence of tokens
	\item The stream of tokens is sent to the parser
	\item When it discovers the lexeme for a new identifier it enters this lexeme into the symbol table
\end{itemize}
\subsection{Lexical Analysis vs Parsing}
Lexical analysis and syntax analysis are separate phases
\begin{itemize}
	\item Simplicity of design
	\begin{itemize}
		\item Simplify each task separately
	\end{itemize}
	\item Improved compiler efficiency
	\begin{itemize}
		\item Apply specialised techniques for each step
	\end{itemize}
	\item Compiler portability
	\begin{itemize}
		\item Different lexical analysers for specific devices
	\end{itemize}
\end{itemize}
\section{Tokens vs Lexemes}
Token
\begin{itemize}
	\item A notation representing the kind of lexical unit, e.g.
	\begin{itemize}
		\item A keyword
		\item An identifier
	\end{itemize}
	\item Consists of a pair of:
	\begin{itemize}
		\item A token name
		\item An (optional) attribute value
	\end{itemize}
	\item The tokens are given as input to the parser
\end{itemize}
Pattern of a token
\begin{itemize}
	\item A description of the form of its lexemes, e.g.
	\begin{itemize}
		\item For a keyword, the sequence of its characters
		\item For an identifier: a description that matches many strings
	\end{itemize}
\end{itemize}
The lexeme of a token
\begin{itemize}
	\item A sequence of characters in the source program that matches the pattern of the token
	\item The lexical analyser identifies a lexeme as an instance of a token
\end{itemize}
\section{Attributes of tokens}
In cases when many lexemes match with a specific token
\begin{itemize}
	\item The compiler must know which lexeme was found in the source program
	\item The lexical analyser provides to the parser
	\begin{itemize}
		\item The token name
		\item Additional information that specifies the particular lexeme represented by the token (attribute value)
	\end{itemize}
\end{itemize}
Token names - influence parsing decisions\\
\\
Attribute names - influence the transition of the token after the parsing (in the semantic analysis)



\end{document}