\documentclass{article}[18pt]
\usepackage{../../../../format}
\lhead{Networks and Systems - Compiler Design}


\begin{document}
\begin{center}
\underline{\huge Top Down Analysis}
\end{center}
Top down constriction of a parse tree:
\begin{itemize}
	\item Start with the root, labelled by the starting symbol
	\item Repeatedly perform the following steps:
	\begin{enumerate}
		\item At internal node N, labelled with non-terminal A
		\begin{itemize}
			\item Select one of the production rules for A
			\item Construct children at N for the symbols in the right part of this production rule
		\end{itemize}
		\item Find the next node to construct a subtree
		\begin{itemize}
			\item Typically the leftmost unexpanded non-terminal of the current tree
		\end{itemize}
	\end{enumerate}
\end{itemize}
During the construction of the parse tree the current terminal of the input that is being scanned is called the lookahead symbol\\
\\
Our aim during top-down parsing - to construct the parse tree, such that the string generated by the parse tree matches the input string\\
\\
For a match to occur - the starting symbol stmt must derive a string that starts with for\\
\\
When a node in the parse tree:
\begin{itemize}
	\item Is labelled with a terminal
	\item Matches the lookahead symbol
\end{itemize}
Then
\begin{itemize}
	\item The lookahead becomes the next terminal in the input
	\item We consider the next child in the parse tree
\end{itemize}
When a node in the parse tree is labelled with a non-terminal then:
\begin{itemize}
	\item We repeat by selecting one of its production rules
	\item Special case: 
\end{itemize}




\end{document}