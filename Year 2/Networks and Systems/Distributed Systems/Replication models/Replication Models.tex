\documentclass{article}[18pt]
\ProvidesPackage{format}
%Page setup
\usepackage[utf8]{inputenc}
\usepackage[margin=0.7in]{geometry}
\usepackage{parselines} 
\usepackage[english]{babel}
\usepackage{fancyhdr}
\usepackage{titlesec}
\hyphenpenalty=10000

\pagestyle{fancy}
\fancyhf{}
\rhead{Sam Robbins}
\rfoot{Page \thepage}

%Characters
\usepackage{amsmath}
\usepackage{amssymb}
\usepackage{gensymb}
\newcommand{\R}{\mathbb{R}}

%Diagrams
\usepackage{pgfplots}
\usepackage{graphicx}
\usepackage{tabularx}
\usepackage{relsize}
\pgfplotsset{width=10cm,compat=1.9}
\usepackage{float}

%Length Setting
\titlespacing\section{0pt}{14pt plus 4pt minus 2pt}{0pt plus 2pt minus 2pt}
\newlength\tindent
\setlength{\tindent}{\parindent}
\setlength{\parindent}{0pt}
\renewcommand{\indent}{\hspace*{\tindent}}

%Programming Font
\usepackage{courier}
\usepackage{listings}
\usepackage{pxfonts}

%Lists
\usepackage{enumerate}
\usepackage{enumitem}

% Networks Macro
\usepackage{tikz}


% Commands for files converted using pandoc
\providecommand{\tightlist}{%
	\setlength{\itemsep}{0pt}\setlength{\parskip}{0pt}}
\usepackage{hyperref}

% Get nice commands for floor and ceil
\usepackage{mathtools}
\DeclarePairedDelimiter{\ceil}{\lceil}{\rceil}
\DeclarePairedDelimiter{\floor}{\lfloor}{\rfloor}

% Allow itemize to go up to 20 levels deep (just change the number if you need more you madman)
\usepackage{enumitem}
\setlistdepth{20}
\renewlist{itemize}{itemize}{20}

% initially, use dots for all levels
\setlist[itemize]{label=$\cdot$}

% customize the first 3 levels
\setlist[itemize,1]{label=\textbullet}
\setlist[itemize,2]{label=--}
\setlist[itemize,3]{label=*}

% Definition and Important Stuff
% Important stuff
\usepackage[framemethod=TikZ]{mdframed}

\newcounter{theo}[section]\setcounter{theo}{0}
\renewcommand{\thetheo}{\arabic{section}.\arabic{theo}}
\newenvironment{important}[1][]{%
	\refstepcounter{theo}%
	\ifstrempty{#1}%
	{\mdfsetup{%
			frametitle={%
				\tikz[baseline=(current bounding box.east),outer sep=0pt]
				\node[anchor=east,rectangle,fill=red!50]
				{\strut Important};}}
	}%
	{\mdfsetup{%
			frametitle={%
				\tikz[baseline=(current bounding box.east),outer sep=0pt]
				\node[anchor=east,rectangle,fill=red!50]
				{\strut Important:~#1};}}%
	}%
	\mdfsetup{innertopmargin=10pt,linecolor=red!50,%
		linewidth=2pt,topline=true,%
		frametitleaboveskip=\dimexpr-\ht\strutbox\relax
	}
	\begin{mdframed}[]\relax%
		\centering
		}{\end{mdframed}}



\newcounter{lem}[section]\setcounter{lem}{0}
\renewcommand{\thelem}{\arabic{section}.\arabic{lem}}
\newenvironment{defin}[1][]{%
	\refstepcounter{lem}%
	\ifstrempty{#1}%
	{\mdfsetup{%
			frametitle={%
				\tikz[baseline=(current bounding box.east),outer sep=0pt]
				\node[anchor=east,rectangle,fill=blue!20]
				{\strut Definition};}}
	}%
	{\mdfsetup{%
			frametitle={%
				\tikz[baseline=(current bounding box.east),outer sep=0pt]
				\node[anchor=east,rectangle,fill=blue!20]
				{\strut Definition:~#1};}}%
	}%
	\mdfsetup{innertopmargin=10pt,linecolor=blue!20,%
		linewidth=2pt,topline=true,%
		frametitleaboveskip=\dimexpr-\ht\strutbox\relax
	}
	\begin{mdframed}[]\relax%
		\centering
		}{\end{mdframed}}
\lhead{Networks and Systems - Distributed Systems}


\begin{document}
\begin{center}
\underline{\huge Replication Models}
\end{center}
\section{Replication}
\begin{itemize}
	\item Provide multiple copies of the same data or functionalities (services) in a distributed system
	\item Improve system capabilities in terms of performance, availability and load distribution
\end{itemize}
\section{Types and Requirements}
Types:
\begin{itemize}
	\item Computation (function/service) replication: multiple instances of the same functional processes are executed (behaviour) - may run on different hardware and implement different algorithms
	\item Data replication: same piece of information is being stored in multiple devices (data integrity)
\end{itemize}
Requirements:\begin{itemize}
	\item Replication transparency: A user sees one logical service, but not its physical copies
	\item Data consistency: The same request will receive the same result even if it is processed by a different copy of the same service
\end{itemize}
\section{System Model - Replication}
\begin{center}
	\includegraphics[scale=0.7]{"System Model"}
\end{center}
Replicas:
\begin{itemize}
	\item Maintain copies of the same data or functions - can be implemented by different technologies
	\item Replicas are not necessarily consistent all the time (some may have received updates, not yet conveyed to others)
\end{itemize}
\section{System Components}
Replicas (R)
\begin{itemize}
	\item Maintain replicas (data/functions) on servers
	\item Process requests or store results (may propagate to other servers)
	\item Dynamic/static: set of Rs is fixed or variable (scalability issues)
\end{itemize}
Clients (C) requests
\begin{itemize}
	\item Those without updates are called read-only requests, the others are called update requests (they may include reads)
	\item Read only: handle by one replica
	\item Update: may involve data propagation/synchronisation, and concurrency control
\end{itemize}
Front End (FE)
\begin{itemize}
	\item Make replication transparent
	\item Monitor and maintain replica availability
	\item Perform request distribution, and collate responses
\end{itemize}
\section{System workflow}
\begin{enumerate}
	\item Incoming request
	\begin{itemize}
		\item Receive by the FE, and the FE will forward to the R(s)
	\end{itemize}
	\item Coordination
	\begin{itemize}
		\item R(s) accepts a request
		\item Decide the ordering request relative to other requests
	\end{itemize}
	\item Execution
	\begin{itemize}
		\item R(s) process the request
	\end{itemize}
	\item Agreement
	\begin{itemize}
		\item R(s) reach consensus on the effect of the requests
	\end{itemize}
	\item Response
	\begin{itemize}
		\item One or more Rs reply to the FE
		\item FE may process the response before returning it to the client
	\end{itemize}
\end{enumerate}
\section{Fault-Tolerance Services}
\begin{itemize}
	\item Provide a correct service despite up to f process failures
	\item Each replica is assumed to behave according to the specification of the distributed system, when they have not crashed
	\item A service based on replication is correct if:
	\begin{itemize}
		\item It keeps responding despite failures
		\item Clients can't tell the difference between the service they obtain from an implementation with replicated data and one provided by a single correct replica manager
	\end{itemize}
\end{itemize}
\subsection{Passive (Primary-Backup) model for fault tolerance}
\begin{center}
	\includegraphics[scale=0.7]{"Passive Fault Tolerance"}
\end{center}
\begin{itemize}
	\item There is at any time a single primary R and one or more secondary (backup, slave) Rs
	\item FEs communicate with the primary which executes the operation and sends copies of the updates data to the result to backups
	\item If the primary fails, one of the backups is promoted to act as the primary
\end{itemize}
\subsubsection{Workflow}
\begin{enumerate}
	\item Request
	\begin{itemize}
		\item An FE issues the request, containing a unique identifier, to the primary R
	\end{itemize}
	\item Coordination
	\begin{itemize}
		\item The primary processes each request, in the order in which it received it relative to other requests (message ordering)
		\item It checks the unique id; if it has already done the request, it re-sends the response (Handle Message loss)
	\end{itemize}
	\item Execution
	\begin{itemize}
		\item The primary executes the request and stores the response
	\end{itemize}
	\item Agreement
	\begin{itemize}
		\item If the request is an update the primary sends the updated state, the response and the unique identifier to all the backups. The backups send an acknowledgement (result propagation)
	\end{itemize}
	\item Response
	\begin{itemize}
		\item The primary responds to the FE, which hands the response back to the client
	\end{itemize}
\end{enumerate}
\subsubsection{Discussion of Passive Replication}
\begin{itemize}
	\item Non-deterministic behaviour at primary replica
	\begin{itemize}
		\item e.g. due to multi-threading
		\item No fatal problem: as other replicas (backups) only slavishly record states determined by the primary's actions
	\end{itemize}
	\item Replica crashes
	\begin{itemize}
		\item Survive up to f replica crash, when the system comprises $f+1$ replicas
	\end{itemize}
	\item Front-end functionality
	\begin{itemize}
		\item Requires little functionality: only need to lookup a new primary replica when the current one isn't available
	\end{itemize}
	\item System overhead
	\begin{itemize}
		\item Relatively large due to data propagation
	\end{itemize}
\end{itemize}
\subsection{Active replication}
\begin{itemize}
	\item The Rs are state machines all playing the same role and organised as a group - all start in the same state and perform the same in the same order so that their state remains identical (synchronisation)
	\item If an R crashes it has no effect on performance of the service because the others continue as normal
\end{itemize}
\begin{center}
	\includegraphics[scale=0.7]{"Active Replication"}
\end{center}




\end{document}