\documentclass{article}[18pt]
\input{../../../../format}
\lhead{Networks and Systems - Security}


\begin{document}
\begin{center}
\underline{\huge Exploits and Mitigations}
\end{center}

\section{Buffer Overflows}
\begin{itemize}
	\item When writing data to a buffer, you overrun into adjacent memory locations
	\item Often results in a crash, but sometimes can be exploited for other malicious behaviour, such as gaining elevated privileges
	\item Can occur on the stack (stack smashing) or on the heap (heap overflow)
\end{itemize}
\subsection{Protection against buffer overflows}
\begin{itemize}
	\item Detect and abort before malicious behaviour occurs
	\item Use input that won't allow you to overflow, such as \texttt{fgets}
\end{itemize}
\section{Heartbleed}
Buffer over-read vulnerability in OpenSSL
\begin{itemize}
	\item Open source code that handles a large proportion of the world's secured web traffic
	\item Clients send heartbeats to server (are you alive?)
	\item Server responds with data
	\item A particular version of OpenSSL didn't check for over-read
	\item Each heartbeat could reveal 64k of application memory
\end{itemize}
\section{Stack smashing}
\begin{itemize}
	\item What if we jumped to somewhere else where we had malicious code?
	\begin{itemize}
		\item If we can use this on a program that has higher privilege than ourself, we can jump to deployed shellcode for that level of privilege
		\item Shell code is executable code inserted as a payload for insertion attacks
	\end{itemize}
	\item Countermeasures
	\begin{itemize}
		\item Check buffer lengths
		\item Use heap memory
		\item Use ASLR
		\item Use a canary
	\end{itemize}
\end{itemize}
\subsection{Canary Value}
\begin{itemize}
	\item Generate random number just before the stack return pointer
	\item After writing to buffer check if the value is same as randomly generated value
	\item If it has been tampered with, exit
\end{itemize}
\section{Heap smashing and NOP slides}
Heap memory rarely contains pointers that influence control flow:
\begin{itemize}
	\item Needs to be combined as part of larger attack
\end{itemize}
Heap sprays
\begin{itemize}
	\item Attempts to put a certain sequence of bytes at a predetermined location in the memory by allocation large blocks on the process' heap and filling the bytes in these blocks with specific values
	\item NOP slides/sleds
	\begin{itemize}
		\item Move onto next instruction - put loads of x90's followed by your shell code. Then your return address is likely to hit one and slide to the malicious code
	\end{itemize}
\end{itemize}
\section{Timing attacks}
Programs take slightly different amounts of time based on if the input is partially correct or not.\\
\\
This can then be used to determine if each character in the password is correct, causing a drastic decrease in the number of combinations that need to be checked
\section{Impact of AI on Cyber Security}
\begin{itemize}
	\item Gaining recent research traction (especially in 2019)
	\item Lots of unethical research taking place in this field
	\begin{itemize}
		\item Predicting sensitive information about people
	\end{itemize}
	\item Could make better anti-virus, or better viruses
\end{itemize}
Adversarial against humans
\begin{itemize}
	\item Chatbots
	\item Can't distinguish human from AI voice
	\item Reinforcement learning
	\item Deep learning fools CAPTCHA
\end{itemize}



\end{document}