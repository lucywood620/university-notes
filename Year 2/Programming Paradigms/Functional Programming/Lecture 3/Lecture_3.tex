\documentclass{article}[18pt]
\usepackage{../../../../format}
\lhead{Programming Paradigms - Functional Programming}
\usepackage{minted}
\begin{document}
\begin{center}
\underline{\huge Types and Classes II}
\end{center}
\section{Simple functions}
\begin{minted}{haskell}
foo :: Int -> Int
foo x = x + 3
\end{minted}
Two parts
\begin{enumerate}
	\item Type declaration \mintinline{haskell}{foo :: Int -> Int}
	\item Definition \mintinline{haskell}{foo x = x + 3}
\end{enumerate}
Any type declaration you write will be checked by the type inference engine. Error if incorrect\\
\\
Reasoning about types is a core part of understanding (and writing) Haskell code so always decorate function definitions with their type\\
\\
Syntax conventions:
\begin{itemize}
	\item Function application is so important that it is written as quietly as possible: with whitespace
	\item All functions are called in prefix form: "\mintinline{haskell}{foo a b}" not "\mintinline{haskell}{a foo b}"
	\item But there is special syntax for binary functions
\end{itemize}
\section{Binary functions}
All binary functions (which have type \mintinline{haskell}{a -> b -> c}) can be written as infix functions
\subsection{Symbol only names}
Names consisting only of symbols
\begin{minted}{haskell}
1 + 2 -- infix notation
(+) 1 2 -- prefix notation
False && True  -- infix notation
(&&) False True -- prefix notation
\end{minted}
\subsection{"Normal" names}
Names with alpha-numeric characters
\begin{minted}{haskell}
mod 3 2 -- prefix notation
3 `mod` 2 -- infix notation using backticks
\end{minted}
\section{Conditional Expressions}
As in most programming languages, functions can be defined using conditional expressions
\begin{minted}[mathescape=true,escapeinside=||]{haskell}
abs :: Int -> Int
abs n = if n >= 0 then n else -n
-- you always have to have the else branch
-- avoids ambiguity problems in the type checker
\end{minted}
\section{Guarded Equations}
As an alternative to conditions, functions can also be defined using guarded equations
\begin{minted}{haskell}
abs n | n >= 0    =n
      | otherwise = -n
\end{minted}
Otherwise is True
\section{Pattern Matching}
Many functions have a particularly clear definition using pattern matching on their arguments
\begin{minted}{haskell}
not :: Bool -> Bool
not False = True
not True = False
\end{minted}
Functions can often be defined in many different ways using pattern matching. For example
\begin{minted}{haskell}
(&&) :: Bool -> Bool -> Bool
True && True = True
False && False = False
False && True = False
False && False = False
\end{minted}
Or more compactly
\begin{minted}{haskell}
True && True = True
_    && _    = False
\end{minted}
And more efficiently
\begin{minted}{haskell}
True  && b = b
False && _ = False
\end{minted}
\_ is a wildcard - matches anything
\section{Polymorphism}
Recall that haskell is strictly typed, which looks like a problem for length
\begin{minted}{haskell}
length [True, False, True] -- :: [Bool] -> Int ?
length [1,2,3] -- :: [Int] -> Int
\end{minted}
Polymorphic types solve this problem
\begin{minted}{haskell}
Prelude> :type length
length :: [a] -> Int
\end{minted}
This says that length takes a list of values of any type a and returns an int
\subsection{Contrast with OO Languages}
\begin{defin}[Parametric Polymorphism]
Write a single implementation of a function that applies generically and identically to values of any type
\end{defin}
\begin{defin}["ad-hoc" polymorphism]
Write multiple implementations of a function, one for each type you wish to support
\end{defin}
\begin{defin}[Subtype polymorphism]
Relate datatypes by some "substitutability". Write a function for a supertype instance. Now all subtypes can use it. This is duck typing
\end{defin}
\newpage
\section{Haskell Classes}
\begin{important}[Class and instance]
	{The words class and instance are the same as in object oriented programming languages, but their meaning is very different}
\end{important}
\begin{defin}[Class]
A collection of types that support certain, specified overloaded operations called methods
\end{defin}
\begin{defin}[Instance]
A concrete type that belongs to a class and provided implementations of the required methods	
\end{defin}
\begin{itemize}
	\item Compare: type "a collection of related values"
	\item This is not like subclassing and inheritance in Java/C++
	\item Closest to a combination of Java interfaces and generics
	\item Also similar to C++ "Concepts"
\end{itemize}
\subsection{Type classes}
\begin{minted}{haskell}
Num -- Numeric Types
Eq -- Equality Types
Ord -- Ordered Types
\end{minted}
\subsection{Defining Classes}
We define the interface the type should support
\begin{minted}{haskell}
class Foo a where:
	isfoo :: a -> Bool
\end{minted}
Now we say how types implement this
\begin{minted}{haskell}
instance Foo Int where
	isfoo _ = False
	
instance Foo Char where
	isfoo c = c `elem` ['a' .. 'c']
\end{minted}
\begin{itemize}
	\item Can add new interfaces to old types, and new types to old interfaces
	\item Contrasted to Java, where if I implement a new interface it is very difficult to make existing classes implement it
\end{itemize}
\begin{itemize}
	\item Classes (interfaces) can provided default implementation
	\item Example, the \mintinline{haskell}{Eq} class representing equality requires both (==) and (/=)
	\item Since \mintinline{haskell}{a == b} $\Leftrightarrow$ \mintinline{haskell}{not (a /= b)}, we can provide default implementations and only require that an instance implements one
\end{itemize}
\begin{minted}{haskell}
class Eq a where
	(==) :: a -> a -> Bool
	x == y = not (x /= y)
	(/=) :: a -> a -> Bool
	x /= y = not (x == y)

-- instance for MyType only needs to provide one of (==) or (/=)
instance Eq MyType where
	x == y = ..
\end{minted}
\end{document}