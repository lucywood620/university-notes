\documentclass{article}[18pt]
\ProvidesPackage{format}
%Page setup
\usepackage[utf8]{inputenc}
\usepackage[margin=0.7in]{geometry}
\usepackage{parselines} 
\usepackage[english]{babel}
\usepackage{fancyhdr}
\usepackage{titlesec}
\hyphenpenalty=10000

\pagestyle{fancy}
\fancyhf{}
\rhead{Sam Robbins}
\rfoot{Page \thepage}

%Characters
\usepackage{amsmath}
\usepackage{amssymb}
\usepackage{gensymb}
\newcommand{\R}{\mathbb{R}}

%Diagrams
\usepackage{pgfplots}
\usepackage{graphicx}
\usepackage{tabularx}
\usepackage{relsize}
\pgfplotsset{width=10cm,compat=1.9}
\usepackage{float}

%Length Setting
\titlespacing\section{0pt}{14pt plus 4pt minus 2pt}{0pt plus 2pt minus 2pt}
\newlength\tindent
\setlength{\tindent}{\parindent}
\setlength{\parindent}{0pt}
\renewcommand{\indent}{\hspace*{\tindent}}

%Programming Font
\usepackage{courier}
\usepackage{listings}
\usepackage{pxfonts}

%Lists
\usepackage{enumerate}
\usepackage{enumitem}

% Networks Macro
\usepackage{tikz}


% Commands for files converted using pandoc
\providecommand{\tightlist}{%
	\setlength{\itemsep}{0pt}\setlength{\parskip}{0pt}}
\usepackage{hyperref}

% Get nice commands for floor and ceil
\usepackage{mathtools}
\DeclarePairedDelimiter{\ceil}{\lceil}{\rceil}
\DeclarePairedDelimiter{\floor}{\lfloor}{\rfloor}

% Allow itemize to go up to 20 levels deep (just change the number if you need more you madman)
\usepackage{enumitem}
\setlistdepth{20}
\renewlist{itemize}{itemize}{20}

% initially, use dots for all levels
\setlist[itemize]{label=$\cdot$}

% customize the first 3 levels
\setlist[itemize,1]{label=\textbullet}
\setlist[itemize,2]{label=--}
\setlist[itemize,3]{label=*}

% Definition and Important Stuff
% Important stuff
\usepackage[framemethod=TikZ]{mdframed}

\newcounter{theo}[section]\setcounter{theo}{0}
\renewcommand{\thetheo}{\arabic{section}.\arabic{theo}}
\newenvironment{important}[1][]{%
	\refstepcounter{theo}%
	\ifstrempty{#1}%
	{\mdfsetup{%
			frametitle={%
				\tikz[baseline=(current bounding box.east),outer sep=0pt]
				\node[anchor=east,rectangle,fill=red!50]
				{\strut Important};}}
	}%
	{\mdfsetup{%
			frametitle={%
				\tikz[baseline=(current bounding box.east),outer sep=0pt]
				\node[anchor=east,rectangle,fill=red!50]
				{\strut Important:~#1};}}%
	}%
	\mdfsetup{innertopmargin=10pt,linecolor=red!50,%
		linewidth=2pt,topline=true,%
		frametitleaboveskip=\dimexpr-\ht\strutbox\relax
	}
	\begin{mdframed}[]\relax%
		\centering
		}{\end{mdframed}}



\newcounter{lem}[section]\setcounter{lem}{0}
\renewcommand{\thelem}{\arabic{section}.\arabic{lem}}
\newenvironment{defin}[1][]{%
	\refstepcounter{lem}%
	\ifstrempty{#1}%
	{\mdfsetup{%
			frametitle={%
				\tikz[baseline=(current bounding box.east),outer sep=0pt]
				\node[anchor=east,rectangle,fill=blue!20]
				{\strut Definition};}}
	}%
	{\mdfsetup{%
			frametitle={%
				\tikz[baseline=(current bounding box.east),outer sep=0pt]
				\node[anchor=east,rectangle,fill=blue!20]
				{\strut Definition:~#1};}}%
	}%
	\mdfsetup{innertopmargin=10pt,linecolor=blue!20,%
		linewidth=2pt,topline=true,%
		frametitleaboveskip=\dimexpr-\ht\strutbox\relax
	}
	\begin{mdframed}[]\relax%
		\centering
		}{\end{mdframed}}
\lhead{Programming Paradigms - Functional Programming}
\usepackage{minted}
\setminted{tabsize=4}

\begin{document}
\begin{center}
\underline{\huge $\lambda$-expressions, list patterns, and comprehensions}
\end{center}
\section{Implementing factorial}
Recursive method:
\begin{minted}{haskell}
factorial :: Int -> Int
factorial 0 = 1
factorial n=n*factorial (n-1)
\end{minted}
Using built in product function
\begin{minted}{haskell}
factorial :: Int -> Int
factorial n = product [1..n]
\end{minted}
Using conditional statements
\begin{minted}{haskell}
factorial :: Int->Int
factorial n = if n==0 then 1
			  else n *factorial n-1

factorial :: Int -> Int
factorial n | n == 0 =1
			|otherwise =n* factorial n-1
\end{minted}
Function application binds more tightly than binary operators. This is implemented:
\begin{minted}{haskell}
n * (factorial n) -1
\end{minted}
$\Rightarrow$ never terminates
\section{Lambda expressions}
\subsection{Nameless function}
\begin{itemize}
	\item As well as giving functions name, we can also construct them without names using lambda expressions
	\begin{minted}{haskell}
-- the nameless function that takes a number x and returns x+x
\x -> x+x
	\end{minted}
	\item Use of a $\lambda$ for nameless functions comes from lambda calculus, which is a theory of functions
	\item There is a whole formal system on reasoning about computation using $\lambda$ calculus
	\item It is also a way of formalising the idea of lazy evaluation
	
\end{itemize}
\subsection{Use cases  for unnamed functions}
\begin{itemize}
	\item Formalises idea of functions defined using currying
\begin{minted}{haskell}
add x y = x + y
-- Equivelently
add = \x -> (\y -> x+y)
\end{minted}
	\item The latter form emphasises the idea that add is a function of one variable that returns a function
	\item Also useful when returning a function as a result
\begin{minted}{haskell}
const :: a -> b -> a
const x _ =x
-- Or, perhaps more naturally
const x = \_ -> x
\end{minted}
In this function const eats an a and returns a function which eats a b and always returns the same a
	\item What good is a function which always returns the same value?
	\item Often when using higher-order functions, we need a base case that always returns the same value
\begin{minted}{haskell}
length' :: [a] -> Int
length' xs = sum(map(const 1) xs)
\end{minted}
The length of a list can be obtained by summing the result of calling const 1 on every item in the list
	\item We will see more of this when we look at higher order functions
	\item Also useful where the function is only used once
\begin{minted}{haskell}
-- Generate the first n positive odd numbers
odds : [Int] -> [Int]
odds n = map f[0..n-1]
	where
		f x = x*2 + 1
\end{minted}
	\item Can be simplified (removing the where clause)
\begin{minted}{haskell}
odds : [Int] -> [Int]
odds n = map(\x -> x*2 +1) [0..n-1]
\end{minted}
\end{itemize}
\subsection{Translating between the two forms}
\begin{itemize}
	\item It is always possible to translate between named functions and arguments, and the approach using $\lambda$ expressions of one argument
	\item Just move the arguments to the right hand side and put it inside a $\lambda$, repeat with the remained until you're done
\begin{minted}{haskell}
f a b c = ...
-- Move formal aguments to right hand side with a lambda
f \a b c ->
-- Move remaining arguments into new lambdas
f = \a -> (\b -> (\c -> ...))
\end{minted}
	\item Which option fits more naturally is often a style choice
	\item Pattern matching is supported in the argument list in exactly the same way as normal functions
\begin{minted}{haskell}
head = \(x:_) -> x
\end{minted}
	\item I sometimes find it easier to think about composing functions or currying by explicitly writing $\lambda$ expressions
\end{itemize}
\section{Lists}
\subsection{Pattern matching}
\subsubsection{Representations of lists}
\begin{itemize}
	\item Every non-empty list is created by repeated use of the (:) operator "construct" that adds an element to the start of a list
\begin{minted}{haskell}
[1,2,3,4] = 1 : (2: (3: (4: [])))
\end{minted}
	\item This is a representation of a linked list
	\item Operations on lists such as indexing, or computing the length must therefore traverse the list
	\item Operations such as reverse, length (!!) are linear in the length of the list
	\item Getting the head and tail is constant time, as is (:) itself
\end{itemize}
\subsubsection{Pattern matching on lists}
\begin{itemize}
	\item Lists can be used for pattern matching in function definitions
\begin{minted}{haskell}
startsWithA :: [Char] -> Bool
startsWithA ['a',_,_] = True
startsWithA _ =False
\end{minted}
	\item Matches 3-element lists and checks if the first entry is the character 'a'
\end{itemize}
\begin{important}[Where to put patterns]
Use patterns in the equations defining a function. Not in the type of a function\\
Pattern matches in the equations don't change the type of the function. They just say how it should act on particular expressions
\end{important}
\begin{itemize}
	\item How match 'a' and not care how long the list is?
	\item Can't use literal list syntax. Instead, use list constructor syntax for matching
\begin{minted}{haskell}
startsWithA :: [Char] -> Bool
startsWithA ('a':_) = True
startsWithA _ = False
\end{minted}
	\item \mintinline{haskell}{('a':_)} matches any list of length at least 1 whose first entry is 'a'
	\item The wildcard match \_ matches anything else
	\item This works with multiple entries too:
\begin{minted}{haskell}
startsWithAB :: [Char] -> Bool
startsWithAB ('a':'b':_) = True
startsWithAB _ = False
\end{minted}
\end{itemize}
\subsubsection{Binding variables in pattern matching}
\begin{itemize}
	\item As well as matching literal values, we can also match a (list) pattern, and bind the values
\begin{minted}{haskell}
sumTwo :: Num a => [a] -> a
sumTwo (x:y:_) = x + y
\end{minted}
	\item Match lists of length at least two and sum their first two entries
\begin{minted}{haskell}
sumTwo [1,2,3,4]
-- introduces the bindings
x = 1
y = 2
_ = [3,4]
\end{minted}
\item Reminder: can't repeat variable names in bindings (exception \_)
\begin{minted}{haskell}
-- Not allowed
sumThree (a:a:b:_) = a + a + b
-- What you'd want to do here would be to have inputs a b and c, but only define through if a==b
sumThree (a:b:c:_) | a==b = a+b+c
				   | otherwise = undefined
-- Allowed
second (_:a:_) = a
\end{minted}
\end{itemize}
\subsubsection{What types of pattern can I match on?}
\begin{itemize}
	\item Patterns are constructed in the same way that we would construct the arguments to the function
\begin{minted}{haskell}
(&&) :: Bool -> Bool -> Bool
True && True = True
False && _ = False
-- Used as
a && b
head :: [a] -> a
head (x:_) = x
-- Used as:
head [1,2,3] == head(1:[2,3])
\end{minted}
	\item This is a general rule in constructing pattern matches "If I were to call the function, what structure do I want to match?"
	\item Caveat: can only match "data constructors"
\begin{minted}{haskell}
-- Not allowed
last :: [a] -> a
last(xs ++ [x]) = x
\end{minted}
\end{itemize}
\subsection{Comprehensions}
\subsubsection{Syntax}
\begin{itemize}
	\item In maths, we often use comprehensions to construct new sets from old ones
\[
\{2,4\}=\{x | x \in\{1.5\} \wedge(x \bmod 2=0)\}
\]
"The set if all integers x between 1 and 5 such that x is even
	\item Haskell supports similar notation for constructing lists
\begin{minted}{haskell}
[x| x <- [1..5], x `mod` 2 ==0]
\end{minted}
"The list of all integers x where x is drawn from [1..5] and x is even"
	\item \mintinline{haskell}{x <- [1..5]} is called a generator
	\item Compare python comprehensions
\begin{minted}{python}
[x for x in range(1,6) if (x%2)==0]
\end{minted}
\end{itemize}
\subsubsection{Generators}
\begin{itemize}
	\item Comprehensions can contain multiple generators, separated by commas
\begin{minted}{haskell}
[(x,y) | x <- [1,2,3], y <- [4,5]]
\end{minted}
	\item Variables in the later generator can change faster, analogous to nested loops
\begin{minted}{python}
l = []
for x in [1,2,3]
	for y in [4,5]
		l.append((x,y))
## or
[(x,y) for x in [1,2,3] for y in [4,5]]
\end{minted}
	\item Later generators can reference variables from earlier generators
\begin{minted}{haskell}
[(x,y) | x <- [1..3], y <- [x..3]]
\end{minted}
"All pairs (x,y) such that $x,y\in \{1,2,3\}$ and $y\geqslant x$
\end{itemize}
\subsubsection{Guards}
\begin{itemize}
	\item As well as binding variables to guards with generators, we can restrict the values using guards
	\item A guard can be any function that returns a Bool
	\item Cards and generators can be freely interspersed, but guards can only refer to variables to their left
\end{itemize}
\begin{minted}{haskell}
[(x,y) | x<- [1..3], even x,y <- [x..3]]
-- [](2,3), 2,3]
[(x,y) | x <- [1..3], y <- [x..3], even x, even y]
-- [(2,2)]
\end{minted}
\subsubsection{Pattern matching in generators}
\begin{itemize}
	\item The left hand side of a generator expression need not be a single variable, but allows pattern matching
	\item We'll illustrate this with the use of the library function zip
\begin{minted}{haskell}
zip :: [a] -> [b] -> [(a,b)]
zip [] _ = []
zip _ [] = []
zip  (x:xs) (y:ys) = (x,y) : zip xs ys
\end{minted}
\end{itemize}
\end{document}