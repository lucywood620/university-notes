\documentclass{article}[18pt]
\usepackage{../../../../format}
\lhead{Programming Paradigms - Functional Programming}
\usepackage{minted}
\setminted{tabsize=4}
\begin{document}
\begin{center}
\underline{\huge Algebraic data types and type classes}
\end{center}
\section{Adding new data types}
\subsection{Defining data types}
\begin{itemize}
	\item So far we've only used the builtin data types
	\item Of course, it often makes sense to define new data types
	\item Multiple reasons to do this
	\begin{itemize}
		\item Hide complexity
		\item Build new abstractions
		\item Type safety
	\end{itemize}
	\item Haskell has three ways to do this
	\begin{itemize}
		\item \mintinline{haskell}{type}
		\item \mintinline{haskell}{data}
		\item \mintinline{haskell}{newtype} (won't cover this)
	\end{itemize}
\end{itemize}
\subsection{Type declarations}
A new name for an existing type can be defined using a type declaration
\begin{minted}{haskell}
type String = [Char]
\end{minted}
We can use these type declarations to make the semantics of our code cleaner
\begin{minted}{haskell}
type Pos  = (Int, Int)
origin :: Pos
origin :: (0,0)
left :: Pos -> Pos
left (i,j) = (i-1,j)
\end{minted}
Just like function definitions, type declarations can be parametrised over type variables
\begin{minted}{haskell}
type Pair a = (a,a)
mult :: Pair Int -> Int
mult (m, n) = m*n

dup :: a -> Pair a
dup x  = (x,x)
\end{minted}
You can't have class constraints in the definition and can't use recursive types
\begin{minted}{haskell}
-- Not allowed
type Tree = (Int, [Tree])
\end{minted}
\subsection{Data Declarations: New types}
We can introduce a completely new type by specifying allowed values using a data declaration
\begin{minted}{haskell}
data Bool = False | True
\end{minted}
Both the type name, and the constructor names, must begin with an upper-case letter
\subsection{Using new types}
\begin{minted}{haskell}
data IsTrue = Yes | No | Perhaps
negate :: IsTrue -> Is True
negate Yes     = No
negate No      = Yes
negate Perhaps = Perhaps
\end{minted}
\subsection{Data declarations with fixed type parameters}
The constructors in a data declaration can take arbitrarily many parameters
\begin{minted}{haskell}
data Shape = Circle Float | Rectangle Float Float
\end{minted}
"A shape is either a Circle, or a Rectangle. The circle is defined by one number, the rectangle by two"
\begin{minted}{haskell}
area :: Shape -> Float
area (Circle r) = pi * r^2
area (Rectangle x y) = x * y
\end{minted}
\subsection{Recursive types}
Data declarations can refer to themselves
\begin{minted}{haskell}
data Nat = Zero | Succ Nat
\end{minted}
"Nat is a new type with constructors \mintinline{haskell}{Zero :: Nat} and \mintinline{haskell}{Succ :: Nat -> Nat}"\\
\\
This type has an infinite set of values\\
\\
We could use this to implement a representation of the natural numbers, and arithmetic\\
\\
This allows for very succinct definitions of data structures
\begin{minted}{haskell}
data List a = Empty | Const a (List a)
intList = Cons 1 (Cons 2 ( Cons 3 Empty))
== [1,2,3]
\end{minted}
\subsection{Some type theory and contrasts}
Haskell's data declarations make Algebraic data types\\
This is a type where we specify the "shape" of each element\\
\\
The two algebraic operations are "sum" and "product"
\begin{defin}[Sum type]
An alternation
\begin{minted}{haskell}
data Foo = A|B
\end{minted}
The value of type Foo can either be A or B, example of this is \mintinline{haskell}{Bool}
\end{defin}

\begin{defin}[Product type]
A combination
	\begin{minted}{haskell}
data Pair = P Int Double
	\end{minted}
A pair of numbers, an Int and Double together
\end{defin}
\subsection{Haskell Types: Pros and Cons}
\begin{minipage}{0.5\textwidth}
\begin{center}
\underline{Classes}
\end{center}
\begin{itemize}
	\item [$\checkmark$] Easy to add new "kinds of things": just make a subclass
	\item [x] Hard to add new "operation on existing things": need to change superclass to add new method and potentially update all subclasses
\end{itemize}
\end{minipage}
\begin{minipage}{0.5\textwidth}
\begin{center}
	\underline{Algebraic Data Types}
\end{center}
	\begin{itemize}
		\item [x] Hard to add new "kinds of things": need to add new constructor and update all functions that use the data type
		\item [$\checkmark$] Easy to add new "operation on existing things": just write a new function
	\end{itemize}
\end{minipage}
% Got up to here
\section{Higher order functions and type classes again}
\subsection{Separating code and data}
\begin{itemize}
	\item When designing software, a good aim is to hide the implementation of data structures
	\item In OO based languages we do this with classes and inheritence
	\item Or with interfaces, which define a contract that a class must implement
	\item Idea is that calling code doesn't know internals and only relies on interface
	\item As a result, we can change the implementation, and the client code still works
\end{itemize}
\subsection{Generic higher order functions}
\begin{itemize}
	\item In Haskell we can realise this idea with generic higher order functions, and type classes
	\item Last time, we saw some examples of higher order functions for list
	\item For example, imagine we want to add two lists pairwise
\begin{minted}{haskell}
-- By hand
addLists _[] = []
addLists [] _ = []
addLists (x:xs) (y:ys) = (x+y) : addLists xs ys
-- Better
addLists xs ys = map (uncurry (+)) $ zip xs ys
-- Best
addLists = zipWith (+)
\end{minted}
	\item If we write our own data types, are we reduced to doing everything "by hand" again
\end{itemize}
\subsection{Using type classes}
\begin{itemize}
	\item Recall, Haskell has a concept of type classes
	\item These describe interfaces that can be used to constrain the polymorphism of functions to those types satisfying the interface
	\item (+) acts on any type, as long as that type implements the Num interface
\begin{minted}{haskell}
(+) :: Num a => a -> a -> a
\end{minted}
	\item $(<)$ acts on any type, as long as that type implements the Ord interface
\begin{minted}{haskell}
(<) :: Ord a => a -> a -> Bool
\end{minted}
	\item Haskell comes with many such type classes encapsulating common patterns
	\item When we implement out own data types, we can "just" implement appropriate instances of these classes
\end{itemize}
\end{document}