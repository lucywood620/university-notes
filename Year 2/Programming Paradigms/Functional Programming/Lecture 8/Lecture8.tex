\documentclass{article}[18pt]
\usepackage{../../../../format}
\lhead{Programming Paradigms - Functional Programming}
\usepackage{minted}[tabsize=4]

\begin{document}
\begin{center}
\underline{\huge IO and Interaction}
\end{center}
\section{Batch programs}
\begin{itemize}
	\item So far, we've only written batch programs
	\item That is, programs that take all their inputs at the start and provide output at the end
	\item To change what we compute, need to change source code a rerun
\end{itemize}
\section{Interactive programs}
\begin{itemize}
	\item What if we want to use haskell to write interactive programs?
	\item These read from the keyboard and write to the screen as they are running
\end{itemize}
\section{A problem}
\begin{itemize}
	\item Haskell problems are pure mathematical functions
	\item [$\Rightarrow$] Haskell programs therefore have no side effects
\end{itemize}
\begin{defin}[Side effect]
	Modifying some (internal/hidden) state as well as returning a value
\end{defin}
\begin{itemize}
	\item Reading from the keyboard and writing to the screen are side effects
	\item [$\Rightarrow$] Interactive programs have side effects
\end{itemize}
\section{Conceptual idea}
\begin{itemize}
	\item We can think of an interactive program as a pure function of type \mintinline{haskell}{World -> World}
	\item That is, it takes the current state of the world as input and produces a modified world as output
	\item New World object reflects any side effects that were performed 
\end{itemize}
\textbf{IO actions}
\begin{minted}{haskell}
type IO a  = World -> (a,World)
\end{minted}
Input/output eats the world and produces a result of type a, along with a new world
\section{Actions}
\begin{itemize}
	\item Copying the world is too expensive in practice
	\item Introduce new types to distinguish pure expressions from impure actions
	\item Use the concept, but Haskell uses a primitive type: implementation details are hidden
	\item These actions may have side effects
	\item Now we can write interactive programs in Haskell and "hide" the side effects behind type
\end{itemize}
\subsection{Basic actions}
\subsubsection{Reading}
\begin{minted}{haskell}
getChar :: IO Char
getChar = ...
\end{minted}
Read a character from the keyboard, echo it to the screen and return it
\subsubsection{Writing}
\begin{minted}{haskell}
putChar :: Char -> IO ()
putChar c = ...
\end{minted}
Write a character to the screen and return noting (indicated by the empty tuple)
\subsection{Bridging from expressions into actions}
\begin{itemize}
	\item For type safety, we need a way of "wrapping" values into actions
	\item Allows us to bring side-effect-free expressions into the "action" world
\end{itemize}
\begin{minted}{haskell}
return :: a -> IO a
return v = ...
\end{minted}
"Lift" a pure expression into an impure action\\
\\
Note: no way of turning an action back again
\begin{important}[Return]
The name return is rather misleading when coming from imperative languages. Calling return does not affect control flow
\end{important}
\subsection{Sequencing actions}
We can combine a sequence of IO actions using do notation
\begin{minted}{haskell}
do v1 <- a1
   v2 <- a2
   ...
   vn <- an
   return (f v1 v2 ... vn)
\end{minted}
Binds results of actions to values then applies f to the values and lifts into "action-land" with return\\
\\
Similarity with list comprehensions
\begin{itemize}
	\item Each expression is called a generator
	\item If we want to execute an action, but don't care about the result, we can use \mintinline{haskell}{_ <- ai} or just ai
\end{itemize}
\subsection{Example: reading characters}
\begin{minted}{haskell}
act :: IO(Char, Char)
act = do x<- getChar
         getChar
         y <- getChar
         return (x,y)
\end{minted}
\begin{itemize}
	\item Read three characters, discard the second, and return the first and third
	\item Note the use of return, without it we would get a type error
\end{itemize}
\subsection{When is an action performed}
\begin{itemize}
	\item Actions never require arguments \mintinline{haskell}{act :: IO} is not a function
	\item Just specify that something will be done
	\item Must run to execute
	\item GHCi knows to run actions at the prompt
	\item Conversely when writing a program to be compiled, GHC only ends up running the main action
\end{itemize}
\section{Pure vs impure}
\begin{minipage}{0.5\textwidth}
\underline{Pure}
\begin{itemize}
	\item Always produces the same result when applied to the same arguments
	\item Never has side effects
	\item Never alters state
\end{itemize}
\end{minipage}
\begin{minipage}{0.5\textwidth}
\underline{Impure}
\begin{itemize}
	\item May produce different results when applied to the same arguments
	\item May have side effects
	\item May alter state
\end{itemize}
\end{minipage}
\begin{defin}[Referential transparency]
	Replacing an expression by its value does not change the behaviour of the program
\end{defin}
\section{Actions as promises}
\begin{itemize}
	\item To fix the issue of referential transparency, \mintinline{haskell}{IO} is introduced
	\item We can think then of a type \mintinline{haskell}{IO Char} as a placeholder for a char that will only materialise once the program executes
	\item Moreover, it encapsulates a promise that this \mintinline{haskell}{Char} will actually appear
	\item Manipulating an \mintinline{haskell}{IO Char} is equivalent to setting up "plans" to be executed when the \mintinline{haskell}{Char} materialises
	\item This way, we maintain type safety "inside" the action
\end{itemize}
\end{document}