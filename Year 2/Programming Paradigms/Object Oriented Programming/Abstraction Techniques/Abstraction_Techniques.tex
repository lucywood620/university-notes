\documentclass{article}[18pt]
\input{../../../../format}
\lhead{Programming Paradigms - OO Programming}
\usepackage{minted}


\begin{document}
\begin{center}
\underline{\huge Abstraction Techniques}
\end{center}
\section{Abstract classes and methods}
\begin{itemize}
	\item Abstract methods have abstract in the signature
	\item Abstract methods have no body
	\item Abstract methods make the class abstract
	\item Abstract classes can't be instantiated
	\item Concrete subclasses complete the implementation
\end{itemize}
\section{Multiple inheritance}
\begin{itemize}
	\item Having a class inherit directly from multiple ancestors
	\item Each language has its own rules
	\item Java forbids it for classes
	\item Java permits it for interfaces - no competing implementation
\end{itemize}
\section{Interfaces as types}
\begin{itemize}
	\item Implementing classes don't inherit code
	\item However implementing classes are subtypes of the interface type
	\item So, polymorphism is available with interfaces as well as classes
\end{itemize}
\section{Features of interfaces}
\begin{itemize}
	\item All methods are abstract
	\item There are no constructors
	\item All methods are public
	\item All fields are public, static and final
\end{itemize}
\section{Interfaces as specifications}
\begin{itemize}
	\item Strong separation of functionality form implementation - though parameter and return types are mandated
	\item Clients interact independently of the implementation - but clients can choose from alternative implementations
\end{itemize}
\section{Example of an interface}
\begin{minted}{java}
public interface Actor
{
	/**
	* Perform the actor's regular behavior.
	* @param newActors A list for storing newly created
	*                  actors.
	*/
	void act(List<Actor> newActors);
	
	/**
	* Is the actor still active?
	* @return true if still active, false if not. 
	*/ 
	boolean isActive(); 
}
\end{minted}

\end{document}
