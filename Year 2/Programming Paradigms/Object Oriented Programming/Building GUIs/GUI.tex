\documentclass{article}[18pt]
\ProvidesPackage{format}
%Page setup
\usepackage[utf8]{inputenc}
\usepackage[margin=0.7in]{geometry}
\usepackage{parselines} 
\usepackage[english]{babel}
\usepackage{fancyhdr}
\usepackage{titlesec}
\hyphenpenalty=10000

\pagestyle{fancy}
\fancyhf{}
\rhead{Sam Robbins}
\rfoot{Page \thepage}

%Characters
\usepackage{amsmath}
\usepackage{amssymb}
\usepackage{gensymb}
\newcommand{\R}{\mathbb{R}}

%Diagrams
\usepackage{pgfplots}
\usepackage{graphicx}
\usepackage{tabularx}
\usepackage{relsize}
\pgfplotsset{width=10cm,compat=1.9}
\usepackage{float}

%Length Setting
\titlespacing\section{0pt}{14pt plus 4pt minus 2pt}{0pt plus 2pt minus 2pt}
\newlength\tindent
\setlength{\tindent}{\parindent}
\setlength{\parindent}{0pt}
\renewcommand{\indent}{\hspace*{\tindent}}

%Programming Font
\usepackage{courier}
\usepackage{listings}
\usepackage{pxfonts}

%Lists
\usepackage{enumerate}
\usepackage{enumitem}

% Networks Macro
\usepackage{tikz}


% Commands for files converted using pandoc
\providecommand{\tightlist}{%
	\setlength{\itemsep}{0pt}\setlength{\parskip}{0pt}}
\usepackage{hyperref}

% Get nice commands for floor and ceil
\usepackage{mathtools}
\DeclarePairedDelimiter{\ceil}{\lceil}{\rceil}
\DeclarePairedDelimiter{\floor}{\lfloor}{\rfloor}

% Allow itemize to go up to 20 levels deep (just change the number if you need more you madman)
\usepackage{enumitem}
\setlistdepth{20}
\renewlist{itemize}{itemize}{20}

% initially, use dots for all levels
\setlist[itemize]{label=$\cdot$}

% customize the first 3 levels
\setlist[itemize,1]{label=\textbullet}
\setlist[itemize,2]{label=--}
\setlist[itemize,3]{label=*}

% Definition and Important Stuff
% Important stuff
\usepackage[framemethod=TikZ]{mdframed}

\newcounter{theo}[section]\setcounter{theo}{0}
\renewcommand{\thetheo}{\arabic{section}.\arabic{theo}}
\newenvironment{important}[1][]{%
	\refstepcounter{theo}%
	\ifstrempty{#1}%
	{\mdfsetup{%
			frametitle={%
				\tikz[baseline=(current bounding box.east),outer sep=0pt]
				\node[anchor=east,rectangle,fill=red!50]
				{\strut Important};}}
	}%
	{\mdfsetup{%
			frametitle={%
				\tikz[baseline=(current bounding box.east),outer sep=0pt]
				\node[anchor=east,rectangle,fill=red!50]
				{\strut Important:~#1};}}%
	}%
	\mdfsetup{innertopmargin=10pt,linecolor=red!50,%
		linewidth=2pt,topline=true,%
		frametitleaboveskip=\dimexpr-\ht\strutbox\relax
	}
	\begin{mdframed}[]\relax%
		\centering
		}{\end{mdframed}}



\newcounter{lem}[section]\setcounter{lem}{0}
\renewcommand{\thelem}{\arabic{section}.\arabic{lem}}
\newenvironment{defin}[1][]{%
	\refstepcounter{lem}%
	\ifstrempty{#1}%
	{\mdfsetup{%
			frametitle={%
				\tikz[baseline=(current bounding box.east),outer sep=0pt]
				\node[anchor=east,rectangle,fill=blue!20]
				{\strut Definition};}}
	}%
	{\mdfsetup{%
			frametitle={%
				\tikz[baseline=(current bounding box.east),outer sep=0pt]
				\node[anchor=east,rectangle,fill=blue!20]
				{\strut Definition:~#1};}}%
	}%
	\mdfsetup{innertopmargin=10pt,linecolor=blue!20,%
		linewidth=2pt,topline=true,%
		frametitleaboveskip=\dimexpr-\ht\strutbox\relax
	}
	\begin{mdframed}[]\relax%
		\centering
		}{\end{mdframed}}
\lhead{Programming Paradigms - OO Programming}
\usepackage{minted}
\setminted[java]{tabsize=4}
\begin{document}
\begin{center}
\underline{\huge Building GUIs}
\end{center}
\begin{definition}[Stage]
The window, along with decorations like a menu bar and window controls
\end{definition}
\begin{definition}[Scene]
The area inside the window in which to put the content
\end{definition}
\section{Creating a stage}
\begin{minted}{java}
public void initialiseGUI1(){
	Stage stage = new Stage();
	stage.setTitle("Hello World");
	stage.show();
}
\end{minted}
\section{Launching the GUI}
\begin{itemize}
	\item Usually just extending the application class, but it is slightly different in Bluej
\end{itemize}
\begin{minted}{java}
public void launchFX(){
	new JFXPanel();
	Platform.setImplicitExit(false);
	Platform.runLater(() -> initialiseGUI1());
}
\end{minted}
\subsection{Lambda expressions}
\begin{itemize}
	\item They were added to Java 8 for defining anonymous methods
	\item e.g
\end{itemize}
\begin{minted}{java}
() -> initialiseGUI1()
\end{minted}
\section{Adding content to the scene}
\begin{minted}{java}
public void initialiseGUI2(){
	Stage stage = new Stage();
	stage.setTitle("Hello World");
	Button btn = new Button();
	btn.setText("Say 'hello world'");
	StackPane root = new StackPane(); 
	root.getChildren().add(btn);
	stage.setScene(new Scene(root, 300, 250));
	stage.show();
}
\end{minted}
\section{Event Handling}
\begin{itemize}
	\item Events correspond to use interactions with components
	\item Clicking on a button causes an ActionEvent
	\item An object implements the EventHandler interface
	\begin{itemize}
		\item Defines a handle method
	\end{itemize}
	\item It registers as a handler with setOnAction
\end{itemize}
\section{Nested class syntax}
Class definitions may be nested
\begin{minted}{java}
public class enclosing
{
	private class Innner
	{
		...
	}
}
\end{minted}
\subsection{Inner classes}
\begin{itemize}
	\item Instances of the inner class are localised within the enclosing class
	\item Instances of the inner class have access to the private members of the enclosing class
\end{itemize}
\subsubsection{Anonymous inner classes}
\begin{itemize}
	\item Obey the rules of inner classes
	\item Used to create one-off objects for which a class name is not required
	\item Use a special syntax
	\item The instance is always referenced by its supertype, as it has no subtype name
\end{itemize}
\section{Anonymous event handler}
\begin{minted}{java}
btn.setOnAction(new EventHandler<ActionEvent>() {
	@Override                
	public void handle(ActionEvent event){
		System.out.println("Hello World");
	}            
});
\end{minted}
\begin{itemize}
	\item Creates object with new
	\item Overrides methods
	\item Can be used with interfaces (only way to use interfaces with new)
	\item @Override annotation is checked by compiler
\end{itemize}




\end{document}
