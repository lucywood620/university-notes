\documentclass{article}[18pt]
\ProvidesPackage{format}
%Page setup
\usepackage[utf8]{inputenc}
\usepackage[margin=0.7in]{geometry}
\usepackage{parselines} 
\usepackage[english]{babel}
\usepackage{fancyhdr}
\usepackage{titlesec}
\hyphenpenalty=10000

\pagestyle{fancy}
\fancyhf{}
\rhead{Sam Robbins}
\rfoot{Page \thepage}

%Characters
\usepackage{amsmath}
\usepackage{amssymb}
\usepackage{gensymb}
\newcommand{\R}{\mathbb{R}}

%Diagrams
\usepackage{pgfplots}
\usepackage{graphicx}
\usepackage{tabularx}
\usepackage{relsize}
\pgfplotsset{width=10cm,compat=1.9}
\usepackage{float}

%Length Setting
\titlespacing\section{0pt}{14pt plus 4pt minus 2pt}{0pt plus 2pt minus 2pt}
\newlength\tindent
\setlength{\tindent}{\parindent}
\setlength{\parindent}{0pt}
\renewcommand{\indent}{\hspace*{\tindent}}

%Programming Font
\usepackage{courier}
\usepackage{listings}
\usepackage{pxfonts}

%Lists
\usepackage{enumerate}
\usepackage{enumitem}

% Networks Macro
\usepackage{tikz}


% Commands for files converted using pandoc
\providecommand{\tightlist}{%
	\setlength{\itemsep}{0pt}\setlength{\parskip}{0pt}}
\usepackage{hyperref}

% Get nice commands for floor and ceil
\usepackage{mathtools}
\DeclarePairedDelimiter{\ceil}{\lceil}{\rceil}
\DeclarePairedDelimiter{\floor}{\lfloor}{\rfloor}

% Allow itemize to go up to 20 levels deep (just change the number if you need more you madman)
\usepackage{enumitem}
\setlistdepth{20}
\renewlist{itemize}{itemize}{20}

% initially, use dots for all levels
\setlist[itemize]{label=$\cdot$}

% customize the first 3 levels
\setlist[itemize,1]{label=\textbullet}
\setlist[itemize,2]{label=--}
\setlist[itemize,3]{label=*}

% Definition and Important Stuff
% Important stuff
\usepackage[framemethod=TikZ]{mdframed}

\newcounter{theo}[section]\setcounter{theo}{0}
\renewcommand{\thetheo}{\arabic{section}.\arabic{theo}}
\newenvironment{important}[1][]{%
	\refstepcounter{theo}%
	\ifstrempty{#1}%
	{\mdfsetup{%
			frametitle={%
				\tikz[baseline=(current bounding box.east),outer sep=0pt]
				\node[anchor=east,rectangle,fill=red!50]
				{\strut Important};}}
	}%
	{\mdfsetup{%
			frametitle={%
				\tikz[baseline=(current bounding box.east),outer sep=0pt]
				\node[anchor=east,rectangle,fill=red!50]
				{\strut Important:~#1};}}%
	}%
	\mdfsetup{innertopmargin=10pt,linecolor=red!50,%
		linewidth=2pt,topline=true,%
		frametitleaboveskip=\dimexpr-\ht\strutbox\relax
	}
	\begin{mdframed}[]\relax%
		\centering
		}{\end{mdframed}}



\newcounter{lem}[section]\setcounter{lem}{0}
\renewcommand{\thelem}{\arabic{section}.\arabic{lem}}
\newenvironment{defin}[1][]{%
	\refstepcounter{lem}%
	\ifstrempty{#1}%
	{\mdfsetup{%
			frametitle={%
				\tikz[baseline=(current bounding box.east),outer sep=0pt]
				\node[anchor=east,rectangle,fill=blue!20]
				{\strut Definition};}}
	}%
	{\mdfsetup{%
			frametitle={%
				\tikz[baseline=(current bounding box.east),outer sep=0pt]
				\node[anchor=east,rectangle,fill=blue!20]
				{\strut Definition:~#1};}}%
	}%
	\mdfsetup{innertopmargin=10pt,linecolor=blue!20,%
		linewidth=2pt,topline=true,%
		frametitleaboveskip=\dimexpr-\ht\strutbox\relax
	}
	\begin{mdframed}[]\relax%
		\centering
		}{\end{mdframed}}
\lhead{Programming Paradigms - Object Oriented Programming}
\usepackage{minted}

\begin{document}
\begin{center}
\underline{\huge Grouping Objects}
\end{center}
\section{Class Libraries}
\begin{itemize}
	\item These are collections of useful classes
	\item We don't have to write everything from scratch
	\item Java calls its libraries packages
	\item Grouping objects is a recurring requirement
\end{itemize}
\section{Collections}
We specify
\begin{itemize}
	\item The type of collection: ArrayList
	\item The type of objects it will contain \mintinline{java}{<String>}
\end{itemize}
\subsection{Features}
\begin{itemize}
	\item Increases capacity as necessary
	\item Keeps a private count (size() accessor)
	\item Keeps the objects in order
	\item Details of how all this is done are hidden
\end{itemize}
\subsection{Using a collection}
\begin{minted}{java}
public class Notebook
{
	private ArrayList<String> notes;
	...
	
	public void storeNote(String note)
	{
		notes.add(note);
	}
	
	public int numberOfNotes()
	{
		return notes.size();
	}
}
\end{minted}
\subsection{Retrieving an object}
\begin{minted}{java}
public void showNote(int noteNumber)
{
	if(noteNumber < 0) {
		// This is not a valid note number.
	}
	else if(noteNumber < numberOfNotes()) {
		System.out.println(notes.get(noteNumber));
	}
	else {
		// This is not a valid note number.
	}
}
\end{minted}
\subsection{Generic Classes}
\begin{itemize}
	\item Collections are known as parameterized or generic types
	\item ArrayList implements list functionality
	\item The type of parameter says what we want a list of
\end{itemize}
\section{For-each loop}
\begin{minted}{java}
/**
* List all notes in the notebook.
*/
public void listNotes()
{
	for(String note : notes) {
		System.out.println(note);
	}
} 
\end{minted}
\section{While loop}
\begin{minted}{java}
/**
* List all notes in the notebook.
*/
public void listNotes()
{
	int index = 0;
	while(index < notes.size()) {
		System.out.println(notes.get(index));
		index++;
	}
} 
\end{minted}
\section{Equality vs Identity}
\begin{minted}{java}
// tests identity
if(input == "bye") {		
	...
}
// tests equality
if(input.equals("bye")) {
	...
}
\end{minted}
Identity tests if a variable holds the same instance as another variable\\
Equality tests if two distinct objects can be used interchangeably
\section{Iterators}
\begin{minted}{java}
Iterator<ElementType> it = myCollection.iterator();
while(it.hasNext()) {
	call it.next() //to get the next object
	//do something with that object
} 
\end{minted}
\newpage
\subsection{Index vs Iterator}
For each loop
\begin{itemize}
	\item Use if we want to process every element
\end{itemize}
While loop
\begin{itemize}
	\item Use if we might want to stop part way through
	\item Use for repetition that doesn't involve a collection
\end{itemize}
Iterator object
\begin{itemize}
	\item Use if we might want to stop part way through
	\item Often used with collections where indexed access if not very efficient, or impossible
\end{itemize}
\section{Arrays}
\begin{minted}{java}
public class LogAnalyzer
{
	private int[] hourCounts; // array variable declaration
	private LogfileReader reader;
	
	public LogAnalyzer()
	{ 
		hourCounts = new int[24]; // array object creation
		reader = new LogfileReader();
	}

}
\end{minted}
\subsection{Array length}
To get the length of an array use
\begin{minted}{java}
private int[] numbers = { 3, 15, 4, 5 };
int n = numbers.length;
\end{minted}
Note here that length is not a method
\section{For loop}
\begin{itemize}
	\item The for loop is often used to iterate a fixed number of times
	\item Often used with a variable that changes a fixed amount on each iteration
\end{itemize}
\begin{minted}{java}
for(int hour = 0; hour < hourCounts.length; hour++) {
	System.out.println(hour + ": " + hourCounts[hour]);
}
\end{minted}











\end{document}