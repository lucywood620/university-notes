\documentclass{article}[18pt]
\input{../../../../format}
\lhead{Programming Paradigms - Object Oriented Programming}
\usepackage{minted}
\setminted[java]{tabsize=4}

\begin{document}
\begin{center}
\underline{\huge Improving Structure with inheritance}
\end{center}
\begin{itemize}
	\item Define one superclass
	\item Define subclasses
	\item The superclass defines common attributes
	\item The subclasses inherit the superclass attributes
	\item The subclasses add their own attributes
\end{itemize}
\begin{minted}{java}
// Superclass
public class Item
{
	private String title;
	private int playingTime;
	private boolean gotIt;
	private String comment;
	
	// constructors and methods omitted.
}
// Subclass
public class CD extends Item
{
	private String artist;
	private int numberOfTracks;
	
	// constructors and methods omitted.
}
\end{minted}
\section{Superclass constructor call}
\begin{itemize}
	\item Subclass constructors must always contain a "super" call
	\item If none is written, the compiler inserts one (without parameters)
	\item Must be the first statement in the subclass constructor
\end{itemize}
\begin{minted}{java}
public class CD extends Item
{
	private String artist;
	private int numberOfTracks;
	
	/**
	* Constructor for objects of class CD
	*/
	public CD(String theTitle, String theArtist, int tracks, int time)
	{
		super(theTitle, time);
		artist = theArtist;
		numberOfTracks = tracks;
	}
	
	// methods omitted
}
\end{minted}
\section{Subclasses and subtyping}
\begin{itemize}
	\item Classes define types
	\item Subclasses define subtypes
	\item Objects of subclasses can be used where objects of supertypes are used (this is called substitution)
\end{itemize}
\section{Polymorphic variables}
\begin{itemize}
	\item Object variables in Java are polymorphic (they can hold objects of more than one type)
	\item They can hold objects of the declared type, or of subtypes of the declared type
\end{itemize}
\section{Casting}
\begin{itemize}
	\item Can assign subtype to supertype
	\item Can't assign supertype to subtype
\end{itemize}
For example, the last line of this causes a compile time error
\begin{minted}{java}
Vehicle v;
Car c = new Car();
v=c; // this is fine
c=v; // this causes a compile time error
\end{minted}
This can be fixed with casting like so
\begin{minted}{java}
c= (Car) v;
\end{minted}
\begin{itemize}
	\item An object type in parentheses
	\item Used to overcome "type loss"
	\item The object is not changed in any way
	\item A runtime check is made to ensure the object really is of that type
\end{itemize}
\section{Polymorphic collections}
\begin{itemize}
	\item All collections are polymorphic
	\item The elements are of type object
\end{itemize}
\section{Collections and primitive types}
All objects can be entered into collections because collections accept elements of type Object and all classes are subtypes of Object
\section{Wrapper classes}
\begin{itemize}
	\item Primitive types (int, char, etc) are not objects. They must be wrapped into an object
\end{itemize}
\begin{tabular}{|c|c|}
	\hline
	Simple Type & Wrapper Class \\
	\hline
	int & Integer \\
	\hline
	float & Float \\
	\hline
	char & Character \\
	\hline
\end{tabular}
\begin{minted}{java}
int i=18;
Integer iwrap = new Integer(i); // wrapping the value
itn value = iwrap.intValue(); //unwrap it
\end{minted}
\section{Static type and dynamic type}
\begin{itemize}
	\item The declared type of a variable is its static type
	\item The type of the object a variable refers to is its dynamic type
	\item The compiler's job is to check for static-type violations
\end{itemize}
\section{Overriding}
\begin{itemize}
	\item Superclass and subclass define methods with the same signature
	\item Each has access to the fields of its class
	\item Superclass satisfies static type check
	\item Subclass method is called at runtime - it overrides the superclass version
\end{itemize}
\section{Method lookup}
\begin{itemize}
	\item The variable is accessed
	\item The object stored in the variable is found
	\item The class of the object is found
	\item The class is searched for a method match
	\item If no match is found, the superclass is searched
	\item This is repeated until a match is found, or the class hierarchy is exhausted
	\item Overriding methods take precedence
\end{itemize}
\section{Super call in methods}
\begin{itemize}
	\item Overriden methods are hidden, but we often still want to be able to call them
	\item An overriden method can be called from the method that overrides it
	\begin{itemize}
		\item \mintinline{java}{super.method(...)}
	\end{itemize}
\end{itemize}
\section{Method polymorphism}
\begin{itemize}
	\item Method calls are polymorphic - the actual method called depends on the dynamic object type
\end{itemize}
\section{The Object class' methods}
\begin{itemize}
	\item Methods in the objects are inherited by all classes
	\item Any of those may be overriden
	\item The \mintinline{java}{toString} method is commonly overriden:
	\begin{itemize}
		\item \mintinline{java}{public String toString()}
		\item Returns a string representation of the object
		\item The programmer might want a custom representation of the data as a string
		\item Note that calls to println with just an object automatically result in toString being called
	\end{itemize}
	\item Equals can also be overridden if the manner in which two things equal each other is not the traditional definition
\end{itemize}
\section{Protected access}
\begin{itemize}
	\item Private access in the superclass may be too restrictive for a subclass
	\item The closer inheritance relationship is supported by protected access
	\item Protected access is more restricted than public access
	\item It is still recommended to keep fields provided and to define protected accessors and mutators 
\end{itemize}
\begin{center}
	\includegraphics[scale=0.6]{"protected access"}
\end{center}




\end{document}
