\documentclass{article}[18pt]
\ProvidesPackage{format}
%Page setup
\usepackage[utf8]{inputenc}
\usepackage[margin=0.7in]{geometry}
\usepackage{parselines} 
\usepackage[english]{babel}
\usepackage{fancyhdr}
\usepackage{titlesec}
\hyphenpenalty=10000

\pagestyle{fancy}
\fancyhf{}
\rhead{Sam Robbins}
\rfoot{Page \thepage}

%Characters
\usepackage{amsmath}
\usepackage{amssymb}
\usepackage{gensymb}
\newcommand{\R}{\mathbb{R}}

%Diagrams
\usepackage{pgfplots}
\usepackage{graphicx}
\usepackage{tabularx}
\usepackage{relsize}
\pgfplotsset{width=10cm,compat=1.9}
\usepackage{float}

%Length Setting
\titlespacing\section{0pt}{14pt plus 4pt minus 2pt}{0pt plus 2pt minus 2pt}
\newlength\tindent
\setlength{\tindent}{\parindent}
\setlength{\parindent}{0pt}
\renewcommand{\indent}{\hspace*{\tindent}}

%Programming Font
\usepackage{courier}
\usepackage{listings}
\usepackage{pxfonts}

%Lists
\usepackage{enumerate}
\usepackage{enumitem}

% Networks Macro
\usepackage{tikz}


% Commands for files converted using pandoc
\providecommand{\tightlist}{%
	\setlength{\itemsep}{0pt}\setlength{\parskip}{0pt}}
\usepackage{hyperref}

% Get nice commands for floor and ceil
\usepackage{mathtools}
\DeclarePairedDelimiter{\ceil}{\lceil}{\rceil}
\DeclarePairedDelimiter{\floor}{\lfloor}{\rfloor}

% Allow itemize to go up to 20 levels deep (just change the number if you need more you madman)
\usepackage{enumitem}
\setlistdepth{20}
\renewlist{itemize}{itemize}{20}

% initially, use dots for all levels
\setlist[itemize]{label=$\cdot$}

% customize the first 3 levels
\setlist[itemize,1]{label=\textbullet}
\setlist[itemize,2]{label=--}
\setlist[itemize,3]{label=*}

% Definition and Important Stuff
% Important stuff
\usepackage[framemethod=TikZ]{mdframed}

\newcounter{theo}[section]\setcounter{theo}{0}
\renewcommand{\thetheo}{\arabic{section}.\arabic{theo}}
\newenvironment{important}[1][]{%
	\refstepcounter{theo}%
	\ifstrempty{#1}%
	{\mdfsetup{%
			frametitle={%
				\tikz[baseline=(current bounding box.east),outer sep=0pt]
				\node[anchor=east,rectangle,fill=red!50]
				{\strut Important};}}
	}%
	{\mdfsetup{%
			frametitle={%
				\tikz[baseline=(current bounding box.east),outer sep=0pt]
				\node[anchor=east,rectangle,fill=red!50]
				{\strut Important:~#1};}}%
	}%
	\mdfsetup{innertopmargin=10pt,linecolor=red!50,%
		linewidth=2pt,topline=true,%
		frametitleaboveskip=\dimexpr-\ht\strutbox\relax
	}
	\begin{mdframed}[]\relax%
		\centering
		}{\end{mdframed}}



\newcounter{lem}[section]\setcounter{lem}{0}
\renewcommand{\thelem}{\arabic{section}.\arabic{lem}}
\newenvironment{defin}[1][]{%
	\refstepcounter{lem}%
	\ifstrempty{#1}%
	{\mdfsetup{%
			frametitle={%
				\tikz[baseline=(current bounding box.east),outer sep=0pt]
				\node[anchor=east,rectangle,fill=blue!20]
				{\strut Definition};}}
	}%
	{\mdfsetup{%
			frametitle={%
				\tikz[baseline=(current bounding box.east),outer sep=0pt]
				\node[anchor=east,rectangle,fill=blue!20]
				{\strut Definition:~#1};}}%
	}%
	\mdfsetup{innertopmargin=10pt,linecolor=blue!20,%
		linewidth=2pt,topline=true,%
		frametitleaboveskip=\dimexpr-\ht\strutbox\relax
	}
	\begin{mdframed}[]\relax%
		\centering
		}{\end{mdframed}}
\lhead{Programming Paradigms - Object Oriented Programming}
\usepackage{minted}

\begin{document}
\begin{center}
\underline{\huge Objects First with Java}
\end{center}
\section{Methods and Parameters}
\begin{itemize}
	\item Objects have operations which can be invoked (Java called them methods)
	\item Methods may have parameters to pass additional information needed to execute
	\item In Java methods are not "first class" (can't pass them around)
	\item Many instances can be created from a single class
	\item An object have attributed: values stored in fields
	\item The class defines what field an object has, but each object stores its own set of values (the state of the object)
\end{itemize}
\section{Basic class structure}
\begin{minted}{java}
public class ClassName
{
	Fields
	Constructors
	Methods
} 
\end{minted}
\section{Fields}
\begin{itemize}
	\item Fields store values for an object
	\item They are also known as instance variables
	\item Use the inspect option to view an object's fields
	\item Fields define the state of an object
\end{itemize}
\section{Constructors}
\begin{itemize}
	\item Constructors initialise an object
	\item They have the same name as their class
	\item They store initial values into the fields
	\item They often receive external parameter values for this
\end{itemize}
\begin{minted}{java}
public TicketMachine(int ticketCost)
{
	price = ticketCost;
	balance = 0;
	total = 0;
}
\end{minted}
\section{Accessor methods}
\begin{itemize}
	\item Methods implement the behaviour of objects
	\item Accessors provide information about an object
	\item Methods have a structure consisting of a header and a body
	\item The header defines the method's signature \mintinline{java}{public int getPrice()}
	\item The body encloses the method's statement
\end{itemize}
\begin{minted}{java}
return price;
\end{minted}
\section{Mutator methods}
\begin{itemize}
	\item Have a similar method structure: header and body
	\item Used to mutate (i.e. change) an object's state
	\item Achieved through changing the value of one or more fields
	\begin{itemize}
		\item Typically contain assignment statements
		\item Typically receive parameters
	\end{itemize}
\end{itemize}
\begin{minted}{java}
balance=balance+amount
\end{minted}
\section{Printing from methods}
\begin{minted}{java}
System.out.println("String")
\end{minted}
\section{If statements}
\begin{minted}{java}
if(perform some test) {
	//Do these statements if the test gave a true result
}
else {
	//Do these statements if the test gave a false result
}
\end{minted}
\section{Local variables}
Fields are one sort of variable
\begin{itemize}
	\item They store values through the life of the object
	\item They are accessible throughout the class
\end{itemize}
Methods can include shorter lived variables
\begin{itemize}
	\item They exist only as long as the method is being executed
	\item They are only accessible from within the method
\end{itemize}
\subsection{Scope and life time}
\begin{itemize}
	\item The scope of a local variable is the block it's declared in
	\item The lifetime of a local variable is the time of execution of the block it's declared in
\end{itemize}
\subsection{Local variable syntax}
\begin{minted}{java}
public int refundBalance()
{
	int amountToRefund; // note there is no visibility modifier here
	amountToRefund = balance;
	balance = 0;
	return amountToRefund;
}
\end{minted}

\end{document}