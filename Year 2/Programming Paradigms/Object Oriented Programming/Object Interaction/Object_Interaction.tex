\documentclass{article}[18pt]
\usepackage{../../../../format}
\lhead{Programming Paradigms - Object Oriented Programming}
\usepackage{minted}


\begin{document}
\begin{center}
\underline{\huge Object Interaction}
\end{center}
\begin{definition}[Abstraction]
The ability to ignore details of parts to focus attention on a higher level of a problem
\end{definition}

\begin{definition}[Modularisation]
The process of dividing a whole into well-defined parts, which can be built and examined separately, and which interact in well-defined ways
\end{definition}
A class can have within it other classes
\begin{minted}{java}
public class NumberDisplay
{
	private int limit;
	private int value;

}
\end{minted}

\begin{minted}{java}
public class ClockDisplay
{
private NumberDisplay hours;
private NumberDisplay minutes;
}
\end{minted}

The variables in NumberDisplay are primitive types\\
The variables in ClockDisplay are object types
\section{Objects creating objects}
Formal parameter:
\begin{minted}{java}
public NumberDisplay(int rollOverLimit);
\end{minted}
Actual Parameter:
\begin{minted}{java}
hours = new NumberDisplay(24);
\end{minted}
\section{Method calling}
\begin{minted}{java}
public void timeTick()
{
	minutes.increment();
	if(minutes.getValue() == 0) { 
		// it just rolled over!
		hours.increment();
	}
	updateDisplay();
}
\end{minted}
\newpage
\section{Internal Methods}
\begin{minted}{java}
/**
* Update the internal string that
* represents the display.
*/
private void updateDisplay()
{
	displayString = 
	hours.getDisplayValue() + ":" + 
	minutes.getDisplayValue();
}
\end{minted}
\section{Method calls}
Internal method calls
\begin{minted}{java}
updateDisplay;
private void updateDisplay()
\end{minted}
External method calls
\begin{minted}{java}
minutes.increment()
\end{minted}




\end{document}