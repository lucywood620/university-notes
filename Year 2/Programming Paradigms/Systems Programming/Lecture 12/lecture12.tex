\documentclass{article}
\usepackage{../../../../format}
\lhead{Programming Paradigms - Systems Programming}
\usepackage{enumerate}
\usepackage{boxedminipage}
\usepackage{xcolor}
\usepackage{tikz}
\usepackage{fancyvrb}
\usepackage{hyperref}
\definecolor{links}{HTML}{2A1B81}
\hypersetup{colorlinks,linkcolor=,urlcolor=blue}
\usetikzlibrary{shapes.misc}
\usetikzlibrary{shapes.geometric, arrows, positioning}
\usepackage{minted}



\begin{document}


\begin{center}
	\underline{\huge Large Programs and External Libraries (Continued)}
\end{center}



\section{Makefiles}
\begin{itemize}
\item When we have a number of files to compile together we need a rule-set to perform this
\item The \verb!make! command provides this
\item Requires a rule-file called the \verb!Makefile!
\item Declarative programming style set of rules for building the program
\item Format of each rule:
\begin{minted}{c}
target [target ...]: [component ...]
        [command 1]
        ...
        [command n]
\end{minted}
\item N.B. Tab character
\item \verb!target! - what you want to make
\item \verb!component! - something which needs to exist (might need another rule)
\end{itemize}



\section{Makefiles: Example}
\begin{itemize}
\item Files: \verb!main.c!, \verb!counter.h!, \verb!counter.c!, \verb!sales.h!, \verb!sales.c!

\begin{minted}{c}
all: counter.o sales.o main.c
        gcc -o program main.c counter.o sales.o

counter.o: counter.c counter.h
        gcc -c counter.c

sales.o: sales.c sales.h
        gcc -c sales.c

clean:
        rm -rf program counter.o sales.o
\end{minted}
\end{itemize}



\section{Makefiles: Macros}
\begin{itemize}
\item Macros can be used to store definitions
\begin{itemize}
\item \verb!AUTHOR = Konrad Dabrowski!
\end{itemize}
\item They can be generated from commands
\begin{itemize}
\item \verb!DATE = `date`!
\end{itemize}
\item And used in the \verb!Makefile!
\begin{minted}{c}
all:
        echo $(AUTHOR) compiled this on $(DATE)
\end{minted}
\item \verb!all:!
\item Running this gives:
\item \verb!echo Konrad Dabrowski compiled this on `date`!
\begin{itemize}
\item \verb!Konrad Dabrowski compiled this on Thu 16! \verb!Jan 10:54:36 GMT 2020!
\end{itemize}
\end{itemize}



\section{Makefiles: Pattern Rules}
\begin{itemize}
\item We can specify a pattern rule which matches multiple files
\item e.g. compile C files into object files:
\begin{minted}{c}
DEPS = counter.h sales.h
%.o: %.c $(DEPS)
        gcc -c $< -o $@
\end{minted}
\item This would change our original Makefile example to:
\begin{minted}{c}
all: counter.o sales.o main.c
        gcc -o program main.c counter.o sales.o

%.o: %.c $(DEPS)
        gcc -c $< -o $@

clean:
        rm -rf program counter.o sales.o
\end{minted}
\end{itemize}



\section{Makefiles: A few comments}
\begin{itemize}
\item Comments - lines starting with \verb!#!
\item Lazy evaluation
\item If a target exists and has a timestamp later than all of its components assume it is up to date and don't bother to re-process
\item Nothing to do with C
\item Although Makefiles are often used with C programs there is no intrinsic link - can use with any code/work
\item You can run any specific rule by invoking its target:
\item \verb!make sales.o!
\end{itemize}




\section{A: Singly-linked list}
\begin{itemize}
\item Implement a program that stores numbers in a linked list and then prints them out.
\item The nodes of your linked list should be an appropriate \verb!struct! type and should be dynamically allocated using \verb!malloc()!.
\item The program should let the user input numbers at runtime and should contain a function that adds nodes to the end of the list.
\item (Note that pressing \verb!Ctrl+D! makes \verb!scanf()! return \verb!EOF!.)
\end{itemize}



\section{B: More advanced linked list}
Add functions to:
\begin{itemize}
\item delete the last number in the list
\item add a number to the start of the list
\item search for a number in the list and return either a pointer to it or \verb!NULL! if the number is not in the list
\end{itemize}



\section{C: Doubly-linked list}
\begin{itemize}
\item Change your code to use a doubly-linked list.
\item Add a function which takes a pointer to a node in the list and deletes the corresponding node from the list. Remember to \verb!free()! the memory used by the node!
\end{itemize}



\section{Implement \texttt{calloc()} and \texttt{realloc()}.}
\begin{itemize}
\item Write functions \verb!calloc2()! and \verb!realloc2()!, that use \verb!malloc()! and \verb!free()! to implement the functionality of \verb!calloc()! and \verb!realloc()!, respectively.
\item Remember that \verb!calloc()! sets the allocated memory to zero (for this exercise, you may ignore testing for integer overflows when multiplying the arguments of \verb!calloc()! together).
\item When implementing the copying part of \verb!realloc()!, recall that \verb!char! is 1 byte; the C standard states that you may use \verb!char *! pointers to access individual bytes of memory.
\end{itemize}



\section{External libraries}
\begin{itemize}
\item One of the reasons why C is so popular is the huge collection of tried and tested libraries available across many different computing platforms.  E.g.  OpenGL
\end{itemize}

\begin{center}
\tikzset{
  myarrow/.style = {line width=0.5mm, draw=black, -triangle 60, fill=white,postaction={draw, line width=2mm, shorten >=3mm, -}}
}
\begin{tikzpicture}
\draw[rounded corners] (-0.9, -0.45) rectangle (0.9, 0.45) {};
\node at (0,0) {C API};
\draw [myarrow] (1.0,0) -- (2.5,0);
\node[align=center] at (1.75,-1) {Commands \&\\Geometry};
\begin{scope}[xshift=100]
\draw[rounded corners] (-0.9, -0.45) rectangle (0.9, 0.45) {};
\node[align=center] at (0,0) {Graphics\\hardware*};
\draw [myarrow] (1.0,0) -- (2.5,0);
\node[align=center] at (1.75,-1) {Pixels};
\begin{scope}[xshift=100]
\draw[rounded corners] (-0.9, -0.45) rectangle (0.9, 0.45) {};
\node[align=center] at (0,0) {Display};
\end{scope}
\end{scope}
\end{tikzpicture}
\end{center}

\begin{itemize}
\item Commands from your program are sent by the API to the graphics hardware which generates pixels for display
\item *in OpenGL the hardware behaves as a state machine
\end{itemize}



\section{OpenGL programming}
\begin{itemize}
\item On its own OpenGL is:
\begin{enumerate}
\item Low level
\item O/S independent
\end{enumerate}
\item Hence it is usually used with:
\begin{itemize}
\item GLU a utility library with high level shape support
\item GLUT utility library for window creation and I/O
\end{itemize}
\end{itemize}



\section{Commonly used C libraries}
\begin{itemize}
\item general: \verb!libglib! / \verb!libgobject! / \verb!libpthread!
\item console: \verb!libncurses!
\item 2D graphics: \verb!libX11! / \verb!libSDL!
\item 3D graphics: \verb!libGL! / \verb!libGLU! / \verb!libGLUT!
\item GUI toolkits: \verb!libgtk! / \verb!libQT!
\item Images: \verb!libjpeg! / \verb!libpng! / \verb!libgif!
\item text rendering: \verb!libpango! / \verb!libfreetype!
\item sound: \verb!libasound! / \verb!libSDL!
\item compression: \verb!libz! (\verb!zlib!) / \verb!libgzip! / \verb!libbz2!
\item encryption: \verb!libcrypt! / \verb!libssl! / \verb!libgssapi! / \verb!libkrb5!
\item XML: \verb!libxml2!
\item web: \verb!libcurl!
\end{itemize}



\section{Usage of libraries}
\begin{itemize}
\item If a library is \emph{statically linked} then a copy of the library is included in the executable
\item C/C++/assembly can be combined
\item Often bound to other languages e.g. php, XML, curl
\item Many of these libraries will be \emph{dynamically linked}
\item LGPL (Lesser Gnu Public License) often used
\item Try \verb!ldd /usr/bin/php! on Linux to list dynamic dependencies
\end{itemize}



\section{Dynamic vs static linking}
\begin{itemize}
\item Dynamic linking takes place at run-time not build-time
\item Reduces filespace demands (bloat) by keeping only one copy of the library
\item Can help with updates e.g. for security
\item Dynamic libraries are called differently by OSs
\item Linux: shared objects (\verb!.so!)
\item Windows: Dynamic Link Libraries (\verb!.dll!)
\item OSX: \verb!.dylib!
\item Can lead to ``DLL Hell'': many versions of the same dynamic library
\item Best to include version number with library
\end{itemize}


\end{document}

