\documentclass{article}
\usepackage{../../../../format}
\lhead{Programming Paradigms - System Programming}
\usepackage{minted}
\usepackage{enumerate}
\usepackage{boxedminipage}
\usepackage{xcolor}
\usepackage{tikz}
\usepackage{fancyvrb}
\usepackage{hyperref}
\definecolor{links}{HTML}{2A1B81}
\hypersetup{colorlinks,linkcolor=,urlcolor=blue}
\usetikzlibrary{shapes.misc}
\usetikzlibrary{shapes.geometric, arrows, positioning}


\begin{document}
	
\begin{center}
	\underline{\huge External Libraries}
\end{center}	
	


\section{External libraries}
\begin{itemize}
\item One of the reasons why C is so popular is the huge collection of tried and tested libraries available across many different computing platforms.  E.g.  OpenGL
\end{itemize}

\begin{center}
\tikzset{
  myarrow/.style = {line width=0.5mm, draw=black, -triangle 60, fill=white,postaction={draw, line width=2mm, shorten >=3mm, -}}
}
\begin{tikzpicture}
\draw[rounded corners] (-0.9, -0.45) rectangle (0.9, 0.45) {};
\node at (0,0) {C API};
\draw [myarrow] (1.0,0) -- (2.5,0);
\node[align=center] at (1.75,-1) {Commands \&\\Geometry};
\begin{scope}[xshift=100]
\draw[rounded corners] (-0.9, -0.45) rectangle (0.9, 0.45) {};
\node[align=center] at (0,0) {Graphics\\hardware*};
\draw [myarrow] (1.0,0) -- (2.5,0);
\node[align=center] at (1.75,-1) {Pixels};
\begin{scope}[xshift=100]
\draw[rounded corners] (-0.9, -0.45) rectangle (0.9, 0.45) {};
\node[align=center] at (0,0) {Display};
\end{scope}
\end{scope}
\end{tikzpicture}
\end{center}

\begin{itemize}
\item Commands from your program are sent by the API to the graphics hardware which generates pixels for display
\end{itemize}
*in OpenGL the hardware behaves as a state machine


\section{OpenGL programming}
\begin{itemize}
\item On its own OpenGL is:
\begin{enumerate}
\item Low level
\item O/S independent
\end{enumerate}
\item Hence it is usually used with:
\begin{itemize}
\item GLU a utility library with high level shape support
\item GLUT utility library for window creation and I/O
\end{itemize}
\end{itemize}



\section{Usage of libraries}
\begin{itemize}
\item If a library is \textbf{statically linked} then a copy of the library is included in the executable
\begin{itemize}
\item No need to worry about what version of the library you have
\end{itemize}
\item C/C++/assembly can be combined
\item Often bound to other languages e.g. php, XML, curl
\item Many of these libraries will be \emph{dynamically linked}
\end{itemize}



\section{Dynamic vs static linking}
\begin{itemize}
\item \emph{Dynamic} linking takes place at run-time not build-time
\begin{itemize}
\item [+] Reduces filespace demands (bloat) by keeping only one copy of the library
\item [+] Only one copy of the library is loaded into memory
\item [+] Can help with updates e.g. for security
\item [-] Using dynamic linking can be slightly slower than static linking
\end{itemize}
\item LGPL (Lesser Gnu Public License) often used
\item Dynamic libraries are called differently by OSs
\item Linux: shared objects (\verb!.so!)
\item Windows: Dynamic Link Libraries (\verb!.dll!)
\item OSX: \verb!.dylib!
\item Can lead to ``DLL Hell'': many versions of the same dynamic library
\item Best to include version number with library
\end{itemize}



\section{Compilation Model}

\begin{center}
\tikzset{
  myarrow/.style = {line width=0.5mm, draw=black, -triangle 60, fill=white,postaction={draw, line width=2mm, shorten >=3mm, -}}
}
\begin{tikzpicture}

\node at (0,0) {source code};
\draw [myarrow] (0,-0.3) -- (0,-1.1);


\begin{scope}[yshift=-40]
\draw[rounded corners] (-1.5, -0.3) rectangle (1.5, 0.3) {};
\node at (0,0) {Pre-processor};
\draw [myarrow] (0,-0.3) -- (0,-1.1);
\begin{scope}[yshift=-20]
\node at (1.5,0) {source code};
\end{scope}
\begin{scope}[yshift=-40]
\draw[rounded corners] (-1.5, -0.3) rectangle (1.5, 0.3) {};
\node at (0,0) {Compiler};
\draw [myarrow] (0,-0.3) -- (0,-1.1);
\begin{scope}[yshift=-20]
\node at (1.5,0) {assembly code};
\end{scope}
\begin{scope}[yshift=-40]
\draw[rounded corners] (-1.5, -0.3) rectangle (1.5, 0.3) {};
\node at (0,0) {Assembler};
\draw [myarrow] (0,-0.3) -- (0,-1.1);
\node at (-3.5,0) {external libraries};
\draw [myarrow] (-2.5,-0.3) -- (-0.5,-1.1);
\begin{scope}[yshift=-20]
\node at (1.5,0) {object code};
\end{scope}
\begin{scope}[yshift=-40]
\draw[rounded corners] (-1.5, -0.3) rectangle (1.5, 0.3) {};
\node at (0,0) {Linker};
\draw [myarrow] (0,-0.3) -- (0,-1.1);
\begin{scope}[yshift=-40]
\node at (0,0) {executable code};
\end{scope}
\end{scope}
\end{scope}
\end{scope}
\end{scope}
\end{tikzpicture}
\end{center}






\section{Creating a static library}
\begin{itemize}
\item A static library is effectively just an archive containing object (\verb!.o!) files and is created with the archiver \verb!ar!.
\item On UNIX, static libraries use the \verb!.a! extension.
\begin{minted}{bash}
gcc -c linkedlist.c -o bin/static/linkedlist.o
gcc -c anotherfile.c -o bin/static/anotherfile.o
ar rcs bin/static/libLL.a bin/static/linkedlist.o \
    bin/static/anotherfile.o
\end{minted}
\item To link statically:
\begin{itemize}
\item Use the \verb!-L! flag to list a (non-standard) directory where the library can be found
\item Use the \verb!-l! flag to give the name of the library. Note that it assumes the library starts with \verb!lib! and has the extension \verb!.a! (static) or \verb!.so! (dynamic)
\end{itemize}
\begin{minted}{bash}
gcc  main.o  -Lbin/static -lLL -o bin/main-static
\end{minted}
\item Can now run
\begin{minted}{bash}
bin/main-static
\end{minted}
\end{itemize}




\section{Loading/unloading a library}
\begin{itemize}
\item We can add code to see when a library is loaded into memory and when it is unloaded.
\begin{itemize}
\item N.B. This is a \verb!gcc! extension and might not work on other compilers.
\end{itemize}
\end{itemize}
\begin{minted}{c}
void __attribute__ ((constructor)) initLibrary(void){
 printf("Library is being loaded\n");
}

void __attribute__ ((destructor)) cleanUpLibrary(void){
 printf("Library is being exited\n");
}
\end{minted}



\section{Creating a shared library}
\begin{itemize}
\item Objects files for a shared library need to be compiled with the \verb!-fPIC! option
\item On UNIX, static libraries use the \verb!.so! extension.
\begin{itemize}
\item PIC=``Position Independent Code'', since we don't know where in memory the library will be loaded at run-time
\end{itemize}
\begin{minted}{bash}
gcc -fPIC -c linkedlist.c -o bin/dynamic/linkedlist.o
gcc -fPIC -c anotherfile.c -o \
    bin/dynamic/anotherfile.o
gcc -shared bin/dynamic/linkedlist.o \
    bin/dynamic/anotherfile.o -o bin/dynamic/libLL.so
\end{minted}
\item To link dynamically:
\begin{minted}{bash}
gcc   main.o  -Lbin/static -lLL -o bin/main-static
\end{minted} 
\end{itemize}



\begin{itemize}
\item If we try to run it, we get an error:
\begin{minted}{bash}
bin/main-shared: error while loading shared libraries:
    libLL.so: cannot open shared object file: No such
    file or directory
\end{minted} 
\item Need to tell the operating system where to find the library:
\begin{minted}{bash}
# In bash:
LD_LIBRARY_PATH=`pwd`/bin/dynamic/:$LD_LIBRARY_PATH
# In tcsh (the default shell on mira):
setenv LD_LIBRARY_PATH \
    `pwd`/bin/dynamic:$LD_LIBRARY_PATH
\end{minted}
\item (Note that \verb!`! above is a backtick)
\end{itemize}



\section{Function Pointers}
\begin{itemize}
\item We've seen pointers to variables. We can also have pointers to functions!
\begin{minted}{c}
#include<stdio.h>
void hello_function(int times);

int main(){
  void (*func_ptr)(int);
  func_ptr=hello_function;
  func_ptr(3);
  return 0;
}

void hello_function(int times){
  for(int i=0;i<times;i++) {
    printf("Hello, World!\n");
  }
}
\end{minted}
\end{itemize}
We want function pointers to pass a function to a function
\section{Using \texttt{qsort()}}
\begin{itemize}
\item \verb!stdlib.h! contains an implementation of the quicksort algorithm. The function declaration is:
\begin{minted}{c}
void qsort(void *base, size_t nmemb, size_t size,
    int (*compar)(const void *, const void *)) 
\end{minted}
\item \verb!void *base! is a pointer to the array
\item \verb!size_t nmemb! is the number of elements in the array
\item \verb!size_t size! is the size of each element
\item \verb!int (*compar)(const void *, const void *)! is a function pointer composed of two arguments and returns '\verb!0! when the arguments have the same value, \verb!<0! when \verb!arg1! comes before \verb!arg2!, and \verb!>0! when \verb!arg1! comes after \verb!arg2!.
\end{itemize}



\begin{minted}{c}
#include <stdio.h>
#include <stdlib.h>
int compare (const void *, const void *); 
int main() {
  int arr[] = {52, 14, 50, 48, 13};
  int num, width, i;
  num = sizeof(arr)/sizeof(arr[0]);
  width = sizeof(arr[0]);
  qsort(arr, num, width, compare);
  for (i = 0; i < 5; i++)
    printf("%d ", arr[i]);
  printf("\n");
  return 0;
}

int compare (const void *arg1, const void *arg2) {
  return *(int *)arg1 - *(int *)arg2;
}
\end{minted}




\section{Implement \texttt{calloc()} and \texttt{realloc()}.}
\begin{itemize}
\item Write functions \verb!calloc2()! and \verb!realloc2()!, that use \verb!malloc()! and \verb!free()! to implement the functionality of \verb!calloc()! and \verb!realloc()!, respectively.
\item Remember that \verb!calloc()! sets the allocated memory to zero (for this exercise, you may ignore testing for integer overflows when multiplying the arguments of \verb!calloc()! together).
\item When implementing the copying part of \verb!realloc()!, recall that \verb!char! is 1 byte; the C standard states that you may use \verb!char *! pointers to access individual bytes of memory.
\end{itemize}



\end{document}
\end{document}
