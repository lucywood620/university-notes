\documentclass{article}
\input{../../../../format}
\usepackage{enumerate}
\usepackage{xcolor}
\usepackage{tikz}
\usepackage{fancyvrb}
\usepackage{minted}
\usepackage{hyperref}
\lhead{Programming Paradigms - Systems Programming}
\definecolor{links}{HTML}{2A1B81}
\hypersetup{colorlinks,linkcolor=,urlcolor=blue}
\usetikzlibrary{shapes.misc}
\usetikzlibrary{shapes.geometric, arrows, positioning}

\title{Systems Programming --- Lecture 4:\\
Data types, \texttt{struct}s and \texttt{union}s (Part II)}
\author{Dr Konrad Dabrowski\\
\href{mailto://konrad.dabrowski@durham.ac.uk}{konrad.dabrowski@durham.ac.uk}
}
\date{E103 Christopherson Building
}

\begin{document}
\begin{center}
\underline{\huge Data types, \texttt{struct}s and \texttt{union}s (Part II)}
\end{center}
\section{Character constants}
\begin{itemize}
\item These are integer values that are written as a character in single quotes
\item e.g. \verb!'0'! = \verb!48! in the ASCII  character set

\item These can also include escape characters:
\begin{tabular}{ll}
\verb!'\n'! & newline character\\
\verb!'\a'! & alert (bell) character\\
\verb!'\t'! & horizontal tab\\
\verb!'\0'! & \verb!NULL! character
\end{tabular}

\item Example:
\begin{minted}{c}
#define BELL '\a'
\end{minted}

\item On UNIX, you can run the \verb!man ascii! command for more information. (Press \verb!q! to exit.)
\end{itemize}



\section{String constants}
\begin{itemize}
\item These are zero or more characters in double quotes
\item Technically this is an array of chars \emph{and} it has a \verb!NULL! character at the end of the string \verb!'\0'!
\begin{minted}{c}
char a[]="Hello";
char a[]={'H','e','l','l','o','\0'};
\end{minted}
\item This means that:
\verb!'x'! is not the same as \verb!"x"! (i.e. \verb!{'x','\0'}!)

\begin{minted}{c}
#include<string.h>
char a[]="x";
char b='x';
strlen(a) = 1 // returns number of characters
sizeof(b) = 1 // returns number of bytes
sizeof(a) = 2
\end{minted}
\end{itemize}



\section{Getting user input}
\begin{itemize}
\item To output text, we use \verb!printf()!
\begin{itemize}
\item e.g. \verb!printf("%d",x);!
\end{itemize}
\item To take user input from the user, we can use \verb!scanf()!
\begin{itemize}
\item e.g. \verb!scanf("%d",&x);!
\end{itemize}
\item Note the \verb!&! (\emph{address-of}) operator in front of \verb!x!. Why do we need this?
\item We can also take in options from the command-line e.g.
\begin{minted}{c}
#include<stdio.h>
int main(int argc, char *argv[]){
  for(int i=0; i<argc; i++) {
    printf("Argument %d is %s\n",i,argv[i]);
  }
  return 0;
}
\end{minted}
\item \verb!argc! is the number of command-line arguments and each \verb!argv[i]! is a pointer to a string
\item \verb!argv[0]! is the name of the program
\item Can convert strings to integers using \verb!atoi()! from \verb!stdlib.h! e.g.
\begin{minted}{c}
char x[]="10";
int y=atoi(x);
\end{minted}
\end{itemize}



\section{Enumerations}
\begin{itemize}
\item In many programs, we'll need variables that have only a small set of meaningful values

\item A variable that stores the suit of a playing card should have only four potential values: ``clubs'', ``diamonds'', ``hearts'', and ``spades''

\item A ``suit'' variable can be declared as an integer, with a set of codes that represent the possible values of the variable:
\begin{minted}{c}
int s; /* s will store a suit */
...
s = 2; /* 2 represents "hearts" */
\end{minted}

\item Problems with this technique:
\begin{itemize}
\item We can't tell that \verb!s! has only four possible values
\item The significance of \verb!2! isn't apparent
\end{itemize}

\item Using macros to define a suit ``type'' and names for the various suits is a step in the right direction:
\begin{minted}{c}
#define SUIT     int
#define CLUBS    0
#define DIAMONDS 1
#define HEARTS   2
#define SPADES   3
\end{minted}

\item An updated version of the previous example:
\begin{minted}{c}
SUIT s;
...
s = HEARTS;
\end{minted}

\item Problems with this technique:
\begin{itemize}
\item There's no indication to someone reading the program that the macros represent values of the same ``type''

\item If the number of possible values is more than a few, defining a separate macro for each will be tedious

\item The names \verb!CLUBS!, \verb!DIAMONDS!, \verb!HEARTS! and \verb!SPADES! will be removed by the preprocessor, so they won't be available during debugging
\end{itemize}

\item C provides a special kind of type designed specifically for variables that have a small number of possible values

\item An enumerated type is a type whose values are listed (``enumerated'') by the programmer

\item Each value must have a name (an enumeration constant)

\item Enumerations are declared like this:
\begin{minted}{c}
enum {CLUBS, DIAMONDS, HEARTS, SPADES} s1, s2;
\end{minted}
\item The names of the constants must  be different from other identifiers declared in the enclosing scope
\item Enumeration constants are similar to \verb!#define! constants directive, but not equivalent
\item If an enumeration is declared inside a function, its constants won't be visible outside the function

\item Behind the scenes, C treats enumeration variables and constants as integers

\item By default, the compiler assigns the integers \verb!0!, \verb!1!, \verb!2!, \ldots to the constants in a particular enumeration

\item In the suit enumeration, \verb!CLUBS!, \verb!DIAMONDS!, \verb!HEARTS! and \verb!SPADES! represent \verb!0!, \verb!1!, \verb!2! and \verb!3!, respectively
\end{itemize}



\section{Enumerations as Integers}
\begin{itemize}
\item The programmer can choose different values for enumeration constants:
\begin{minted}{c}
enum suit {CLUBS = 1, DIAMONDS = 2, HEARTS = 3,
  SPADES = 4};
\end{minted}

\item The values of enumeration constants may be arbitrary integers, listed in no particular order:
\begin{minted}{c}
enum dept {RESEARCH = 20, PRODUCTION = 10,
  SALES = 25};
\end{minted}

\item It's even legal for two or more enumeration constants to have the same value

\item When no value is specified for an enumeration constant, its value is one greater than the value of the previous constant
\item The first enumeration constant has the value \verb!0! by default

\item Example:
\begin{minted}{c}
enum EGA_colors {BLACK, LT_GRAY = 7, DK_GRAY,
  WHITE = 15};
\end{minted}
\item \verb!BLACK! has the value \verb!0!, \verb!LT_GRAY! is \verb!7!, \verb!DK_GRAY! is \verb!8! and \verb!WHITE! is \verb!15!
\end{itemize}



\section{Structures}
\begin{itemize}
\item Collections of one or more variables forming a new data structure, the closest thing C has to an O-O class

\item The elements of a structure (its \emph{members}) aren't required to have the same type

\item The members of a structure have names; to select a particular member, we specify its name

\item In some languages, structures are called records, and members are known as fields
\end{itemize}



\section{Structure: example}
\begin{itemize}
\item For example a 2D point has \verb!x! and \verb!y! components but it is useful to create a single data structure to group them: 


\item Declares template for a point

\begin{minted}{c}
struct point {
  int x;
  int y;
};
\end{minted}

\item With members \verb!x! and \verb!y!
\end{itemize}



\section{Structures}
\begin{minted}{c}
struct point {
  int x;
  int y;
};
\end{minted}

\begin{itemize}
\item Create an instance of the point data structure:
\begin{minted}{c}
struct point a_point;
\end{minted}

\item Initialise a \verb!struct!:
\begin{minted}{c}
struct point a_point = {5, 6};
\end{minted}

\item Access to variable members of the structure:
\begin{minted}{c}
a_point.x = 4;
a_point.y = 3;
\end{minted}
\end{itemize}



\section{Structure and scope}
\begin{minted}{c}
struct point {
  int x;
  int y;
};
\end{minted}

\begin{itemize}
\item Each structure represents a new scope

\item Any names declared in that scope won't conflict with other names in a program

\item In C terminology, each structure has a separate name space for its members
\end{itemize}



\section{Operations on structures}
\begin{itemize}
\item The \verb!.! used to access a structure member is actually a C operator
\item It takes precedence over nearly all other operators
\item Example:

\begin{minted}{c}
z = 20*a_point.x;
\end{minted}

\item The \verb!.! operator takes precedence over the \verb!*! operator
\end{itemize}



\section{Assignment of structures}
\begin{itemize}
\item The other major structure operation is assignment:
\begin{minted}{c}
point2 = point1;
\end{minted}

\item The effect of this statement is to copy \verb!point1.x! into \verb!point2.x!, \verb!point1.y! into \verb!point2.y! and so on

\item The structures must have compatible types

\end{itemize}



\section{Nested structures}
\begin{itemize}
\item Declare a template for a rect(angle)
\begin{minted}{c}
struct rect{
  struct point pt1;
  struct point pt2;
};
\end{minted}

\item Create an instance of the point data structure:
\begin{minted}{c}
struct rect a_window;
\end{minted}

\item Access to variable members of the structure:
\begin{minted}{c}
a_window.pt1.x = 4;
\end{minted}

\item What is the \verb!sizeof(a_window)!?
\end{itemize}




\section{Unions}
\begin{itemize}
\item A \verb!union!, like a structure, consists of one or more members, possibly of different types

\item The compiler allocates only enough space for the largest of the members, which overlay each other within this space

\item Assigning a new value to one member alters the values of the other members as well
\end{itemize}



\subsection{Memory use}
\begin{itemize}
\item The structure \verb!s! and the union \verb!u! differ in just one way
\item The members of \verb!s! are stored at different addresses in memory
\item The members of \verb!u! are stored at the same address

\begin{minted}{c}
union {
  int i;
  double d;
} u;

struct {
  int i;
  double d;
} s;
\end{minted}
\end{itemize}



\subsection{Accessing members}
\begin{itemize}
\item Members of a union are accessed in the same way as members of a structure:
\begin{minted}{c}
u.i = 82;
u.d = 74.8;
\end{minted}
\item Changing one member of a union alters any value previously stored in any of the other members
\item Storing a value in \verb!u.d! causes any value previously stored in \verb!u.i! to be lost
\item Changing \verb!u.i! corrupts \verb!u.d!
\end{itemize}



\subsection{Properties}
\begin{itemize}
\item The properties of unions are almost identical to the properties of structures

\item Like structures, unions can be copied using the \verb!=! operator, passed to functions and returned by functions
\end{itemize}



\subsection{Initialisation}
\begin{itemize}
\item By default, only the first member of a union can be given an initial value
\item How to initialize the i member of u to 0:
\begin{minted}{c}
union {
  int i;
  double d;
} u = {0};
\end{minted}
\end{itemize}



\subsection{Designated initialisers}
\begin{itemize}
\item Designated initializers can also be used with unions
\item A designated initializer allows us to specify which member of a union should be initialized:
\begin{minted}{c}
union {
  int i;
  double d;
} u = {.d = 10.0};
\end{minted}
\item Only one member can be initialized, but it doesn't have to be the first one
\end{itemize}



\subsection{For space saving}
\begin{itemize}
\item Unions can be used to save space in structures
\item Suppose that we're designing a structure that will contain information about an item that's sold through a gift catalog
\item Each item has a stock number and a price, as well as other information that depends on the type of the item:
\begin{tabular}{ll}
Books: & Title, author, number of pages\\
Mugs: & Design\\
Shirts: & Design, colors available, sizes available\\
\end{tabular}

\item A first attempt at designing the \verb!catalog_item! using \verb!struct!:
\begin{minted}{c}
struct s_catalog_item {
  int stock_number;
  double price;
  int item_type;
  char title[TITLE_LEN+1];
  char author[AUTHOR_LEN+1];
  int num_pages;
  char design[DESIGN_LEN+1];
  int colors;
  int sizes;
};
\end{minted}
\end{itemize}



\begin{minted}{c}
struct u_catalog_item {
  int stock_number;
  double price;
  int item_type;
  union {
    struct {
      char title[TITLE_LEN+1];
      char author[AUTHOR_LEN+1];
      int num_pages;
    } book;
    struct {
      char design[DESIGN_LEN+1];
    } mug;
    struct {
      char design[DESIGN_LEN+1];
      int colors;
      int sizes;
    } shirt;
  } item;
};
\end{minted}



\subsection{Accessing nested structure}
\begin{itemize}
\item This nesting of unions does make accessing the struct fields a little more complex: 
\begin{minted}{c}
struct s_catalog_item  c; 

c.title

struct u_catalog_item c;

c.item.book.title
\end{minted}

\item Enumerations can be used to mark which member of a union was the last to be assigned

\item In the number structure, we can make a kind member an enumeration instead of an \verb!int!:
\begin{minted}{c}
struct number {
  enum {INT_KIND, DOUBLE_KIND} kind;
  union {
    int i;
    double d;
  } u;
};
\end{minted}
\end{itemize}



\section{Using Enumerations to Declare ``Tag Fields''}
\begin{minted}{c}
struct number a_number ={INT_KIND, {10}};

if (a_number.kind == INT_KIND)
  printf("a_number is %d value %d \n", 
     a_number.kind, a_number.u.i );

a_number.kind = DOUBLE_KIND;

a_number.u.d = 150.03;
if (a_number.kind == DOUBLE_KIND)
  printf("a_number is %d value %6.3f \n", 
     a_number.kind, a_number.u.d );
\end{minted}



\section{Creating new types}
\begin{itemize}
\item \verb!typedef! can be used to assign names to types
\begin{minted}{c}
typedef unsigned char byte;
byte b1 = 12;
\end{minted}
\item You can use this with \verb!struct!s and \verb!union!s too
\begin{minted}{c}
typedef struct coords {
  int x;
  int y;
} point;
point p1={5,4};

typedef union id_thing {
  int i;
  double d;
} number;
number n = {.d =10.0};
\end{minted}
\end{itemize}





\end{document}
