\documentclass{article}[18pt]
\usepackage{../../../../format}
\lhead{Software Engineering - Software Devopment}


\begin{document}
\begin{center}
\underline{\huge Refactoring code and scaling up}
\end{center}
\section{Code Complexity}
Metrics
\begin{itemize}
	\item Lines of code
	\begin{itemize}
		\item Easy to count and automate
		\item Focuses on implementation
		\item Counts vary between language
		\item Can encourage counter-productive coding practices
		\item More a measure of effort rather than complexity
		\item Remember that readability counts
	\end{itemize}
	\item Cyclomatic complexity
	\begin{itemize}
		\item Treats the program as a graph
		\item Measures how many paths there are through your code
		\item Node(N), actions
		\item Edge(E), program flow between actions
		\item P is the number of connected components (for a single piece of code P is equal to 1)
		$$C=E-N+2P$$
		
		\[
		\begin{array}{ll}
		{1-10} & {\text { Easy program, low risk }} \\
		{11-20} & {\text { Complex program, tolerable risk }} \\
		{21-50} & {\text { Complex program, high risk }} \\
		{50+} & {\text { Impossible to test }}
		\end{array}
		\]
		
	\end{itemize}
	\item Halstead metrics
	\begin{itemize}
		\item Relates to the size of the codebase
		\begin{itemize}
			\item n1 = Number of distinct operators
			\item n2 = Number of distinct operands
			\item N1 = Total number of occurrences of operators
			\item N2 = Total number of occurrences of operands
		\end{itemize}
		\item Halstead volume
		$$V=N\times \log_2(n)$$
		Where:
		\begin{itemize}
			\item N=N1+N2
			\item n=n1+n2
		\end{itemize}
	\end{itemize}
	\item Maintainability index
	\begin{itemize}
		\item A combination of the previous metrics
		$$MI = 171-5.2\ln(V)-0.23(C)-16.2\ln(LoC)+50\sin\sqrt{2.4CM}$$
		Where
		\begin{itemize}
			\item V = Halstead Volume
			\item C = Cyclomatic Complexity
			\item LoC = Lines of code
			\item CM = average percentage of comment lines
		\end{itemize}
		\item High MI ($>75$) - Good maintainability
		\item Low MI ($<25$) - Code restructuring necessary
	\end{itemize}
\end{itemize}
\section{Complexity and Refactoring}
Why refactor?
\begin{itemize}
	\item Greater predictability - plan for amount of maintenance required
	\item Software Risk Mitigation - Can assess and mitigate the risks of introducing bugs into existing code
	\item Reduced costs - Keeping the software relatively simple can reduce financial overhead in terms of maintenance
	\item Extended value - reducing complexity means that the software will retain value for longer
	\item Decision support - having the data to hand to help make a retirement decision (cost of re-write vs new code)
\end{itemize}
\section{Refactoring}
Refactoring is changing a software system by improving its internal structure without changing its external behaviour
\begin{itemize}
	\item When refactoring you should not add functionality
	\item Refactoring improves the understandability of the software and so reduces the need for documentation
	\item Changes are easier to make because the code is well structured and clear
\end{itemize}
Refactoring improves the code to slow down the degradation of the code through change
\subsection{Where refactoring helps}
\begin{itemize}
	\item Code smell - a surface indication that there might be a deeper problem in the system, a subjective assessment
	\item Unhealthy dependencies between classes and packages
	\item Duplicate code - similar code may be included at different places in a program
	\item Long methods
	\item Data clumping - same group of data items reoccur in different places
	\item Untidy code and poorly named attributes and methods
\end{itemize}
\subsection{Why refactor?}
\begin{itemize}
	\item Improves the design of software
	\item Easier to understand
	\item To find the bugs
	\item To program faster
\end{itemize}
\subsection{Resistance to refactoring}
Maintenance is hard:
\begin{itemize}
	\item Programmers don't want to touch an existing codebase
	\item Often hard to follow another programmers code
	\item Code maintenance is boring so avoided if possible
\end{itemize}
Stakeholders
\begin{itemize}
	\item Project manager - It doesn't add any new desired functionality, nor does it fix any bugs
	\item Senior developers - Too much investment in the system and its easier if nothing changes
	\item Junior developers - Don't have enough experience and not fun or leads to promotion
\end{itemize}
\subsection{When to refactor}
\begin{itemize}
	\item Best to do it continuously, but if not possible, then set an MI boundary that you and your team wish to maintain
\end{itemize}




\end{document}