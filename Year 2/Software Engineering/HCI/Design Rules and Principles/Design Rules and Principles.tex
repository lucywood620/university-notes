\documentclass{article}[18pt]
\usepackage{../../../../format}
\lhead{Software Engineering - HCI}


\begin{document}
\begin{center}
\underline{\huge Design Rules and Principles}
\end{center}
3 Types:
\begin{itemize}
	\item Design standards
	\item Design principles
	\begin{itemize}
		\item Universal/heuristic
		\item Specific
	\end{itemize}
\end{itemize}
\section{Standards}
"Rules" - with high authority. Specific to a particular type of design \\
\\
Set by national or international bodies to ensure compliance by a large community of designers\\
\\
Hardware standards more common than software\\
\\
Standards have high authority and a great amount of detail. In some domains (e.g. safety critical design) failure to apply standards can lead to disaster (or a legal case)
\section{Principles}
\begin{itemize}
	\item Over the years many principles of good interactive system design have been developed
	\item Design principles can be very broad such as "make things visible"
	\item They can be more specific such as "provide clearly marked exits"
	\item There are also good design principles that derive from psychology such as "minimise memory load"
	\item General, high level, design knowledge
	\item Principles are based on knowledge from many fields, particularly psychology, graphic design, cultural studies
	\item Abstract guidelines (universal principles) applicable during early design activities
	\item Detailed guidelines (specific principles) applicable during later design activities
\end{itemize}
\section{Principles and patterns}
The application of design principles has lead to established patterns of interaction in certain circumstances\\
\\
Design principles can
\begin{itemize}
	\item Guide the designer during the design process
	\item Can be used to evaluate and critique prototype design ideas
\end{itemize}
\newpage
\section{Twelve principles for good human-centred interactive systems design}
Learnability - A system should not be difficult to learn how to use
\begin{itemize}
	\item Visibility
	\begin{itemize}
		\item Try to ensure that things are visible so that people can see what functions are available and what the system is currently doing
	\end{itemize}
	\item Consistency
	\begin{itemize}
		\item Be consistent in the use of design features
		\item Be consistent with similar systems and standard ways of working
		\item This involves being consistent both internally to the system and externally as the system relates to things outside of it
	\end{itemize}
	\item Familiarity
	\begin{itemize}
		\item Use language and symbols that the intended audience will be familiar with
		\item Where this is not possible because the concepts are quite different from those people know about, use a metaphor
	\end{itemize}
	\item Affordance
	\begin{itemize}
		\item Design things so it is clear what they are for
		\item Affordance refers to the properties that things have and how these relate to how things could be used
		\item Affordances are culturally determined
	\end{itemize}
\end{itemize}
Ease of use
\begin{itemize}
	\item Navigation
	\begin{itemize}
		\item Provide support to enable people to move around parts of the system
	\end{itemize}
	\item Control
	\begin{itemize}
		\item Make it clear who or what is in control and allow people to take control
		\item Control is enhanced if there is a clear, logical mapping between controls and the effect they have
	\end{itemize}
	\item Feedback
	\begin{itemize}
		\item Rapidly feed back information from the system to people so that they know what affect their actions had
	\end{itemize}
\end{itemize}
Robustness
\begin{itemize}
	\item Recovery
	\begin{itemize}
		\item Enable recovery from actions, particularly mistakes and errors, quickly and effectively
	\end{itemize}
	\item Constraints
	\begin{itemize}
		\item Provide constraints so that people do not try to do things that are inappropriate 
	\end{itemize}
\end{itemize}
Accommodation
\begin{itemize}
	\item Flexibility
	\begin{itemize}
		\item Allow multiple ways of doing things to accommodate users with different levels of experience and interest in the systems
		\item Provide people with the opportunity to change the way things look or behave
	\end{itemize}
	\item Style
	\begin{itemize}
		\item Designs should be stylish and attractive
	\end{itemize}
	\item Conviviality
	\begin{itemize}
		\item Interactive systems should be polite, friendly and generally pleasant
	\end{itemize}
\end{itemize}
\end{document}