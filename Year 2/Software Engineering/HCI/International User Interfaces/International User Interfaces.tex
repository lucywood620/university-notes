\documentclass{article}[18pt]
\ProvidesPackage{format}
%Page setup
\usepackage[utf8]{inputenc}
\usepackage[margin=0.7in]{geometry}
\usepackage{parselines} 
\usepackage[english]{babel}
\usepackage{fancyhdr}
\usepackage{titlesec}
\hyphenpenalty=10000

\pagestyle{fancy}
\fancyhf{}
\rhead{Sam Robbins}
\rfoot{Page \thepage}

%Characters
\usepackage{amsmath}
\usepackage{amssymb}
\usepackage{gensymb}
\newcommand{\R}{\mathbb{R}}

%Diagrams
\usepackage{pgfplots}
\usepackage{graphicx}
\usepackage{tabularx}
\usepackage{relsize}
\pgfplotsset{width=10cm,compat=1.9}
\usepackage{float}

%Length Setting
\titlespacing\section{0pt}{14pt plus 4pt minus 2pt}{0pt plus 2pt minus 2pt}
\newlength\tindent
\setlength{\tindent}{\parindent}
\setlength{\parindent}{0pt}
\renewcommand{\indent}{\hspace*{\tindent}}

%Programming Font
\usepackage{courier}
\usepackage{listings}
\usepackage{pxfonts}

%Lists
\usepackage{enumerate}
\usepackage{enumitem}

% Networks Macro
\usepackage{tikz}


% Commands for files converted using pandoc
\providecommand{\tightlist}{%
	\setlength{\itemsep}{0pt}\setlength{\parskip}{0pt}}
\usepackage{hyperref}

% Get nice commands for floor and ceil
\usepackage{mathtools}
\DeclarePairedDelimiter{\ceil}{\lceil}{\rceil}
\DeclarePairedDelimiter{\floor}{\lfloor}{\rfloor}

% Allow itemize to go up to 20 levels deep (just change the number if you need more you madman)
\usepackage{enumitem}
\setlistdepth{20}
\renewlist{itemize}{itemize}{20}

% initially, use dots for all levels
\setlist[itemize]{label=$\cdot$}

% customize the first 3 levels
\setlist[itemize,1]{label=\textbullet}
\setlist[itemize,2]{label=--}
\setlist[itemize,3]{label=*}

% Definition and Important Stuff
% Important stuff
\usepackage[framemethod=TikZ]{mdframed}

\newcounter{theo}[section]\setcounter{theo}{0}
\renewcommand{\thetheo}{\arabic{section}.\arabic{theo}}
\newenvironment{important}[1][]{%
	\refstepcounter{theo}%
	\ifstrempty{#1}%
	{\mdfsetup{%
			frametitle={%
				\tikz[baseline=(current bounding box.east),outer sep=0pt]
				\node[anchor=east,rectangle,fill=red!50]
				{\strut Important};}}
	}%
	{\mdfsetup{%
			frametitle={%
				\tikz[baseline=(current bounding box.east),outer sep=0pt]
				\node[anchor=east,rectangle,fill=red!50]
				{\strut Important:~#1};}}%
	}%
	\mdfsetup{innertopmargin=10pt,linecolor=red!50,%
		linewidth=2pt,topline=true,%
		frametitleaboveskip=\dimexpr-\ht\strutbox\relax
	}
	\begin{mdframed}[]\relax%
		\centering
		}{\end{mdframed}}



\newcounter{lem}[section]\setcounter{lem}{0}
\renewcommand{\thelem}{\arabic{section}.\arabic{lem}}
\newenvironment{defin}[1][]{%
	\refstepcounter{lem}%
	\ifstrempty{#1}%
	{\mdfsetup{%
			frametitle={%
				\tikz[baseline=(current bounding box.east),outer sep=0pt]
				\node[anchor=east,rectangle,fill=blue!20]
				{\strut Definition};}}
	}%
	{\mdfsetup{%
			frametitle={%
				\tikz[baseline=(current bounding box.east),outer sep=0pt]
				\node[anchor=east,rectangle,fill=blue!20]
				{\strut Definition:~#1};}}%
	}%
	\mdfsetup{innertopmargin=10pt,linecolor=blue!20,%
		linewidth=2pt,topline=true,%
		frametitleaboveskip=\dimexpr-\ht\strutbox\relax
	}
	\begin{mdframed}[]\relax%
		\centering
		}{\end{mdframed}}
\lhead{Software Engineering - HCI}


\begin{document}
\begin{center}
\underline{\huge International User Interfaces}
\end{center}
\section{What is cross-cultural design?}
Designing technology for different cultures, languages and economic standings to ensure usability and user experience across cultural boundaries\\
\\
We use western design patterns and mostly use English as the mean language. But to make sure our designs are effective we must know that everything is interpreted by users based on their cultural backgrounds and values
\section{Some issues}
Factors that effect an individual's response to a system
\begin{itemize}
	\item Age, gender, race, sexuality, class, religion and political persuasion
\end{itemize}
Exported software frequently required modifications to suit local customs; laws or conventions\\
\\
Most experiences have previously concentrated on the technical issues such as the spelling dictionary and translation\\\\\
However some applications blend technical and social facilities such as groupware - this increases the complexity of the design issues\\
\\
The development of multiple interfaces is costly so it is important to make generic, and easily modifiable, as much of the interface as possible
\section{Three levels of interface specialisation}
\textbf{Globalisation}
\begin{itemize}
	\item A fully globalized app or website is fully available and functional in multiple languages
	\item Globalisation is the end goal and to achieve this have to consider the other two processes within globalisation
\end{itemize}
\textbf{Internationalisation}
\begin{itemize}
	\item Internationalisation is a task that has to be completed in order for the end goal to be achieved. Internationalisation is not the easiest change to make to a website and it takes a lot more effort than just translating the website's language. It is the process of designing and building an application to facilitate localisation
\end{itemize}
\textbf{Localisation}
\begin{itemize}
	\item Refers to the adaptation of a product, application or document content to meet the language, cultural requirements of a specific target market (a locale) developing specific interfaces to meet a particular market
\end{itemize}
\section{Effective design}
Effective design involves recognising the cultural elements in a given application\\
\\
Cultural diversity makes it even more unrealistic for designers to rely in intuition or personal experience for interface design\\
\\
Adaptation of shared interfaces requires identification of user factors, including:
\begin{itemize}
	\item Objective factors: gender, age, ethnic background, mother-tongue
	\item Subjective factors: which cannot be directly measured or identified such as cognitive style
\end{itemize}
\section{Iceberg model}
\textbf{Surface}: visible, obvious rules such as number, currency, time and date formats\\
\\
\textbf{Unspoken rules}: Obscured, need context of situation to understand the rules\\
\\
\textbf{Unconscious rules}: Rules out of conscious awareness and difficult to study
\section{Multi-cultural interface design}
Some practical advice on factors that are affected by culture, language and local conventions
\begin{itemize}
	\item Character sets and collating sequences
	\item Format conventions for numbers, dates, times and currency
	\item Layout conventions e.g. for names, addresses and telephone numbers
	\item Icons, symbols, graphics and colours; screen text
\end{itemize}
\begin{defin}[Collating sequences]
These define the value and position of each character with respect to other characters. Alphabetic and alphanumeric lists are sorted according to these sequences. However, different cultures have different sequences and rules for sorting these characters
\end{defin}
\subsection{Icons, fonts and symbols}
Not all icons translate across cultures, in Japan tick means incorrect and a circle means correct
\section{Cultural issues}
Issues that impact the user interface from locale to locale include:
\begin{itemize}
	\item \textbf{Nationalism} - What is considered an inherent part of a nation or culture and what is considered a threat to it
	\item \textbf{Language} - People are insulted by "low status" dialect
	\item \textbf{Social context} - Who made a statement is just as important as what the statement is in some parts of the world
\end{itemize}
\section{User customisation}
\begin{itemize}
	\item Keep sentences as short and simple as possible
	\item Allow users to select the date and time format
	\item Allow users to select calendar format
	\item Allow users to select paper size
	\item Allow users to select numeric and monetary formats
\end{itemize}
\end{document}