\documentclass{article}[18pt]
\usepackage{../../../../format}
\lhead{Software Engineering - HCI}


\begin{document}
\begin{center}
\underline{\huge Prototypes and Evaluation}
\end{center}
\section{Prototypes}
Prototypes can come in all shapes and forms
\begin{itemize}
	\item Early in the development process they may be made of paper, post-its or cardboard
	\item Later they may be screen mock-ups or products with restricted functionality
	\item Finally they may be more polished pieces of software, metal or plastic that represent the final product
\end{itemize}
\begin{defin}[Prototyping]
The process of building the interactive versions of the proposed product
\end{defin}
A prototype allows users to
\begin{itemize}
	\item Interact with envisioned product
	\item Gain experience of using it in a realistic setting
	\item Explore imagined uses
\end{itemize}
A prototype is a limited representation that allows interaction and helps users explore its suitability
\subsection{What is a prototype}
An interaction design it can be (among other things):
\begin{itemize}
	\item A series of screen sketches
	\item A storyboard
	\item A powerpoint slide show
	\item A video simulating the use of a system
	\item A lump of wood
	\item A cardboard mock-up
	\item A piece of software with limited functionality
\end{itemize}
\subsection{What to prototype}
Technical issues:
\begin{itemize}
	\item How feasible is the design
	\item If you can't make a mock up what hope is there for the full product
\end{itemize}
Screen layouts and information display
\begin{itemize}
	\item What will the screens look like?
	\item How will information be displayed?
	\item How does this affect the proposed interaction?
\end{itemize}
Difficult, controversial, critical areas
\begin{itemize}
	\item Remove ambiguity from the design
	\item Clarify vague requirements
	\item Force the developers to make concrete design decisions and see how they work
\end{itemize}
$ $
\begin{itemize}
	\item Can be difficult to get users to articulate exactly what they want
	\item When presented with examples Stakeholders can see, hold, interact with a prototype more easily than a document or drawing
	\item Allows developers to verify requirements
	\item Prototypes answer questions, and support designers in choosing between alternatives
	\item Team members can communicate effectively
	\item It encourages reflection: very important aspect of design
	\item You can test out ideas yourself
\end{itemize}
\subsection{Wireframe vs Mockup vs Prototype}
\begin{defin}[Wireframe]
Basic illustrations of structure and components of a web page
\end{defin}
\begin{defin}[Mockups]
Close or identical to the actual final web site design and include graphics - generally just image files
\end{defin}
\begin{defin}[Prototypes]
Semi functional and generally give the client the ability to click around and simulate the way the site will eventually work
\end{defin}
\subsection{Prototyping process}
\begin{itemize}
	\item Very early in the process, you should develop paper prototypes- wireframes of screen designs - and walk through this with end-users
	\item You then refine your design and develop increasingly sophisticated automated prototypes, then make them available to users for testing
\end{itemize}
\subsection{Low-fidelity Prototyping}
Uses a medium which is unlike the final medium\\
\\
Is quick, cheap and easily changes
\begin{itemize}
	\item Supports the exploration of alternative designs and ideas
	\item Important for early stage of design
	\item They are exploratory only
\end{itemize}
\subsection{Storyboards}
Often used with scenarios, bringing more detail, and a chance to role play\\
\\
It is a series of sketches showing how a user might progress through a task using the device
\begin{itemize}
	\item A series of sketched screens for a GUI interface
	\item A series of scene sketches showing the user performing an activity
	\item Used early in design
\end{itemize}
\subsection{High-fidelity programming}
\begin{itemize}
	\item Uses materials that you would expect to be in the final product
	\item Prototype looks more like the final system than a low-fidelity version
	\item Danger that users think they have a full system - need to manage user expectations
\end{itemize}
Problems with high fidelity prototypes
\begin{itemize}
	\item They take too long to build
	\item Reviewers and testers tend to comment on superficial aspects rather than content
	\item Developers are reluctant to change something they have crafted for hours
	\item A software prototype can set expectations too high
	\item Just one bug in a high-fidelity prototype can bring testing to a fault
\end{itemize}
\subsection{Compromises in prototyping}
\begin{itemize}
	\item All prototypes involve compromises
	\item For software based prototyping
	\begin{itemize}
		\item Slow response
		\item Sketchy icons
		\item Limited functionality
	\end{itemize}
	\item Two common types of compromise
	\begin{itemize}
		\item Horizontal - lots of functions, little detail
		\item Vertical - Lots of detail, few functions
	\end{itemize}
	\item Compromises in prototypes mustn't be ignored, product still needs to be produced
\end{itemize}
\subsection{Management of problems for prototyping}
Time
\begin{itemize}
	\item Building prototypes take time
	\item Needs to be fast - rapid prototyping. Careful this does not lead to rushed evaluation and erroneous results
\end{itemize}
Planning
\begin{itemize}
	\item Hard to adequately and cost a design involving prototyping
\end{itemize}
Non functional requirements
\begin{itemize}
	\item Such features sacrificed in prototypes
	\item May be critical to product acceptance
\end{itemize}
Contracts
\begin{itemize}
	\item Design often bound by contractual agreement on technical and managerial issues
	\item Prototypes are quick to change but this needs to be reflected in the requirements document which serves as the binding development agreement
\end{itemize}
\subsection{Key points}
\begin{itemize}
	\item Prototyping is the process of building interactions of a proposed product
	\item Used to gather formative evaluation results and feedback from users
	\item Low fidelity prototypes are cheap and quick to develop and should be used early in the design for formalise requirements
	\item High fidelity prototypes provide the look and feel of the final product and serve as a living specification
	\item Need early consideration of the nature of the prototype
	\item Prototyping can be difficult to plan and cost and we must be aware of the non-functional components that may be missing
\end{itemize}
\section{Evaluation}
\subsection{Goals of evaluation}
\begin{itemize}
	\item To assess the extent of the systems functionality
	\item To assess the effect of the interface on the user
	\item To identify any specific problems with the system
\end{itemize}
\subsection{Why, what, where and when to evaluate}
\begin{itemize}
	\item Why: to check users' requirements and that users can use the product and they like it
	\item What: a conceptual model, early prototypes of a new system and later, more complete prototypes
	\item Where: in natural and laboratory settings
	\item When: Throughout design; finished products can be evaluated to collect information and to inform new products
\end{itemize}
\subsection{Types of evaluation}
\begin{itemize}
	\item \textbf{Controlled settings} involving users
	\begin{itemize}
		\item Usability testing
		\item Living labs
	\end{itemize}
	\item \textbf{Natural settings} involving users
	\begin{itemize}
		\item Evaluate people in natural settings
		\item Goal to be unobtrustive
	\end{itemize}
	\item \textbf{Any settings} not involving users
	\begin{itemize}
		\item Researcher has to imagine or model how an interaction will take place
		\item Inspection methods used to predict user behaviours and identify usability problems
	\end{itemize}
\end{itemize}
\subsubsection{Inspection methods}
\begin{itemize}
	\item Heuristic evaluation
	\begin{itemize}
		\item Guides design or critique a decision. Critique based on principles and guidelines
	\end{itemize}
	\item Cognitive walkthrough
	\begin{itemize}
		\item Analysis of action sequences the user interface requires the user to perform
	\end{itemize}
	\item Analytics
	\begin{itemize}
		\item Measurement, collection, analysis and reporting of internet data
	\end{itemize}
\end{itemize}
\subsection{Key points}
\begin{itemize}
	\item Evaluation and design are closely integrated in user-centred design
	\item Some of the same data gathering methods are used in evaluation as for establishing requirements but they are used differently
	\item Three types of evaluation. Controlled - laboratory based with users, Less controlled - in the field with users, studies that do not involve users
	\item Usability testing and experiments enable they evaluator to have high level of control over what gets tested, whereas evaluations typically impose little or no control on participants in field studies 
\end{itemize}
\end{document}