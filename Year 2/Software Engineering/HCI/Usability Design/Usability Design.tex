\documentclass{article}[18pt]
\usepackage{../../../../format}
\lhead{Software Enginering - HCI}


\begin{document}
\begin{center}
\underline{\huge Usability Design}
\end{center}
\section{The design of everyday thing}
How can we use knowledge of human psychology and physiology to help us design better products
\begin{enumerate}
	\item Accordances
	\item Mappings
	\item Constraints
	\item Conventions
\end{enumerate}
\section{Knowledge in the Head and in the World}
Not all the knowledge required for accurate behaviour has to be in the head. It can be distributed
\begin{itemize}
	\item Partly in the head
	\item Partly in the world
\end{itemize}
Placing knowledge in the world (as part of the design of objects)
\begin{itemize}
	\item Having knowledge in the world reduced the load on human memory
	\item An example of the input format can be provided in the interface "Please enter the date (yyyy/mm/dd)
\end{itemize}
\section{Embedded Knowledge}
Good design should "embed" knowledge of how it is used in the design itself\\
\\
If a user has to remember "how to use" they will forget\\
\\
The design of these things "suggest" how you use them as part of their design
\section{Perceived Affordance}
The perceived properties of the object that suggest how one could use it\\
When simple things need pictures, labels or instructions, the design has failed
\section{GUI Affordances}
For screen-based interfaces, the computer already has built in physical affordances
\begin{itemize}
	\item Screen affords touching
	\item Mouse affords pointing
	\item Mouse buttons afford clicking
	\item Keyboard affords typing
\end{itemize}
Computer software/web pages can suggest on screen affordances by using raised buttons, icons in the shape of sliders or knobs
\section{Mappings}
Mappings are the relationships between controls and their effects on a system\\
Natural mapping take advantage of physical analogies and cultural standards\\
Examples:
\begin{itemize}
	\item Turn steering wheel clockwise to turn a car right
\end{itemize}
\section{Constraints}
Constraints are physical, cultural or logical limits on the number of possibilities for an object's use
\begin{itemize}
	\item Physical constraints such as pegs and holes limit possible operations
	\item Cultural constraints rely upon accepted cultural conventions
\end{itemize}
Where affordances suggest the range of possibilities, constraints limit the number of alternatives
\section{Conventions}
Conventions are cultural constraints . They are initially arbitrary, but evolve and become accepted over time. They can however still vary enormously across different cultures
\section{Conceptual Models}
Conceptual models are formed from:
\begin{itemize}
	\item Affordances
	\item Mapping
	\item Constraints
	\item Familiarity with similar devices (transfer of previous experience)
	\item Instructions
	\item Interactions (trial and error)
\end{itemize}
\end{document}