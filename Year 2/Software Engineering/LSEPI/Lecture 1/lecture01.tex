\documentclass{article}
\ProvidesPackage{format}
%Page setup
\usepackage[utf8]{inputenc}
\usepackage[margin=0.7in]{geometry}
\usepackage{parselines} 
\usepackage[english]{babel}
\usepackage{fancyhdr}
\usepackage{titlesec}
\hyphenpenalty=10000

\pagestyle{fancy}
\fancyhf{}
\rhead{Sam Robbins}
\rfoot{Page \thepage}

%Characters
\usepackage{amsmath}
\usepackage{amssymb}
\usepackage{gensymb}
\newcommand{\R}{\mathbb{R}}

%Diagrams
\usepackage{pgfplots}
\usepackage{graphicx}
\usepackage{tabularx}
\usepackage{relsize}
\pgfplotsset{width=10cm,compat=1.9}
\usepackage{float}

%Length Setting
\titlespacing\section{0pt}{14pt plus 4pt minus 2pt}{0pt plus 2pt minus 2pt}
\newlength\tindent
\setlength{\tindent}{\parindent}
\setlength{\parindent}{0pt}
\renewcommand{\indent}{\hspace*{\tindent}}

%Programming Font
\usepackage{courier}
\usepackage{listings}
\usepackage{pxfonts}

%Lists
\usepackage{enumerate}
\usepackage{enumitem}

% Networks Macro
\usepackage{tikz}


% Commands for files converted using pandoc
\providecommand{\tightlist}{%
	\setlength{\itemsep}{0pt}\setlength{\parskip}{0pt}}
\usepackage{hyperref}

% Get nice commands for floor and ceil
\usepackage{mathtools}
\DeclarePairedDelimiter{\ceil}{\lceil}{\rceil}
\DeclarePairedDelimiter{\floor}{\lfloor}{\rfloor}

% Allow itemize to go up to 20 levels deep (just change the number if you need more you madman)
\usepackage{enumitem}
\setlistdepth{20}
\renewlist{itemize}{itemize}{20}

% initially, use dots for all levels
\setlist[itemize]{label=$\cdot$}

% customize the first 3 levels
\setlist[itemize,1]{label=\textbullet}
\setlist[itemize,2]{label=--}
\setlist[itemize,3]{label=*}

% Definition and Important Stuff
% Important stuff
\usepackage[framemethod=TikZ]{mdframed}

\newcounter{theo}[section]\setcounter{theo}{0}
\renewcommand{\thetheo}{\arabic{section}.\arabic{theo}}
\newenvironment{important}[1][]{%
	\refstepcounter{theo}%
	\ifstrempty{#1}%
	{\mdfsetup{%
			frametitle={%
				\tikz[baseline=(current bounding box.east),outer sep=0pt]
				\node[anchor=east,rectangle,fill=red!50]
				{\strut Important};}}
	}%
	{\mdfsetup{%
			frametitle={%
				\tikz[baseline=(current bounding box.east),outer sep=0pt]
				\node[anchor=east,rectangle,fill=red!50]
				{\strut Important:~#1};}}%
	}%
	\mdfsetup{innertopmargin=10pt,linecolor=red!50,%
		linewidth=2pt,topline=true,%
		frametitleaboveskip=\dimexpr-\ht\strutbox\relax
	}
	\begin{mdframed}[]\relax%
		\centering
		}{\end{mdframed}}



\newcounter{lem}[section]\setcounter{lem}{0}
\renewcommand{\thelem}{\arabic{section}.\arabic{lem}}
\newenvironment{defin}[1][]{%
	\refstepcounter{lem}%
	\ifstrempty{#1}%
	{\mdfsetup{%
			frametitle={%
				\tikz[baseline=(current bounding box.east),outer sep=0pt]
				\node[anchor=east,rectangle,fill=blue!20]
				{\strut Definition};}}
	}%
	{\mdfsetup{%
			frametitle={%
				\tikz[baseline=(current bounding box.east),outer sep=0pt]
				\node[anchor=east,rectangle,fill=blue!20]
				{\strut Definition:~#1};}}%
	}%
	\mdfsetup{innertopmargin=10pt,linecolor=blue!20,%
		linewidth=2pt,topline=true,%
		frametitleaboveskip=\dimexpr-\ht\strutbox\relax
	}
	\begin{mdframed}[]\relax%
		\centering
		}{\end{mdframed}}
\lhead{Software Engineering - LSEPI}
\usepackage{enumerate}
\usepackage{hyperref}

\begin{document}
\begin{center}
	\underline{\huge Introduction to English Law}
\end{center}	
	





\section{Overview of LSEPI}
\begin{itemize}
\item Among other things, you will be expected to know the various laws mentioned that apply to the UK (note that this includes EU laws); you will not be specifically tested on US laws.
\item You do \emph{not} need to memorize historical dates (e.g. what year the first patent was granted).
\end{itemize}

\begin{itemize}
\item Subject specific knowledge
\begin{itemize}
\item Knowledge about and understanding why Legal, Social, Ethical, and Professional Issues are essential in software production and management.
\item A good appreciation of the complexities and impact of legislation on the professional work environment, to which you are aspiring to enter. 
\end{itemize}
\item Subject specific skills
\begin{itemize}
\item Have gained the ability to analyse and discuss a number of current professional  issues, e.g.: data protection and use; professional responsibilities; freedom of information; the impact of the digital economy. 
\item Ability to discuss and explain key legislation, for example:
\item Computer Misuse Act 1990: \url{https://www.legislation.gov.uk/ukpga/1990/18/contents}
\item Data Protection Act 2018: \url{https://www.legislation.gov.uk/ukpga/2018/12/contents}
\end{itemize}
\end{itemize}

\begin{itemize}
\item Key skills
\begin{itemize}
\item Carefully reading / listening / judging the value of on-line materials
\item Questioning and probing complex issues in a public forum  
\end{itemize}
\end{itemize}



Pandora's Box: 
social and professional issues of the information age, 
by Adams and McCrindle, 2008
\begin{itemize}
\item ``due to the ubiquity of computer technology, computing professionals must now consider their work in the same way as the nuclear physicists in terms of its potential impact on individuals and society''.
\item ``Despite the huge growth of the computer usage and the Internet access during the last 10 years, to the point where a majority of the population (in industrialized countries) uses a computer and accesses the Internet at least weekly, computers and the Internet remain a mystery to most of those users.''
\item ``Where mystery holds sway, there is always the potential for social reactions (economic, legal and individual) to produce adverse effects on the lives of computing professionals.''
\end{itemize}



\section{Law}
\begin{itemize}
\item ``The body of rules, whether proceeding from formal enactment or from custom, which a particular state or community recognizes as binding on its members or subjects...'' \cite{oed}
\item \ldots ``a set of rules which form a pattern of behaviour of a given society''
\item \ldots `a set of rules that can be enforced in a court' \cite{shears_james_2005}
\item The laws are different in different countries. Examples:
\begin{itemize}
\item Divorce
\item Sale of alcohol
\end{itemize}
\end{itemize}



\section{Criminal Law}
\begin{itemize}
\item ``represents society's view of the minimum standard of acceptable behaviour'' \cite{shears_james_2005}

\item Criminal law defines:
\begin{itemize}
\item What constitutes a crime
\begin{itemize}
\item In general: behaviours prohibited by the state because it threatens / harms / endangers the public safety / welfare
\end{itemize}
\item The mechanisms for deciding innocence / guilt of an accused person
\item Appropriate punishment:
\begin{itemize}
\item A range from which an authority makes a selection 
\item Person must be found guilty beyond a reasonable doubt
\end{itemize}
\end{itemize}
\end{itemize}




\section{Civil Law}
\begin{itemize}
\item Deals with disputes between legal persons
\begin{itemize}
\item A compensation may be awarded to the victim that has gone through the process of `incorporation' by:
\begin{itemize}
\item Royal Charter \url{http://privycouncil.independent.gov.uk/royal-charters/}
(example: Durham University, 1837)
\item Act of Parliament \url{http://www.parliament.uk}
\item Registering as company 
\end{itemize}
\end{itemize}

\item Q. What constitutes a legal person?

\begin{itemize}
\item A legal person could be companies and organisations, hence civil laws apply to them in the same way it applies to a natural person
\end{itemize}
\end{itemize}

\begin{itemize}
\item Court proceedings are known as \emph{litigation}
\begin{itemize}
\item Must be initiated by the party (\emph{plaintiff}) of the dispute who feels wronged by the \emph{defendant}
\item For a plaintiff to win a case, he/she/it must show that his/her/its claim is correct \emph{on the balance of probabilities} (i.e. $> 50\%$)
\end{itemize}
\end{itemize}



\section{Criminal Law vs Civil Law}
\begin{itemize}
\item Differences between criminal law and civil law relate to the standard of proof and burden of proof:
\begin{itemize}
\item In criminal law, the court must prove the offender is guilty beyond all reasonable doubt. Where in civil law the plaintiff has to proof his/her/its claim is correct on the balance of probabilities.
\item In a criminal case, the burden of proof lies on the prosecution. The defendant is innocent until proved guilty. Whereas in a civil case, both parties present their argument and must the convince the court of their correctness. (probability of correctness is used to determine the winner)
\end{itemize}
\end{itemize}



\section{Where Does the Law Come From?}
\begin{itemize}
\item Common Law
\begin{itemize}
\item Not written
\item Based on previous rulings (precedents) by judges
\item Not found in other countries, e.g. in continental Europe
\item Experience shared between some countries (e.g. commonwealth countries)
\end{itemize}

\item Statute Law 
\begin{itemize}
\item Written 
\item Set by governments
\begin{itemize}
\item Acts of Parliament - legislation
\end{itemize}
\item Some common law became statute law (e.g. the Theft Act 1968)
\end{itemize}

\item Think of an example of a statute law related to computing
\end{itemize}



\section{The Legislative Process in the UK}
\begin{itemize}
\item Two-chamber legislature
\item The British  legislature is know as ``parliament''
\item House of Commons (the lower house)
\begin{itemize}
\item Democratically elected (650 MPs) every five years
\item Usually the initiator of a bill (but not always)
\item Will accept or reject changes made by the House of Lords
\item \url{http://www.parliament.uk/education/about-your-parliament/mps-lords-monarch/}
\item \url{https://www.youtube.com/watch?v=SlPSAOa4vR4}
\end{itemize}
\item House of Lords (the upper house)
\begin{itemize}
\item Aristocrats, life peers, bishops, hereditary peers
\item Approval of a bill will result in an Act of Parliament
\item Modification will result in the bill go back to the House of Commons
\item Democracy: it cannot disapprove a bill that has passed 2 times from the House of Commons
\item Normally small in size
\end{itemize}
\end{itemize}



\section{Public input to statute law}
\begin{itemize}
\item Green Paper 
\begin{itemize}
\item Also known as consultation document
\item A discussion document intended to initiate public consultation on a particular topic: (participatory democracy)
\item Explains why the government wants to create a new law
\item Invites opinions / ideas / proposals from interested individuals / organizations (e.g. BCS) who have expertise in the topic
\item Example: \url{http://ec.europa.eu/internal_market/copyright/docs/copyright-infso/greenpaper_en.pdf}
\end{itemize}

\item White Paper
\begin{itemize}
\item More public consultation
\item However: more specific and detailed than the green paper
\item Signifies a clear intention of the government to pass a new law
\item \url{http://www.out-law.com/page-10147}
\end{itemize}
\end{itemize}



\section{Act of Parliament}
\begin{itemize}
\item A Bill (\url{https://www.parliament.uk/education/teaching-resources-lesson-plans/from-bill-to-law/}
\begin{itemize}
\item A proposal for discussion in Parliament 
\begin{itemize}
\item Available for debate and modification
\item Intended for approval (by voting)
\end{itemize}
\item Primary Legislation
\begin{itemize}
\item Law made by the legislative branch of the government (Parliament)
\item Discussed / debated
\item Parliament cannot examine every law in detail
\end{itemize}
\item Secondary Legislation (or delegated legislation)
\begin{itemize}
\item Detailed regulation introduced without full discussion
\item Follows the marriage method (proposals placed in the House of Commons library)
\item Law allows the government to slightly change / update a law without pushing through a completely new Act of Parliament
\item e.g. Copyright, Design and Patents Act 1988 (protecting semi-conductor chip design): \url{https://www.legislation.gov.uk/ukpga/1988/48/pdfs/ukpga_19880048_en.pdf}
\end{itemize}
\end{itemize}
\end{itemize}

\begin{itemize}
\item Royal Assent
\begin{itemize}
\item The final step for a bill to become law (Act of Parliament) is the formal approval by the Queen
\begin{itemize}
\item Power of veto (however: last veto in 1708)
\end{itemize}
\item \url{http://www.parliament.uk/about/how/laws/passage-bill/lords/lrds-royal-assent/}
\item \url{https://www.youtube.com/watch?v=RUkSyuw6f7s}
\end{itemize}
\item The people
\begin{itemize}
\item Can challenge laws in the courts
\end{itemize}

\item \url{https://www.youtube.com/watch?v=iM4CKYCrW7Y}
\end{itemize}


\bibliographystyle{unsrt}
\bibliography{bibliography}


\end{document}
