\documentclass{article}
\usepackage{../../../../format}
\lhead{Software Engineering - LSEPI}
\usepackage{enumerate}
\usepackage{hyperref}
%\definecolor{links}{HTML}{2A1B81}
\hypersetup{colorlinks,linkcolor=,urlcolor=blue}


\begin{document}


\begin{center}
	\underline{\huge Data Protection}
\end{center}

\section{Data Protection Overview}

{\em What kind of data?}
\bigskip
\begin{itemize}
\item Data relating to an {\em identified} or {\em identifiable} {\bf living} individual should not be collected without:
\begin{itemize}
\item an explicit {\em purpose}
\item a {\em lawful basis} for collecting and using the data.
\end{itemize}
\end{itemize}


\bigskip
\begin{itemize}
\item There are two strands of data protection and privacy regulation in the Western world: 
\begin{itemize}
\item the US approach (self-regulation and market forces)
\item the European approach (government regulation and strict laws)
\end{itemize}
\end{itemize}



\section{Data Protection Act 2018 (UK)
\small \href{http://www.legislation.gov.uk/ukpga/2018/12/pdfs/ukpga_20180012_en.pdf}{www.legislation.gov.uk/ukpga/2018/12/pdfs/ukpga\_20180012\_en.pdf}}

\begin{itemize}
\item A way for individuals to control their personal information
\item Protects individuals from:
\begin{itemize}
\item use of personal information:
\begin{itemize}
\item by unauthorized persons
\item for purposes other than those identified when it was collected
\item for a longer period than needed
\end{itemize}
\item inaccurate personal information 
\item irrelevant personal information
\begin{itemize}
\item i.e. individuals have the right to have incorrect / irrelevant information corrected / deleted (e.g. Google case)
\end{itemize}
\end{itemize}
\end{itemize}

\bigskip
\begin{itemize}
\item Information Commissioner's Office (\href{https://ico.org.uk/}{ico.org.uk})
\begin{itemize}
\item regulates compliance with the Data Protection Act
\end{itemize}
\end{itemize}




\section{GDPR Terminology (EU)}
\begin{itemize}
 \item European Union (EU) General Data Protection Regulation (GDPR) 
 \begin{itemize}
   \item Common European minimum standards that were agreed in 2016 and came into force 25th May 2018
   \item Applies to organisations, not to personal use
 \end{itemize}
 \item Data
 \begin{itemize}
  \item Information that is
  \begin{itemize}
   \item being processed automatically, or 
   \item is collected to be processed automatically, or 
   \item is recorded as part of a relevant filing systems.
  \end{itemize}
 \end{itemize}
 \item Personal data
 \begin{itemize}
  \item Information that is or can be linked directly to an individual
  \begin{itemize}
   \item Name, address, etc.
  \end{itemize}
  \item ID codes that can be linked to an individual
  \begin{itemize}
   \item National Insurance Number
  \end{itemize}
  \item Must ``relate to'' the individual
 \end{itemize}
\end{itemize}




\section{GDPR}
\begin{itemize}
\item The GDPR applies to processing carried out by organisations operating within the EU. It also applies to organisations outside the EU that offer goods or services to individuals in the EU.
\item Infringements of the basic principles for processing personal data could mean a fine of up to 20 million euros, or 4\% of your total worldwide annual turnover, whichever is higher.
\end{itemize}




\section{GDPR Terminology (EU)}
\begin{itemize}
\item Data controller
\begin{itemize}
\item The individual / authority who determines the purposes and means of the processing of personal data 
\item May need to pay an annual fee to the Information Commissioner's Office, depending on the type of processing
\end{itemize}

\item Data processor
\begin{itemize}
\item Anyone who obtains, records, stores, organises, transforms, transports, discloses, deletes, combines etc. data on behalf of the data controller
\end{itemize}

\item Data subject
\begin{itemize}
\item An individual identified or identifiable by personal information
\end{itemize}
\end{itemize}

\bigskip
What's new: \href{https://www.youtube.com/watch?v=7mMnDsp7Weg}{www.youtube.com/watch?v=7mMnDsp7Weg}




\section{GDPR}
\begin{itemize}
\item Member states must:
\begin{itemize}
\item create legislation to ensure:
\begin{itemize}
\item use limitation
\item purpose specification 
\end{itemize}

\item set up a supervisory authority that will:
\begin{itemize}
\item monitor the data protection level in the country
\item give advice to the government 
\item start legal proceedings when regulation is violated
\end{itemize}
\end{itemize}

\item Data Protection Act 2018 (updated from 1984, 1998) (UK)
\begin{itemize}
\item Information Commissioner's Office (\href{https://ico.org.uk/}{ico.org.uk}) 
\item Data controllers' and processors' responsibility
\item 7 data protection principles
\begin{enumerate}[label=\alph*]
\item Lawfulness, fairness and transparency
\item Purpose limitation
\item Data minimisation
\item Accuracy
\item Storage limitation
\item Integrity and confidentiality (security)
\item Accountability
\end{enumerate}
\end{itemize}
\end{itemize}



\section{GDPR}
Certain types of data have extra protections:
\begin{itemize}
\item Criminal offence data
\item Special category data e.g. information about an individual's
\begin{itemize}
\item race
\item ethnic origin
\item politics
\item religion
\item trade union membership
\item genetics
\item biometrics (where used for ID purposes)
\item health
\item sex life
\item sexual orientation
\end{itemize}
\end{itemize}



\section{Principle (a): Lawfulness, Fairness and Transparency}
\begin{itemize}
\item You must have a valid reason (a {\em lawful basis}) to collect and process personal data
\item You must not do anything with the data that is illegal
\item You must use personal data in a way that is {\em fair} i.e. you must not use data in a way that is
\begin{itemize}
\item detrimental in an unjustified way
\item unexpected or
\item misleading to the individual concerned.
\end{itemize}
\item You must be clear, open and honest with people from the start about how you will use their data
\end{itemize}



\section{Lawful Basis}
\begin{itemize}
\item There are six lawful bases for collecting and processing personal data:
\begin{itemize}
\item {\bf Consent:} individual has given clear consent for you to process their data for a specific purpose (must be opt-in; can withdraw their consent at any time)
\item {\bf Contract:} necessary to carry out a contract or if they have asked you to take steps before entering into a contract
\item {\bf Legal obligation:} so you can comply with the law
\item {\bf Vital interests:} protecting someone's life (could be a third party, only applies if consent cannot be given)
\item {\bf Public task:} to perform a task in the public interest that has a basis in law
\item {\bf Legitimate interests:} necessary for your legitimate interests or those of a third party, unless there is a good reason to protect the individual's interests that overrides these legitimate interests
\end{itemize}
\item You need to determine which is most appropriate
\item It may be difficult to switch to a different one later
\end{itemize}



\section{Principle (b): Purpose Limitation}
\begin{itemize}
\item You must be clear from the start about {\em why} you are collecting personal data and {\em what} you intend to do with it.
\item You need to document your purposes and include them in your privacy information for individuals
\item You can only use the personal data for a new purpose if either this is compatible with your original purpose, you get consent, or you have a clear basis in law
\end{itemize}




\section{Principles (c)-(e)}
\begin{enumerate}[label=\alph*]
\setcounter{enumi}{2}
\item Data Minimization 
\begin{itemize}
\item You must ensure the personal data you are processing is:
\begin{itemize}
\item adequate - sufficient to properly fulfil your stated purpose;
\item relevant - has a rational link to that purpose; and
\item limited to what is necessary - you do not hold more than you need for that purpose.
\end{itemize}
\end{itemize}
\item Accuracy
\begin{itemize}
\item You must take reasonable steps to ensure that personal data you hold is correct, up-to-date and not misleading
\end{itemize}

\item Storage Limitation
\begin{itemize}
\item data shall {\em not} be kept longer than necessary
\item you must have a policy setting standard retention periods if possible 
\item you should periodically review data you hold and erase or anonymise it when you no longer need it
\item exceptions apply if you are keeping the data for
\begin{itemize}
\item archiving in the public interest
\item scientific or historical research, or
\item statistical purposes
\end{itemize}
\end{itemize}
\end{enumerate}



\section{Principle (f): Integrity and Confidentiality}
\begin{itemize}
\item Appropriate technical / organisational measures must be taken against unauthorised / unlawful processing and against accidental loss / damage / destruction of personal data
\item In particular, you will need to:
\begin{itemize}
\item manage your {\em security} to fit the nature of the personal data you hold and the harm that may result from a security breach
\item be clear about {\em who} in your organisation is responsible for ensuring {\em information security}
\item make sure you have the right physical / technical security, backed up by robust policies / procedures
\end{itemize}
\end{itemize}



\section{Principle (g): Accountability}
\begin{itemize}
\item You are responsible for complying with the GDPR and you must be able to demonstrate you compliance
\item There are a number of measures that you can, and in some cases must, take including:
\begin{itemize}
\item adopting and implementing data protection policies
\item taking a `data protection by design and default' approach
\item putting written contracts in place with organisations that process personal data on your behalf
\item maintaining documentation of your processing activities
\item implementing appropriate security measures
\item recording and, where necessary, reporting personal data breaches
\item carrying out data protection impact assessments for uses of personal data that are likely to result in high risk to individuals' interests;
\item appointing a Data Protection Officer and
\item adhering to relevant codes of conduct and signing up to certification schemes.
\end{itemize}
\item You must review and, where necessary, update the measures you put in place.
\end{itemize}



\section{Data protection - Not an Impediment to Life}
In the autumn of 2003 a UK utility company stopped the supply of gas to an elderly couple's homes for non-payment of \pounds140 --- they died

\bigskip   
``Data Act `not to blame' for deaths'' (\href{http://news.bbc.co.uk/1/hi/england/london/3342977.stm}{http://news.bbc.co.uk/1/hi/england/london/3342977.stm})

\bigskip
The Information Commissioner's Office made it plain that in their view the protection of personal information such as unpaid bills could not be seen as overriding the protection of life offered by the social services.



\section{Rights}
The GDPR provides the following rights to individuals:
\begin{enumerate}
\item The right to be informed (about collection and use of their data)
\item The right of access (a copy of their data)
\item The right to rectification (correct inaccurate data and complete incomplete data)
\item The right to erasure (to have personal data deleted, also known as ``the right to be forgotten'')
\item The right to restrict processing (e.g. while considering a correction to their data)
\item The right to object
\begin{itemize}
\item In particular, individuals have an absolute right to stop their data being used for direct marketing
\end{itemize}
\item Rights in relation to automated decision making and profiling
\end{enumerate}



\begin{enumerate}
\setcounter{enumi}{7}
\item The right to data portability (move, copy or transfer personal data from one IT environment to another)
\begin{itemize}
\item only applies in certain circumstances e.g. if the lawful basis is consent or in performance of a contract and the processing is carried out by automated means
\item must be provided in a format that is
\begin{itemize}
\item structured,
\item commonly used and
\item machine-readable (e.g. CSV, JSON, XML).
\end{itemize}
\end{itemize}
\end{enumerate}

\bigskip
Requests to access, rectify, erase, restrict or object must be:
\begin{itemize}
\item replied to within 1 calendar month
\item for free 
\item may be denied in certain circumstances
\end{itemize}



\section{Data Breaches}
\begin{itemize}
\item Breaches must be reported to the ICO within 72 hours if there is risk to a person's rights or freedoms
\item All breaches should be documented regardless of whether they are reported
\item Individuals must be informed of breaches without undue delay
\end{itemize}



\section{International Transfers}
\begin{itemize}
\item The GDPR primarily applies to controllers and processors located in the European Economic Area (the EEA)

\item Personal data can not be transferred to a country outside of the European Economic Area (EEA), unless that country ensures an adequate level of protection for the rights / freedoms of data subjects in relation to the processing of personal data. 

\item Before making a transfer:
\begin{itemize}
\item consider whether you can achieve your aims without actually processing personal data
\item Example: if data is made anonymous, so that it is not possible to identify individuals from it (now or at any point in the future), then the data protection principles will not apply and you are free to transfer the information outside the EEA.
\end{itemize}
\end{itemize}

\bigskip
{\em What will happen after brexit?}





\section{Scenario}
You have set up an online business and hold personal details of your customers in a database. You use the data to email your customers regarding forthcoming offers. An unauthorised person accesses the database and alters product prices and some customers' details.

\bigskip
{\em Discuss how the UK Data Protection Act 2018 relates to this scenario.}



\section{Scenario (Possible points to consider)}
\begin{itemize}
\item You should collect the data in a fair, lawful and transparent manner
\item The data subjects (customers) should give you their consent to processing the data
\item Make it clear to them how you are collecting their information
\item Make it clear why you are collecting their information
\item Offer them the chance to request the removal of their details
\item Offer an opt-in tick box for them to confirm their consent to use their data
\item Take appropriate security precautions
\item Do not transfer the data unless customers gave you the permission to do so
\item Can you share the data with anyone and anywhere in the world?
\end{itemize}



\section{You Should be Able To:}
\begin{itemize}
\item Identify a source that will help you gain an understanding of the UK's data protection policy (Acts 1984, 1998, 2018)
\item Write at least 5 key points about the Data Protection Act 2018
\begin{itemize}
\item 3 fairly general - could include the influence of Europe
\item 2 very detailed - perhaps further investigation into two of the Data protection principles 
\item Write out examples of the principles that make sense to you
\end{itemize}
\item Compare the UK to another country's data protection policy or legislation.
\end{itemize}




\end{document}
