\documentclass{article}
\usepackage{../../../../format}
\lhead{Software Engineering - LSEPI}
\usepackage{hyperref}
%\PassOptionsToPackage{hyphens}{url}
\usepackage{enumerate}

\makeatletter
\g@addto@macro{\UrlBreaks}{\UrlOrds}
\makeatother

%\definecolor{links}{HTML}{2A1B81}
\hypersetup{colorlinks,linkcolor=,urlcolor=blue}


\begin{document}


\begin{center}
	\underline{\huge Privacy}
\end{center}


\section{What is privacy?}
\begin{itemize}
\item Privacy related to notion of access \& ownership
\item Access:
\begin{itemize}
\item Physical proximity to a person
\item Knowledge about a person
\end{itemize}
\item Edmund Byrne: \emph{Privacy is a ``zone of inaccessibility''}
\item Edward Bloustein: \emph{Privacy violations are an affront to human dignity}.
\item Too much individual privacy can harm society.
\end{itemize}



\section{History of privacy}
\begin{itemize}
\item Internal walls: most people didn't have walls.
\item Solo beds: One bed shared with the entire family, and guests.
\item In 1776, US president's comment on British right to search homes sparked the fight for independence: unjustified violation of privacy.
\item Information privacy (1900s)
\begin{itemize}
\item information about citizens were public (the first  American census). 
\item Post office and privacy: post card without envelope (cheap)
\end{itemize}
\end{itemize}



\section{The world has changed}
\begin{itemize}
\item Privacy is the right to ``being left alone''.
\item We have our privacy as long as we are being left alone by other people, organisations or governments.
\item In the era of technology, privacy is more about personal information, not physical boundaries.
\end{itemize}



\section{Do we have a right to privacy?}
\begin{itemize}
\item ``No one shall be subjected to arbitrary interference with his privacy, family, home or correspondence, nor to attacks upon his honour and reputation.''
\item ``Everyone has the right to the protection of the law against such interference or attacks.''
\begin{itemize}
\item Article 12 - Respect for privacy, the home and the family. Universal Declaration of Human Rights, United Nations, 1948. \url{http://www.ohchr.org/EN/UDHR/Documents/UDHR_Translations/eng.pdf}
\end{itemize}
\item ``Everyone has the right to respect for his or her private and family life, home and communications.''
\begin{itemize}
\item Article 7 - Respect for private and family life. Charter of Fundamental Rights of the European Union \url{http://eur-lex.europa.eu/legal-content/EN/TXT/HTML/?uri=CELEX:12012P/TXT&from=EN}
\end{itemize}
\end{itemize}



\section{Exaggeration of privacy}
\begin{itemize}
\item Secrecy versus privacy: what is the difference between them? 
\item Solving crimes: would information disclosure violate privacy principles?
\item Pressure on the nuclear family
\begin{itemize}
\item Loss of pre-industrial age support structures
\end{itemize}
\item Isolation and seclusion
\begin{itemize}
\item meditation in isolation or privately?
\end{itemize}
\item Loneliness
\end{itemize}



\section{Benefits from privacy}
\begin{itemize}
\item Individuality
\begin{itemize}
\item Separate `moral agent' (capable of acting with reference to right and wrong)
\item Being yourself
\item Time without the public mask
\end{itemize}
\item Invention and creativity
\begin{itemize}
\item Focus
\item Silence
\end{itemize}
\item Building relationships
\begin{itemize}
\item Trust and friendship
\begin{itemize}
\item Ladder of privacy (different amounts of personal information to different types of people)
\item Intimacy (How can companies meet the privacy requirements while developing an intimate relationship with their customers?)
\end{itemize}
\end{itemize}
\end{itemize}



\section{Privacy and cyber-technology}
\begin{itemize}
\item Our privacy is affected by cyber-technology whether we owned or even used a networked computer.
\item How much personal information can be acquired about us every day?
\item Consider the way we have been watched every day almost everywhere.
\item Also, consider, web-based applications such as Google Street View.
\item Personal data, including our web-browsing interests, can be easily acquired by organisations. Do they always need our personal information?
\begin{itemize}
\item What is the risk? Information can be sold to third parties!
\end{itemize}
\item Privacy concerns now affect many aspects of our day-to-day lives (commerce, healthcare, work, etc.)
\end{itemize}



\begin{itemize}
\item There are many types of privacy:
\begin{itemize}
\item Consumer privacy (privacy-related threats in e-commerce)
\item Employee/workplace privacy (invisible line managers)
\item Location privacy (RFID or GPS)
\item Medical/healthcare privacy (centralized electronic registry of medical records)
\end{itemize}
\end{itemize}



\section{Identify some of the ways that cyber-technology makes you concerned about your privacy.}
\begin{itemize}
\item Amount of personal information that can now be collected
\item Speed at which personal information can now be transferred and exchanged
\item Duration of time in which personal information can now be retained
\item Kinds of personal information (such as transactional information) that can be acquired
\end{itemize}



\section{Public safety or privacy}
\begin{itemize}
\item Edward Snowden revealed US National Security Agency's violation of privacy law.
\item Prism (a surveillance programme) used for tracking online communication.
\item UK spy agency taps fiber-optic cables.
\item Angela Merkel `warns' Barack Obama! (\url{http://www.bbc.co.uk/news/world-us-canada-23123964})
\item Watchdog warns about the collection of Big Data with the proliferation of CCTV (\url{https://www.theguardian.com/uk-news/2017/mar/14/public-faces-mass-invasion-of-privacy-as-big-data-and-surveillance-merge}) 
\begin{itemize}
\item in UK, 1 CCTV for every 14 people (we are caught 300 times by them daily).  China has 1 CCTV for every 427,000. (\url{http://www.dailymail.co.uk/news/article-1205607/Shock-figures-reveal-Britain-CCTV-camera-14-people--China.html})
\end{itemize}
\end{itemize}



\section{Technologies and Privacy}
\begin{itemize}
\item Loyalty programs (club cards, rewards card, points cards)
\begin{itemize}
\item data collection
\end{itemize}
\item Body scanners
\begin{itemize}
\item Image creation
\item Video: \url{https://www.youtube.com/watch?v=BwsWAxo2xmo}
\end{itemize}
\item `Black boxes'
\begin{itemize}
\item Not just planes
\end{itemize}
\end{itemize}



\section{Surveillance}
\begin{itemize}
\item CCTV
\item Oyster Cards in London
\item Loyalty Cards
\item Email
\begin{itemize}
\item Monitored  
\item Public key cryptography - make the key available upon request
\begin{itemize}
\item The assumption is refusal $\Rightarrow$ guilt
\item always morally correct?
\end{itemize}
\end{itemize}
\end{itemize}



\section{Monitoring you}
\begin{itemize}
\item To ensure that organisation's regulations and procedures are being kept
\item To establish facts, e.g. by
\begin{itemize}
\item cameras
\item agents
\item witnesses
\end{itemize}
\item To prevent / detect crime
\item To investigate / detect unauthorised use of telecommunication systems
\item To ensure the effective operation of the system
\begin{itemize}
\item phone services
\end{itemize}
\item To determine whether a communication within business is personal
\end{itemize}



\section{Biometrics}
\begin{itemize}
\item What might we measure
\begin{itemize}
\item Easily visible characteristics
\begin{itemize}
\item Height, weight, structure
\item Eye colour (Hair colour??)
\item Glasses, hearing aid, wheelchair user\ldots
\end{itemize}
\item Getting a bit more personal
\begin{itemize}
\item Fingerprints \& handprints: recently: ``ear-prints'', ``walk-prints''?
\item Retinal scans
\item Ear lobe capillary patterns
\item Skull measurement
\item DNA
\item Voice recognition
\item article: \url{https://skift.com/2013/05/11/dublin-airport-now-using-biometric-gates-to-quickly-process-passengers/} and video: \url{https://www.sita.aero/resources/type/videos/sita-smart-path}
\end{itemize}
\end{itemize}
\end{itemize}



\section{Regulation of Investigatory Powers Act 2000}
\begin{itemize}
\item \url{http://www.legislation.gov.uk/ukpga/2000/23/contents}
\item A framework for controlling the lawful interception of computer, telephone and postal and Internet communication.
\item Government security and law enforcement authorities can intercept, monitor and investigate electronic data.
\item Under the Act and the associated regulations, organisations (those providing telecommunication services) can monitor and record communications.
\end{itemize}



\section{Regulation of Investigatory Powers Act 2000}
\begin{itemize}
\item \url{http://www.legislation.gov.uk/ukpga/2000/23/contents}
\item The main purpose of the Act is to ensure that the relevant investigatory powers are used in accordance with human rights. 
\item These powers are:
\begin{itemize}
\item the interception of communications;
\item the acquisition of communications data (e.g. billing data);
\item intrusive surveillance (in private homes/private vehicles);
\item covert surveillance in the course of specific operations;
\item the use of covert human intelligence sources (agents, informants, undercover officers);
\item access to encrypted data
\end{itemize}
\end{itemize}



\section{Regulation of Investigatory Powers Act 2000}
\begin{itemize}
\item \url{http://www.legislation.gov.uk/ukpga/2000/23/contents}
\item For each of these powers, the Act will ensure that the law clearly covers:
\begin{itemize}
\item the purposes for which they may be used;
\item which authorities can use the powers;
\item who should authorise each use of the power;
\item the use that can be made of the material gained;
\item independent judicial oversight;
\item a means of redress for the individual
\end{itemize}
\item Not all of these matters need be dealt with in this Act - in many cases existing legislation already covers the ground
\item The Act will work in conjunction with existing legislation, in particular the \emph{Intelligence Services Act 1994}, the \emph{Police Act 1997} and the \emph{Human Rights Act 1998}
\end{itemize}



\section{The act and the regulations are criticised}
\begin{itemize}
\item Security experts and some sectors of the telecommunications industry argued that:
\begin{itemize}
\item there are ways in which the act can be rendered ineffective
\item allowing the seizure of encryption keys will undermine the effectiveness of encryption systems.
\end{itemize}
\end{itemize}




\section{You should be able to:}
\begin{itemize}
\item Identify a source that helps you gain an understanding of the UK's public and government view on privacy
\item Write at least 5 key points about privacy including:
\begin{itemize}
\item Pros and cons of a society that favours privacy (with examples)
\item A summary of the Regulation of Investigatory Powers Act 2000
\end{itemize}
\item Examine your own view on privacy. 
\item Contrast and compare it to others outside of your immediate circle of family and friends. Perhaps different cultures learned from literature, travel, etc.
\end{itemize}

\end{document}
