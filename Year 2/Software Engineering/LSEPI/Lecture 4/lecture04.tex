\documentclass{article}
\usepackage{../../../../format}
\lhead{Software Engineering - LSEPI}
\usepackage{enumerate}
\usepackage{hyperref}
\makeatletter
\g@addto@macro{\UrlBreaks}{\UrlOrds}
\makeatother



\hypersetup{colorlinks,linkcolor=,urlcolor=blue}

\title{Software Engineering --- LSEPI Lecture 4: Intellectual Property Rights}
\author{Dr Konrad Dabrowski\\
\href{mailto://konrad.dabrowski@durham.ac.uk}{konrad.dabrowski@durham.ac.uk}
}
\date{E103 Christopherson Building}

\begin{document}

\begin{center}
	\underline{\huge Intellectual Property Rights}
\end{center}




\section{Objectives}
\begin{enumerate}
\item Become familiar with the main types of intellectual property rights
\item Understand the way these rights can be used to protect software
\item Have an awareness of the limitation of this protection
\end{enumerate}
\begin{itemize}
\item You should use the videos \& training material for Intellectual Properties on DUO:
\begin{itemize}
\item \url{https://duo.dur.ac.uk/bbcswebdav/institution/E-tutorials/Intellectual\%20Property\%20Rights/index.html}
\end{itemize}
\end{itemize}



\section{Intellectual Property}
\begin{itemize}
\item An expression of ideas
\begin{itemize}
\item Brand
\item Invention
\item Design
\item Song
\item Computer Program
\end{itemize}
\item Governed by laws concerned with the control about:
\begin{itemize}
\item who has \emph{access} to your ideas
\item what they can \emph{do} with your ideas
\end{itemize}
\end{itemize}



\section{Intellectual Property Rights (IPR)}
\begin{itemize}
\item Copyright, Designs and Patents Act 1988
\begin{itemize}
\item \url{http://www.legislation.gov.uk/ukpga/1988/48/contents}
\item Copyright
\item Design rights
\item Patent (Next Lecture)
\end{itemize}
\item Trade Marks Act 1994 
\begin{itemize}
\item \url{http://www.legislation.gov.uk/ukpga/1994/26/contents}
\item Trademarks registration (e.g. domain name issues)
\item A domain name can qualify as a trademark, if it is used in connection with a website that offers services to the public
 (yahoo.com, ebay.com, amazon.co.uk \ldots)
\begin{itemize}
\item must be a \emph{distinctive} name
\item \emph{not} ``common'' terms, as dictionary.com, healthanswers.com
\end{itemize}
\end{itemize}
\end{itemize}



\section{Copyright}
\begin{itemize}
\item Is concerned with the right to copy something 
\begin{itemize}
\item The `something' is known as the `work'
\begin{itemize}
\item \emph{literary works}, including novels, instruction manuals, computer programs, song lyrics, newspaper articles and some types of databases 
\item \emph{dramatic works}, including dance / mime 
\item \emph{musical works}
\item \emph{artistic works}, including paintings, engravings, photographs, sculptures, collages, architecture, technical drawings, diagrams, maps and logos 
\item \emph{layouts / typographical arrangements} used to publish a work (e.g. a book)
\item \emph{recordings} of a work (e.g. sound / film)
\item \emph{broadcasts} of a work (e.g. radio show)
\end{itemize}
\end{itemize}
\end{itemize}



\section{Copyright}
\begin{itemize}
\item Automatic right
\begin{itemize}
\item Best to be sure
\begin{itemize}
\item Copyright $\copyright$ $<$company / your name$>$ $<$year it was written$>$
\end{itemize}
\end{itemize}
\item Last for 70 years after the death of the last surviving author
\end{itemize}



\section{Copyright}
\begin{itemize}
\item It protects:
\begin{itemize}
\item Code (including comments)
\begin{itemize}
\item large / small
\item Full suite of programs
\item Fix of a bug
\item Stored on a disc / server
\item Available for download
\item Any language (high / low level)
\end{itemize}
\item Documentation
\begin{itemize}
\item Comments in the code
\item Manuals
\end{itemize}
\item Packaging (may be covered by other rights as well)
\end{itemize}
\end{itemize}



\section{Ownership}
\begin{itemize}
\item Owner of the copyright
\begin{itemize}
\item Initial author(s)
\item Employer (e.g. commissioned work)
\end{itemize}
\item Licences (making money from your copyright)
\begin{itemize}
\item Single user
\item Specified maximum number of users
\item Site licences (e.g. in the labs of the Engineering building)
\item Licence to sell
\item Licence to develop
\begin{itemize}
\item Interoperable products 
\item \url{https://msdn.microsoft.com/en-gb/openspecifications/dn646764}
\end{itemize}
\end{itemize}
\item Transfer of copyright
\begin{itemize}
\item Must be done in writing
\end{itemize}
\end{itemize}



\section{Software Copyright}
\begin{itemize}
\item Copyright Regulations:
\begin{itemize}
\item EU Directive on the Protection of Computer Programs 1991
\end{itemize}
\item What an individual (who has the program licence) can do:
\begin{itemize}
\item Make copies / backup of a program 
\begin{itemize}
\item only for own use
\end{itemize}
\item Alter the program if necessary
\begin{itemize}
\item e.g. to correct errors
\end{itemize}
\item Decompile if necessary
\begin{itemize}
\item to ensure it operates with another program
\item the results of the de-compilation cannot be used for other purposes
\end{itemize}
\end{itemize}
\end{itemize}



\section{Copyright infringement (software piracy)}
\begin{itemize}
\item Primary Infringement of the Copyright
\begin{itemize}
\item Also applies if done unintentionally
\item When an individual breaches the exclusive rights of the owner
\begin{itemize}
\item Do any of the following without the copyright owner's permission: copy / use / sell / adapt the program
\end{itemize}
\item Civil Court
\begin{itemize}
\item Compensation for damages
\end{itemize}
\end{itemize}
\item Secondary Infringement of the Copyright
\begin{itemize}
\item Only intentional infringement
\item When business breaches the exclusive rights of the owner
\begin{itemize}
\item Using / selling unlicensed copies of software products
\end{itemize}
\item Criminal Court
\begin{itemize}
\item Substantial fines, imprisonment, confiscation of copying equipment
\end{itemize}
\end{itemize}
\end{itemize}



\section{Copyright infringement cases:}
\begin{itemize}
\item Recent lawsuit against Spotify: \url{https://www.reuters.com/article/us-spotify-lawsuit/spotify-hit-with-1-6-billion-copyright-lawsuit-idUSKBN1ER1RX}
\item Ed Sheeran may regret Photograph that led to \$20m copyright case: \url{https://www.theguardian.com/music/2017/apr/11/ed-sheeran-20m-dollar-copyright-claim-matt-cardle-x-factor}
\item If a monkey takes a photo, who owns the copyright? \url{https://en.wikipedia.org/wiki/Monkey_selfie_copyright_dispute}
\end{itemize}



\section{Special case for Databases}
\begin{itemize}
\item Copyright applies when: 
\begin{itemize}
\item Contents are original
\item Money and effort were required 
\begin{itemize}
\item geographical data
\end{itemize}
\end{itemize}
\item Database Right (\url{http://www.legislation.gov.uk/uksi/1997/3032/part/III/made})
\begin{itemize}
\item regulation introduced in 1997
\item if there has been ``substantial'' investment in obtaining / verifying / presenting the contents of the database
\item prevents the reuse / extraction of all / a substantial part of the database
\item lasts for up to 15 years
\end{itemize}
\end{itemize}



\section{Registered Designs}
\begin{itemize}
\item Gives exclusive rights to the \emph{look and appearance} of a product
\begin{itemize}
\item others cannot make / offer / market / import / export your design
\item in the UK up to 25 years
\end{itemize}
\item Protects the overall visual appearance of an object (3-dimensional and 2-dimensional)
\begin{itemize}
\item Lines, contours, colour, texture, patterns / ornamentation
\end{itemize}
\item To be registered a design must be:
\begin{itemize}
\item original: not already known
\item Unique: overall impression is that it is different from \emph{any} other known design
\end{itemize}
\item Protection is limited to a geographical region
\begin{itemize}
\item UK, Europe
\end{itemize}
\end{itemize}



\section{Design Rights}
\begin{itemize}
\item The specific legal protection available to unregistered designs in the UK
\item Internal and external shape of an original design
\begin{itemize}
\item only its \emph{3-dimensional} shape
\item no protection for any 2-dimensional parts (e.g. surface patterns)
\end{itemize}
\item Protection will last
\begin{itemize}
\item 10 years from when the product is first marketed or 
\item 15 years from when it was created (whichever is earlier)
\end{itemize}
\item Limited to the UK only
\begin{itemize}
\item \url{https://www.gov.uk/search-registered-design}
\end{itemize}
\end{itemize}



\section{Trademark}
\begin{itemize}
\item Symbol distinguishing some company's goods and services from those of the competitors in a particular geographic area 
\begin{itemize}
\item But more recently will include a \emph{word} rather than an icon
\begin{itemize}
\item Apple 
\item Microsoft, Windows, Office 365
\end{itemize}
\end{itemize}
\item Must be: 
\begin{itemize}
\item Distinctive 
\item Not descriptive of the goods or services
\end{itemize}
\end{itemize}



\section{Trademark}
\begin{itemize}
\item Protects your exclusive right to use the symbol in the UK
\item A registered trademark must have an \textregistered\ after it 
\begin{itemize}
\item Using the symbol if it is not registered is an offence!
\end{itemize}
\item Associated with a Classification
\begin{itemize}
\item Class 9 (computers \& software)
\item Class 42 (sales and services of computers and software)
\item \url{https://www.gov.uk/guidance/how-to-classify-trade-marks}
\item \url{https://www.wipo.int/classifications/nice/nclpub/en/fr/?explanatory_notes=show&lang=en&menulang=en&notion=class_headings}
\end{itemize}
\item Registration renewed every 10 years
\end{itemize}



\section{Trade marks may not be registered if they:}
\begin{itemize}
\item describe your goods / services or any of their characteristics
\item are not distinctive
\item are 3 dimensional shapes
\item are specially protected emblems
\item are offensive
\item are against the law, e.g. promoting illegal drugs
\item are deceptive: there should be nothing in the mark which would lead the public to think that your goods and services have a quality which they do not
\end{itemize}



\section{The 1994 Trade Marks Act makes it an offence to:}
\begin{itemize}
\item apply an unauthorized registered trademark to goods 
\begin{itemize}
\item or the packaging of those goods
\item sell / offer for sale / hire goods that bear an unauthorised trademark
\item import / export goods that bear an unauthorised trademark
\item have in the course of business goods for sale / hire bearing unauthorized trademark
\end{itemize}
\item infringement punishable (by the criminal office) 
\begin{itemize}
\item with fine or up to 2 years imprisonment
\item civil proceeding may also take place as plaintiff may want damages (compensation)
\end{itemize}
\item Trade marks cases:
\begin{itemize}
\item Apple Corps vs. Apple Inc: Who has the right to trademark the word ``apple,'' the Beatles or Apple Inc.?
\url{https://www.business.com/articles/5-trademark-cases-and-what-you-should-learn-from-them/}
\end{itemize}
\end{itemize}



\section{Domain names}
\begin{itemize}
\item Internet Corporation for Assigned Names and Numbers (ICANN)
\begin{itemize}
\item non-profit international organisation
\item responsible for ensuring the `universal resolvability' of Internet addresses 
\item \url{http://www.icann.org/en/system/files/files/participating-08nov13-en.pdf}
\end{itemize}
\item Domain names are intended to be globally unique
\item Potential conflicts between trademarks and domains lead to ``cybersquatting''
\item US academy vs. GoDaddy case
\item \url{https://domainnamewire.com/2020/01/31/breaking-california-attorney-general-delays-org-sale/}
\end{itemize}

\end{document}
