\documentclass{article}
\ProvidesPackage{format}
%Page setup
\usepackage[utf8]{inputenc}
\usepackage[margin=0.7in]{geometry}
\usepackage{parselines} 
\usepackage[english]{babel}
\usepackage{fancyhdr}
\usepackage{titlesec}
\hyphenpenalty=10000

\pagestyle{fancy}
\fancyhf{}
\rhead{Sam Robbins}
\rfoot{Page \thepage}

%Characters
\usepackage{amsmath}
\usepackage{amssymb}
\usepackage{gensymb}
\newcommand{\R}{\mathbb{R}}

%Diagrams
\usepackage{pgfplots}
\usepackage{graphicx}
\usepackage{tabularx}
\usepackage{relsize}
\pgfplotsset{width=10cm,compat=1.9}
\usepackage{float}

%Length Setting
\titlespacing\section{0pt}{14pt plus 4pt minus 2pt}{0pt plus 2pt minus 2pt}
\newlength\tindent
\setlength{\tindent}{\parindent}
\setlength{\parindent}{0pt}
\renewcommand{\indent}{\hspace*{\tindent}}

%Programming Font
\usepackage{courier}
\usepackage{listings}
\usepackage{pxfonts}

%Lists
\usepackage{enumerate}
\usepackage{enumitem}

% Networks Macro
\usepackage{tikz}


% Commands for files converted using pandoc
\providecommand{\tightlist}{%
	\setlength{\itemsep}{0pt}\setlength{\parskip}{0pt}}
\usepackage{hyperref}

% Get nice commands for floor and ceil
\usepackage{mathtools}
\DeclarePairedDelimiter{\ceil}{\lceil}{\rceil}
\DeclarePairedDelimiter{\floor}{\lfloor}{\rfloor}

% Allow itemize to go up to 20 levels deep (just change the number if you need more you madman)
\usepackage{enumitem}
\setlistdepth{20}
\renewlist{itemize}{itemize}{20}

% initially, use dots for all levels
\setlist[itemize]{label=$\cdot$}

% customize the first 3 levels
\setlist[itemize,1]{label=\textbullet}
\setlist[itemize,2]{label=--}
\setlist[itemize,3]{label=*}

% Definition and Important Stuff
% Important stuff
\usepackage[framemethod=TikZ]{mdframed}

\newcounter{theo}[section]\setcounter{theo}{0}
\renewcommand{\thetheo}{\arabic{section}.\arabic{theo}}
\newenvironment{important}[1][]{%
	\refstepcounter{theo}%
	\ifstrempty{#1}%
	{\mdfsetup{%
			frametitle={%
				\tikz[baseline=(current bounding box.east),outer sep=0pt]
				\node[anchor=east,rectangle,fill=red!50]
				{\strut Important};}}
	}%
	{\mdfsetup{%
			frametitle={%
				\tikz[baseline=(current bounding box.east),outer sep=0pt]
				\node[anchor=east,rectangle,fill=red!50]
				{\strut Important:~#1};}}%
	}%
	\mdfsetup{innertopmargin=10pt,linecolor=red!50,%
		linewidth=2pt,topline=true,%
		frametitleaboveskip=\dimexpr-\ht\strutbox\relax
	}
	\begin{mdframed}[]\relax%
		\centering
		}{\end{mdframed}}



\newcounter{lem}[section]\setcounter{lem}{0}
\renewcommand{\thelem}{\arabic{section}.\arabic{lem}}
\newenvironment{defin}[1][]{%
	\refstepcounter{lem}%
	\ifstrempty{#1}%
	{\mdfsetup{%
			frametitle={%
				\tikz[baseline=(current bounding box.east),outer sep=0pt]
				\node[anchor=east,rectangle,fill=blue!20]
				{\strut Definition};}}
	}%
	{\mdfsetup{%
			frametitle={%
				\tikz[baseline=(current bounding box.east),outer sep=0pt]
				\node[anchor=east,rectangle,fill=blue!20]
				{\strut Definition:~#1};}}%
	}%
	\mdfsetup{innertopmargin=10pt,linecolor=blue!20,%
		linewidth=2pt,topline=true,%
		frametitleaboveskip=\dimexpr-\ht\strutbox\relax
	}
	\begin{mdframed}[]\relax%
		\centering
		}{\end{mdframed}}
\lhead{Software Engineering - LSEPI}
\usepackage{enumerate}
\usepackage{hyperref}
\hypersetup{colorlinks,linkcolor=,urlcolor=blue}

\begin{document}

\begin{center}
	\underline{\huge Patents}
\end{center}

\section{Intellectual Property}
\begin{itemize}
\item An expression of ideas
\begin{itemize}
\item Brand
\item Invention
\item Design
\item Song
\item Computer Program
\end{itemize}
\item Governed by laws concerned with your right to control who has access to your ideas and what they can do with your ideas
\end{itemize}



\section{Intellectual Property Rights (IPR)}
\begin{itemize}
\item Copyright, Designs and Patents Act 1988
 \url{http://www.legislation.gov.uk/ukpga/1988/48/contents}
\begin{itemize}
\item Copyright
\item Design rights: protects 3-D \& 2-D objects.
\item Patents (Today's Lecture)
\end{itemize}
\item Trade Marks Act 1994 
\url{http://www.legislation.gov.uk/ukpga/1994/26/contents}
\begin{itemize}
\item Trademarks registration (e.g. domain name issues)
\item A domain name can qualify as a trademark, if it is used in connection with a website that offers services to the public
\item (yahoo.com, ebay.com, amazon.co.uk \ldots)
\begin{itemize}
\item must be a \emph{distinctive} name
\item \emph{not} ``common'' terms, as dictionary.com, healthanswers.com
\end{itemize}
\end{itemize}
\end{itemize}



\section{Brief History of Patents in the UK}
\begin{itemize}
\item `Letters patent'
\begin{itemize}
\item An open document where the Crown granted a monopoly 
\begin{itemize}
\item to skilled individuals with new techniques for the production / sale of goods
\item Weaving of woollen cloths (1331)
\end{itemize}
\end{itemize}
\item `Patent'
\begin{itemize}
\item First recorded patent 1449 for the manufacture of coloured glass by John Utyman. Was first used in Eton College.
\end{itemize}
\item Statutes of Monopolies 1624
\begin{itemize}
\item An Act of the Parliament of England (to reduce the Crown abuse)
\item The basis for the UK's intellectual property law 
\item Restricted patents to inventions
\begin{itemize}
\item 14 years in length
\item stipulated that the patent should not cause harm to trade
\item \url{https://www.uh.edu/engines/epi2002.htm}
\end{itemize}
\end{itemize}
\end{itemize}



\section{Patent Law in the UK}
\begin{itemize}
\item Patents Act 1977
\begin{itemize}
\item The foundation of our current patent laws based on the European Patent Convention (1973)
\begin{itemize}
\item Signed by 27 countries
\end{itemize}
\item \url{https://www.epo.org/law-practice/legal-texts/html/epc/1973/e/ma1.html}
\end{itemize}
\item Copyright, Designs and Patents Act 1988
\begin{itemize}
\item \url{http://www.legislation.gov.uk/ukpga/1988/48/contents}
\end{itemize}
\item Regulatory Reform (Patents) Order 2004
\begin{itemize}
\item \url{http://www.legislation.gov.uk/uksi/2004/2357/pdfs/uksi_20042357_en.pdf}
\end{itemize}
\item Patents Act 2004
\begin{itemize}
\item \url{http://www.legislation.gov.uk/ukpga/2004/16/contents}
\end{itemize}
\end{itemize}



\section{Modern Patents}
\begin{itemize}
\item Most countries have patent laws for the protection of `inventions' 
\begin{itemize}
\item New 
\begin{itemize}
\item Not been ``thought'' of before, whether previously patented or not
\end{itemize}
\item Non-obvious
\item Protection for a ``reasonable'' period of time
\begin{itemize}
\item Use
\item Manufacture
\item Import
\item Sell
\end{itemize}
\end{itemize}
\item A patent can be also:
\begin{itemize}
\item an ``improvement'' of an existing patent
\item an innovation that provides a ``new use'' for an existing invention 
\end{itemize}
\end{itemize}



\section{Patent laws in the UK}
\begin{itemize}
\item In the UK, patents last for at most 20 years
\begin{itemize}
\item To get the full 20 years, a renewal fee must be paid every year, starting 5 years after the filing date.
\end{itemize}
\item An invention can be registered for a patent if it:
\begin{itemize}
\item is new
\item involves an inventive step-invention-shouldn't be obvious
\item is capable of industrial application (technical effect)
\begin{itemize}
\item not an idea or theory, a discovery, a work of art
\end{itemize}
\item is not in an area specifically excluded by  the Patent Act 1977.
\end{itemize}
\end{itemize}



\subsection{Areas specifically excluded}
\begin{itemize}
\item Scientific theories and mathematical methods
\begin{itemize}
\item Gravity and floating point arithmetic
\end{itemize}
\item Aesthetic creations
\begin{itemize}
\item Literary, dramatic, musical or artistic work 
\end{itemize}
\item Presentation of information
\item A newly discovered animal or plant
\item A method of medical treatment or diagnosis
\item A scheme / rule / method for performing a mental act, playing a game or doing business
\item \emph{Some} computer programs
\item Against public policy or morality
\end{itemize}



\section{Requirements for patent}
\begin{itemize}
\item The inventor must not disclose information about the invention before the date of the patent application.
\item Nobody, including the inventor, should use the invention before the patent application date.
\item Officials at the Patent Office must search existing patent and literature. 
\end{itemize}



\section{Benefits of patent protection}
\begin{itemize}
\item the right to stop others from copying / manufacturing / selling / importing your invention
\item make money by selling / licensing your invention
\item safely discuss your invention with others
\item others gain advanced knowledge of technology and can prepare and develop new inventions / patents
\end{itemize}





\section{Obtaining a patent}
\begin{itemize}
\item Applying for patent can be expensive and time consuming. The World Intellectual Property Organisation (WIPO) offers a simplified process.
\item The invention must be novel. The date of the patent application is crucial.
\item Full patent specification must follow within 12 months of the initial application.
\item Full patent specification needs to be prepared by a specialist patent attorney.
\item Can take several years!
\item Should you get a Patent? \url{https://www.gov.uk/guidance/before-you-apply-for-a-patent}
\item Guidance to Patent: \url{https://www.gov.uk/guidance/patents-step-by-step}
\item Before you apply for a patent:
\begin{itemize}
\item check if your idea / invention can be called ``truly novel'' / if it has already been patented / can be considered as obvious
\item Novelty search (or: Patentability search) 
\begin{itemize}
\item done by you and/or professionals (specialized patent examiners)
\end{itemize}
\item UK Intellectual Property Office: \url{https://www.gov.uk/government/organisations/intellectual-property-office}
\item US Patent and Trademark Office: \url{http://www.uspto.gov/} 
\end{itemize}
\item Application for a patent to the Intellectual Property Office
\begin{itemize}
\item Full description of the invention
\item Drawings
\item A set of claims defining your invention
\item Summary of all the technical features
\end{itemize}
\end{itemize}



\section{Enforcing a patent}
If people infringe on your patent, they may challenge you for many things
\begin{itemize}
\item Challenges:
\begin{itemize}
\item no inventive step
\item prior art (background art, state of the art)
\begin{itemize}
\item all information, in any form, publicly known before your application
\end{itemize}
\item information kept secret (usually) does not count as prior art (``trade secret'', ``confidential / classified information'')
\begin{itemize}
\item however, they \emph{may} claim ``prior user rights''  - been using before patent was granted, 
\end{itemize}
\item and thus gain the right to continue using the invention
\item Resolving challenges
\begin{itemize}
\item ask for a non-binding opinion (e.g. by the Intellectual Property Office)
\item mediation (professional patent attorneys)
\item hearings (to resolve a dispute between inventor and IPO or someone else)
\end{itemize}
\end{itemize}
\end{itemize}



\section{Software Patents}
\begin{itemize}
\item In the USA patents are granted to software if: 
\begin{itemize}
\item it is part of a product that is itself eligible for a patent
\item it controls a process that has some physical effect
\item it processes data that arise from the physical world
\end{itemize}
\item In Europe
\begin{itemize}
\item European Patent Office (EPO) \url{https://web.archive.org/web/20180312052803/https://www.epo.org/news-issues/issues/software.html}
\item Under the EPC, a computer program claimed ``as such'' is not a patentable invention (Article 52(2)(c) and (3) EPC). Patents are not granted merely for program listings. Program listings as such are protected by copyright. For a patent to be granted for a computer-implemented invention, a technical problem has to be solved in a novel and non-obvious manner. 
\end{itemize}

\item Software patents have been controversial.
\item The argument for and against software:
\begin{itemize}
\item For: it is illegal and unfair that something is patentable if implemented in hardware but is not patentable if implemented in software
\item Against: the software industry has been very productive and successful mainly due to the efforts of small companies, where Patents are not helpful to them. 
\end{itemize}
\item Many patents that are granted are `bad'. Much software was written before the software patents were thought possible.
\end{itemize}




\section{A few Patent myths}
\begin{itemize}
\item No point in getting a patent, can't sue big companies.
\item Can get worldwide patent
\item The patent police can protect my invention
\item A patent will make me lots of money
\end{itemize}



\section{You should be able to:}
\begin{itemize}
\item Describe in detail patent rights and how to obtain protection in the UK
\begin{itemize}
\item Include the influence of European directives
\item Include some discussion of how the USA approaches patenting of software
\end{itemize}

\item Explain why there is confusion in the software development industry about when and when not to apply for patents.

\item Useful videos \& training material for Patents:
\begin{itemize}
\item \url{https://duo.dur.ac.uk/bbcswebdav/institution/E-tutorials/Intellectual\%20Property\%20Rights/index.html}
\end{itemize}
\end{itemize}

\end{document}
