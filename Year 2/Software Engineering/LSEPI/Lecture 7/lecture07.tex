\documentclass{article}
\ProvidesPackage{format}
%Page setup
\usepackage[utf8]{inputenc}
\usepackage[margin=0.7in]{geometry}
\usepackage{parselines} 
\usepackage[english]{babel}
\usepackage{fancyhdr}
\usepackage{titlesec}
\hyphenpenalty=10000

\pagestyle{fancy}
\fancyhf{}
\rhead{Sam Robbins}
\rfoot{Page \thepage}

%Characters
\usepackage{amsmath}
\usepackage{amssymb}
\usepackage{gensymb}
\newcommand{\R}{\mathbb{R}}

%Diagrams
\usepackage{pgfplots}
\usepackage{graphicx}
\usepackage{tabularx}
\usepackage{relsize}
\pgfplotsset{width=10cm,compat=1.9}
\usepackage{float}

%Length Setting
\titlespacing\section{0pt}{14pt plus 4pt minus 2pt}{0pt plus 2pt minus 2pt}
\newlength\tindent
\setlength{\tindent}{\parindent}
\setlength{\parindent}{0pt}
\renewcommand{\indent}{\hspace*{\tindent}}

%Programming Font
\usepackage{courier}
\usepackage{listings}
\usepackage{pxfonts}

%Lists
\usepackage{enumerate}
\usepackage{enumitem}

% Networks Macro
\usepackage{tikz}


% Commands for files converted using pandoc
\providecommand{\tightlist}{%
	\setlength{\itemsep}{0pt}\setlength{\parskip}{0pt}}
\usepackage{hyperref}

% Get nice commands for floor and ceil
\usepackage{mathtools}
\DeclarePairedDelimiter{\ceil}{\lceil}{\rceil}
\DeclarePairedDelimiter{\floor}{\lfloor}{\rfloor}

% Allow itemize to go up to 20 levels deep (just change the number if you need more you madman)
\usepackage{enumitem}
\setlistdepth{20}
\renewlist{itemize}{itemize}{20}

% initially, use dots for all levels
\setlist[itemize]{label=$\cdot$}

% customize the first 3 levels
\setlist[itemize,1]{label=\textbullet}
\setlist[itemize,2]{label=--}
\setlist[itemize,3]{label=*}

% Definition and Important Stuff
% Important stuff
\usepackage[framemethod=TikZ]{mdframed}

\newcounter{theo}[section]\setcounter{theo}{0}
\renewcommand{\thetheo}{\arabic{section}.\arabic{theo}}
\newenvironment{important}[1][]{%
	\refstepcounter{theo}%
	\ifstrempty{#1}%
	{\mdfsetup{%
			frametitle={%
				\tikz[baseline=(current bounding box.east),outer sep=0pt]
				\node[anchor=east,rectangle,fill=red!50]
				{\strut Important};}}
	}%
	{\mdfsetup{%
			frametitle={%
				\tikz[baseline=(current bounding box.east),outer sep=0pt]
				\node[anchor=east,rectangle,fill=red!50]
				{\strut Important:~#1};}}%
	}%
	\mdfsetup{innertopmargin=10pt,linecolor=red!50,%
		linewidth=2pt,topline=true,%
		frametitleaboveskip=\dimexpr-\ht\strutbox\relax
	}
	\begin{mdframed}[]\relax%
		\centering
		}{\end{mdframed}}



\newcounter{lem}[section]\setcounter{lem}{0}
\renewcommand{\thelem}{\arabic{section}.\arabic{lem}}
\newenvironment{defin}[1][]{%
	\refstepcounter{lem}%
	\ifstrempty{#1}%
	{\mdfsetup{%
			frametitle={%
				\tikz[baseline=(current bounding box.east),outer sep=0pt]
				\node[anchor=east,rectangle,fill=blue!20]
				{\strut Definition};}}
	}%
	{\mdfsetup{%
			frametitle={%
				\tikz[baseline=(current bounding box.east),outer sep=0pt]
				\node[anchor=east,rectangle,fill=blue!20]
				{\strut Definition:~#1};}}%
	}%
	\mdfsetup{innertopmargin=10pt,linecolor=blue!20,%
		linewidth=2pt,topline=true,%
		frametitleaboveskip=\dimexpr-\ht\strutbox\relax
	}
	\begin{mdframed}[]\relax%
		\centering
		}{\end{mdframed}}
\lhead{Software Engineering - LSEPI}


\begin{document}

\begin{center}
	\underline{\huge Professionalism}
\end{center}






\section{Common Characteristics of a Profession}
\begin{itemize}
\item Education and training are required prior to being able to practice the profession
\item Members of the profession:
\begin{itemize}
\item decide the nature of this training
\item control the entry of new members to the profession
\item decide about certain standards of conduct to be followed
\begin{itemize}
\item design mechanisms to enforce them
\end{itemize}
\end{itemize}
\item The profession has one (or more) professional bodies
\item A combination of formal education and training (experience) is necessary to ``rise'' within the profession
\end{itemize}



\section{Professional Bodies}
\begin{itemize}
\item Started by people coming together because of shared interests in some activity, e.g.
\item computers, law, science, photography, mathematics \ldots
\begin{itemize}
\item (a list of bodies in the UK: \url{http://www.careercompanion.co.uk/files/page_element/page_element/List_of_Professional_Bodies_mapped_to_career_interests.pdf})
\end{itemize}
\item BCS (Chartered Institute for IT) started in 1957
\item The main goals of a body are to:
\begin{itemize}
\item transfer knowledge / experience to its junior members 
\item build stronger connections between its members (networking)
\begin{itemize}
\item organize meetings / conferences
\end{itemize}
\item help members find jobs
\item provide professional advice to the members from other senior members
\end{itemize}
\item Other main goals of a professional body:
\begin{itemize}
\item to establish code of professional conduct
\begin{itemize}
\item take disciplinary measures against those who breach the code
\end{itemize}
\item to represent the common interests against others, or protect them from others, who:
\begin{itemize}
\item try to enter it without the proper knowledge, or practice it dishonestly
\end{itemize}
\item establish mechanisms for disseminating knowledge of good practice
\item setting standards of education and experience to be met by members

\item to advise the government about matters within their expertise, e.g.
\begin{itemize}
\item to set up security standards for using some new technologies
\item to certify new products in the market
\end{itemize}
\item being member of a body ``guarantees'' that you satisfy some minimum professional quality standards
\end{itemize}
\end{itemize}


\section{Professional Bodies}
\begin{itemize}
\item IET - \emph{Institution of Engineering and Technology} ``Professional home for life for Engineers and Technologists''
\item IEE - Institute of Electrical Engineers in the UK
\item IEEE -  Institute of Electrical and Electronics Engineers ``The world's largest professional association for the advancement of technology''
\item ACM - Association for Computer Machinery ``Advancing Computing as a Science \& Profession''
\item BCS - the Chartered Institute for IT (Royal Charter in 1984) Formerly the British Computer Society (since 1957)
\item EATCS - \emph{European Association for Theoretical Computer Science}
\item Royal Academy of Engineering
\item ACL - Association for Computational Linguistics
\end{itemize}



\section{Protecting the public}
\begin{itemize}
\item in the past, professional bodies received a royal charter (formal document) to be recognised, e.g. BCS was awarded a royal charter in 1984.
\item In cases of public interest, Parliament may grant the members of a professional body some sort of ``legal monopoly''
\item This can be done in two ways:
\begin{itemize}
\item Reservation of title
\item Reservation of function
\end{itemize}
\begin{definition}[Reservation of title]
You are not allowed to use some professional titles, if you are not a member of a particular professional body
\end{definition}
\begin{definition}[Reservation of function]
You are not allowed to perform certain professional activities, if you are not ``qualified'' by a particular professional body
\end{definition}

\item In England \& Wales, you are not allowed to audit the accounts of \emph{public companies} if you are not a member of:
\begin{itemize}
\item the Institute of Chartered Accountants and the Association of Certified Accountants
\end{itemize}

\item Reservation of title and function could be required
\begin{itemize}
\item Veterinary Surgeons Act 1996
\begin{itemize}
\item you are not allowed to call yourself veterinary surgeon nor carry out surgical operation unless registered with the Royal College of Veterinary Surgeons (RCVS)
\end{itemize}
\item in the USA, title and function reserved not to members of professional bodies but to those registered by a state government
\item in the UK, membership of the Architects Registration Board now replaces membership in the Royal Institute of British Architects as a requirement for calling yourself an architect
\end{itemize}
\end{itemize}






\section{Software Development as Engineering}
\begin{itemize}
\item Software development and information systems usually regarded as a branch of engineering
\begin{itemize}
\item people practicing it are called engineers
\end{itemize}
\item Engineering and software development share some characteristics
\begin{itemize}
\item Engineers design and build a wide variety of objects
\item Designing and building objects must work properly and meet predetermined requirements (functionality, performance, and reliability)
\item designing and building objects must be completed within constrained time and budget
\end{itemize}
\end{itemize}



\section{Are we engineers in America?}
\begin{itemize}
\item In the USA, Engineering is reserved both in title and function
\item In the USA, it is illegal: 
\begin{itemize}
\item to call yourself an engineer in a state unless you are registered with the State Engineers Registration Board
\item do engineering work without the supervision of a registered engineer
\item for a company to have the word ``engineering'' in its name, unless they employ one or more registered engineers
\end{itemize}
\item Academic degree programs with ``engineering'' in the title must be taught (mostly) by registered engineers
\item Must have approved bachelor's degree, 4 years experience working for a registered engineer, passed 2 eight hour public exams
\end{itemize}



\section{Are we Engineers in the UK?}
\begin{itemize}
\item Neither the title nor the function of Engineer are reserved in the UK
\item Engineering Council:   \url{http://www.engc.org.uk/}
\begin{itemize}
\item the UK regulatory body for the engineering profession 
\item if you are registered, then you are ``Chartered Engineer''
\item sets standards of education, experience and competency for:
\begin{itemize}
\item initial registration
\item continuing registration
\end{itemize}
\end{itemize}

\item Main objectives of the Engineering Council:
\begin{itemize}
\item to advance education in science and engineering
\item to promote the practice of engineering for the ``public good''
\item to advise the government and represent the UK internationally
\item to maintain a register of accredited programmes in Higher Education
\item license appropriate bodies to admit members to the register
\end{itemize}
\item Useful link with many documents describing the standards of the Engineering Council:
\begin{itemize}
\item \url{http://www.engc.org.uk/standards-guidance/standards/uk-spec/}
\end{itemize}
\end{itemize}



\section{International Recognition of Engineering qualification}
\begin{itemize}
\item EU mobility directives
\begin{itemize}
\item movement of qualified professionals in EU countries
\item This done through FEANI 1951 (European Federation of National Engineering Associations) \url{https://www.feani.org/feani/what-feani}
\begin{itemize}
\item maintains a register of EU engineers (EurIng)
\end{itemize}
\end{itemize}
\item Washington Accord 1989 (\url{https://web.archive.org/web/20160306142535/http://washingtonaccord.org/})
\begin{itemize}
\item Australia, UK, New Zealand, USA, Canada, Ireland etc.
\item standards and procedures are used to accredit academic education of an engineer
\item the agreement is fairly limited to undergraduate levels
\end{itemize}
\end{itemize}



\section{Compulsory Registration of Software Engineers (SE)}
\begin{itemize}
\item many disasters directly related to lack of professional competence of SEs
\begin{itemize}
\item system Therac-25 in the USA 
\item the London Ambulance System in the UK 
\end{itemize}
\item Dangers arising from professional incompetence led to calls for compulsory registration of SEs
\item All software must be written by registered SEs
\item Little progress in this direction has been made
\end{itemize}





\end{document}
