\documentclass{article}
\input{../../../../format}
\lhead{Software Engineering - LSEPI}
\usepackage{enumerate}
\usepackage{hyperref}

\hypersetup{colorlinks,linkcolor=,urlcolor=blue}

\makeatletter
\g@addto@macro{\UrlBreaks}{\UrlOrds}
\makeatother


\title{Software Engineering --- LSEPI Lecture 8: Equality}
\author{Dr Konrad Dabrowski\\
\href{mailto://konrad.dabrowski@durham.ac.uk}{konrad.dabrowski@durham.ac.uk}
}
\date{E103 Christopherson Building}

\begin{document}

\begin{center}
	\underline{\huge Equality}
\end{center}




\section{Equality Act 2010}
\url{http://www.legislation.gov.uk/ukpga/2010/15/contents}
\begin{itemize}
\item The Equality Act combines and strengthens a number of previous laws to help tackle discrimination and inequality
\item N.B. It does {\bf not} apply in Northern Ireland
\item It protects against discrimination from {\bf nine} {\em protected characteristics}.
\item What are they?

\begin{itemize}
\item age
\item disability
\item gender reassignment
\item being married and or in a civil partnership
\item being pregnant or on maternity leave
\item race, including colour, nationality, ethnic or national origin
\item religious or philosophical belief
\item sex
\item sexual orientation
\end{itemize}
\end{itemize}



\section{Philosophical Belief}
\begin{itemize}
\item To be recognised as a philosophical belief, a belief must satisfy the Grainger criteria:

\begin{itemize}
\item Must be genuinely held
\item Must be a belief, as opposed to an opinion or viewpoint based on information
\item Must be a belief as to a weighty and substantial aspect of human life and behaviour
\item Must have a level of cogency, seriousness, cohesion and importance and
\item Must be worthy of respect in a democratic society and not be incompatible with human dignity or conflict with the fundamental rights of others
\end{itemize}
\item Is ``ethical veganism'' a philosophical belief?
\url{https://www.bbc.co.uk/news/uk-50981359}
\end{itemize}



\begin{itemize}
\item The act protects from discrimination:

\begin{itemize}
\item at work
\item in education
\item as a consumer
\item when using public services
\item when buying or renting property
\item as a member or guest of a private club or association
\item if you're associated with someone who has a protected characteristic, for example a family member or friend
\item if you've complained about discrimination or supported someone else’s claim
\end{itemize}
\end{itemize}



\section{Types of Discrimination}
\begin{itemize}
\item Discrimination can come in a number of forms:
\begin{definition}[Direct discrimination]
Treating someone with a protected characteristic less favourably than others
\end{definition}
\begin{definition}[Indirect discrimination]
Putting rules or arrangements in place that apply to everyone, but that put someone with a protected characteristic at an unfair disadvantage
\end{definition}
\begin{definition}[Harassment]
Unwanted behaviour linked to a protected characteristic that violates someone's dignity or creates an offensive environment for them
\end{definition}
\begin{definition}[Victimisation]
Treating someone unfairly because they've complained about discrimination or harassment
\end{definition}

\item However, it {\bf can} be lawful to have specific rules or arrangements in place, as long as they can be justified (if it is a ``proportionate means of achieving a legitimate aim'')
\end{itemize}





\section{Positive Action}
\begin{itemize}
\item You can do something voluntarily to help people with a protected characteristic. This is called {\em positive action}.

\item Taking positive action is legal if people with a protected characteristic:

\begin{itemize}
\item     are at a disadvantage
\item     have particular needs
\item     are under-represented in an activity or type of work
\end{itemize}
\item Positive action can even be used in hiring practices.
\end{itemize}




\begin{itemize}
\item The law protects you against discrimination at work, including:

\begin{itemize}
\item dismissal
\item employment terms and conditions
\item pay and benefits
\item promotion and transfer opportunities
\item training
\item recruitment
\item redundancy
\end{itemize}

\item Some forms of discrimination are only allowed if they’re needed for the way the organisation works, for example:

\begin{itemize}
\item a Roman Catholic school restricting applications for admission of pupils to Catholics only
\item employing only women in a health centre for Muslim women
\end{itemize}

\item You're also protected from being treated unfairly because of:

\begin{itemize}
\item trade union membership or non-membership
\item being a fixed-term or part-time worker
\end{itemize}

\end{itemize}



\section{Disability}

\begin{itemize}
\item If you're disabled you have the same rights as other workers. Employers should also make {\em reasonable adjustments} to help disabled employees and job applicants with:

\begin{itemize}
\item application forms, for example providing forms in Braille or audio formats
\item aptitude tests, for example giving extra time to complete the tests
\item dismissal or redundancy
\item discipline and grievances
\item interview arrangements, such as providing wheelchair access, communicator support
\item making sure the workplace has the right facilities and equipment for disabled workers or someone offered a job
\item promotion, transfer and training opportunities
\item terms of employment, including pay
\item work-related benefits like access to recreation or refreshment facilities
\end{itemize}
\end{itemize}





\begin{itemize}
\item If you think you've been unfairly discriminated against, what can you do?
 

\begin{itemize}
\item complain directly to the person or organisation
\item use someone else to help you sort it out (called `mediation' or `alternative dispute resolution')
\item make a claim in a court or tribunal
\item Can also talk to:
\begin{itemize}
\item Acas (Advisory, Conciliation and Arbitration Service),
\item Citizens Advice or
\item a trade union representative.
\end{itemize}
\end{itemize}
\end{itemize}



\section{Exceptions}
\begin{itemize}
\item     Priests, monks, nuns, rabbis and ministers of religion.
\item     Actors and models in the film, television and fashion industries (a British Chinese actress for a specific role, for instance).
\item     Special employment training programmes aimed at ethnic minorities, ex-offenders, young adults, the long term unemployed, or people with physical or learning disabilities.
\item     Employment where there are cultural sensitivities (such as a documentary where male victims of domestic violence need to be interviewed by a male researcher, or a gay men's domestic violence helpline).
\item     Where safety or operational efficiency could be jeopardised.
\end{itemize}



\section{Exceptions}
\begin{itemize}
\item     Political parties who use `protected characteristics' (age, race, religion, sex, sexual orientation) as candidate selection criteria; though these `Selection arrangements do not include short-listing only such persons as have a particular protected characteristic' -- other than sex, which may still be used to prejudice selection in some circumstances (e.g. all-women/all-men shortlists).
\item     Local support staff who work in embassies and high commissions, by virtue of diplomatic immunity.
\item     Where national security could be jeopardised.
\end{itemize}






\end{document}
