\documentclass{article}[18pt]
\input{../../../../format}
\lhead{Software Engineering - Measurement and Evaluation}


\begin{document}
\begin{center}
\underline{\huge Code metrics}
\end{center}
\section{Characteristics of a metric}
\begin{itemize}
	\item A metric not only needs to have a clear and relevant association with an attribute, its values need to change in accordance with our expectations when the attribute changes
	\item The problem here is that the attributes we tend to associate with describing software are often rather imprecise, and can mean different things in different contexts - so the rules for combining them might not always be clear cut
\end{itemize}
\section{What things do we need to measure?}
Some examples of what we want to know about include:
\begin{itemize}
	\item Code structure
	\item Control flow through code
	\item Information flow (and relationships)
	\item "object" properties
	\item Execution of code
	\item Steps in development
	\item Specifications
	\item Design prope
\end{itemize}





\end{document}
