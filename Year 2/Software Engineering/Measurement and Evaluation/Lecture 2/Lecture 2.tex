\documentclass{article}[18pt]
\input{../../../../format}
\lhead{Software Engineering - Measurement and Evaluation}


\begin{document}
\begin{center}
\underline{\huge Code metrics}
\end{center}
\section{Characteristics of a metric}
\begin{itemize}
	\item A metric not only needs to have a clear and relevant association with an attribute, its values need to change in accordance with our expectations when the attribute changes
	\item The problem here is that the attributes we tend to associate with describing software are often rather imprecise, and can mean different things in different contexts - so the rules for combining them might not always be clear cut
\end{itemize}
\section{What things do we need to measure?}
Some examples of what we want to know about include:
\begin{itemize}
	\item Code structure
	\item Control flow through code
	\item Information flow (and relationships)
	\item "object" properties
	\item Execution of code
	\item Steps in development
	\item Specifications
	\item Design properties
	\item Testing outcomes
\end{itemize}
\section{Attributes and Measures for Code}
Two key attributes when thinking about development and evolution are:
\begin{itemize}
	\item Size
	\item Relationships between elements
\end{itemize}
\subsection{Halstead's Measures}
Focused on counting lexical tokens in programs to extract measure like
\begin{itemize}
	\item Number of unique operators
	\item Number of unique operands
	\item Total occur
	\item Total occurrences of operands
	\item Size of a program	
\end{itemize}
\subsection{Lines of code}
Widely used as a size metric as it is easy to measure, however it has limitations:
\begin{itemize}
	\item Do you count empty lines?
	\item Do you count comments
	\item Statements can be expanded onto more lines for readability
\end{itemize}
\subsection{Cyclomatic Complexity}
Defined as
$$V(G)=e-n+2$$
or can also be computed as number of decision points+1\\
\\
This means there is no penalty for writing more legible code
\subsubsection{Compound conditions}
\begin{itemize}
	\item Compound conditions are a problem for this as different routes could require different numbers of decisions
	\item The original rule treated any compound condition as a single decision
	\item There has been a suggestion of treating each element in a compound condition as a separate decision to get an upper bound
\end{itemize}

\end{document}
