\documentclass{article}[18pt]
\input{../../../../format}
\lhead{Software Engineering}


\begin{document}
\begin{center}
\underline{\huge OO and Usability Metrics}
\end{center}
\section{What do we want to measure?}
A key goal is one of assessing attributes that relate to such concepts as "separation of concerns" and "information hiding" such as
\begin{itemize}
	\item The interactions between elements
	\item The way that elements are grouped
\end{itemize}
Complications added by the OO paradigm include:
\begin{itemize}
	\item Inheritance
	\item Polymorphism
	\item The class-instance distinction
\end{itemize}
\section{Chidamber and Kemerer's metrics}
\begin{itemize}
	\item The six C\&K metrics are the most widely used OO metrics. Defined by employing a model related to the major features of the "object model" as well as to established concepts such as coupling and cohesion
	\item The concepts provide a set of indirect measures used to assess different object attributes. The C\&K metrics then provide direct measures that are meant to act as surrogates for the concepts
\end{itemize}
\begin{definition}[Surrogate]
	Something we can measure that we believe relates to the property of interest
\end{definition}
\subsection{What are they used for?}
\begin{itemize}
	\item To identify the classes that are most likely to contain faults
	\item Identify where changes may have increased the likelihood of errors occuring
\end{itemize}
\subsection{WMC Weighted methods/class}
Formula for this is
$$WMC=\sum_{i=1}^{n}c_i$$
where $c_i$ is the complexity of method i
\begin{itemize}
	\item Main rationale for this metric is that methods are properties of objects, and so the complexity of the whole is a function of the set of individual properties
	\item C\&K suggest that the number of methods and their combined complexity reflects the effort required to develop and maintain the object + possible impact on children
	\item Weights are measures that are considered to relate to the static complexity of each method by using such attributes as length, and metrics such as cyclomatic complexity
	\item If all weights are set to 1, this reduces to a count of methods
\end{itemize}
Usefulness of WMC:
\begin{itemize}
	\item For weights a key issue is the need to devise some way of assigning meaningful values to these that can be extracted from the design/code
	\item Commonly used are V(G), LOC or simply a value of 1
	\item An increase in WMC is a reasonably good indicator of the likelihood of there being an increase in defects for that class
\end{itemize}
\subsection{DIT: Depth of inheritance tree}
DIT is basically a count of tree height from a node to the root of a tree:
\begin{itemize}
	\item A measure that identifies how many ancestor classes can potentially affect a given class
	\item Deeper trees implicitly constitute greater design complexity since they require an understanding of more super-classes
	\item Wider trees are more loosely coupled, but may also indicate that the commonality between classes is not exploited well
	\item DIT offers no significant predictive ability for fault proneness
\end{itemize}



\end{document}