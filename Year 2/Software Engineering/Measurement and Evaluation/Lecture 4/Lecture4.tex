\documentclass{article}[18pt]
\input{../../../../format}
\lhead{Software Engineering}


\begin{document}
\begin{center}
\underline{\huge Empirical studies}
\end{center}
\begin{definition}[Evaluate]
	To judge the quality of
\end{definition}
\section{Empirical vs Experimental}
\begin{definition}[Empirical]
Relying on observation and experiment rather than theory
\end{definition}
\begin{definition}[Experiment]
A study in which an intervention is deliberately controlled to observe its effects
\end{definition}
\section{Forms of measure}
\begin{definition}[Quantitative evaluation]
Used to determine whether a cause effect relationship exists
\begin{itemize}
	\item May test the effect of some intervention
	\item Uses measures based on "counting" scales
	\item Can employ statistical forms to aid analysis
\end{itemize}
\end{definition}
\begin{definition}[Qualitative evaluation]
Studies entities in their natural setting, usually through observation:
\begin{itemize}
	\item Analysis involves interpretation based on explanations
	\item Recognises that there may be different interpretations
\end{itemize}
\end{definition}
\section{Primary or secondary}
\begin{definition}[Primary study]
Directly study the entity of interest by making observations and measurements
\end{definition}
\begin{definition}[Secondary study]
Seek to aggregate the outcomes of many different primary studies
\end{definition}
\section{The research protocol}
A good evaluation process needs to be:
\begin{itemize}
	\item Objective
	\item Unbiased
\end{itemize}
And should avoid "fishing" for results from the outcomes.



\end{document}