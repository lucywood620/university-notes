\documentclass{article}[18pt]
\ProvidesPackage{format}
%Page setup
\usepackage[utf8]{inputenc}
\usepackage[margin=0.7in]{geometry}
\usepackage{parselines} 
\usepackage[english]{babel}
\usepackage{fancyhdr}
\usepackage{titlesec}
\hyphenpenalty=10000

\pagestyle{fancy}
\fancyhf{}
\rhead{Sam Robbins}
\rfoot{Page \thepage}

%Characters
\usepackage{amsmath}
\usepackage{amssymb}
\usepackage{gensymb}
\newcommand{\R}{\mathbb{R}}

%Diagrams
\usepackage{pgfplots}
\usepackage{graphicx}
\usepackage{tabularx}
\usepackage{relsize}
\pgfplotsset{width=10cm,compat=1.9}
\usepackage{float}

%Length Setting
\titlespacing\section{0pt}{14pt plus 4pt minus 2pt}{0pt plus 2pt minus 2pt}
\newlength\tindent
\setlength{\tindent}{\parindent}
\setlength{\parindent}{0pt}
\renewcommand{\indent}{\hspace*{\tindent}}

%Programming Font
\usepackage{courier}
\usepackage{listings}
\usepackage{pxfonts}

%Lists
\usepackage{enumerate}
\usepackage{enumitem}

% Networks Macro
\usepackage{tikz}


% Commands for files converted using pandoc
\providecommand{\tightlist}{%
	\setlength{\itemsep}{0pt}\setlength{\parskip}{0pt}}
\usepackage{hyperref}

% Get nice commands for floor and ceil
\usepackage{mathtools}
\DeclarePairedDelimiter{\ceil}{\lceil}{\rceil}
\DeclarePairedDelimiter{\floor}{\lfloor}{\rfloor}

% Allow itemize to go up to 20 levels deep (just change the number if you need more you madman)
\usepackage{enumitem}
\setlistdepth{20}
\renewlist{itemize}{itemize}{20}

% initially, use dots for all levels
\setlist[itemize]{label=$\cdot$}

% customize the first 3 levels
\setlist[itemize,1]{label=\textbullet}
\setlist[itemize,2]{label=--}
\setlist[itemize,3]{label=*}

% Definition and Important Stuff
% Important stuff
\usepackage[framemethod=TikZ]{mdframed}

\newcounter{theo}[section]\setcounter{theo}{0}
\renewcommand{\thetheo}{\arabic{section}.\arabic{theo}}
\newenvironment{important}[1][]{%
	\refstepcounter{theo}%
	\ifstrempty{#1}%
	{\mdfsetup{%
			frametitle={%
				\tikz[baseline=(current bounding box.east),outer sep=0pt]
				\node[anchor=east,rectangle,fill=red!50]
				{\strut Important};}}
	}%
	{\mdfsetup{%
			frametitle={%
				\tikz[baseline=(current bounding box.east),outer sep=0pt]
				\node[anchor=east,rectangle,fill=red!50]
				{\strut Important:~#1};}}%
	}%
	\mdfsetup{innertopmargin=10pt,linecolor=red!50,%
		linewidth=2pt,topline=true,%
		frametitleaboveskip=\dimexpr-\ht\strutbox\relax
	}
	\begin{mdframed}[]\relax%
		\centering
		}{\end{mdframed}}



\newcounter{lem}[section]\setcounter{lem}{0}
\renewcommand{\thelem}{\arabic{section}.\arabic{lem}}
\newenvironment{defin}[1][]{%
	\refstepcounter{lem}%
	\ifstrempty{#1}%
	{\mdfsetup{%
			frametitle={%
				\tikz[baseline=(current bounding box.east),outer sep=0pt]
				\node[anchor=east,rectangle,fill=blue!20]
				{\strut Definition};}}
	}%
	{\mdfsetup{%
			frametitle={%
				\tikz[baseline=(current bounding box.east),outer sep=0pt]
				\node[anchor=east,rectangle,fill=blue!20]
				{\strut Definition:~#1};}}%
	}%
	\mdfsetup{innertopmargin=10pt,linecolor=blue!20,%
		linewidth=2pt,topline=true,%
		frametitleaboveskip=\dimexpr-\ht\strutbox\relax
	}
	\begin{mdframed}[]\relax%
		\centering
		}{\end{mdframed}}
\lhead{Software Engineering}


\begin{document}
\begin{center}
\underline{\huge Requirements Engineering}
\end{center}
\section{Gathering Requirements}
\begin{itemize}
	\item A software requirement is a functional or non-functional need to be implemented in the system
	\item Functional means providing particular service to the user
	\begin{itemize}
		\item For example, in context to an online purchase application the functional requirement will be when customer selects "Checkout" they must be able to look at the latest items in their basket before paying
	\end{itemize}
	\item A software requirement can also be non-functional, it can be a performance requirement
	\begin{itemize}
		\item For example, a non-functional requirement is where a click on a link results in a $<$0.5s delay to give a response
	\end{itemize}
\end{itemize}
\section{Functional Requirements}
\begin{itemize}
	\item Descriptions of data to be entered into the system
	\item Descriptions of operations performed on each system
	\item Descriptions of work flows performed by the system
	\item Descriptions of system reports or other outputs
	\item Who can enter data into the system
	\item How the system meets applicable regulatory requirements
	\item The user can get some meaningful work done
\end{itemize}
\section{Non-Functional Requirements}
\begin{itemize}
	\item Characteristics of the system which cannot be expressed as functions:
	\begin{itemize}
		\item Maintainability, portability, usability, security, safety, reliability, performance ..
		\item These characteristics can be measurable
		\item E.g a system sub-component that needs to have a response time or $<$1s 95\% of the time
		\item Constraints are NFR
		\item PM issues are NFR (costs, schedule, logistics)
	\end{itemize}
\end{itemize}
\section{RE: Elicitation/discovery}
Develop a plan
\begin{itemize}
	\item What information should be discovered?
	\item What sources should be used?
	\item What mechanisms or techniques should be employed
\end{itemize}
The information to discover:
\begin{itemize}
	\item A description of the problem domain
	\item The basic type of application
	\item The identity of the stakeholders
	\item The main motivation behind the development
	\item List of problems requiring solutions (the requirements)
	\item Any stakeholder imposed constraints upon behaviour or structure of solution system
\end{itemize}
\section{Techniques for discovery}
Interviewing:
\begin{itemize}
	\item Different types of interviews
	\item Good for getting an overall understanding of what, how they interact with the system and any difficulties
	\item Not good for understanding specific domain requirements
\end{itemize}
Scenarios
\begin{itemize}
	\item Useful for adding detail to an outline requirements description
	\item Starts with an outline of the interaction and adds detail to create a complete description - must include a description of:
	\begin{itemize}
		\item What system and users expect when scenario starts
		\item Normal flow of events
		\item What can go wrong and how it is handled
		\item Information on other activities which might be going on at the same time
		\item System state when scenario finishes
	\end{itemize}
\end{itemize}
\section{RE: Analysis}
Having determined the basic list of requirements we now analyse them
\begin{itemize}
	\item Check completeness
	\item Remove inconsistencies
	\item Conflict resolution
	\item Revisit assumptions
	\item Prioritise
\end{itemize}
\section{Prioritise}
\begin{itemize}
	\item Must have
	\begin{itemize}
		\item Fundamental requirements, without which the system will not work
		\item Define the MVP
	\end{itemize}
	\item Should have
	\begin{itemize}
		\item Considered important but the lack of these can be worked around
	\end{itemize}
	\item Could have
	\begin{itemize}
		\item Extended functionality but can be left out without undermining the project
	\end{itemize}
	\item Won't have
	\begin{itemize}
		\item The waiting list, requirements that can wait until a later development date
	\end{itemize}
\end{itemize}
\end{document}