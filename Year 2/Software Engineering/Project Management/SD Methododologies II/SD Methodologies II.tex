\documentclass{article}[18pt]
\usepackage{../../../../format}
\lhead{Software Engineering}


\begin{document}
\begin{center}
\underline{\huge SD Methodologies II}
\end{center}
\section{SCRUM}
The Scrum process is suited for projects with rapidly changing or highly emergent requirements. Its focus is on managing iterative development rather than specific agile practices.\\
\\
Scrum is empirical in that it provides a means for teams to establish a hypothesis of how they think something works, try it out, reflect on the experience, and make the appropriate adjustments\\
\\
When to use:
\begin{itemize}
	\item SCRUM is best suited in the base where a cross functional team is working in a product development setting where there is a non trivial amount of work that lends itself to being split into more than one 1-4 week iteration
\end{itemize}
\subsection{Roles}
\begin{defin}[Product Owner (PO)]
Single person who has the vision behind the product dev; makes a final call on what the priorities of the work are: what is top of the list in the sprint backlog; makes business decisions focused on what not how and people go through him/her to the team
\end{defin}

\begin{defin}[Scrum Team]
No hierarchy; 4-9 people ideally; build a shippable product in each sprint; each sprint they improve and collaboration increases between the team
\end{defin}


\begin{defin}[Scrum Master (SM)]
Has NO managerial authority over the team; facilitates team needs without authority.  Protects teams from distractions and interruptions; and teaches people to use scrum, good engineering practices, and enforces time boxes.
\end{defin}






\end{document}