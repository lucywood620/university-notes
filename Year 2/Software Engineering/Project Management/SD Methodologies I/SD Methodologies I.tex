\documentclass{article}[18pt]
\usepackage{../../../../format}
\lhead{Software Engineering}


\begin{document}
\begin{center}
\underline{\huge SD Methodologies I (and OOP)}
\end{center}
\section{Waterfall model}
\begin{itemize}
	\item Plan vs course change
	\item The waterfall model is often considered out of date
	\item Not an inflexible process but often created an inflexible instance (process vs people)
	\item Planning allows for more oversight and control
\end{itemize}
When to use:
\begin{itemize}
	\item Requirements very well documented, clear and fixed
	\item Product definition is stable
	\item Technology is understood and is not dynamic
	\item There are no ambiguous requirements
	\item Ample resources with required expertise are available to support the product
	\item The product is short
\end{itemize}
\subsection{Advantages}
\begin{itemize}
	\item Simple and easy to understand and use
	\item Easy to manage due to the rigidity of the model. Each phase has specific deliverables and a review process
	\item Phases are processed and completed one at a time
	\item Works well for smaller projects where requirements are very well understood
	\item Clearly defined stages
	\item Well understood milestones
	\item Easy to arrange tasks
	\item Process and results are well documented
	\item Iteration occurs within activities 
\end{itemize}
\subsection{Disadvantages}
\begin{itemize}
	\item No working software is produced until late during the life cycle
	\item High amounts of risk and uncertainty
	\item Not a good model for complex and object-oriented project
	\item Poor model for long and ongoing projects
	\item Not suitable for the projects where requirements are at a moderate to high risk of changing. So, risk and uncertainty is high with this process model
	\item It is difficult to measure progress within stages
	\item Can't accommodate changing requirements
	\item Adjusting scope during the life cycle can end a project
	\item Integration is done as a "big-bang" at the very end, which doesn't allow identifying any technological or business bottleneck or challenges early
\end{itemize}
\section{Agile}
\subsection{Manifesto}
We are uncovering better ways of developing software by doing it and helping others do it. Through this work we have come to value:
\begin{itemize}
	\item Individuals and interactions over processes and tools
	\item Working software over comprehensive documentation
	\item Customer collaboration over contract negotiation
	\item Responding to change over following a plan
\end{itemize}
\subsection{Principles}
\begin{enumerate}
	\item Our highest priority is to satisfy the customer through early and continuous delivery of valuable software
	\item Welcome changing requirements, even late in development. Agile processes harness change for the customer's competitive advantage
	\item Deliver working software frequently, from a couple of weeks to a couple of months, with a preference to the shorter timescale
	\item Business people and developers must work together daily throughout the project
	\item Build projects around motivates individuals. Give them the environment and support they need, and trust them to get the job done
	\item The most efficient and effective way of conveying information to and within a development team is face-to-face conversation
	\item Working software is the primary measure of progress
	\item Agile processes promote sustainable development. The sponsors, developers, and users should be able to maintain a constant pace indefinitely
	\item Continuous attention to technical excellence and good design enhances agility
	\item Simplicity - the art of maximising the work not done - is essential
	\item The best architectures, requirements and designs emerge from self-organising teams
	\item At regular intervals, the team reflects on how to become more effective, then tunes and adjusts its behaviour accordingly
\end{enumerate}





\end{document}