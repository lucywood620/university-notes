\documentclass{article}[18pt]
\input{../../../../format}
\lhead{Software Engineering}


\begin{document}
\begin{center}
\underline{\huge Scheduling and Risks II}
\end{center}
\section{Scheduling}
\subsection{Tracking Progress}
\begin{itemize}
	\item Do we understand customer's needs?
	\item Can we design a system to solve customer's problems or satisfy customer's needs?
	\item How long will it take to develop the system?
	\item How much will it cost to develop the system
\end{itemize}
\subsection{Project Scheduling}
\begin{enumerate}
	\item Capturing and Sequencing Activities
	\item Assigning resources and establishing durations
	\item Verifying the schedule and critical path
	\item Conducting a schedule risk analysis
	\item Updating the schedule
	\item Maintaining a baseline schedule
\end{enumerate}
\begin{itemize}
	\item Understand how to plan, monitor and control projects using PERT/CPM
	\begin{itemize}
		\item Program Evaluation and Review Technique
		\begin{itemize}
			\item A graphic representation of a project's schedule
			\item Shows the sequence of tasks
			\item Shows which tasks can be performed simultaneously
			\item Pert applies a 3-point weighted average duration estimate
$$T_e=(T_o+4\times T_m + T_p)\div 6$$
$T_e$ = Expected Duration\\
$T_o$ = Optimistic Duration\\
$T_m$ = Most Likely Duration\\
$T_p$ = Pessimistic Duration
		\end{itemize}
	
		\item Critical path method
		\begin{itemize}
			\item Analysis of path among milestones
			\item Identify minimum amount of time to complete the project
			\item Reveals those activities that are most critical to completing the project on time
			\item Calculate available time, real time and slack time as well as latest and earliest start time
		\end{itemize}
	\end{itemize}
	\item Determine the earliest finish, the latest start, the latest finish and slack times for each activity
	\item Understand the impact of variability in activity times on the project completion time
\end{itemize}
\subsection{Schedule Risks}
A schedule is one of the project drivers that is most susceptible to risk
\begin{itemize}
	\item Schedules comprise multiple inter-dependant activities, each of which usually contain multiple uncertainties
	\item \textit{Proactive scheduling}: Add time up front to deal with uncertainties
	\item \textit{Reactive scheduling}: Incorporate a process to react to and absorb disruptions
	\item Mainly affects the critical path - other disruptions can often be easily absorbed
	\item Therefore it is important that the critical path is protected
\end{itemize} 
\section{Risks}


\end{document}