\documentclass{article}[18pt]
\ProvidesPackage{format}
%Page setup
\usepackage[utf8]{inputenc}
\usepackage[margin=0.7in]{geometry}
\usepackage{parselines} 
\usepackage[english]{babel}
\usepackage{fancyhdr}
\usepackage{titlesec}
\hyphenpenalty=10000

\pagestyle{fancy}
\fancyhf{}
\rhead{Sam Robbins}
\rfoot{Page \thepage}

%Characters
\usepackage{amsmath}
\usepackage{amssymb}
\usepackage{gensymb}
\newcommand{\R}{\mathbb{R}}

%Diagrams
\usepackage{pgfplots}
\usepackage{graphicx}
\usepackage{tabularx}
\usepackage{relsize}
\pgfplotsset{width=10cm,compat=1.9}
\usepackage{float}

%Length Setting
\titlespacing\section{0pt}{14pt plus 4pt minus 2pt}{0pt plus 2pt minus 2pt}
\newlength\tindent
\setlength{\tindent}{\parindent}
\setlength{\parindent}{0pt}
\renewcommand{\indent}{\hspace*{\tindent}}

%Programming Font
\usepackage{courier}
\usepackage{listings}
\usepackage{pxfonts}

%Lists
\usepackage{enumerate}
\usepackage{enumitem}

% Networks Macro
\usepackage{tikz}


% Commands for files converted using pandoc
\providecommand{\tightlist}{%
	\setlength{\itemsep}{0pt}\setlength{\parskip}{0pt}}
\usepackage{hyperref}

% Get nice commands for floor and ceil
\usepackage{mathtools}
\DeclarePairedDelimiter{\ceil}{\lceil}{\rceil}
\DeclarePairedDelimiter{\floor}{\lfloor}{\rfloor}

% Allow itemize to go up to 20 levels deep (just change the number if you need more you madman)
\usepackage{enumitem}
\setlistdepth{20}
\renewlist{itemize}{itemize}{20}

% initially, use dots for all levels
\setlist[itemize]{label=$\cdot$}

% customize the first 3 levels
\setlist[itemize,1]{label=\textbullet}
\setlist[itemize,2]{label=--}
\setlist[itemize,3]{label=*}

% Definition and Important Stuff
% Important stuff
\usepackage[framemethod=TikZ]{mdframed}

\newcounter{theo}[section]\setcounter{theo}{0}
\renewcommand{\thetheo}{\arabic{section}.\arabic{theo}}
\newenvironment{important}[1][]{%
	\refstepcounter{theo}%
	\ifstrempty{#1}%
	{\mdfsetup{%
			frametitle={%
				\tikz[baseline=(current bounding box.east),outer sep=0pt]
				\node[anchor=east,rectangle,fill=red!50]
				{\strut Important};}}
	}%
	{\mdfsetup{%
			frametitle={%
				\tikz[baseline=(current bounding box.east),outer sep=0pt]
				\node[anchor=east,rectangle,fill=red!50]
				{\strut Important:~#1};}}%
	}%
	\mdfsetup{innertopmargin=10pt,linecolor=red!50,%
		linewidth=2pt,topline=true,%
		frametitleaboveskip=\dimexpr-\ht\strutbox\relax
	}
	\begin{mdframed}[]\relax%
		\centering
		}{\end{mdframed}}



\newcounter{lem}[section]\setcounter{lem}{0}
\renewcommand{\thelem}{\arabic{section}.\arabic{lem}}
\newenvironment{defin}[1][]{%
	\refstepcounter{lem}%
	\ifstrempty{#1}%
	{\mdfsetup{%
			frametitle={%
				\tikz[baseline=(current bounding box.east),outer sep=0pt]
				\node[anchor=east,rectangle,fill=blue!20]
				{\strut Definition};}}
	}%
	{\mdfsetup{%
			frametitle={%
				\tikz[baseline=(current bounding box.east),outer sep=0pt]
				\node[anchor=east,rectangle,fill=blue!20]
				{\strut Definition:~#1};}}%
	}%
	\mdfsetup{innertopmargin=10pt,linecolor=blue!20,%
		linewidth=2pt,topline=true,%
		frametitleaboveskip=\dimexpr-\ht\strutbox\relax
	}
	\begin{mdframed}[]\relax%
		\centering
		}{\end{mdframed}}
\lhead{Software Engineering}


\begin{document}
\begin{center}
\underline{\huge Agile Model}
\end{center}
\section{Manifesto}
We are uncovering better ways of developing software by doing it and helping others do it. Through this work we have come to value:
\begin{itemize}
	\item Individuals and interactions over processes and tools
	\item Working software over comprehensive documentation
	\item Customer collaboration over contract negotiation
	\item Responding to change over following a plan
\end{itemize}
\section{Principles}
\begin{enumerate}
	\item Our highest priority is to satisfy the customer through early and continuous delivery of valuable software
	\item Welcome changing requirements, even late in development. Agile processes harness change for the customer's competitive advantage
	\item Deliver working software frequently, from a couple of weeks to a couple of months, with a preference to the shorter timescale
	\item Business people and developers must work together daily throughout the project
	\item Build projects around motivates individuals. Give them the environment and support they need, and trust them to get the job done
	\item The most efficient and effective way of conveying information to and within a development team is face-to-face conversation
	\item Working software is the primary measure of progress
	\item Agile processes promote sustainable development. The sponsors, developers, and users should be able to maintain a constant pace indefinitely
	\item Continuous attention to technical excellence and good design enhances agility
	\item Simplicity - the art of maximising the work not done - is essential
	\item The best architectures, requirements and designs emerge from self-organising teams
	\item At regular intervals, the team reflects on how to become more effective, then tunes and adjusts its behaviour accordingly
\end{enumerate}
\section{User stories}
\begin{definition}[User stories]
	User stories are part of an agile approach that helps shift the focus from writing about requirements to talking about them. All agile user stories include a written sentence or two and, more importantly, a series of conversations about the desired functionality
\end{definition}
\begin{itemize}
	\item Captures the spirit
	\item Ignores details
	\item Make sense to customer
	\item Delivers value to customer
	\item End to end (full stack)
	\item Independent
	\item Testable
	\item Small (1-5 days) so easy to estimate
\end{itemize}
\subsection{Behaviour Driven Development}
\begin{definition}[Behaviour Driven Development]
	An agile process what supports and encourages collaborative development
\end{definition}
Built on TDD (Test Driven Development) and ATDD (Acceptance TDD), plus:
\begin{itemize}
	\item Where to start in the process
	\item What to test and what not to test
	\item How much to test in one go
	\item What to call the tests
	\item How to understand why a test fails
\end{itemize}
\end{document}





\end{document}