\documentclass{article}[18pt]
\input{../../../../format}
\lhead{Software Engineering}


\begin{document}
\begin{center}
\underline{\huge Software Maintenance}
\end{center}
\section{Types of maintenance}
\begin{definition}[Corrective Maintenance]
	Fixed/repaired
\end{definition}
\begin{definition}[Adaptive maintenance]
	Adapted to changing needs
\end{definition}
\begin{definition}[Perfective maintenance]
	Improved in performance or maintainability
\end{definition}
\begin{definition}[Preventive maintenance]
	Improved by fixing bugs before they activate
\end{definition}
\section{Maintenance problems}
Staff problems:
\begin{itemize}
	\item Limited understanding
	\item Management priorities
	\item Morale
\end{itemize}
Technical problems:
\begin{itemize}
	\item Artifacts and paradigms
	\item Testing difficulties
\end{itemize}
\section{Process metrics}
May be used to assess maintainability:
\begin{itemize}
	\item Number of requests for corrective maintenance
	\item Average time required for impact analysis
	\item Average time taken to implement a change request
	\item Number of outstanding change requests
\end{itemize}
\section{Techniques for maintenance}
\begin{definition}[Program comprehension]
	Reading and understanding programs in order to implement change
\end{definition}
\begin{definition}[Reverse engineering]
Analyse software to identify the components and their inter-relationships to produce call graphs and control flow graphs
\end{definition}
\begin{definition}[Migration]
Modify to run in a different environment
\end{definition}
\begin{definition}[Re-engineering]
Restructuring or rewriting part or all of a legacy system without changing its functionality to make it easier to maintain
\end{definition}
\section{Re-engineering}
\begin{center}
	\includegraphics[scale=0.7]{re-engineering}
\end{center}
\subsection{Refactoring vs re-engineering}
Refactoring is a continuous process of improvement throughout the development and evolution process. It is intended to avoid the structure and code degradation that increases the costs and difficulties of maintaining a system.\\
\\
Re-engineering takes place after a system has been maintained for some time and maintenance costs are increasing. You use automated tools to process and re-engineer a legacy system to create a new system that is more maintainable. 
\section{Legacy systems}
Multiple strategies for legacy systems:
\begin{itemize}
	\item Scrap the system completely
	\item Continue maintaining the system
	\item Transform the system by re-engineering to improve its maintainability
	\item Replace the system with a new system
\end{itemize}
\subsection{Legacy system categories}
{\renewcommand{\arraystretch}{2}
\begin{tabularx}{\textwidth}{|X|X|X|}
\hline
& Low quality & High quality\\
\hline
Low business value& Scrap the system & Replace with COTS, scrap or maintain\\
\hline
High business value& Re-engineer or replace& Continue in operation with maintenance\\
\hline
\end{tabularx}}

\end{document}