\documentclass{article}[18pt]
\input{../../../../format}
\lhead{Software Engineering}


\begin{document}
\begin{center}
\underline{\huge Software Reuse}
\end{center}
\begin{definition}[Software reuse]
Using existing software artifacts to construct a new software system
\end{definition}
\begin{definition}[Artifact]
A piece of formalised knowledge that can contribute to the software development process
\end{definition}
There are two types of software artifacts
\begin{itemize}
	\item Software products that are created as deliverables during the development process
	\item Development knowledge that is applied to the process
\end{itemize}
\section{Reuse levels}
Abstraction level
\begin{itemize}
	\item Use the concepts, not actual code
\end{itemize}
Object level:
\begin{itemize}
	\item Reuse objects or functions from code
\end{itemize}
Component level:
\begin{itemize}
	\item Reuse a collection of objects and object classes
\end{itemize}
System level:
\begin{itemize}
	\item Reuse the whole system
\end{itemize}
\section{Reusable assets in the software lifecycle}
Requirements:
\begin{itemize}
	\item Analysis: Project models, histories, system simulations
	\item Requirements specifications, trade studies, cost models
\end{itemize}
Design:
\begin{itemize}
	\item Designs, standards, frameworks, consumer reports, historical engineering costs, design simulations
\end{itemize}
Develop:
\begin{itemize}
	\item Code modules, programmer's manuals, operating manuals, unit test cases, codes simulations, operating training programs
\end{itemize}
Integrate and test:
\begin{itemize}
	\item System and subsystem test plans, procedures and cases; detailed interface simulations
\end{itemize}
Maintain:
\begin{itemize}
	\item Problem/trouble report tracking systems, regression test plans, procedures, cases
\end{itemize}
\section{Benefits of software reuse}
\begin{itemize}
	\item Reduced development costs - fewer components to spec, designed, implemented etc
	\item Accelerated development - reduction in development and validation time
	\item Increased dependability - tried and tested
	\item Reduced process risk - reduced margin of error in project cost estimation
	\item Effective use of specialists - encapsulation of their knowledge; not re-inventing the wheel
	\item Standards compliance
\end{itemize}
\section{Problems with reuse}
\begin{itemize}
	\item Increased maintenance costs - source code of reused component etc not available
	\item "Not-invented-here syndrome" - software engineers prefer to write own code as it has better quality
	\item Creating, maintaining and using a component library - populating and structuring can be expensive. The development processes used need to be adapted so they can use the library
	\item Finding, understanding and adapting reusable components- components in library need to be understood and adapted for new environment
\end{itemize}
\section{COTS product reuse}
\begin{definition}[COTS]
Commercial off the shelf - a software system that can be adapted for different customers without changing the source code of the system
\end{definition}
COTS systems have generic features and so can be used/reused in different environments\\
\\
COTS products are adapted by using built in configuration mechanisms that allow the functionality of the system to be tailored to specific customer needs
\subsection{Benefits of COTS reuse}
As with other types of reuse, more rapid deployment of a reliable system may be possible\\
\\
It is possible to see what functionality is provided by the applications and so it is easier to judge whether or not they are likely to be suitable\\
\\
Some development risks are avoided by using existing software. However, this approach has its own risks\\
\\
Businesses can focus on their core activity without having to devote a lot of resources to IT systems development\\
\\
As operating platforms evolve, technology updates may be simplified as these are the responsibility of the COTS product vendor rather than the customer
\subsection{Problems of COTS reuse}
Requirements usually have to be adapted to reflect the functionality and mode of operation of the COTS product\\
\\
The COTS product may be based on assumptions that are practically impossible to change\\
\\
Choosing the right COTS system for an enterprise can be a difficult process, especially as many COTS products are not well documented\\
\\
There may be a lack of local expertise to support systems development\\
\\
The COTS product vendor controls system support and evaluation
\subsection{Risks of using COTS}
Vendor risks:
\begin{itemize}
	\item Failure of vendor to provide support when required
	\item Vendor goes out of business or drops product from its portfolio
\end{itemize}
Product risks:
\begin{itemize}
	\item Incompatible event/data model with other systems
	\item Inadequate performance when integrated with other systems
	\item Product is undependable in intended operating environment
\end{itemize}
Process risk
\begin{itemize}
	\item Time required to understand how to integrate product is higher than expected
\end{itemize}
\subsubsection{Risks reduction}
\begin{itemize}
	\item Only dealing with vendors that allow access to source code if they go out of business
	\item By extensive research and testing of product capabilities before using, discussion with other users etc
	\item In general though, because COTS are provided by external vendors, risk reduction is difficult
\end{itemize}
\section{Reuse costs}
\begin{itemize}
	\item The costs of the time spent in looking for software to reuse and assessing whether or not it meets your needs
	\item Where applicable, the costs of buying the reusable software for large off the shelf systems can be very high
	\item The costs of adapting and configuring the reusable software components or systems to reflect the requirements of the system that you are developing
	\item The costs of integrating reusable software elements with each other and with the new code that you have developed
\end{itemize}


\end{document}