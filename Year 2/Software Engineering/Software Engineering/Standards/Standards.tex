\documentclass{article}[18pt]
\input{../../../../format}
\lhead{Software Engineering}


\begin{document}
\begin{center}
\underline{\huge Standards}
\end{center}

Why standards?
\begin{itemize}
	\item Quality
	\item Shared communication
	\item Shared understanding
	\item Influence, from understanding to creation/development
	\item Profit
	\item Collaboration
	\item Reputation
	\item Regulation (assurance)
	\item Flexibility
\end{itemize}
Importance
\begin{itemize}
	\item They encapsulate best practice (normally)
	\item Framework for QA
	\item Provide continuity
	\begin{itemize}
		\item Record of decision making process
		\item Organisational memory
		\item New staff save time
	\end{itemize}
\end{itemize}
Issues:
\begin{itemize}
	\item Standards are considered too large, unwieldy and difficult to adopt for SMEs
	\item Focus is on large organisations
	\item Concerns over cost and documentation
	\item Difficult to justify
\end{itemize}
\section{Software standards}
Standards are about providing rules, guidelines and heuristics which, if followed, deliver an assurance of good practice - they are not intended to be about best practice\\
\\
Standards may be international, national, organisational or project standards.\\
\\
\textbf{Product} standards define characteristics that all software components should exhibit\\
\\
\textbf{Process} standards define how the software process should be enacted





\section{ISO SC7}
\subsection{Structure}
\begin{center}
	\includegraphics[scale=0.7]{SC7}
\end{center}
\subsection{Domains}
\begin{center}
	\includegraphics[scale=0.7]{Domains}
\end{center}
\subsection{Standards}
\begin{center}
	\includegraphics[scale=0.7]{Standards1}
\end{center}
Standards of particular interest
\begin{itemize}
	\item ISO 9000, family of standards for quality management systems
	\item ISO 12207, defines the software engineering process, activity, and tasks that are associated with a software life cycle process from conception through retirement
	\item ISO 15504, also known as SPICE (Software Process Improvement and Capability Determination), is a framework for the assessment of processes
\end{itemize}
\section{ISO 9000}
\begin{center}
	\includegraphics[scale=0.7]{ISO9000}
\end{center}
QSM:
\begin{itemize}
	\item ISO9001 – QSM for Quality Assurance in design, development, production, installation and service
	\item ISO9002 – QSM for Quality Assurance in production, installation, and servicing
	\item ISO9003 – QSM for Quality Assurance in final inspection and test
\end{itemize}
Quality: refers to all features of a product (such as software) which are required by a customer\\
\\
Quality management: covers the organisations approach to ensuring that it produces quality products and complies with the appropriate regulations
\section{ISO 12207}
\begin{itemize}
	\item Created to supply a common structure so that the buyers, suppliers, developers, maintainers, operators, managers and technicians involved with the software development use a common language
	\item It is the standard that defines all the tasks required for developing and maintaining software
	\item Created in ’95, last updated in ’17 (ISO 12207:2017)
	\item Covers the process in the life cycle of software:
	\begin{itemize}
		\item High level process architecture
		\item Activities and tasks
		\item Tailored for any organization or project (inc. SME et al)
		\item An ‘inventory’ of processes from which to choose
	\end{itemize}
	\item This standard does not create a standardised way to create a product
	\item It is not prescriptive
	\item Nor does it advocate or enforce a standardised methodology
\end{itemize}
\subsection{ISO 12207:17}
\begin{center}
	\includegraphics[scale=0.7]{ISO12207:17}
\end{center}
\section{Process Implementation}
\begin{itemize}
	\item Define or select software life cycle model appropriate to the scope, magnitude, and complexity of the project;
	\item Select, tailor, and use standards, methods, tools, and programming languages (if not stipulated in  contract);
	\item Develop plans for conducting the activities of the Development process.
\end{itemize}
\section{ISO 15504}
Process assessment: What is it?
\begin{itemize}
	\item A disciplined examination of the processes by an organisation against a set of criteria to determine capability of those processes to perform within quality, cost and schedule goals
	\item Focus here is on continual, self-improvement
\end{itemize}
Why bother?
\begin{itemize}
	\item Identify strengths and weaknesses in current utilisation of processes
	\item Ongoing development of systems, maturity and growth
	\item Feeds into the future
\end{itemize}
\begin{center}
	\includegraphics[scale=0.7]{ISO15504}
\end{center}




\end{document}