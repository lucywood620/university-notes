\documentclass{article}[18pt]
\ProvidesPackage{format}
%Page setup
\usepackage[utf8]{inputenc}
\usepackage[margin=0.7in]{geometry}
\usepackage{parselines} 
\usepackage[english]{babel}
\usepackage{fancyhdr}
\usepackage{titlesec}
\hyphenpenalty=10000

\pagestyle{fancy}
\fancyhf{}
\rhead{Sam Robbins}
\rfoot{Page \thepage}

%Characters
\usepackage{amsmath}
\usepackage{amssymb}
\usepackage{gensymb}
\newcommand{\R}{\mathbb{R}}

%Diagrams
\usepackage{pgfplots}
\usepackage{graphicx}
\usepackage{tabularx}
\usepackage{relsize}
\pgfplotsset{width=10cm,compat=1.9}
\usepackage{float}

%Length Setting
\titlespacing\section{0pt}{14pt plus 4pt minus 2pt}{0pt plus 2pt minus 2pt}
\newlength\tindent
\setlength{\tindent}{\parindent}
\setlength{\parindent}{0pt}
\renewcommand{\indent}{\hspace*{\tindent}}

%Programming Font
\usepackage{courier}
\usepackage{listings}
\usepackage{pxfonts}

%Lists
\usepackage{enumerate}
\usepackage{enumitem}

% Networks Macro
\usepackage{tikz}


% Commands for files converted using pandoc
\providecommand{\tightlist}{%
	\setlength{\itemsep}{0pt}\setlength{\parskip}{0pt}}
\usepackage{hyperref}

% Get nice commands for floor and ceil
\usepackage{mathtools}
\DeclarePairedDelimiter{\ceil}{\lceil}{\rceil}
\DeclarePairedDelimiter{\floor}{\lfloor}{\rfloor}

% Allow itemize to go up to 20 levels deep (just change the number if you need more you madman)
\usepackage{enumitem}
\setlistdepth{20}
\renewlist{itemize}{itemize}{20}

% initially, use dots for all levels
\setlist[itemize]{label=$\cdot$}

% customize the first 3 levels
\setlist[itemize,1]{label=\textbullet}
\setlist[itemize,2]{label=--}
\setlist[itemize,3]{label=*}

% Definition and Important Stuff
% Important stuff
\usepackage[framemethod=TikZ]{mdframed}

\newcounter{theo}[section]\setcounter{theo}{0}
\renewcommand{\thetheo}{\arabic{section}.\arabic{theo}}
\newenvironment{important}[1][]{%
	\refstepcounter{theo}%
	\ifstrempty{#1}%
	{\mdfsetup{%
			frametitle={%
				\tikz[baseline=(current bounding box.east),outer sep=0pt]
				\node[anchor=east,rectangle,fill=red!50]
				{\strut Important};}}
	}%
	{\mdfsetup{%
			frametitle={%
				\tikz[baseline=(current bounding box.east),outer sep=0pt]
				\node[anchor=east,rectangle,fill=red!50]
				{\strut Important:~#1};}}%
	}%
	\mdfsetup{innertopmargin=10pt,linecolor=red!50,%
		linewidth=2pt,topline=true,%
		frametitleaboveskip=\dimexpr-\ht\strutbox\relax
	}
	\begin{mdframed}[]\relax%
		\centering
		}{\end{mdframed}}



\newcounter{lem}[section]\setcounter{lem}{0}
\renewcommand{\thelem}{\arabic{section}.\arabic{lem}}
\newenvironment{defin}[1][]{%
	\refstepcounter{lem}%
	\ifstrempty{#1}%
	{\mdfsetup{%
			frametitle={%
				\tikz[baseline=(current bounding box.east),outer sep=0pt]
				\node[anchor=east,rectangle,fill=blue!20]
				{\strut Definition};}}
	}%
	{\mdfsetup{%
			frametitle={%
				\tikz[baseline=(current bounding box.east),outer sep=0pt]
				\node[anchor=east,rectangle,fill=blue!20]
				{\strut Definition:~#1};}}%
	}%
	\mdfsetup{innertopmargin=10pt,linecolor=blue!20,%
		linewidth=2pt,topline=true,%
		frametitleaboveskip=\dimexpr-\ht\strutbox\relax
	}
	\begin{mdframed}[]\relax%
		\centering
		}{\end{mdframed}}
\lhead{Software Engineering - Testing}


\begin{document}
\begin{center}
\underline{\huge Integration and System Testing}
\end{center}
\section{Integration Testing}
\begin{itemize}
	\item When unit testing has demonstrated a suitable level of correctness for our components, we need to start combining these
	\item \textbf{Big Bang} - Stick all the components together and hope it workd
	\item \textbf{Phased} - Begin testing before all components are ready
\end{itemize}
\subsection{Bottom-up integration}
\begin{itemize}
	\item Aim to complete unit testing for components at the lowest level of hierarchy first
	\item Test the next level of components, using the lowest ones
	\item Continue with this to complete system level
\end{itemize}
This does assume that there is a component hierarchy\\
Need to create a set of component drivers to test each level by providing the necessary calls\\
Devising an oracle for this is often relatively tractable
\subsubsection{Issues}
\begin{itemize}
	\item Helps identify sources of problems quite well
	\item Lower level components get tested first and key ones at the top level only get tested later
\end{itemize}
\subsection{Top-down integration}
\begin{itemize}
	\item Involves testing with the key components at the top of the hierarchy
	\item Since lower level elements may not be ready or tested, can use a stub which emulates the missing component in a simplified manner for each one.
	\item Testing of components in the middle may need stubs and drivers
\end{itemize}
\subsubsection{Issues}
\begin{itemize}
	\item Writing the stubs and drivers may be quite complex
	\item Needs the support of an effective test harness to aid configuration, and also collection of test outputs. (call the correct stubs and drivers at the right time)
	\item Devising an oracle can be quite challenging
\end{itemize}
\subsection{Sandwich integration}
\begin{itemize}
	\item Combine top down and bottom up to work from both ends, reducing the number of stubs needed
\end{itemize}
\section{Continuous builds}
\begin{itemize}
	\item Maintain a single source repository
	\item Automate the build
	\item Make the build self testing
	\item Require everyone to commit every day
	\item Keep the build fast
	\item Ensures visibility to all participants
\end{itemize}
\section{System Testing}
System testing is much more concerned with conformance to the specification (requirements) than with finding bugs. Precedes and underpins UAT
\begin{itemize}
	\item Unit testing and integration testing are concerned with whether the coding conforms to the design
	\item System testing is concerned with whether the design conforms to the requirements
\end{itemize}
Steps in system testing include:
\begin{enumerate}
	\item Function testing
	\item Performance testing
	\item Acceptance testing
	\item Installation testing
\end{enumerate}
\subsection{Function tests}
\begin{itemize}
	\item Driven by the list of requirements
	\item Can be documented in a tabular form by listing the requirement and then recording how it has been tested and the outcomes of that test
	\item Can form a preliminary stage for UAT, ensuring that the system works and that it conforms to the requirements as stated. UAT then assesses whether the stated requirements are the real ones
\end{itemize}
\subsection{Performance tests}
Address the non functional issues such as
\begin{itemize}
	\item Security
	\item Speed
	\item Accuracy
	\item Reliability
\end{itemize}
The ordering here is important. It is a good principle to get the system working then address these issues.






\end{document}