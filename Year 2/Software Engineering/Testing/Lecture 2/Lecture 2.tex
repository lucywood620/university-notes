\documentclass{article}[18pt]
\ProvidesPackage{format}
%Page setup
\usepackage[utf8]{inputenc}
\usepackage[margin=0.7in]{geometry}
\usepackage{parselines} 
\usepackage[english]{babel}
\usepackage{fancyhdr}
\usepackage{titlesec}
\hyphenpenalty=10000

\pagestyle{fancy}
\fancyhf{}
\rhead{Sam Robbins}
\rfoot{Page \thepage}

%Characters
\usepackage{amsmath}
\usepackage{amssymb}
\usepackage{gensymb}
\newcommand{\R}{\mathbb{R}}

%Diagrams
\usepackage{pgfplots}
\usepackage{graphicx}
\usepackage{tabularx}
\usepackage{relsize}
\pgfplotsset{width=10cm,compat=1.9}
\usepackage{float}

%Length Setting
\titlespacing\section{0pt}{14pt plus 4pt minus 2pt}{0pt plus 2pt minus 2pt}
\newlength\tindent
\setlength{\tindent}{\parindent}
\setlength{\parindent}{0pt}
\renewcommand{\indent}{\hspace*{\tindent}}

%Programming Font
\usepackage{courier}
\usepackage{listings}
\usepackage{pxfonts}

%Lists
\usepackage{enumerate}
\usepackage{enumitem}

% Networks Macro
\usepackage{tikz}


% Commands for files converted using pandoc
\providecommand{\tightlist}{%
	\setlength{\itemsep}{0pt}\setlength{\parskip}{0pt}}
\usepackage{hyperref}

% Get nice commands for floor and ceil
\usepackage{mathtools}
\DeclarePairedDelimiter{\ceil}{\lceil}{\rceil}
\DeclarePairedDelimiter{\floor}{\lfloor}{\rfloor}

% Allow itemize to go up to 20 levels deep (just change the number if you need more you madman)
\usepackage{enumitem}
\setlistdepth{20}
\renewlist{itemize}{itemize}{20}

% initially, use dots for all levels
\setlist[itemize]{label=$\cdot$}

% customize the first 3 levels
\setlist[itemize,1]{label=\textbullet}
\setlist[itemize,2]{label=--}
\setlist[itemize,3]{label=*}

% Definition and Important Stuff
% Important stuff
\usepackage[framemethod=TikZ]{mdframed}

\newcounter{theo}[section]\setcounter{theo}{0}
\renewcommand{\thetheo}{\arabic{section}.\arabic{theo}}
\newenvironment{important}[1][]{%
	\refstepcounter{theo}%
	\ifstrempty{#1}%
	{\mdfsetup{%
			frametitle={%
				\tikz[baseline=(current bounding box.east),outer sep=0pt]
				\node[anchor=east,rectangle,fill=red!50]
				{\strut Important};}}
	}%
	{\mdfsetup{%
			frametitle={%
				\tikz[baseline=(current bounding box.east),outer sep=0pt]
				\node[anchor=east,rectangle,fill=red!50]
				{\strut Important:~#1};}}%
	}%
	\mdfsetup{innertopmargin=10pt,linecolor=red!50,%
		linewidth=2pt,topline=true,%
		frametitleaboveskip=\dimexpr-\ht\strutbox\relax
	}
	\begin{mdframed}[]\relax%
		\centering
		}{\end{mdframed}}



\newcounter{lem}[section]\setcounter{lem}{0}
\renewcommand{\thelem}{\arabic{section}.\arabic{lem}}
\newenvironment{defin}[1][]{%
	\refstepcounter{lem}%
	\ifstrempty{#1}%
	{\mdfsetup{%
			frametitle={%
				\tikz[baseline=(current bounding box.east),outer sep=0pt]
				\node[anchor=east,rectangle,fill=blue!20]
				{\strut Definition};}}
	}%
	{\mdfsetup{%
			frametitle={%
				\tikz[baseline=(current bounding box.east),outer sep=0pt]
				\node[anchor=east,rectangle,fill=blue!20]
				{\strut Definition:~#1};}}%
	}%
	\mdfsetup{innertopmargin=10pt,linecolor=blue!20,%
		linewidth=2pt,topline=true,%
		frametitleaboveskip=\dimexpr-\ht\strutbox\relax
	}
	\begin{mdframed}[]\relax%
		\centering
		}{\end{mdframed}}
\lhead{Software Engineering - Testing}


\begin{document}
\begin{center}
\underline{\huge User Acceptance Testing and Testing Objects}
\end{center}
Consist of a set of tests:
\begin{itemize}
	\item Drawn up by the customer's test designers/end-users
	\item Derived from the requirements specification
	\item Forming the final test activity before the system is approved for delivery
\end{itemize}
\section{Why do UAT?}
\begin{itemize}
	\item To provide confidence that the system delivered to the customer is the one that they need. So the customer acts as the test oracle
	\item Testing is done from the customer's perspective and based upon their understanding of the requirements
	\item Ensures that the set of acceptance criteria have a pivotal role for driving the project
	\begin{itemize}
		\item The criteria should be well understood by the customer, development team and project manager
		\item Ideally they are made explicit in the contract
	\end{itemize}
\end{itemize}
\section{Operational benefits of UAT}
\begin{itemize}
	\item Reduces the risk of subsequent operational system failure
	\item Validates manuals and other documentation
	\item Checks handling of error conditions
	\item Tests the system on the operational platform so wa can assess the impact our software has on existing systems and resources, and vice versa
\end{itemize}
\section{UAT Focus}
\begin{itemize}
	\item Key goal is to gain acceptance of the system by its end-user stakeholders
	\item To achieve this, the system should:
	\begin{itemize}
		\item Fulfil the intended purpose(s) and fit the business case
		\item Provide evidence and results under the specified conditions of use
		\item Be clearly and correctly documented
		\item Be reliable and stable
		\item Have no unintended side-effects
	\end{itemize}
\end{itemize}
\section{Types of UAT}
UAT is typically approached in three ways:
\begin{itemize}
	\item For a \textbf{benchmark test}, the customer prepares a set of test cases that represent typical operational scenarios - the tests may then be performed with actual users or a specified testing team, who also evaluate the outcomes
	\item A \textbf{pilot test} involves installing the system on an experimental basis and letting users employ it as if it were permanently installed, relying upon everyday use to test all the functions - so less formal and structured than a benchmark test
	\item In \textbf{parallel testing} the new system runs alongside an existing one, addressing compatibility and function testing
\end{itemize}
\section{Prerequisites}
\subsection{Determine whether to start UAT}
Earlier testing must be successfully completed:
\begin{itemize}
	\item Unit tests
	\item Integration tests
	\item System tests
\end{itemize}
The developers must have confidence that the system is operational and ready for delivery\\
\\
Managers must have confidence that there will be no embarrassment, so operate on the principle of least surprise
\subsection{Assign roles}
\begin{itemize}
	\item The development team and customer need to agree about assigning roles and responsibilities
	\item In particular, should decide on who has the following roles
	\begin{itemize}
		\item UAT team leader
		\item Development team leader
		\item Senior User/ User Representative
		\item Testers
	\end{itemize}
\end{itemize}
\section{Key steps}
Phase 1: Planning:
\begin{itemize}
	\item Prepare a test plan
	\item Review the test plan with participants and stakeholders
	\item Arrange sign-off procedures
\end{itemize}
Phase 2: Preparing tests, Test Data and Training
\begin{itemize}
	\item Prepare the tests (test cases)
	\item Prepare the scripts (test scenarios)
	\item Prepare the test data
	\item Conduct user testing
	\item Establish the test environment
	\item Confirm availability of resources
\end{itemize}
Phase 3: Executing and Controlling:
\begin{itemize}
	\item Run the tests (scripts and test cases)
	\item Record the results
	\item Log problems and monitor the resolution
	\item Fix problems and re-test (not a good thing for UAT)
\end{itemize}
Phase 4: Closure:
\begin{itemize}
	\item Arrange for formal acceptance of the system
	\item Perform system hand-over
\end{itemize}
\section{Challenges of UAT}
\textbf{Coverage} - Ensuring that all aspects of a system are covered by the tests. Difficult to have a really formal record of this\\
\\
\textbf{Regression testing} - There end users are employed to exercise the system, it can be difficult to ensure that regression testing is performed rigorously\\
\\
\textbf{Training} - The end users do need to have a good understanding of their role and the actitivities that they need to perform




\end{document}