\documentclass{article}[18pt]
\ProvidesPackage{format}
%Page setup
\usepackage[utf8]{inputenc}
\usepackage[margin=0.7in]{geometry}
\usepackage{parselines} 
\usepackage[english]{babel}
\usepackage{fancyhdr}
\usepackage{titlesec}
\hyphenpenalty=10000

\pagestyle{fancy}
\fancyhf{}
\rhead{Sam Robbins}
\rfoot{Page \thepage}

%Characters
\usepackage{amsmath}
\usepackage{amssymb}
\usepackage{gensymb}
\newcommand{\R}{\mathbb{R}}

%Diagrams
\usepackage{pgfplots}
\usepackage{graphicx}
\usepackage{tabularx}
\usepackage{relsize}
\pgfplotsset{width=10cm,compat=1.9}
\usepackage{float}

%Length Setting
\titlespacing\section{0pt}{14pt plus 4pt minus 2pt}{0pt plus 2pt minus 2pt}
\newlength\tindent
\setlength{\tindent}{\parindent}
\setlength{\parindent}{0pt}
\renewcommand{\indent}{\hspace*{\tindent}}

%Programming Font
\usepackage{courier}
\usepackage{listings}
\usepackage{pxfonts}

%Lists
\usepackage{enumerate}
\usepackage{enumitem}

% Networks Macro
\usepackage{tikz}


% Commands for files converted using pandoc
\providecommand{\tightlist}{%
	\setlength{\itemsep}{0pt}\setlength{\parskip}{0pt}}
\usepackage{hyperref}

% Get nice commands for floor and ceil
\usepackage{mathtools}
\DeclarePairedDelimiter{\ceil}{\lceil}{\rceil}
\DeclarePairedDelimiter{\floor}{\lfloor}{\rfloor}

% Allow itemize to go up to 20 levels deep (just change the number if you need more you madman)
\usepackage{enumitem}
\setlistdepth{20}
\renewlist{itemize}{itemize}{20}

% initially, use dots for all levels
\setlist[itemize]{label=$\cdot$}

% customize the first 3 levels
\setlist[itemize,1]{label=\textbullet}
\setlist[itemize,2]{label=--}
\setlist[itemize,3]{label=*}

% Definition and Important Stuff
% Important stuff
\usepackage[framemethod=TikZ]{mdframed}

\newcounter{theo}[section]\setcounter{theo}{0}
\renewcommand{\thetheo}{\arabic{section}.\arabic{theo}}
\newenvironment{important}[1][]{%
	\refstepcounter{theo}%
	\ifstrempty{#1}%
	{\mdfsetup{%
			frametitle={%
				\tikz[baseline=(current bounding box.east),outer sep=0pt]
				\node[anchor=east,rectangle,fill=red!50]
				{\strut Important};}}
	}%
	{\mdfsetup{%
			frametitle={%
				\tikz[baseline=(current bounding box.east),outer sep=0pt]
				\node[anchor=east,rectangle,fill=red!50]
				{\strut Important:~#1};}}%
	}%
	\mdfsetup{innertopmargin=10pt,linecolor=red!50,%
		linewidth=2pt,topline=true,%
		frametitleaboveskip=\dimexpr-\ht\strutbox\relax
	}
	\begin{mdframed}[]\relax%
		\centering
		}{\end{mdframed}}



\newcounter{lem}[section]\setcounter{lem}{0}
\renewcommand{\thelem}{\arabic{section}.\arabic{lem}}
\newenvironment{defin}[1][]{%
	\refstepcounter{lem}%
	\ifstrempty{#1}%
	{\mdfsetup{%
			frametitle={%
				\tikz[baseline=(current bounding box.east),outer sep=0pt]
				\node[anchor=east,rectangle,fill=blue!20]
				{\strut Definition};}}
	}%
	{\mdfsetup{%
			frametitle={%
				\tikz[baseline=(current bounding box.east),outer sep=0pt]
				\node[anchor=east,rectangle,fill=blue!20]
				{\strut Definition:~#1};}}%
	}%
	\mdfsetup{innertopmargin=10pt,linecolor=blue!20,%
		linewidth=2pt,topline=true,%
		frametitleaboveskip=\dimexpr-\ht\strutbox\relax
	}
	\begin{mdframed}[]\relax%
		\centering
		}{\end{mdframed}}
\lhead{Software Methodologies}


\begin{document}
\begin{center}
\underline{\huge State Spaces}
\end{center}
\section{State spaces}
State space - a set of states\\
\\
Two types of search problems
\begin{itemize}
	\item Global search problems - solutions are paths (state spaces are states and transitions. Get from initial to goal state by a sequence of transitions. Each transition has a step cost. Different actions can be used for the same transitions. The aim is to find a path from initial to goal state where the total step cost for the path is either max or min depending on the problem)
	\item Local search problems - Have state space and state transitions but no actions or goal states. Each state has a cost associated called an objective cost. Find the state of either smallest or largest cost depending on if the problem is to minimise or maximise
\end{itemize}
\textit{How can you deal with an infinite state space?} - Only ever retain in memory what has already been visited in the state space. Build states when you have to
\section{Real world problems}
\subsubsection{8-Puzzle Problems}
\begin{itemize}
	\item Move tiles into the empty cell to reach a specific tile configuration
	\item Not always a solution for n problem, but can find if there is a solution in P
	\item If there is a solution, finding one is in P
	\item Finding a solution with the fewest moves is in NP
\end{itemize}
\textit{What is a state?} - A particular tile configuration\\
\textit{How might we represent this state?}
\begin{itemize}
	\item As a \textbf{tuple} $x\in \{0,1,...,8\}^9$ so that all components are distinct (sequence of 9 integers in a list, numbers between 1 and 9)
	\item As a $3\times 3$ \textbf{matrix} M over $\{0,1,...,8\}$ so that all components are distinct (just a matrix looking like the grid)
\end{itemize}
\textit{What is a state transition} - Moving a tile into the empty space\\
\textit{How might we represent this transition?}
\begin{itemize}
	\item $(x,y)$ is a transition iff
	\begin{itemize}
		\item $x_i=y_i \ \forall i \in \{1,2,...,9\}$ except $\exists j,k$ s.t. $0=x_j\neq y_j$ and $x_k\neq y_k=0$
	\end{itemize}
	\item $(M_1,M_2)$ is a transition iff
	\begin{itemize}
		\item $M_1[1,j]=M_2[i,j] \ \forall i, j \in \{1,2,3\}$ except $\exists (a,b), (c,d)$  s.t.$0=M_1[a,b]\neq M_2[a,b]$ and $M_1[c,d]\neq M_2[c,d]=0$
		\item Either ($a=c$ and $|b=d|=1$) or ($b=d$ and $|a-c|=1$)
	\end{itemize}
	\item The initial state and the goal state are as required
\end{itemize}
\subsubsection{Rush hour}
\begin{itemize}
	\item Move car to the exit
	\item Can generalise to $n\times n$
	\item Finding if there is a solution is PSPACE-complete - harder than NP complete
	\item If there are no lorries, still PSPACE-complete
\end{itemize}
\subsection{Path Problems}
\subsubsection{Travelling Salesman Problem}
\begin{itemize}
	\item Given a set of cities and the distance between each pair of cities, find a shortest tour of all the cities
	\item "cities" don't have to be cities and "distances" don't have to be distances. Could model anything
	\item City-to-city distances don't have to be symmetric ($A\rightarrow B$ doesn't have to be $B\rightarrow A$ (might have a hill)) nor Euclidean (where you place them on the plane doesn't have to be scaled and stuff)
	\begin{itemize}
		\item Computing the optimal solution - \textbf{NP Hard}
		\item Computing the optimal solution for the symmetric Euclidean TSP - \textbf{NP Hard}
	\end{itemize} 
\end{itemize}
\textbf{Modelled as a search problem}

\textit{What is a state?}
	\begin{itemize}
		\item A list of distinct cities (a partial tour)
		\item A list containing each of the n cities exactly once
	\end{itemize}
\textit{How might we represent this state}
\begin{itemize}
	\item As a list of i cities, where $0\leqslant i \leqslant n$
	\item As a list of n cities 
\end{itemize}
\textit{What might the initial state be?}
\begin{itemize}
	\item An empty list?
\end{itemize}
\textit{What is a state transition}
\begin{itemize}
	\item An ordered pair of states so that the second state is the first state incremented with an as yet unvisited city
	\item An ordered pair of states so that the second state is the first state except that the cities in two locations have been swapped
\end{itemize}
\textit{How might we represent this transition?}
\begin{itemize}
	\item An ordered pair of lists $(p,q)$ so that $p=(p_1,p_2,...,p_i)$, for some $0\leqslant i < n$ and $q=(p_1,p_2,...,p_i,c)$, where $c\neq p_j$, $\forall j \in \{1,2,...,i\}$ (add unvisited cities)
	\item An ordered pair of lists $(p,q)$ so that $q=(q_1,q_2,...,q_n)$ and $p=(p_1,p_2,...,p_n)$ where $p_i=q_i, \ \forall i\in \{1,2,..,n\}$ except $\exists j\neq k$ s.t. $p_j=q_k$ and $p_k=q_j$
\end{itemize}
\textit{What are the initial state and a goal state?}
\begin{itemize}

\item \textit{Idea 1}
\begin{itemize}
	\item Initial state: the empty list
	\item Goal State: A list of all cities
\end{itemize}
\item \textit{Idea 2}
\begin{itemize}
	\item Initial state: list of cities in an arbitrary order
	\item Goal state: Every state
\end{itemize}
\end{itemize}
\textit{What is the cost to move from n-1 to n cities?}\\
The cost of moving from n-1 to nth cities, and the cost from the nth city back to the 1st

\subsubsection{Disjoint Connecting Paths}
\begin{itemize}
	\item In a labyrinth of rooms connected by corridors, each person $p_i$, where $i=1,2,...,k$ has to reach their own exit-room $t_i$ starting at their own start-room $s_i$
	\item Start and exit rooms can only appear as terminal rooms on paths and no other room lies on more than one person's path
\end{itemize}
\section{Some considerations (and subtleties)}
\textit{The state space might be very large or even infinite}
\begin{itemize}
	\item There are $n!$ states in an instance of the n-puzzle problem
	\item There are $> 2n!$ and $n!$ states, respectively, in our two n-city TSP abstractions
\end{itemize}
\textit{So how do we store the state space?}
\begin{itemize}
	\item We describe states succinctly using appropriate data structures
	\item State transitions are described using roles for application. So, states are built as and when we need them
	\item There is a tension between how we represent states and state transitions versus the "implementation cost" of building states and establishing state transitions
\end{itemize}
\textit{In an optimization problem, how do we know we have an optimal path to a goal state}
\begin{itemize}
	\item This will be down to the search algorithm that we employ. Maybe we'll have to settle for non-optimal solutions
\end{itemize}
\textit{We need to ensure that our abstraction of the real world problem is:}
\begin{itemize}
	\item Such that solutions to the search problem and real-world problem correspond
	\item Such that the level of real-world abstraction is appropriate. A balance between "abstraction" complexity and "computational" complexity
\end{itemize}
\section{In summary}
\textit{To formally describe a real-world problem, we need to undertake the following}
\begin{enumerate}
	\item Define a state space that contains representations of all possible essential configurations of the objects of the real-world problem
	\begin{itemize}
		\item One need not enumerate all states, but one should be able to "build" a state as and when required
	\end{itemize}
	\item Specify an initial state corresponding to the configuration in which the real world problem starts
	\item Specify goal states corresponding to the real-world configurations that are acceptable from which to derive solutions to the problem
	\item Specify a set of rules to describe the transitions, possibly via actions between states in the state space, being sure to consider
	\begin{itemize}
		\item All real world assumptions made in the abstraction of the problem
		\item The generality of the rules in relation to the adequacy in yielding solutions
		\item How much work should be represented within the rules
	\end{itemize}
	\item Establish a notion of cost to measure the relative quality of solutions.
\end{enumerate}
\section{Definition of a "global path-based" search problem - KNOW FOR EXAM}
\textit{Every (search) problem has six essential components}
\begin{enumerate}
	\item An \textbf{initial state} in which a problem-solving agent starts
	\item A description of the possible \textbf{actions} available to the agent, via a \textbf{transition}(or \textbf{successor}) function $\phi$ which for each state x in the \textbf{state space} and each action $\alpha$, details a set of states (possibly empty) reachable from $x$ via action $\alpha$
	\begin{itemize}
		\item The state space is implicit in the description of the transition function
		\item The pair $(y,\alpha)$ is a successor of $x$ if $y\in \phi(x,a)$
		\item This is going from state x to state y by action $\alpha$
	\end{itemize}
	\item A goal test which determines whether a state is a goal state
	\item A non negative step-cost function $\sigma$ which details the cost $\sigma (x,\alpha, y)$ of a transition from any state $x$ to any state $y$ via any action $\alpha$ (if possible)

\item \textit{A solution to a problem}
\begin{itemize}
	\item A path from the initial state to a goal state
\end{itemize}
\item \textit{An optimal solution}
\begin{itemize}
	\item A solution with the lowest/highest path-cost among all solutions where path-cost of some path is the sum of the constituent step-costs
\end{itemize}
\end{enumerate}
\section{Searching for solution paths - KNOW FOR EXAM}
\textit{We're interested in search strategies that generate/reveal the search tree}
\textit{Initially, the search tree consists of a root-node}
\begin{itemize}
	\item The associated state is the initial state
\end{itemize}
\textit{A search tree is generated/revealed by iteratively expanding, one by one, the hitherto unexpanded leaves of the tree so that}
\begin{itemize}
	\item A leaf is augmented with new (tree-) nodes so that the states associated with the new leaves are successor states of the state associated with the old leaf
\end{itemize}
\textit{If $y\in \phi(x,\alpha)$ and $y\in \phi(x,\beta)$, where $\alpha$ and $\beta$ are different actions}
\begin{itemize}
	\item Then a tree-node with associated state $x$ will have two children each of whose associated state is $y$. One for the transition via $\alpha$ and one from the transition via $\beta$
	\item Each child node $y$ will have an associated action - the action undertaken to get from its parent node 
\end{itemize}
\textit{The unexpanded leaves form the fringe} - the choice of which fringe leaf to expand is determined by the \textbf{search strategy}
A goal-node is a tree node for which the associated state is a goal state
\section{Generating/revealing the search tree}
\textit{Generating the search tree}
\begin{itemize}
	\item Suppose in the state space $\phi(x_0,\alpha)=\{x_1,x_2\},\phi(x_0,\beta)=\{x_2\}$ and $\phi(x_0,\gamma)=\varnothing$
	\item Expand the root node
	\item Suppose in the state space $\phi(x_1,\alpha)=\{x_1\},\phi(x_1,\beta)=\varnothing$ and $\phi(x_1,\gamma)=\{x_3\}$
	\item Expand the left-most fringe node (because the search strategy says so)
\end{itemize}
\textit{Revealing the search tree}
\begin{itemize}
	\item Fringe equates to the tree-nodes that have been revealed
\end{itemize}
\section{Search Tree vs State Space}
\textit{The search tree can also be seen as the "unrolling" of the state space}
\begin{itemize}
	\item For every "walk" w in the state space from the initial state there is a path \textbf{p} in the tree starting at the root node so that the states associated with the tree-nodes of this path \textbf{p} form the walk \textbf{w}
	\item For every path \textbf{p} in the tree starting at the root node there is a walk \textbf{w} in the state space from the initial state so that the states of \textbf{w} form the sequence of states associated with the nodes of \textbf{p}
\end{itemize}
\end{document}