\documentclass{article}[18pt]
\usepackage{../../../../format}
\lhead{Software Methodologies - AI Search}
\lstset{language=Python,
	basicstyle=\ttfamily,
	keywordstyle=\bfseries,
	showstringspaces=false,
	morekeywords={if, else, then, print, end, for, do, while},
	commentstyle=\color{red},
	tabsize=4
}

\begin{document}
\begin{center}
\underline{\huge Uninformed Search}
\end{center}
\section{Breadth First Search}
\textbf{Uninformed Search (Blind Search)} - Agents have no additional information beyond that provided in the definition of the search problem\\
\textbf{Breadth First Search} - Given the fringe of the search tree, every node on the fringe is expanded in each phase. I.e. a node with the smallest (search-tree) depth is expanded (in the case of a tie, order of expansion is an implementation detail)\\
\\
Algorithm issues
\begin{itemize}
	\item Need to create tree-nodes as required where tree-nodes need to be distinct but different tree-nodes might have the same associated state or action
	\item Need to retain enough information to reconstruct a path through the state space to a goal state, once we find one. Equivalently, a path from the root-node to a goal-node in the search tree
	\item We cannot be sure of the size of the state space or how much of the search tree we'll need to expand when we start
	\item We'll use specific data structures as and when they are needed
\end{itemize}
\section{Data Structures of our search tree}
Every node of our search tree has a unique identity, an integer id, and has associated with it the following data structure:
\begin{center}
	\texttt{(id, state, parent\_id, action, path\_cost, depth)}
\end{center}
where
\begin{itemize}
	\item \textbf{id} is the unique integral identifier of the node (1 for root)
	\item \textbf{state} is the associated state from the state space (initial state for root)
	\item \textbf{parent\_id} is the identifier of the parent of the node in the search tree (empty for root)
	\item \textbf{action} is the action which enabled the transition from the parent-node to the node in question (empty for root)
	\item \textbf{path\_cost} is the cumulative cost of the step-costs in the path from the root of the search tree to the node in question (0 for root)
	\item \textbf{depth} is the depth of the node in question in the search tree (0 for root)
\end{itemize}
The transition function, the initial state, the goal test, and so on, all come as part of the search problem instance
We reiterate:
\begin{itemize}
	\item There could be two children with the same parent in the search tree, where the associated states of the children are identical but where the associated actions are different, and vice versa
	\item These children will have unique identifiers
\end{itemize}
\section{Breadth-first search}
\begin{lstlisting}[mathescape=true,tabsize=2]
newid = 1 #initialize node identifier
register node:
S[1]=initial_state; #Associate initial state with root
P[1] = -; #Root has no parent
A[1] = -; # No action to get to root
PC[1] = 0; #No path cost to get to root
D[1] = 0 #Depth of root is 0

initialize fifo queue (of tuples):
fringe F = [(newid, S[1], P[1], A[1], PC[1], D[1])]
if node 1 is a goal-node then return 1 #Halt when find goal node on fringe
else
	while $F \neq \varnothing$ do
		pop (id, state, parent_id, action, path_cost, depth) from F
		#Child reachable in 1 step
		for each state child and action $\alpha$ for which we have that child $\in$$\phi$(state, $\alpha$) do
			newid = newid + 1 #increment identifier
			register node: 
			S[newid]= child; #associated state is child
			P[newid] = id; #parent is id
			A[newid] = $\alpha$ #action is $\alpha$
			#Path cost is path-cost to parent id+ step-cost to new id
			PC[newid] = PC[id]+$\sigma$(state, $\alpha$, child) 
			D[newid]=D[id] + 1 #depth is depth to parent + 1
			if node newid is a goal-node then return newid
			# Put a non-goal node on fringe
			else add (newid, S[newid], P[newid], A[newid], A[newid], PC[newid], D[newid]) to F
return 0
\end{lstlisting}
\textit{Can we ever delete some of the search-tree from memory?}
\begin{itemize}
	\item Expand all nodes from root
	\item Keep expanding
	\item Can only remove from memory if there are no more nodes beneath it as you can't know if there would be a goal node beneath it, in which case, the node would form part of the path
\end{itemize}
\textit{Why do we include PC and D when we can reconstruct them?}
\begin{itemize}
	\item The function that determines if a goal is a goal node depends on them
\end{itemize}
\section{Measuring performance}
We evaluate a search strategy using four parameters
\begin{enumerate}
	\item \textbf{Completeness} - Is the algorithm guaranteed to find a solution if there is one?
	\item \textbf{Optimality} - If the strategy finds a solution, does it find an optimal solution?
	\item \textbf{Time complexity} - How long does it take to find a solution?
	\item \textbf{Space complexity} - How much memory does it take to perform the search
\end{enumerate}	
\begin{itemize}
	\item Time/space complexity associated with specific notions of "size" or "difficulty" - we need these notions as our state space or search trees might be infinite
	\begin{itemize}
		\item \textit{b}: the \textbf{branching factor}, namely the maximum number of children of a tree node - the maximum number of successors of any state
		\item \textit{d}: the depth of the shallowest goal-node in the search tree - the smallest number of transitions from the initial state to a goal in the state space
		\item \textit{m}: the maximum length of a path from the root in the search tree - the maximum length of a walk in the state space
	\end{itemize}
	\item The parameters b and m might be infinite
	\item d might not be defined
\end{itemize}
\textbf{Time}: The number of tree-nodes generated (or revealed)\\
\textbf{Space}: The number of tree-nodes held in memory at any one time
\section{Performance of BFS}
\begin{itemize}
	\item Suppose that our state space is such that every state has exactly b successors i.e. the branching factor is \textit{b}(and so is bounded)
	\item The root generates \textit{b} nodes (at depth 1), and each of these \textit{b} nodes generates \textit{b} nodes (at depth 2)
	\item Each of these $b^2$ nodes (at depth 2) generates \textit{b} nodes at depth 3, and so on
	\item Hence the number of nodes generated at depth at most d is
	$$1+b+b^2+b^3+...+ b^d$$
	$$=(b^{d+1}-1)/(b-1)=\Omega(b^d)$$
	\item Note that all of these nodes need to be held in storage as potentially any one of them could appear on a path from some goal node to the root
	\item Thus, both the time and the space complexity is exponential in the shallowest depth d of a goal node. Equivalently, the nearest goal state to the initial state in the search space, in terms of the number of transitions
	\item BFS is complete (assuming bounded branching factor, if infinite, spend all time on first level of tree) but not necessarily optimal (might be shorter path cost at larger depth, which would then be more optimal)
\end{itemize}
\section{Depth-First Search}
\textbf{Depth First search} - Given the fringe of the search tree, a node of greatest depth is expanded\\
\\
DFS is implemented by identical code to that implementing BFS, except that the queue is now a lifo queue (also called a stack)\\
DFS also has a (more well-known) recursive implementation
\subsection{DFS Performance}
\begin{itemize}
	\item DFS is not complete as could go down a tree forever
	\item However, DFS requires considerably less memory than DFS. Once we "back up" from a node, the register data corresponding to that node (and to all its ancestors) can be expunged from memory as we know that the node will not be involved in any solution
	\item This means only having to store all nodes on the path p from the root up to the current node and the other children of all nodes on path p
	\item Thus, the amount of memory required for a DFS is $\mathcal{O}(bm)$, where b is the branching factor and m is the maximum depth of any node (not the shallowest depth of any goal node) in the search tree (if it exists). Even in such a search tree, DFS may take time exponential in m
\end{itemize}
\section{Iterative deepening}
\begin{itemize}
	\item We can implement BFS using DFS
	\item Iterative deepening:
	\begin{itemize}
		\item Undertake DFS but cut paths off at depth 1, if goal node found then return
		\item Undertake DFS but cut paths off at depth 2, if goal node found then return
		\item and so on
	\end{itemize}
	\item Note that
	\begin{itemize}
		\item We only ever keep a linear ($\mathcal{O}(bh)$) number of nodes in memory but in order to check paths of length k+1 we have to recompute all paths of length k
	\end{itemize}
	\item Complexity
	\begin{itemize}
		\item Complete (assuming bounded branching factor) but not necessarily optimal
		\item The number of nodes held in memory is $\mathcal{O}(bd)$
		\item The number of nodes generated when goal is at depth d is
		$$db+(d-1)b^2+...+2b^{d-1}+b^d=\Omega(b^{d+1})$$
	\end{itemize}
	
	
\end{itemize}
\section{Assignment}
\begin{itemize}
	\item Implement 2 of the algorithms we cover in the course
	\item Solve the TSP
	\item Find the "best tour" using the algorithms
	\item More marks for harder algorithms
	\item Tweaking the algorithms to give better results
	\item Also ran on secret city sets
\end{itemize}
\end{document}