\documentclass{article}[18pt]
\usepackage{../../../../format}
\lhead{Software Methodologies - AI Search}


\begin{document}
\begin{center}
\underline{\huge A* Search}
\end{center}
He might ask general stuff about this, but wouldn't have a whole exam question relating to proving this.
\section{A* Search Completeness}
\textbf{Theorem 1}:\\
If
\begin{itemize}
	\item There is a fixed $\epsilon>0$ such that all step costs exceed $\epsilon$
	\item The branching factor is bounded by b
\end{itemize}
Then A* search is complete (terminates having found a goal-node if there is one)\\
\\
\textbf{Proof}:\\
Suppose that there is a goal-node but A* search doesn't find it
\begin{itemize}
	\item So, A* search does not terminate having found a goal-node
	\item So, A* search terminates without finding a goal-node or A* does not terminate
\end{itemize}
Case (a): suppose A* search terminates without finding a goal-node (which exists by assumption)
\begin{itemize}
	\item So, the search tree is finite and every goal has been expanded
	\item So, some goal-node must have been on the fringe so that it has minimal f-value at some point
	\item So, we can't have this case
\end{itemize}
Case (b): suppose A* search does not terminate
\begin{itemize}
	\item some nodes are expanded - having been on the fringe
	\item some nodes might be placed on the fringe but not expanded
	\item some nodes might never be placed on the fringe, so they are not expanded
\end{itemize}
In particular, every goal-node is
\begin{itemize}
	\item either never placed on the fringe, or
	\item is placed on the fringe but remains there throughout - it can't be a node of minimal f value from amongst the fringe nodes
\end{itemize}
Let's pause the main proof for a moment and prove a useful lemma\\
\\
\textbf{Lemma 2}: Let $\delta>0$ be any chosen value. There are only finitely many nodes of the search tree with f-value(path cost+heuristic cost) at most $\delta$\\
\\
Proof: 
\begin{itemize}
	\item Let z be any node in the search tree, of depth d, say
	\item The cost $g(z)$ of the path from root to z is no less that $d\epsilon$ (every step cost is at least $\epsilon$, by assumption) (each step cost is at least $\epsilon$, and d steps)
	\item Hence, $f(z)=g(z)+h(z)\geqslant d\epsilon + h(z) \geqslant d\epsilon$
	\item If $f(z)\leqslant \delta$, then $d\epsilon \leqslant \delta$; that is, $d\leqslant \delta \backslash \epsilon$
	\item So, all nodes z for which $f(z)\leqslant \delta$ have depth at most $\delta \backslash \epsilon$ (a fixed value)
	\item But as the branching factor is bounded by b, there are finitely many nodes of depth $\delta \backslash \epsilon$ and so also f-value at most $\delta$
\end{itemize}
Recall we are in case (b) (Suppose A* search does not terminate)\\
\\
Suppose there is a non-goal-node z that is not expanded where the search tree path p from z to the root doesn't contain a goal-node
\begin{itemize}
	\item We may assume that all nodes on p from the root to z are expanded. If not then just take z to be the closest node to the root on this path p that is not expanded (z must be a non-goal-node as no goal-node lies on the path p)
\end{itemize}
As the parent of p is expanded, z appears on the fringe at some point\\
As z is not expanded
\begin{itemize}
	\item When z is placed on the fringe, it does not have minimal f-value (if it did, then it would be expanded) from amongst the fringe nodes and is such thereafter
\end{itemize}
By lemma 2, there are finitely many search tree nodes with f-value at most f(z)
\begin{itemize}
	\item So, at some point z will have minimal f-value from amongst the fringe nodes and so be expanded
\end{itemize}
Hence, every non-goal-node z where path from the root to z does not contain a goal-node is necessarily expanded\\
\\
\\
Let w be a goal-node so that the path from the root to w contains only non-goal-nodes\\
By above, every node of this path is expanded, so at some point w will appear on the fringe\\
But by lemma 2 with the value f(w)
\begin{itemize}
	\item There are finitely many search tree nodes with f-value at most f(w)
	\item So, at some point w will have minimal f-value from amongst the fringe nodes
\end{itemize}
So the A* search algorithm terminates(a contradiction)\\
\\
So, neither case(a) or case(b) holds
\begin{itemize}
	\item Which means our very first assumption "Suppose that there is a goal-node but A* search doesn't find it" does not hold
\end{itemize}
Hence, if there is a goal-node then A* search will find it, assuming a bounded branching factor and a lower bound on step-costs\\
\\
If the branching factor is infinite then lemma 2 will not hold
\section{A* search optimality}
\begin{defin}[Admissible heuristic]
The value $h(z)$ of any node z in the search tree is always at most the cost of a minimal cost path from z to a goal-node
\end{defin}
In "geographic" problems, not that the straight-line distance between two locations is an admissible heuristic\\
\\
\textbf{Theorem 3}:\\
If the heuristic function h is admissible and A* search terminates through finding a goal-node then we always obtain an optimal solution\\
\\
\textbf{Proof}\\
Suppose that A* search terminates because a goal-node w appears on the fringe with minimal f-value
\begin{itemize}
	\item but where the value f(w) is strictly greater than the cost c* of an optimal path to some goal-node (c* is optimal)
\end{itemize}
In particular, at termination every other fringe node z is such that $f(z)\geqslant f(w)$\\
Also at termination, at least one node on the fringe, call it z*, is on an optimal path in the search tree to some "optimal" goal-node W*
\begin{itemize}
	\item So we have $f(w*)=g(w*)+h(w*)=g(w*)=c*$
\end{itemize}
Note that no goal-node appears "above" the fringe\\
\\
The optimal path in the search tree from the root to w* is formed by
\begin{itemize}
	\item a path from the root to z* of cost g(z*)
	\item followed by a path from z* to w* of cost c, say. So c*=g(z*)+c
\end{itemize}
As our heuristic is admissible
\begin{itemize}
	\item h(z*)$\leqslant$ c\\
	and so
	\item $f(z*)=g(z*)+h(z*)\leqslant g(z) +c =c*$
\end{itemize}
But at termination
\begin{itemize}
	\item w was a node of minimal f-value on the fringe, with $f(w)>c*$
	\item z* was on the fringe with $f(z*)\leqslant c*$
\end{itemize}
Hence, if A* search terminates through finding a goal-node then we always obtain an optimal solution - assuming that h is admissible
\section{A* search optimally efficient}
Not only is A* search complete and optimal (under our reasonable conditions) but A* search can be forced to be \textbf{optimally efficient}
\begin{itemize}
	\item Every complete and optimal "search-tree-path-extended-from-root" algorithm necessarily expands all nodes whose f-value is less than the optimal path-cost c* (plus maybe some of f-value c*)
	\begin{itemize}
		\item i.e., the nodes expanded by A* search (plus maybe some of f-value c*)
	\end{itemize}
\end{itemize}
A heuristic function h is a \textbf{consistent} heuristic if:
\begin{itemize}
	\item for every node z in the search tree and for every child node z' of z
	\begin{itemize}
		\item the step-cost c of the transition from z to z' is such that $h(z)\leqslant c+h(z')$
	\end{itemize}
\end{itemize}
\textbf{Theorem 4}:\\
If:
\begin{itemize}
	\item h is consistent
	\item there is a fixed $\epsilon >0$ such that all step-costs exceed $\epsilon$
	\item the branching factor is bounded by b
\end{itemize}
Then A* search is optimally efficient\\
\begin{important}[Consistency vs Admissibility]
If a heuristic is consistent then it is admissible
\end{important}
\section{Practical limitations of A* search}
Whilst out A* search is complete, optimal and optimally efficient, it turns out that in practice there are still exponentially-many (in the depth of an optimal goal-node) nodes under the potential expansion in many fringes\\
\\
The potentially exponential sizes of fringes, allied with the fact that all fringe nodes must be stored in memory, means that A* search is memory-demanding\\
\\
The error in the heuristic function has a significant impact on A* search
\begin{itemize}
	\item Unless the error in the heuristic function h is such that
	$$|h(z)-h*(z)|=\mathcal{O}(\log(h*(z)))$$
	where $h*(z)$ is the true optimal cost of getting from node z to a goal-node
	\begin{itemize}
		\item there can be an exponential number of nodes for potential expansion
	\end{itemize}
\end{itemize}
We can use DFS+iterative deepening to "implement" A* search as IDA*
\begin{itemize}
	\item do a DFS but so that no node with f-value above some threshold is expanded
	\item if no goal-node is found then increase the threshold and repeat
	\begin{itemize}
		\item otherwise, if a goal-node z is found then set the threshold to f(z), repeat a DFS, and return the goal-node found with minimal f-value
	\end{itemize}
\end{itemize}
IDA* is complete and optimal under the conditions of Theorem 4
\end{document}