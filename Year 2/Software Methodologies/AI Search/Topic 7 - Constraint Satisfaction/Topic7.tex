\documentclass{article}[18pt]
\ProvidesPackage{format}
%Page setup
\usepackage[utf8]{inputenc}
\usepackage[margin=0.7in]{geometry}
\usepackage{parselines} 
\usepackage[english]{babel}
\usepackage{fancyhdr}
\usepackage{titlesec}
\hyphenpenalty=10000

\pagestyle{fancy}
\fancyhf{}
\rhead{Sam Robbins}
\rfoot{Page \thepage}

%Characters
\usepackage{amsmath}
\usepackage{amssymb}
\usepackage{gensymb}
\newcommand{\R}{\mathbb{R}}

%Diagrams
\usepackage{pgfplots}
\usepackage{graphicx}
\usepackage{tabularx}
\usepackage{relsize}
\pgfplotsset{width=10cm,compat=1.9}
\usepackage{float}

%Length Setting
\titlespacing\section{0pt}{14pt plus 4pt minus 2pt}{0pt plus 2pt minus 2pt}
\newlength\tindent
\setlength{\tindent}{\parindent}
\setlength{\parindent}{0pt}
\renewcommand{\indent}{\hspace*{\tindent}}

%Programming Font
\usepackage{courier}
\usepackage{listings}
\usepackage{pxfonts}

%Lists
\usepackage{enumerate}
\usepackage{enumitem}

% Networks Macro
\usepackage{tikz}


% Commands for files converted using pandoc
\providecommand{\tightlist}{%
	\setlength{\itemsep}{0pt}\setlength{\parskip}{0pt}}
\usepackage{hyperref}

% Get nice commands for floor and ceil
\usepackage{mathtools}
\DeclarePairedDelimiter{\ceil}{\lceil}{\rceil}
\DeclarePairedDelimiter{\floor}{\lfloor}{\rfloor}

% Allow itemize to go up to 20 levels deep (just change the number if you need more you madman)
\usepackage{enumitem}
\setlistdepth{20}
\renewlist{itemize}{itemize}{20}

% initially, use dots for all levels
\setlist[itemize]{label=$\cdot$}

% customize the first 3 levels
\setlist[itemize,1]{label=\textbullet}
\setlist[itemize,2]{label=--}
\setlist[itemize,3]{label=*}

% Definition and Important Stuff
% Important stuff
\usepackage[framemethod=TikZ]{mdframed}

\newcounter{theo}[section]\setcounter{theo}{0}
\renewcommand{\thetheo}{\arabic{section}.\arabic{theo}}
\newenvironment{important}[1][]{%
	\refstepcounter{theo}%
	\ifstrempty{#1}%
	{\mdfsetup{%
			frametitle={%
				\tikz[baseline=(current bounding box.east),outer sep=0pt]
				\node[anchor=east,rectangle,fill=red!50]
				{\strut Important};}}
	}%
	{\mdfsetup{%
			frametitle={%
				\tikz[baseline=(current bounding box.east),outer sep=0pt]
				\node[anchor=east,rectangle,fill=red!50]
				{\strut Important:~#1};}}%
	}%
	\mdfsetup{innertopmargin=10pt,linecolor=red!50,%
		linewidth=2pt,topline=true,%
		frametitleaboveskip=\dimexpr-\ht\strutbox\relax
	}
	\begin{mdframed}[]\relax%
		\centering
		}{\end{mdframed}}



\newcounter{lem}[section]\setcounter{lem}{0}
\renewcommand{\thelem}{\arabic{section}.\arabic{lem}}
\newenvironment{defin}[1][]{%
	\refstepcounter{lem}%
	\ifstrempty{#1}%
	{\mdfsetup{%
			frametitle={%
				\tikz[baseline=(current bounding box.east),outer sep=0pt]
				\node[anchor=east,rectangle,fill=blue!20]
				{\strut Definition};}}
	}%
	{\mdfsetup{%
			frametitle={%
				\tikz[baseline=(current bounding box.east),outer sep=0pt]
				\node[anchor=east,rectangle,fill=blue!20]
				{\strut Definition:~#1};}}%
	}%
	\mdfsetup{innertopmargin=10pt,linecolor=blue!20,%
		linewidth=2pt,topline=true,%
		frametitleaboveskip=\dimexpr-\ht\strutbox\relax
	}
	\begin{mdframed}[]\relax%
		\centering
		}{\end{mdframed}}
\lhead{Software Methodologies - AI Search}


\begin{document}
\begin{center}
\underline{\huge Constraint Satisfaction}
\end{center}

\begin{defin}[Constraint Satisfaction problems]
Structured global path based search problems
\begin{itemize}
	\item There are variables $X_1,X_2,...,X_n$ - each variable $X_i$ has a non-empty domain of values $D_i$
	\item There are constraints $C_1,C_2,...,C_m$
	\begin{itemize}
		\item Each constraint $C_j$ involves some tuple of variables $(X_{i1}, X_{i2},...,X_{ik})$ called the scope
		\item Specifies the allowable combinations of values for scope variables
		\item $C_j$ is usually given as a relation of k-tuples from $D_{i1}\times D_{i2}\times \ldots \times D_{ik}$ detailing these allowable simultaneous values
	\end{itemize}
	\item An assignment associates a value from $D_i$ to the variable $X_i$, for some or all of the variables
	\begin{itemize}
		\item It is called \textbf{complete} if every variable is assigned a value
		\item It is called \textbf{consistent} (or legal) if no constraint is violated
	\end{itemize}
	\item A solution is a complete, consistent assignment
	\item Some CSPs require a solution to maximise/minimize some objective function defined on the set of arguments
\end{itemize}
\end{defin}
\section{Examples of CSPs}
\subsection{Graph 3-colourability problem (G3COL)}
\begin{itemize}
	\item Variable $X_i$ for every vertex $v_i$ in the given graph G
	\item Domain of values for each variable is \{R,W,B\} (Red, White, Blue)
	\item For every pair of variables $(X_i,X_j)$ for which the corresponding vertices are joined by an edge $e=(v_i,v_j)$ in G
	\begin{itemize}
		\item There is a constraint $C_e$ with scope $(X_i,X_j)$
		\item The tuples of simultaneous values in the constraint $C_e$ consist of the set of all pairs of distinct colours
	\end{itemize}
\end{itemize}
\subsection{Satisfiability(SAT)}
\begin{itemize}
	\item Variable $X_i$ foe every propositional variable $X_i$ appearing in the given c.n.f formula $\phi_1 \land \phi_2 \land ... \land \phi_m$
	\item Domain of values for each variable is \{true,false\}
	\item Constraint $C_j$ for every clause $\phi_j$
	\begin{itemize}
		\item The scope of $C_j$ is the set of propositional variables involved in $\phi_j$
		\item The tuples of simultaneous values in the constraint $C_j$ are the truth assignments on the corresponding propositional variables making $\phi_j$ evaluate to true
	\end{itemize}
\end{itemize}
\section{CSPs as search problems}
We can realise CSPs as "global path-based" search problems canonically
\begin{itemize}
	\item A state in the state space is an assignment
	\item The initial state consists of the empty assignment
	\item The state x' is the successor (we assume only one action) of some state x if
	\begin{itemize}
		\item x is consistent(we can move to inconsistent states, but once there, are stuck) and the assignment x' is the extension of x by the assignment of some value to some previously unassigned variable
	\end{itemize}
	\item A goal state is a complete, consistent assignment (lies at depth n in the search tree where n is the number of variables)
	\item We have a constant step-cost of 1 per transition
\end{itemize}
We can also realise a CSP as a "local state-based" search problem, the same as above, but $f(x)=1$ if x is a complete consistent assignment and $f(x)=0$ otherwise\\
\\
The CSP framework lends itself to the development of specific heuristic methods
\begin{itemize}
	\item Not available in the general setting
	\item So that the inherent structure of a CSP leads to an "exponentially simplified" search algorithm (coming up)
\end{itemize}
\section{Commutativity}
The most crucial observation to make about CSPs - commutativity
\begin{itemize}
	\item In order to come up with an assignment it does not matter in which order we choose to assign values to variables
\end{itemize}
Hence in every CSP search algorithm we expand a node of the search tree by considering possible assignments for only a single variable - it's up to us to choose which variable\\
\\
Consequently we no longer speak of "the" search tree with CSPs but talk of "a" search tree, for we will build different search trees depending on which variable we choose to expand a node with respect to\\
\\
We are essentially "pruning" the search tree of a vast number of branches but do not risk missing a goal-node if there is one\\
\\
We also have leeway in the order in which we choose to assign variables - judicious orderings can yield improved performance
\section{Real-world CSPs}
In reality CSPs can be much more general e.g., domains of values need not be finite nor take discrete values - many real world problems have as their domains the natural numbers or real numbers\\
\\
In the case of infinite domains of values, constraints might not be finitely describable, they might contain an infinite set of allowable value combinations\\
\\
In such a circumstance we need to develop constraint languages
\begin{defin}[Constraint Languages]
Languages that allow us to describe infinite sets of objects
\end{defin}
Many real world constraint satisfaction problems involve preference constraints:
\begin{itemize}
	\item Lectures A and B both prefer only teaching in the morning but are willing to teach in the afternoon
\end{itemize}
\section{Back-tracing search for CSPs}
\begin{defin}[Back-tracking search]
A depth-first search w.r.t chosen orderings of both variables and values
\end{defin}
Commutativity: we make assignments to one variable at a time so that
\begin{itemize}
	\item Every search tree node that is constructed (apart from the root) is labelled with the assignment of some value to some variable
	\item The children of any node correspond to the assignment of a different value to the same variable
\end{itemize}
If we ever have some fringe node for which the assignment obtained by tracing the path from the fringe node violates a constraint (inconsistent) then this fringe node has no successors\\
\\
So we obtain different search trees not only depending on the order in which we choose variables to assign values to but also depending upon the order in which we choose to assign values to a chosen variable
\section{Example of a back-tracking search}
Solve the following CSP
\begin{itemize}
	\item Variables B,A,C with domain of values \{1,2,3,4\}
	\item Constraints:
\[
\begin{array}{l}{(A, B) \in\{(1,2),(1,3),(1,4),(2,3),(2,4),(3,4)\}} \\ {(B, C) \in\{(1,2),(1,3),(1,4),(2,3),(2,4),(3,4)\}}\end{array}
\]
\end{itemize}
Expand the nodes until there is an acceptable one
\section{Back tracking algorithm}
\begin{lstlisting}[caption=Backtrack-Search(CSP)]
answer=Recursive-Backtrack(ass=$\varnothing$)
\end{lstlisting}
\begin{lstlisting}[caption=Recursive-Backtrack(ass)]
if ass is complete and consistent then return ass
if ass is consistent than
	var=Select-Variable(ass)
	for each value in Order-Domain(var,ass)
		ass = ass + {var=value}
		result = Recursive-Backtrack(ass)
		if result $\neq$ failure then
			return result
		else
			ass = ass - {var=value}
return failure
\end{lstlisting}
\section{Minimum remaining values (MRV) heuristic}
Can often do better than just leaving the functions Select-Variable and Order-Domain to be implementation dependent - we can employ heuristics to choose variables and values
\begin{defin}[Minimum remaining values heuristic]
\begin{itemize}
	\item Decides upon the next variable to assign a value to
	\item Intuition: find variable that are likely to "cause problems" as soon as we can as otherwise we'll have a lengthy backtrack to undertake
	\item Maintains, for every unassigned variable X, the number of legitimate values
	\item The next variable chosen has the smallest set of legitimate values with ties broken arbitrarily
\end{itemize}
\end{defin}
Extreme scenario
\begin{itemize}
	\item We are at point p in the execution and the set of legitimate values for X is empty
	\item Immediately attempt to assign a value to X, fail and recursively back-track
	\begin{itemize}
		\item Otherwise, if we were to make alternative assignments from point p then at some future time we would still have to back-track all the way to point p
	\end{itemize}
\end{itemize}
\section{Least constraining value heuristic}
\begin{defin}[Least constraining value heuristic]
\begin{itemize}
	\item Assists in the selection of a value to an already chosen variable X
	\item Any choice of value for X will, in general, rule out certain values for other unassigned variables (from their set of legitimate values)
	\item The total number of all such ruled-out values over all unassigned variables is maintained (in reality we just need to worry about those with unassigned variable sharing a constraint with X)
	\item Values for X are ordered in decreasing numbers of ruled-out values - "illegal" values are not considered and ties are broken arbitrarily 
\end{itemize}
\end{defin}
Intuition - Try to leave the maximum flexibility for subsequent variable assignments\\
\\
When we choose some value for our variable, we do not touch the sets of legitimate values for other unassigned variables even though we might know some of these values can be ruled out as this requires additional maintenance 
\end{document}