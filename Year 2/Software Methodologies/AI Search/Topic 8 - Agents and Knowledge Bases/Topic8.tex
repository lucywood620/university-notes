\documentclass{article}[18pt]
\usepackage{../../../../format}
\lhead{Software Methodologies - AI Search}


\begin{document}
\begin{center}
\underline{\huge Agents and Knowledge Bases}
\end{center}
\section{Knowledge-based agents}
\begin{defin}[Knowledge-based agents]
Knowledge based agents are structured around their knowledge-base (KB)
\begin{itemize}
	\item Collection of logical formulae from some knowledge presentation language
	\item Used to express assertions representing the agent's knowledge
	\item Can implement the notion of "state"
\end{itemize}
\end{defin}
Logical agents can:
\begin{itemize}
	\item Query their knowledge base
	\item Infer new formulae from the knowledge-base and percepts using some inference system
\end{itemize}
A generic knowledge-based agent program is structured as follows
\begin{lstlisting}[caption=KB-Agent(percept)]
Tell(KB, Make-percept-formula(percept,t))
action=Ask(KB, Make-action-query(t))
Rell(KB, Make-action-formula(action,t))
t = t +1
\end{lstlisting}
\section{Our working example - Wumpus World}
We shall work with Wumpus World to illustrate issues regarding the design of intelligent agents\\
\\
Wumpus World is a cave system consisting of rooms connected by passages\\
\\
Lurking somewhere in some room is a Wumpus - a beast that eats anyone who enters its room - so, the room with the Wumpus should be avoided\\
\\
The Wumpus can be shot by an agent, but the agent only has one arrow\\
\\
Some rooms contain bottomless pits which will trap anyone who wanders into these rooms excepts for the Wumpus which is too big to fall in, so again these rooms should be avoided\\
\\
However in one room there is a heap of gold - agent's greed for gold is greater than the fear of the Wumpus

\end{document}