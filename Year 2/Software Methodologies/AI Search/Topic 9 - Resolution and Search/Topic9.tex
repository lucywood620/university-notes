\documentclass{article}[18pt]
\ProvidesPackage{format}
%Page setup
\usepackage[utf8]{inputenc}
\usepackage[margin=0.7in]{geometry}
\usepackage{parselines} 
\usepackage[english]{babel}
\usepackage{fancyhdr}
\usepackage{titlesec}
\hyphenpenalty=10000

\pagestyle{fancy}
\fancyhf{}
\rhead{Sam Robbins}
\rfoot{Page \thepage}

%Characters
\usepackage{amsmath}
\usepackage{amssymb}
\usepackage{gensymb}
\newcommand{\R}{\mathbb{R}}

%Diagrams
\usepackage{pgfplots}
\usepackage{graphicx}
\usepackage{tabularx}
\usepackage{relsize}
\pgfplotsset{width=10cm,compat=1.9}
\usepackage{float}

%Length Setting
\titlespacing\section{0pt}{14pt plus 4pt minus 2pt}{0pt plus 2pt minus 2pt}
\newlength\tindent
\setlength{\tindent}{\parindent}
\setlength{\parindent}{0pt}
\renewcommand{\indent}{\hspace*{\tindent}}

%Programming Font
\usepackage{courier}
\usepackage{listings}
\usepackage{pxfonts}

%Lists
\usepackage{enumerate}
\usepackage{enumitem}

% Networks Macro
\usepackage{tikz}


% Commands for files converted using pandoc
\providecommand{\tightlist}{%
	\setlength{\itemsep}{0pt}\setlength{\parskip}{0pt}}
\usepackage{hyperref}

% Get nice commands for floor and ceil
\usepackage{mathtools}
\DeclarePairedDelimiter{\ceil}{\lceil}{\rceil}
\DeclarePairedDelimiter{\floor}{\lfloor}{\rfloor}

% Allow itemize to go up to 20 levels deep (just change the number if you need more you madman)
\usepackage{enumitem}
\setlistdepth{20}
\renewlist{itemize}{itemize}{20}

% initially, use dots for all levels
\setlist[itemize]{label=$\cdot$}

% customize the first 3 levels
\setlist[itemize,1]{label=\textbullet}
\setlist[itemize,2]{label=--}
\setlist[itemize,3]{label=*}

% Definition and Important Stuff
% Important stuff
\usepackage[framemethod=TikZ]{mdframed}

\newcounter{theo}[section]\setcounter{theo}{0}
\renewcommand{\thetheo}{\arabic{section}.\arabic{theo}}
\newenvironment{important}[1][]{%
	\refstepcounter{theo}%
	\ifstrempty{#1}%
	{\mdfsetup{%
			frametitle={%
				\tikz[baseline=(current bounding box.east),outer sep=0pt]
				\node[anchor=east,rectangle,fill=red!50]
				{\strut Important};}}
	}%
	{\mdfsetup{%
			frametitle={%
				\tikz[baseline=(current bounding box.east),outer sep=0pt]
				\node[anchor=east,rectangle,fill=red!50]
				{\strut Important:~#1};}}%
	}%
	\mdfsetup{innertopmargin=10pt,linecolor=red!50,%
		linewidth=2pt,topline=true,%
		frametitleaboveskip=\dimexpr-\ht\strutbox\relax
	}
	\begin{mdframed}[]\relax%
		\centering
		}{\end{mdframed}}



\newcounter{lem}[section]\setcounter{lem}{0}
\renewcommand{\thelem}{\arabic{section}.\arabic{lem}}
\newenvironment{defin}[1][]{%
	\refstepcounter{lem}%
	\ifstrempty{#1}%
	{\mdfsetup{%
			frametitle={%
				\tikz[baseline=(current bounding box.east),outer sep=0pt]
				\node[anchor=east,rectangle,fill=blue!20]
				{\strut Definition};}}
	}%
	{\mdfsetup{%
			frametitle={%
				\tikz[baseline=(current bounding box.east),outer sep=0pt]
				\node[anchor=east,rectangle,fill=blue!20]
				{\strut Definition:~#1};}}%
	}%
	\mdfsetup{innertopmargin=10pt,linecolor=blue!20,%
		linewidth=2pt,topline=true,%
		frametitleaboveskip=\dimexpr-\ht\strutbox\relax
	}
	\begin{mdframed}[]\relax%
		\centering
		}{\end{mdframed}}
\lhead{Software Methodologies - AI Search}


\begin{document}
\begin{center}
\underline{\huge Resolution as Search}
\end{center}
\section{Proof systems}
The proof system resolution:
\begin{itemize}
	\item All formulae are clauses, i.e. disjunctions of literals
	\item One rule of inference, the resolution rule
\end{itemize}
We may assume that our knowledge base KB is a set of clauses as follows:
\begin{itemize}
	\item Think of KB as a conjunction of all its formulae
	\item Find an equivalent formula in conjunctive normal form
	\item Split the c.n.f conjunction into a set of clauses, so we have moved back to the KB being a set of formula
\end{itemize}
\section{The resolution inference algorithm}
Suppose that we want to decide whether $KB \models \phi$
\begin{itemize}
	\item Convert $\lnot \phi$ to cnf
	\item Add the resulting clauses to (the set of clauses) KB
	\item Iteratively apply the resolution rule to produce new clauses which are added to the set of clauses if they are not already present
	\item This iterative process continues until either
	\begin{itemize}
		\item There are no new clauses to be added - in which case our algorithm answers that KB \textbf{does not} entail $\phi$
		\item Two clauses resolve to yield the empty clause, so these clauses must be X and $\lnot X$, for some boolean variable X, in which case our algorithm answers that KB \textbf{does} entail $\phi$
	\end{itemize}
	\item We can also factor, that is, replace $\alpha\lor \alpha$ with $\alpha$ ($\alpha$ is some literal)
\end{itemize}
The resolution inference algorithm is both sound and complete, so in the worst case it is exponential
\subsection{Example}
Recall our wumpus world KB from earlier
\[
\left\{\neg P_{1,1}, B_{1,1} \Leftrightarrow\left(P_{1,2}{\lor} P_{2,1}\right), B_{1,2} \Leftrightarrow\left(P_{1,1}{\lor} P_{2,2}{\lor} P_{1,3}\right), \neg B_{1,1}, B_{1,2}\right\}
\]
Rewrite these formula so that KB is in c.n.f and take the clauses
\[
\begin{array}{l}{\left\{\neg P_{1,1},-B_{1,1}\lor P_{1,2}\lor P_{2,1}, \neg P_{1,2}\lor B_{1,1}, \neg P_{2,1}\lor B_{1,1}\right.,} \\ {\neg B_{1,2}\lor P_{1,1}\lor P_{2,2}\lor P_{1,3}, \neg P_{1,1}\lor B_{1,2}, \neg P_{2,2}\lor B_{1,2},} \\ {\left.\neg P_{1,3}\lor B_{1,2}, \neg B_{1,1}, B_{1,2}\right\}}\end{array}
\]
Suppose that the agent returns from room [1,2] to room [1,1] and then moves to room [2,1]\\
As a consequence, suppose the following formulae are added to KB
\[
B_{2,1} \Leftrightarrow\left(P_{1,1}\lor P_{2,2}\lor P_{3,1}\right) \text { and } \neg B_{2,1}
\]
Which converted to c.n.f is:
\[
B_{2,1}\lor P_{1,1}\lor P_{2,2}\lor P_{3,1}, \neg P_{1,1}\lor B_{2,1}, \neg P_{2,2}\lor B_{2,1}, \neg P_{3,1}\lor B_{2,1}, \neg B_{2,1}
\]
Suppose we want to know whether $KB \models P_{1,3}$ (which is finding if there is a pit in $P_{1,3}$)\\
Add $\lnot P_{1,3}$ to our set of clauses and apply Resolution\\
\\
We get the empty set, so there is definitely a pit in room [1,3]\\
\\
This is a bit of a cheat as the resolution is just being applied, where in reality it is difficult for an algorithm to do this. It is especially difficult as the set of clauses could get very large-
\section{Realising Resolution via "global path-based" search}
\begin{itemize}
	\item In our first illustration we "magically" found the derivation of the empty clause
	\item In general, what we'll need to do is apply a search algorithm in order to try and "find" the empty clause
\end{itemize}
{\renewcommand{\arraystretch}{2}
\begin{tabularx}{\textwidth}{|X|X|}
\hline
State & Set of clauses\\
\hline
Initial state & The clauses of $KB \lor \lnot \phi$\\
\hline
Actions & Resolve and Factor\\
\hline
Goal state & Any set of clauses containing the empty clause\\
\hline
State Transition & 
($\Sigma$, action, $\Sigma$') such that
\begin{itemize}
	\item \textbf{Resolve}: the target of clauses $\Sigma'$ is the result of applying the resolution rule of inference (once) to the source set of clauses $\Sigma$
	\item \textbf{Factor}: we factor a clause $\Sigma$ to obtain the set of clauses $\Sigma'$
\end{itemize}
\\
\hline
Step cost & 1\\
\hline

\end{tabularx}}\\
\\
The path from the initial state to a goal state is a "proof" with an optimal path being a "shortest proof"
\section{Realising Resolution via "local state-based" search}
Here is a "local state-based" search formulation of Resolution\\
\\
{\renewcommand{\arraystretch}{2}
\begin{tabularx}{\textwidth}{|X|X|}
\hline
State & Set of clauses\\
\hline
Initial state & The clauses of $KB \lor \lnot \phi$\\
\hline
State transition & 
$(\Sigma,\Sigma')$ is such that the target set of clauses $\Sigma'$ is
\begin{itemize}
	\item The result of applying the resolution rule of inference (once) to the source set of clauses $\Sigma$, or
	\item The result of factoring a clause $\Sigma$
\end{itemize}
\\
\hline
Objective function &
f such that $f(\Sigma)$ is
\begin{itemize}
	\item The number of clauses in $\Sigma$ if the empty clause $\varnothing$ is in $\Sigma$
	\item $\infty$ otherwise, with $\infty$ a number bigger than the total number of clauses
	\item Or a better objective function might be the size of the smallest clause in $\Sigma$
\end{itemize}
\\
\hline
\end{tabularx}
}



\end{document}