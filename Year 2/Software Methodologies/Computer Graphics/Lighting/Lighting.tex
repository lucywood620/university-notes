\documentclass{article}[18pt]
\ProvidesPackage{format}
%Page setup
\usepackage[utf8]{inputenc}
\usepackage[margin=0.7in]{geometry}
\usepackage{parselines} 
\usepackage[english]{babel}
\usepackage{fancyhdr}
\usepackage{titlesec}
\hyphenpenalty=10000

\pagestyle{fancy}
\fancyhf{}
\rhead{Sam Robbins}
\rfoot{Page \thepage}

%Characters
\usepackage{amsmath}
\usepackage{amssymb}
\usepackage{gensymb}
\newcommand{\R}{\mathbb{R}}

%Diagrams
\usepackage{pgfplots}
\usepackage{graphicx}
\usepackage{tabularx}
\usepackage{relsize}
\pgfplotsset{width=10cm,compat=1.9}
\usepackage{float}

%Length Setting
\titlespacing\section{0pt}{14pt plus 4pt minus 2pt}{0pt plus 2pt minus 2pt}
\newlength\tindent
\setlength{\tindent}{\parindent}
\setlength{\parindent}{0pt}
\renewcommand{\indent}{\hspace*{\tindent}}

%Programming Font
\usepackage{courier}
\usepackage{listings}
\usepackage{pxfonts}

%Lists
\usepackage{enumerate}
\usepackage{enumitem}

% Networks Macro
\usepackage{tikz}


% Commands for files converted using pandoc
\providecommand{\tightlist}{%
	\setlength{\itemsep}{0pt}\setlength{\parskip}{0pt}}
\usepackage{hyperref}

% Get nice commands for floor and ceil
\usepackage{mathtools}
\DeclarePairedDelimiter{\ceil}{\lceil}{\rceil}
\DeclarePairedDelimiter{\floor}{\lfloor}{\rfloor}

% Allow itemize to go up to 20 levels deep (just change the number if you need more you madman)
\usepackage{enumitem}
\setlistdepth{20}
\renewlist{itemize}{itemize}{20}

% initially, use dots for all levels
\setlist[itemize]{label=$\cdot$}

% customize the first 3 levels
\setlist[itemize,1]{label=\textbullet}
\setlist[itemize,2]{label=--}
\setlist[itemize,3]{label=*}

% Definition and Important Stuff
% Important stuff
\usepackage[framemethod=TikZ]{mdframed}

\newcounter{theo}[section]\setcounter{theo}{0}
\renewcommand{\thetheo}{\arabic{section}.\arabic{theo}}
\newenvironment{important}[1][]{%
	\refstepcounter{theo}%
	\ifstrempty{#1}%
	{\mdfsetup{%
			frametitle={%
				\tikz[baseline=(current bounding box.east),outer sep=0pt]
				\node[anchor=east,rectangle,fill=red!50]
				{\strut Important};}}
	}%
	{\mdfsetup{%
			frametitle={%
				\tikz[baseline=(current bounding box.east),outer sep=0pt]
				\node[anchor=east,rectangle,fill=red!50]
				{\strut Important:~#1};}}%
	}%
	\mdfsetup{innertopmargin=10pt,linecolor=red!50,%
		linewidth=2pt,topline=true,%
		frametitleaboveskip=\dimexpr-\ht\strutbox\relax
	}
	\begin{mdframed}[]\relax%
		\centering
		}{\end{mdframed}}



\newcounter{lem}[section]\setcounter{lem}{0}
\renewcommand{\thelem}{\arabic{section}.\arabic{lem}}
\newenvironment{defin}[1][]{%
	\refstepcounter{lem}%
	\ifstrempty{#1}%
	{\mdfsetup{%
			frametitle={%
				\tikz[baseline=(current bounding box.east),outer sep=0pt]
				\node[anchor=east,rectangle,fill=blue!20]
				{\strut Definition};}}
	}%
	{\mdfsetup{%
			frametitle={%
				\tikz[baseline=(current bounding box.east),outer sep=0pt]
				\node[anchor=east,rectangle,fill=blue!20]
				{\strut Definition:~#1};}}%
	}%
	\mdfsetup{innertopmargin=10pt,linecolor=blue!20,%
		linewidth=2pt,topline=true,%
		frametitleaboveskip=\dimexpr-\ht\strutbox\relax
	}
	\begin{mdframed}[]\relax%
		\centering
		}{\end{mdframed}}
\lhead{Software Methodologies - Computer Graphics}


\begin{document}
\begin{center}
\underline{\huge Lighting in Computer Graphics}
\end{center}
\section{Shading}
\begin{itemize}
	\item Generally, the process for re-creating the phenomenon where colour differ from surface to surface due to lighting
	\item In CG, shading is the process of altering the colour of an object/surface/polygon, based on
	\begin{itemize}
		\item The type of light source that is emitting light
		\item How the light is reflected from object surfaces and enters the eye
	\end{itemize}
to create a photo realistic effect
\end{itemize}
\section{Normal Vector}
\begin{itemize}
	\item The orientation of a surface is specified by the direction perpendicular to the surface and is called a normal (normal vector)
	\item A surface has a front and back face, each side has its own normal
\end{itemize}
\section{Types of shading}
Flat shading
\begin{itemize}
	\item Assign a single colour to each face (triangle) of an object
\end{itemize}
Gouraud (smooth shading)
\begin{itemize}
	\item Apply lighting against the normal vector at each vertex to calculate the vertex colour (vertex shader)
	\item Colours across a face are generated by interpolating colours obtained at the corner vertices of the face (rasterisation)
\end{itemize}
Phong Shading
\begin{itemize}
	\item Normal vector at each point over an object surface is obtained by interpolating normal vectors of the corner vertices of the surface (rasterisation)
	\item Colouring of each surface point will be calculated by applying lighting against the interpolated normal vector at the point (fragment shader)
\end{itemize}
\section{Types of light source}
\begin{definition}[Directional light]
	Like the sun that emits light naturally (from very far away, generating parallel light rays)
\end{definition}
\begin{definition}[Point light]
	Like a light bulb that emits light artificially in all directions from a point
\end{definition}
\begin{definition}[Ambient light]
	Represents indirect light, that is, light emitted from all light sources and reflected by walls or other
\end{definition}
\begin{center}
	\includegraphics[scale=0.7]{"Types of Light"}
\end{center}
\section{Types of reflected light}
\begin{itemize}
	\item Illuminate objects: How light is reflected by the surface of an object and then enters the eye
	\item Colour of the surface determined by:
	\begin{itemize}
		\item Type of the light (colour and direction)
		\item Type of surface of the object (colour and orientation)
	\end{itemize}
\end{itemize}
Surface colour by diffuse and ambient reflection = surface colour by diffuse reflection + surface colour by ambient reflection
\subsection{Ambient reflection}
\begin{itemize}
	\item Ambient reflection is reflection of light from indirect light sources
	\item Illuminates an object equally from all directions with the same intensity, its brightness is the same at any position
\end{itemize}
\begin{center}
	Surface colour by ambient reflection  = light colour $\times$ base colour of surface
\end{center}
\begin{center}
	\includegraphics[scale=0.7]{"Ambient Reflection"}
\end{center}
\subsection{Diffuse reflection}
\begin{itemize}
	\item Reflection of light from a directional light or a point light
	\item Light is reflected equally in all directions from where it hits (due to rough surface)
	\item $\theta$: Angle between light direction and surface orientation (direction "perpendicular" to the surface)
\end{itemize}
\begin{center}
	Surface colour by diffuse reflection = light colour $\times$ base colour of surface $\times \cos\theta$
\end{center}
\begin{center}
	\includegraphics[scale=0.7]{"Diffuse Refection"}
\end{center}
\subsubsection{Calculating $\cos\theta$}
$\cos\theta$ is derived by calculating the dot product of the light direction and the orientation of a surface

$\cos\theta$=light direction $\cdot$ orientation of a surface
\section{Using a point light object}
\begin{itemize}
	\item In contract to a directional light, the direction of the light from a point light source differs at each position in the 3D scene
	\item So, when calculating shading, you need to calculate the light direction at the specific position on the surface where the light hits
	\item Light direction changes: pass the position of the light source and then calculate the light direction at each vertex position
\end{itemize}

\end{document}
