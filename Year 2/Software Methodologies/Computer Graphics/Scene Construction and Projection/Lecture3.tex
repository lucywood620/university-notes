\documentclass{article}[18pt]
\ProvidesPackage{format}
%Page setup
\usepackage[utf8]{inputenc}
\usepackage[margin=0.7in]{geometry}
\usepackage{parselines} 
\usepackage[english]{babel}
\usepackage{fancyhdr}
\usepackage{titlesec}
\hyphenpenalty=10000

\pagestyle{fancy}
\fancyhf{}
\rhead{Sam Robbins}
\rfoot{Page \thepage}

%Characters
\usepackage{amsmath}
\usepackage{amssymb}
\usepackage{gensymb}
\newcommand{\R}{\mathbb{R}}

%Diagrams
\usepackage{pgfplots}
\usepackage{graphicx}
\usepackage{tabularx}
\usepackage{relsize}
\pgfplotsset{width=10cm,compat=1.9}
\usepackage{float}

%Length Setting
\titlespacing\section{0pt}{14pt plus 4pt minus 2pt}{0pt plus 2pt minus 2pt}
\newlength\tindent
\setlength{\tindent}{\parindent}
\setlength{\parindent}{0pt}
\renewcommand{\indent}{\hspace*{\tindent}}

%Programming Font
\usepackage{courier}
\usepackage{listings}
\usepackage{pxfonts}

%Lists
\usepackage{enumerate}
\usepackage{enumitem}

% Networks Macro
\usepackage{tikz}


% Commands for files converted using pandoc
\providecommand{\tightlist}{%
	\setlength{\itemsep}{0pt}\setlength{\parskip}{0pt}}
\usepackage{hyperref}

% Get nice commands for floor and ceil
\usepackage{mathtools}
\DeclarePairedDelimiter{\ceil}{\lceil}{\rceil}
\DeclarePairedDelimiter{\floor}{\lfloor}{\rfloor}

% Allow itemize to go up to 20 levels deep (just change the number if you need more you madman)
\usepackage{enumitem}
\setlistdepth{20}
\renewlist{itemize}{itemize}{20}

% initially, use dots for all levels
\setlist[itemize]{label=$\cdot$}

% customize the first 3 levels
\setlist[itemize,1]{label=\textbullet}
\setlist[itemize,2]{label=--}
\setlist[itemize,3]{label=*}

% Definition and Important Stuff
% Important stuff
\usepackage[framemethod=TikZ]{mdframed}

\newcounter{theo}[section]\setcounter{theo}{0}
\renewcommand{\thetheo}{\arabic{section}.\arabic{theo}}
\newenvironment{important}[1][]{%
	\refstepcounter{theo}%
	\ifstrempty{#1}%
	{\mdfsetup{%
			frametitle={%
				\tikz[baseline=(current bounding box.east),outer sep=0pt]
				\node[anchor=east,rectangle,fill=red!50]
				{\strut Important};}}
	}%
	{\mdfsetup{%
			frametitle={%
				\tikz[baseline=(current bounding box.east),outer sep=0pt]
				\node[anchor=east,rectangle,fill=red!50]
				{\strut Important:~#1};}}%
	}%
	\mdfsetup{innertopmargin=10pt,linecolor=red!50,%
		linewidth=2pt,topline=true,%
		frametitleaboveskip=\dimexpr-\ht\strutbox\relax
	}
	\begin{mdframed}[]\relax%
		\centering
		}{\end{mdframed}}



\newcounter{lem}[section]\setcounter{lem}{0}
\renewcommand{\thelem}{\arabic{section}.\arabic{lem}}
\newenvironment{defin}[1][]{%
	\refstepcounter{lem}%
	\ifstrempty{#1}%
	{\mdfsetup{%
			frametitle={%
				\tikz[baseline=(current bounding box.east),outer sep=0pt]
				\node[anchor=east,rectangle,fill=blue!20]
				{\strut Definition};}}
	}%
	{\mdfsetup{%
			frametitle={%
				\tikz[baseline=(current bounding box.east),outer sep=0pt]
				\node[anchor=east,rectangle,fill=blue!20]
				{\strut Definition:~#1};}}%
	}%
	\mdfsetup{innertopmargin=10pt,linecolor=blue!20,%
		linewidth=2pt,topline=true,%
		frametitleaboveskip=\dimexpr-\ht\strutbox\relax
	}
	\begin{mdframed}[]\relax%
		\centering
		}{\end{mdframed}}
\lhead{Software Methodologies - Computer Graphics}
\usepackage{minted}


\begin{document}
\begin{center}
\underline{\huge Scene Construction and Projection}
\end{center}
\section{World and Local Coordinate Systems}
\begin{definition}[Local Coordinates]
Each object is constructed on a dedicated coordinate system
\end{definition}
\begin{definition}[World Coordinates]
Apply a single coordinate system to all objects globally.
\end{definition}
The purpose of this is to reduce complication for 3D scene construction
\section{View Transform}
Shift the origin of the world coordinate system to the view origin. The view origin is where our eye (or virtual camera) is located with respect to the world origin\\
\\
The purpose of this is to allow a user (application) to specify how 2D rendered images of a 3D scene will be generated
\section{Projection Transform}
\textbf{Visible Region} - Define which part of a 3D scene will be currently visible\\
\\
\textbf{Object appearance} - Modify or preserve object shape properties
\begin{center}
	\includegraphics[scale=0.7]{"Projection Transform"}
\end{center}
\section{Define a view frustum}
Projection transform is done based on a view frustum and it is defined by six planes (near, far, top, bottom, right and left)

\begin{center}
	\includegraphics[scale=0.3]{"View Frustum"}
\end{center}
\section{Types of View Frustum}
\begin{itemize}
	\item The shape and extent of the frustum determines the type of view projection from the 3D scene space to the 2D screen
	\item If the far and near planes have the same dimensions, then the frustum will determine an orthographic projection. Otherwise, it will be a perspective projection
\end{itemize}

\begin{center}
	\includegraphics[scale=0.5]{"Frustum Shape"}
\end{center}
\section{Viewport Transform}
Map projected view to the available space in the computer screen, i.e. viewport, typically referring to the canvas
\begin{center}
	\includegraphics[scale=0.7]{"Viewport Transform"}
\end{center}
\textbf{NDC} - Normalized Device Coordinates. Its x and y coordinates represent the location of your vertices on a normalised 2D screen space
\section{Model-View-Projection Transformation}
\begin{center}
	\includegraphics[scale=0.7]{"Model-View-Projection Transformation"}
\end{center}
\section{Scene Graph}
Scene Graph
\begin{itemize}
	\item Is a collection of nodes in a graph or tree structure
	\item A tree node may have many children but often only a single parent, with the effect of a parent applied to all its child nodes
	\item An operation performed on a group automatically propagates its effect to all of its members
\end{itemize}









\end{document}