\documentclass{article}[18pt]
\usepackage{../../../../format}
\lhead{Software Methodologies - Image Processing}


\begin{document}
\begin{center}
\underline{\huge Point intensity transforms for contrast enhancement}
\end{center}
\section{Functional point transforms}
Processing each pixel value, p, individually using a mathematical function
$$p'=f(p)$$
as a point transform operator transforms the image from one state to another
\begin{center}
	\includegraphics[scale=0.7]{functional}
\end{center}
Also known as intensity transform functions
\section{Image enhancement}
Goal: make the image look better so we can view and process the visual information with greater clarity\\
\\
Image enhancement is subjective. It depends on
\begin{itemize}
	\item The information required by the visual task
	\item The physical characteristics of the image
	\item The user's prior knowledge/experience
	\item The user's intuition and judgement
\end{itemize}
Evaluation methodology: perceived quality of results\\
\\
Common poor image characteristics: poor lighting and noise
\section{Dynamic range}
The \textbf{range of a sensor} is the set of all possible intensity values of the images it captures\\
\\
A good image should utilise the full (or most of the) sensor's range\\
\\
Dynamic range of a sensor, display or image, is the largest (possible) signal value divided by the smallest (possible) signal value\\
\\
Increasing the dynamic range improves contrast
\section{Logarithmic transform}
The logarithmic transform replaces each pixel value with its logarithm
\[
I_{\text {output}}(i, j)=\log I_{\text {input}}(i, j)
\]
In practice, we control the range using the function:
\[
I_{\text {output}}(i, j)=c \cdot \log \left[1+\left(e^{\sigma}-1\right) I_{\text {input}}(i, j)\right]
\]
with scaling parameters $\sigma$ and c
\section{Logarithmic transform}
$\sigma$ controls the range of values onto which the logarithmic function is applied\\
\\
c normalises the output to the range [0,255]. That is,
$$c=\dfrac{255}{R}$$
Where R is the maximum of
$$\log \left[1+\left(e^{\sigma}-1\right) I_{\text {input}}(i, j)\right]$$
\begin{center}
	\includegraphics[scale=0.7]{log}
\end{center}
The dynamic range of this scene exceeded that of the sensor/camera (dark foreground - bright background)\\
\\
As a result of a (usually automatic) decision on the camera exposure, the dynamic range of the dark parts was compressed
\begin{center}
	\includegraphics[scale=0.7]{log1}
\end{center}
The logarithmic transform in this example:
\begin{itemize}
	\item brightens the foreground (which consists of dark pixels) spreading low pixel values over a wider range
	\item compresses the background pixel range (the bright high values)
\end{itemize}
Applying the logarithmic transform yields poor results: information loss and worse visualisation
\section{Application: X-ray images}
X-ray sensors return values given by an exponential function
\[
I(i, j)=I_{o} \cdot \exp (-f(i, j))
\]
$I_o$ is the X-ray source intensity\\
f = material attenuating properties (object thickness and material density)\\
\\
The logarithmic transform cancels the exponent
\[
\log I_{o u t}=\log \left[I_{o} \cdot(\exp (-f(i, j))]\right.
\]
As a result, the image gives a linear mapping of the material properties
\section{Exponential transform}
The exponential transform is the inverse of the logarithmic transform\\
\\
It replaces each pixel value with its exponent
\[
I_{\text {output}}(i, j)=\exp \left(I_{\text {input}}(i, j)\right)
\]
In practice, we use a variable basis and scaling
\[
I_{\text {output}}(i, j)=\mathrm{c} \cdot\left[(1+\alpha)^{I_{\text {input}}(i, j)}-1\right]
\]
where $1+\alpha$ is the basis and c the scaling factor\\
\\
$I_{input}(i,j)=0$ gives
\[
I_{\text {output}}(i, j)=\mathrm{c} \cdot\left[(1+\alpha)^{I_{\text {input}}(i, j)}\right]=c
\]
We subtract 1 to prevent offset in output\\
\\
Basis $>1$ is required for functions suitable for out purpose (decrease the dynamic range of dark regions - increase the dynamic range of bight regions)
\begin{center}
	\includegraphics[scale=0.7]{exp}
\end{center}
The exponential transform decreases the dynamic range of dark regions whilst increasing the dynamic range in light regions\\
Enhances detail in high value (bright) areas
\section{Power-law ('raise to power') transform}
Raise each pixel value to a fixed power
\[
I_{\text {output}}(i, j)=c \cdot\left(I_{\text {input}}(i, j)\right)^{r}
\]
for $r>1$ it enhances (spreads) high value intensities whilst compressing low value intensities\\
\\
For $r<1$ it enhances (spreads) low value intensities whilst compressing high value intensities\\
\\
The 'power-law' transform has similar effect the logarithmic (when $r<1$) or to the exponential (when $r>1$)
\section{Application: gamma correction}
The power-law transform is used in digital photography to correct the tonality of an image\\
\\
r is traditionally called the gamma value, and the process gamma correction\\
\\
The transform $f(x)=x^\gamma$ with $\gamma<1$ weights the intensities towards higher (brighter) values
\begin{itemize}
	\item an underexposed photo can be corrected using gamma correction with $\gamma<1$
\end{itemize} 
The transform $f(x)=x^y$ with $\gamma>1$ weights the intensities toward lower(darker) values
\begin{itemize}
	\item An overexposed photo can be corrected using gamma correction with $\gamma>1$
\end{itemize}
\begin{center}
	\includegraphics[scale=0.7]{gamma}
\end{center}

\end{document}