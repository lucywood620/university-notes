\documentclass{article}[18pt]
\ProvidesPackage{format}
%Page setup
\usepackage[utf8]{inputenc}
\usepackage[margin=0.7in]{geometry}
\usepackage{parselines} 
\usepackage[english]{babel}
\usepackage{fancyhdr}
\usepackage{titlesec}
\hyphenpenalty=10000

\pagestyle{fancy}
\fancyhf{}
\rhead{Sam Robbins}
\rfoot{Page \thepage}

%Characters
\usepackage{amsmath}
\usepackage{amssymb}
\usepackage{gensymb}
\newcommand{\R}{\mathbb{R}}

%Diagrams
\usepackage{pgfplots}
\usepackage{graphicx}
\usepackage{tabularx}
\usepackage{relsize}
\pgfplotsset{width=10cm,compat=1.9}
\usepackage{float}

%Length Setting
\titlespacing\section{0pt}{14pt plus 4pt minus 2pt}{0pt plus 2pt minus 2pt}
\newlength\tindent
\setlength{\tindent}{\parindent}
\setlength{\parindent}{0pt}
\renewcommand{\indent}{\hspace*{\tindent}}

%Programming Font
\usepackage{courier}
\usepackage{listings}
\usepackage{pxfonts}

%Lists
\usepackage{enumerate}
\usepackage{enumitem}

% Networks Macro
\usepackage{tikz}


% Commands for files converted using pandoc
\providecommand{\tightlist}{%
	\setlength{\itemsep}{0pt}\setlength{\parskip}{0pt}}
\usepackage{hyperref}

% Get nice commands for floor and ceil
\usepackage{mathtools}
\DeclarePairedDelimiter{\ceil}{\lceil}{\rceil}
\DeclarePairedDelimiter{\floor}{\lfloor}{\rfloor}

% Allow itemize to go up to 20 levels deep (just change the number if you need more you madman)
\usepackage{enumitem}
\setlistdepth{20}
\renewlist{itemize}{itemize}{20}

% initially, use dots for all levels
\setlist[itemize]{label=$\cdot$}

% customize the first 3 levels
\setlist[itemize,1]{label=\textbullet}
\setlist[itemize,2]{label=--}
\setlist[itemize,3]{label=*}

% Definition and Important Stuff
% Important stuff
\usepackage[framemethod=TikZ]{mdframed}

\newcounter{theo}[section]\setcounter{theo}{0}
\renewcommand{\thetheo}{\arabic{section}.\arabic{theo}}
\newenvironment{important}[1][]{%
	\refstepcounter{theo}%
	\ifstrempty{#1}%
	{\mdfsetup{%
			frametitle={%
				\tikz[baseline=(current bounding box.east),outer sep=0pt]
				\node[anchor=east,rectangle,fill=red!50]
				{\strut Important};}}
	}%
	{\mdfsetup{%
			frametitle={%
				\tikz[baseline=(current bounding box.east),outer sep=0pt]
				\node[anchor=east,rectangle,fill=red!50]
				{\strut Important:~#1};}}%
	}%
	\mdfsetup{innertopmargin=10pt,linecolor=red!50,%
		linewidth=2pt,topline=true,%
		frametitleaboveskip=\dimexpr-\ht\strutbox\relax
	}
	\begin{mdframed}[]\relax%
		\centering
		}{\end{mdframed}}



\newcounter{lem}[section]\setcounter{lem}{0}
\renewcommand{\thelem}{\arabic{section}.\arabic{lem}}
\newenvironment{defin}[1][]{%
	\refstepcounter{lem}%
	\ifstrempty{#1}%
	{\mdfsetup{%
			frametitle={%
				\tikz[baseline=(current bounding box.east),outer sep=0pt]
				\node[anchor=east,rectangle,fill=blue!20]
				{\strut Definition};}}
	}%
	{\mdfsetup{%
			frametitle={%
				\tikz[baseline=(current bounding box.east),outer sep=0pt]
				\node[anchor=east,rectangle,fill=blue!20]
				{\strut Definition:~#1};}}%
	}%
	\mdfsetup{innertopmargin=10pt,linecolor=blue!20,%
		linewidth=2pt,topline=true,%
		frametitleaboveskip=\dimexpr-\ht\strutbox\relax
	}
	\begin{mdframed}[]\relax%
		\centering
		}{\end{mdframed}}
\lhead{Image Processing}


\begin{document}
\begin{center}
\underline{\huge Spatial Filtering II}
\end{center}
\section{Convolution}
\begin{defin}[Mask]
A matrix of coefficients defining a linear filter for processing an image\\
Also called: filter, kernel, template window
\end{defin}
The size of the mask is that of the local neighbourhood of the filter\\
\\
A linear filter operates on the image linearly, i.e. component-wise multiplication and summation\\
\\
The spatial linear filtering of a matrix A by the mask M is called the convolution of A by M\\
\\
Intuitively, the mechanics of convolution work as follows\\
\\
For each pixel of the input image $I_{input}$
\begin{itemize}
	\item Place the centre of the mast M over the pixel $I_{input}(i,j)$
	\item Do the component-wise multiplication of corresponding elements of the mask and the neighbourhood of $I_{input} (i,j)$
\end{itemize} 
The matrix should be updated only after all responses have been computed\\
\\
If the mask has odd dimensions it has a well-defined centre. Otherwise, we have to arbitrarily designate an element of the mask as its centre\\
\\
For a mask of size $(2N+1)\times (2N+1)$ convolution can be written:
\[
I_{o u t p u t}(i, j)=\sum_{k=1}^{2 N+1} \sum_{l=1}^{N+1} I_{i n p u t}(i-N-1+k, j-N-1+l) m_{k l}
\]
Or more conveniently
\[
I_{\text {output}}(i, j)=\sum_{k=-N}^{N} \sum_{l=-N}^{N} I_{\text {input}}(i+k, j+l) m_{k l}
\]
\subsection{Example}
Find the convolution of A by M
\[
A=\left[\begin{array}{lllllll}{0} & {0} & {0} & {0} & {0} & {0} & {0} \\ {0} & {0} & {0} & {0} & {0} & {0} & {0} \\ {0} & {0} & {1} & {1} & {0} & {0} & {0} \\ {0} & {0} & {1} & {1} & {0} & {0} & {0} \\ {0} & {0} & {0} & {0} & {0} & {0} & {0} \\ {0} & {0} & {0} & {0} & {0} & {0} & {0} \\ {0} & {0} & {0} & {0} & {0} & {0} & {0}\end{array}\right] \quad M=\left[\begin{array}{ccc}{1 / 9} & {1 / 9} & {1 / 9} \\ {1 / 9} & {1 / 9} & {1 / 9} \\ {1 / 9} & {1 / 9} & {1 / 9}\end{array}\right]
\]
\section{Gaussian filters}
The value of the element p' of the mask is given by the Gaussian function g of the distance, $d=|p-p'|$ between the centre of the mask and the element p'\\
\\
We construct the $3\times 3$ Gaussian filter for $\sigma=0.5$ by dividing the matrix
\[
\left[\begin{array}{ccc}{g_{\sigma}(\sqrt{2})} & {g_{\sigma}(1)} & {g_{\sigma}(\sqrt{2})} \\ {g_{\sigma}(1)} & {g_{\sigma}(0)} & {g_{\sigma}(1)} \\ {g_{\sigma}(\sqrt{2})} & {g_{\sigma}(1)} & {g_{\sigma}(\sqrt{2})}\end{array}\right]
\]
by the sum of its elements, getting
\[
M_{3}=\left[\begin{array}{ccc}{0.011} & {0.083} & {0.011} \\ {0.083} & {0.619} & {0.083} \\ {0.011} & {0.083} & {0.011}\end{array}\right]
\]
In image processing we often require the elements of a mask to sum up to 1, in which case convolution does not change the average intensity of the image\\
\\
The gaussian filter gives more weight to the centre of the mask and, gradually, less weight to the values away from the centre\\
\\
The Gaussian function is positive everywhere\\
\\
In principle a Gaussian mask would be infinitely large\\
\\
We can define the size of the mask, essentially trim the very small weights, so that we do not waste time on computations that will not affect the final result\\
\\
The elements of the Gaussian masks are coming from the values of a Gaussian function\\
\\
That allows a description of Gaussian filtering without explicit reference to masks
\[
I_{p}^{\text {output }}=\frac{\sum_{p^{\prime} \in \Omega} g\left(\left|p-p^{\prime}\right|\right) I_{p \prime}}{\sum_{p^{\prime} \in \Omega} g\left(\left|p-p^{\prime}\right|\right)}
\]
The new intensity $I_p^{output}$ of the pixel p is the weighted sum of the intensities of the pixels p' in a neighbourhood $\Omega$\\
\\
The denominator of the fraction normalises the expression, making the sum of the weights 1\\
\\
Gaussian filters are used in practice to remove noise from an image, but often removes the fine features too\\
\\
Performs blurring/smoothing of the image\\
\\
Increasing the standard deviation $\sigma$ makes it more blurred
\section{Solution to the blurring}
\subsection{Non-local means}
\textbf{Main idea}:\\
\\
In a larger neighbourhood, perhaps even the whole image, search for neighbourhoods similar to the one pixel being processed\\
\\
Replace the current pixel with a weighted mean of pixels with similar neighbourhoods\\
\\
Weights computed according to similarity between neighbourhoods\\
\\
The averaging spans pixels that are not local within the image to the target pixel - hence the name\\
\\
Output pixel = weighted sum of other pixels (all, or a subset of them) weighted a similarity measure of the corresponding neighbourhoods.
\[
I_{o u t p u t}(i, j)=\frac{\sum W_{n e i g h b u b u r n o o d}(k, l) I_{i n p u t}(k, l)}{\sum W_{n e i g h b o u r n o o d}(k, l)}
\]
A commonly used similarity measure is obtained by applying an exponential function on the squared distance between neighbourhoods (seen as $N^2$ vectors)\\
\\
Performance:
\begin{itemize}
	\item Good on Gaussian noise removal
	\item Dependent on neighbourhood size $N\times N$
	\item Slow, but several optimisations exist
	\item Easily parallelised
\end{itemize}
\section{Laplacian filtering}
In continuous mathematics, the Laplacian of a function f(x,y) is defined as a sum of partial derivatives
\[
\Delta f=\frac{\partial^{2} f}{\partial x^{2}}+\frac{\partial^{2} f}{\partial y^{2}}
\]
In image processing, thinking of the image as a function $I(x,y)$ from pixel coordinates to intensities, the Laplacian is commonly implemented via either of the discrete convolution filters
\[
\begin{array}{ccccccc}{0} & {1} & {0} & {} & {1} & {1} & {1} \\ {1} & {-4} & {1} & {\text { or }} & {1} & {-8} & {1} \\ {0} & {1} & {0} & {} & {1} & {1} & {1}\end{array}
\]
Notice that both mass have some negative elements and they sum up to 0\\
\\
When we discrete a continuous function, derivatives and the second order derivatives become second order differences (i.e. differences of differences)\\
\\
Consider a 1-dimensional image with three pixels with intensities a,b,c
$$[a  \ b \ c]$$ 
The intensity differences are
$$[b-a,c-b]$$
The difference of the intensity differences is
$$(c-b)=(b-a)=c-2b+a$$
Which is the response at the central pixel b in the convolution of the image
$$[a \ b \ c]$$
By the mask
$$[1 \ -2 \ 1]$$
In 2D images, the differences of the intensity differences in the x and y directions are given my the responses to the masks
\[
\begin{array}{ccccccc}{0} & {0} & {0} & {} & {0} & {1} & {0} \\ {1} & {-2} & {1} & {\text { and }} & {0} & {-2} & {0} \\ {0} & {0} & {0} & {} & {0} & {1} & {0}\end{array}
\]
respectively, and by adding them we get the Laplacian filter
\[
\begin{array}{ccccccc c ccc}
{0} & {0} & {0} & {} & {0} & {1} & {0} & &0&1&0 \\
 {1} & {-2} & {1} & {\text {+}} & {0} & {-2} & {0} & = &1&-4&1 \\
  {0} & {0} & {0} & {} & {0} & {1} & {0} & & 0&1&0
  \end{array}
\]
The response to the Laplacian filter is zero at the areas of the image where the intensity changes smoothly\\
\\
Indeed, when intensities change smoothly, intensity differences are equal, and the second order differences are zero\\
\\
When the variation of the intensities is not smooth, the responses to the Laplacian filter can be non-zero\\
\\
The less smooth the variation of the intensities, the higher in absolute value in the response of the Laplacian filter.\\
\\
Thus, the highest in absolute value responses are at the edges of the image\\
\\
Laplacian filter operation: highlights areas of rapid intensity change i.e., the edges of the image\\
\\
Laplacian filter is sensitive to noise. Both edges and noise were amplified\\
\\
Solution: smooth the image with a Gaussian kernel to remove noise and compute the Laplacian of the smoothed image (known as Laplacian of Gaussian (LoG))
\section{General side note}
Image boundaries - what do we do here?
\begin{itemize}
	\item Extrapolate values from inside the image (one of the best options)
	\item Or, use padding with zeros or any other value (often visually poor results)
	\item Or, do not perform convolution in these regions (the resulting convolved image will be smaller than the original)
\end{itemize}


\end{document}