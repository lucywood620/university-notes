\documentclass{article}[18pt]
\ProvidesPackage{format}
%Page setup
\usepackage[utf8]{inputenc}
\usepackage[margin=0.7in]{geometry}
\usepackage{parselines} 
\usepackage[english]{babel}
\usepackage{fancyhdr}
\usepackage{titlesec}
\hyphenpenalty=10000

\pagestyle{fancy}
\fancyhf{}
\rhead{Sam Robbins}
\rfoot{Page \thepage}

%Characters
\usepackage{amsmath}
\usepackage{amssymb}
\usepackage{gensymb}
\newcommand{\R}{\mathbb{R}}

%Diagrams
\usepackage{pgfplots}
\usepackage{graphicx}
\usepackage{tabularx}
\usepackage{relsize}
\pgfplotsset{width=10cm,compat=1.9}
\usepackage{float}

%Length Setting
\titlespacing\section{0pt}{14pt plus 4pt minus 2pt}{0pt plus 2pt minus 2pt}
\newlength\tindent
\setlength{\tindent}{\parindent}
\setlength{\parindent}{0pt}
\renewcommand{\indent}{\hspace*{\tindent}}

%Programming Font
\usepackage{courier}
\usepackage{listings}
\usepackage{pxfonts}

%Lists
\usepackage{enumerate}
\usepackage{enumitem}

% Networks Macro
\usepackage{tikz}


% Commands for files converted using pandoc
\providecommand{\tightlist}{%
	\setlength{\itemsep}{0pt}\setlength{\parskip}{0pt}}
\usepackage{hyperref}

% Get nice commands for floor and ceil
\usepackage{mathtools}
\DeclarePairedDelimiter{\ceil}{\lceil}{\rceil}
\DeclarePairedDelimiter{\floor}{\lfloor}{\rfloor}

% Allow itemize to go up to 20 levels deep (just change the number if you need more you madman)
\usepackage{enumitem}
\setlistdepth{20}
\renewlist{itemize}{itemize}{20}

% initially, use dots for all levels
\setlist[itemize]{label=$\cdot$}

% customize the first 3 levels
\setlist[itemize,1]{label=\textbullet}
\setlist[itemize,2]{label=--}
\setlist[itemize,3]{label=*}

% Definition and Important Stuff
% Important stuff
\usepackage[framemethod=TikZ]{mdframed}

\newcounter{theo}[section]\setcounter{theo}{0}
\renewcommand{\thetheo}{\arabic{section}.\arabic{theo}}
\newenvironment{important}[1][]{%
	\refstepcounter{theo}%
	\ifstrempty{#1}%
	{\mdfsetup{%
			frametitle={%
				\tikz[baseline=(current bounding box.east),outer sep=0pt]
				\node[anchor=east,rectangle,fill=red!50]
				{\strut Important};}}
	}%
	{\mdfsetup{%
			frametitle={%
				\tikz[baseline=(current bounding box.east),outer sep=0pt]
				\node[anchor=east,rectangle,fill=red!50]
				{\strut Important:~#1};}}%
	}%
	\mdfsetup{innertopmargin=10pt,linecolor=red!50,%
		linewidth=2pt,topline=true,%
		frametitleaboveskip=\dimexpr-\ht\strutbox\relax
	}
	\begin{mdframed}[]\relax%
		\centering
		}{\end{mdframed}}



\newcounter{lem}[section]\setcounter{lem}{0}
\renewcommand{\thelem}{\arabic{section}.\arabic{lem}}
\newenvironment{defin}[1][]{%
	\refstepcounter{lem}%
	\ifstrempty{#1}%
	{\mdfsetup{%
			frametitle={%
				\tikz[baseline=(current bounding box.east),outer sep=0pt]
				\node[anchor=east,rectangle,fill=blue!20]
				{\strut Definition};}}
	}%
	{\mdfsetup{%
			frametitle={%
				\tikz[baseline=(current bounding box.east),outer sep=0pt]
				\node[anchor=east,rectangle,fill=blue!20]
				{\strut Definition:~#1};}}%
	}%
	\mdfsetup{innertopmargin=10pt,linecolor=blue!20,%
		linewidth=2pt,topline=true,%
		frametitleaboveskip=\dimexpr-\ht\strutbox\relax
	}
	\begin{mdframed}[]\relax%
		\centering
		}{\end{mdframed}}
\lhead{Software Methodologies - Image Processing}


\begin{document}
\begin{center}
\underline{\huge Histograms and Histogram based image enhancement}
\end{center}
\section{What is a histogram?}
\begin{defin}[Histogram function]
A function defined over all possible intensity levels. For each intensity level, its value is equal to the number of pixels with that intensity
\end{defin}
\section{Constructing a histogram}
Simply count occurrences of each intensity value
\begin{lstlisting}[caption=Constructing a histogram]
initialise all histogram array entries to 0
for each pixel I(i,j) within the image I
	histogram(I(i,j)) = hostogram(I(i,j)) +1
end
\end{lstlisting}
\section{Contrast stretch/normalisation}
\textbf{Operation}: Stretches the pixel range over a larger dynamic range\\
\\
\textbf{Approach}: use four intensity values
\begin{enumerate}[label=\alph*:]
	\item upper pixel quantisation limit
	\item lower pixel quantisation limit
	\item maximum pixel value present
	\item minimum pixel value present
\end{enumerate}
$$I_{output}(i,j)=(I_{input}(i,j)-c)\Bigg(\dfrac{a-b}{c-d}\Bigg)+a$$
Potential problem - outliers in the image
\begin{itemize}
	\item Possible that $c\sim a$ and $d\sim b$ (or even $c=a$ and $d=b$)
	\item Result: contrast stretch has no effect on the image
\end{itemize}
\subsection{Solutions}
Use a robust against outliers method to select c and d, instead of the min and max values in the image\\
\textbf{Method 1}
\begin{itemize}
	\item Select c and d at fixed percentile points of the histogram distribution
	\item If any of new intensity values are below b or above a, map them to b and a respectively
\end{itemize}
\textbf{Method 2}
\begin{itemize}
	\item Find the most frequent image value (histogram peak)
	\item Select a cut-off as a percentage of the peak
	\item Scan down from peak in either direction until last values above cut-off are reached and select these as c and d
	\item If any of new intensity values below b or above a, map them to b and a respectively
\end{itemize}
Method 2 is marginally weaker for complex, multi-peak histograms
\section{Histogram modelling}
\begin{defin}[Histogram modelling]
Modify an image so that its histogram conforms to a given shape
\end{defin}
\begin{defin}[Histogram equalisation]
Histogram modelling via an intensity transformation function aiming at producing an output image with uniform histogram distribution
\end{defin}
\section{Cumulative histogram function}
Let the dynamic range of a grayscale image be
$$i=1,2,...,L$$
For a histogram function $h(i)$ we construct the cumulative histogram function $C(i)$
$$C(i)=\sum_{j=1}^{i}h(j)$$
That is, the values of $C(i)$ record the sum of the occurrence of each grey level up to and including i\\
\\
$C(i)$ is a monotonically increasing function
\section{Histogram equalisation}
In an ideally equalised image, all intensity values would appear the same number of times, i.e. N/L each, where N is the number of pixels in the image\\
\\
The cumulative function would then be
$$C_{ideallyEqualised}(i)=(N/L)i$$
Histogram equalisation corresponds to the intensity transform
$$t(i)=(L/N)C_{input}(i)$$
Which computes the cumulative histogram at intensity i, and maps i to the intensity of the ideally equalised image for that value of the cumulative histogram\\
\\
A schematic example of how histogram equalisation works:
\begin{itemize}
	\item Let L = 100 (dynamic range of 100 levels)
	\item Let $C(50) = 0.8\cdot N$ (80\% of the N pixels have value 50 or lower)
	\item $t(50)=(L/N)\cdot 0.8\cdot N = 100 \cdot 0.8 = 80$
	\item 50 is mapped 80 and this, 80\% of the pixel of the equalised image have value 80 or lower
\end{itemize}
\section{Implementation}
Technical issues requiring attention in practice:
\begin{itemize}
	\item a dynamic range with L levels usually consists of the values $i=0,1,..,L-1$
	\item L/N might not be an integer, thus some of the values of t() might not be integers
\end{itemize}
As a solution we can instead use the formula
$$t(i)=\lfloor ((L-1)/N)\cdot C_{input}(i) \rfloor$$
\section{Limitations}
As a fully automated technique (no parameters) the effect of histogram equalisation is highly input dependant\\
\\
In some image the global contrast can be over-exposed or under-exposed
\subsection{Solution}
Use the histogram from (a well balanced) sub-part of original image as the input histogram of the equalisation algorithm
\section{Localised histogram equalisation}
Split the image into a set of discrete, non-overlapping neighbourhoods of size $N\times N$\\
\\
Histogram equalisation of each neighbourhood in isolation (tiling)
\section{Adaptive histogram equalisation}
Perform histogram equalisation at each pixel (rather than neighbourhood) using overlapping local $N\times N$ neighbourhoods\\
\\
Adaptive histogram equalisation is slower than localised histogram equalisation\\
\\
Tiling artefacts can be avoided but choice of neighbourhood size N crucial
\section{Localised/adaptive equalisation}
In terms of the aesthetics of the output, performance is poor because a global transform (of the entire dynamic range) is computed and applied locally
\end{document}