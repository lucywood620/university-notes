\documentclass{article}[18pt]
\ProvidesPackage{format}
%Page setup
\usepackage[utf8]{inputenc}
\usepackage[margin=0.7in]{geometry}
\usepackage{parselines} 
\usepackage[english]{babel}
\usepackage{fancyhdr}
\usepackage{titlesec}
\hyphenpenalty=10000

\pagestyle{fancy}
\fancyhf{}
\rhead{Sam Robbins}
\rfoot{Page \thepage}

%Characters
\usepackage{amsmath}
\usepackage{amssymb}
\usepackage{gensymb}
\newcommand{\R}{\mathbb{R}}

%Diagrams
\usepackage{pgfplots}
\usepackage{graphicx}
\usepackage{tabularx}
\usepackage{relsize}
\pgfplotsset{width=10cm,compat=1.9}
\usepackage{float}

%Length Setting
\titlespacing\section{0pt}{14pt plus 4pt minus 2pt}{0pt plus 2pt minus 2pt}
\newlength\tindent
\setlength{\tindent}{\parindent}
\setlength{\parindent}{0pt}
\renewcommand{\indent}{\hspace*{\tindent}}

%Programming Font
\usepackage{courier}
\usepackage{listings}
\usepackage{pxfonts}

%Lists
\usepackage{enumerate}
\usepackage{enumitem}

% Networks Macro
\usepackage{tikz}


% Commands for files converted using pandoc
\providecommand{\tightlist}{%
	\setlength{\itemsep}{0pt}\setlength{\parskip}{0pt}}
\usepackage{hyperref}

% Get nice commands for floor and ceil
\usepackage{mathtools}
\DeclarePairedDelimiter{\ceil}{\lceil}{\rceil}
\DeclarePairedDelimiter{\floor}{\lfloor}{\rfloor}

% Allow itemize to go up to 20 levels deep (just change the number if you need more you madman)
\usepackage{enumitem}
\setlistdepth{20}
\renewlist{itemize}{itemize}{20}

% initially, use dots for all levels
\setlist[itemize]{label=$\cdot$}

% customize the first 3 levels
\setlist[itemize,1]{label=\textbullet}
\setlist[itemize,2]{label=--}
\setlist[itemize,3]{label=*}

% Definition and Important Stuff
% Important stuff
\usepackage[framemethod=TikZ]{mdframed}

\newcounter{theo}[section]\setcounter{theo}{0}
\renewcommand{\thetheo}{\arabic{section}.\arabic{theo}}
\newenvironment{important}[1][]{%
	\refstepcounter{theo}%
	\ifstrempty{#1}%
	{\mdfsetup{%
			frametitle={%
				\tikz[baseline=(current bounding box.east),outer sep=0pt]
				\node[anchor=east,rectangle,fill=red!50]
				{\strut Important};}}
	}%
	{\mdfsetup{%
			frametitle={%
				\tikz[baseline=(current bounding box.east),outer sep=0pt]
				\node[anchor=east,rectangle,fill=red!50]
				{\strut Important:~#1};}}%
	}%
	\mdfsetup{innertopmargin=10pt,linecolor=red!50,%
		linewidth=2pt,topline=true,%
		frametitleaboveskip=\dimexpr-\ht\strutbox\relax
	}
	\begin{mdframed}[]\relax%
		\centering
		}{\end{mdframed}}



\newcounter{lem}[section]\setcounter{lem}{0}
\renewcommand{\thelem}{\arabic{section}.\arabic{lem}}
\newenvironment{defin}[1][]{%
	\refstepcounter{lem}%
	\ifstrempty{#1}%
	{\mdfsetup{%
			frametitle={%
				\tikz[baseline=(current bounding box.east),outer sep=0pt]
				\node[anchor=east,rectangle,fill=blue!20]
				{\strut Definition};}}
	}%
	{\mdfsetup{%
			frametitle={%
				\tikz[baseline=(current bounding box.east),outer sep=0pt]
				\node[anchor=east,rectangle,fill=blue!20]
				{\strut Definition:~#1};}}%
	}%
	\mdfsetup{innertopmargin=10pt,linecolor=blue!20,%
		linewidth=2pt,topline=true,%
		frametitleaboveskip=\dimexpr-\ht\strutbox\relax
	}
	\begin{mdframed}[]\relax%
		\centering
		}{\end{mdframed}}
\lhead{Software Methodologies - Image Processing}


\begin{document}
\begin{center}
\underline{\huge Colour spaces and adding colour to grayscale images}
\end{center}
\section{RGB Colour Space}
Colours represented as R,G,B intensities.\\
\\
The space is modelled geometrically by a cube with axes R,G and B\\
\\
RGB is used as it is based on the human perception of the visible spectrum - the human eye has R, G and B receptors with different sensitivity at different wavelengths
{\renewcommand{\arraystretch}{2}
\section{HSV colour space}
\begin{tabularx}{\textwidth}{|X|X|X|}
\hline
H(ue)& Specifies the dominant wavelength & $0 \rightarrow 360^\circ$\\
\hline
S(aturation)& High values correspond to vibrant pure colours& $0\rightarrow 1$\\
\hline
V(alue)&Brightness of colour& $0\rightarrow 1$\\
\hline
\end{tabularx}}
\subsection{The geometry of the HSV space}
The hue is mapped cyclically, so red corresponds to both 0 and 360, so the HSV space can be modelled by a cylinder\\
\\
When we map the RGB cube onto the HSV cylinder we get a subset of this cylinder in the shape of a pyramid with hexagonal base, which in turn gives rise to the cone model
\subsection{Thresholding}
Thresholding produces a binary image from a grayscale image\\
\\
Easier to colour threshold in HSV space than RGB
\begin{defin}[Colour slicing]
	A portion of the Hue colour channel (range of wavalengths) can be isolated. For better results, also use S and V in conjunctoon
\end{defin}
Isolate objects by colour:
\begin{itemize}
	\item Use an upper and lower threshold on the Hue channel to isolate by colour
	\item Use a threshold on the Saturation channel to isolate pure colours
	\item Combine the two outputs with logical AND
\end{itemize}
\textbf{Specular highlights}
\begin{itemize}
	\item RGB: very different position in space to original colour
	\item HSV: similar position in Hue as specular present in Value
\end{itemize}
\textbf{Shadows}:
\begin{itemize}
	\item HSV: isolated in the Saturation/Value components
\end{itemize}
\section{CMY colour space}


\end{document}