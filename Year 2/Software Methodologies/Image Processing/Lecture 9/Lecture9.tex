\documentclass{article}[18pt]
\usepackage{../../../../format}
\lhead{Software Methodologies - Image Processing}


\begin{document}
\begin{center}
\underline{\huge Colour spaces and adding colour to grayscale images}
\end{center}
\section{RGB Colour Space}
Colours represented as R,G,B intensities.\\
\\
The space is modelled geometrically by a cube with axes R,G and B\\
\\
RGB is used as it is based on the human perception of the visible spectrum - the human eye has R, G and B receptors with different sensitivity at different wavelengths
{\renewcommand{\arraystretch}{2}
\section{HSV colour space}
\begin{tabularx}{\textwidth}{|X|X|X|}
\hline
H(ue)& Specifies the dominant wavelength & $0 \rightarrow 360^\circ$\\
\hline
S(aturation)& High values correspond to vibrant pure colours& $0\rightarrow 1$\\
\hline
V(alue)&Brightness of colour& $0\rightarrow 1$\\
\hline
\end{tabularx}}
\subsection{The geometry of the HSV space}
The hue is mapped cyclically, so red corresponds to both 0 and 360, so the HSV space can be modelled by a cylinder\\
\\
When we map the RGB cube onto the HSV cylinder we get a subset of this cylinder in the shape of a pyramid with hexagonal base, which in turn gives rise to the cone model
\subsection{Thresholding}
Thresholding produces a binary image from a grayscale image\\
\\
Easier to colour threshold in HSV space than RGB
\begin{defin}[Colour slicing]
	A portion of the Hue colour channel (range of wavalengths) can be isolated. For better results, also use S and V in conjunctoon
\end{defin}
Isolate objects by colour:
\begin{itemize}
	\item Use an upper and lower threshold on the Hue channel to isolate by colour
	\item Use a threshold on the Saturation channel to isolate pure colours
	\item Combine the two outputs with logical AND
\end{itemize}
\textbf{Specular highlights}
\begin{itemize}
	\item RGB: very different position in space to original colour
	\item HSV: similar position in Hue as specular present in Value
\end{itemize}
\textbf{Shadows}:
\begin{itemize}
	\item HSV: isolated in the Saturation/Value components
\end{itemize}
\section{CMY colour space}


\end{document}