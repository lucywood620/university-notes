\documentclass{article}[18pt]
\ProvidesPackage{format}
%Page setup
\usepackage[utf8]{inputenc}
\usepackage[margin=0.7in]{geometry}
\usepackage{parselines} 
\usepackage[english]{babel}
\usepackage{fancyhdr}
\usepackage{titlesec}
\hyphenpenalty=10000

\pagestyle{fancy}
\fancyhf{}
\rhead{Sam Robbins}
\rfoot{Page \thepage}

%Characters
\usepackage{amsmath}
\usepackage{amssymb}
\usepackage{gensymb}
\newcommand{\R}{\mathbb{R}}

%Diagrams
\usepackage{pgfplots}
\usepackage{graphicx}
\usepackage{tabularx}
\usepackage{relsize}
\pgfplotsset{width=10cm,compat=1.9}
\usepackage{float}

%Length Setting
\titlespacing\section{0pt}{14pt plus 4pt minus 2pt}{0pt plus 2pt minus 2pt}
\newlength\tindent
\setlength{\tindent}{\parindent}
\setlength{\parindent}{0pt}
\renewcommand{\indent}{\hspace*{\tindent}}

%Programming Font
\usepackage{courier}
\usepackage{listings}
\usepackage{pxfonts}

%Lists
\usepackage{enumerate}
\usepackage{enumitem}

% Networks Macro
\usepackage{tikz}


% Commands for files converted using pandoc
\providecommand{\tightlist}{%
	\setlength{\itemsep}{0pt}\setlength{\parskip}{0pt}}
\usepackage{hyperref}

% Get nice commands for floor and ceil
\usepackage{mathtools}
\DeclarePairedDelimiter{\ceil}{\lceil}{\rceil}
\DeclarePairedDelimiter{\floor}{\lfloor}{\rfloor}

% Allow itemize to go up to 20 levels deep (just change the number if you need more you madman)
\usepackage{enumitem}
\setlistdepth{20}
\renewlist{itemize}{itemize}{20}

% initially, use dots for all levels
\setlist[itemize]{label=$\cdot$}

% customize the first 3 levels
\setlist[itemize,1]{label=\textbullet}
\setlist[itemize,2]{label=--}
\setlist[itemize,3]{label=*}

% Definition and Important Stuff
% Important stuff
\usepackage[framemethod=TikZ]{mdframed}

\newcounter{theo}[section]\setcounter{theo}{0}
\renewcommand{\thetheo}{\arabic{section}.\arabic{theo}}
\newenvironment{important}[1][]{%
	\refstepcounter{theo}%
	\ifstrempty{#1}%
	{\mdfsetup{%
			frametitle={%
				\tikz[baseline=(current bounding box.east),outer sep=0pt]
				\node[anchor=east,rectangle,fill=red!50]
				{\strut Important};}}
	}%
	{\mdfsetup{%
			frametitle={%
				\tikz[baseline=(current bounding box.east),outer sep=0pt]
				\node[anchor=east,rectangle,fill=red!50]
				{\strut Important:~#1};}}%
	}%
	\mdfsetup{innertopmargin=10pt,linecolor=red!50,%
		linewidth=2pt,topline=true,%
		frametitleaboveskip=\dimexpr-\ht\strutbox\relax
	}
	\begin{mdframed}[]\relax%
		\centering
		}{\end{mdframed}}



\newcounter{lem}[section]\setcounter{lem}{0}
\renewcommand{\thelem}{\arabic{section}.\arabic{lem}}
\newenvironment{defin}[1][]{%
	\refstepcounter{lem}%
	\ifstrempty{#1}%
	{\mdfsetup{%
			frametitle={%
				\tikz[baseline=(current bounding box.east),outer sep=0pt]
				\node[anchor=east,rectangle,fill=blue!20]
				{\strut Definition};}}
	}%
	{\mdfsetup{%
			frametitle={%
				\tikz[baseline=(current bounding box.east),outer sep=0pt]
				\node[anchor=east,rectangle,fill=blue!20]
				{\strut Definition:~#1};}}%
	}%
	\mdfsetup{innertopmargin=10pt,linecolor=blue!20,%
		linewidth=2pt,topline=true,%
		frametitleaboveskip=\dimexpr-\ht\strutbox\relax
	}
	\begin{mdframed}[]\relax%
		\centering
		}{\end{mdframed}}
\lhead{Software Methodologies - Machine Learning}


\begin{document}
\begin{center}
\underline{\huge Linear Regression, Training and Loss}
\end{center}
\section{Linear regression}
\begin{definition}[Linear regression]
A method for finding the straight line or hyperplane that best fits a set of points
\end{definition}
$$y=b+w_1x_1$$
y - the predicted label\\
b - the bias, sometimes referred to as $w_0$\\
$w_1$ - the weight of feature 1\\
$x_1$ - a feature
\section{Training and loss}
\begin{definition}[Training a model]
Learning good values for all weights and the bias from labelled examples
\end{definition}
\begin{definition}[Loss]
The penalty for a bad prediction
\end{definition}
\begin{definition}[Empirical Risk Minimisation]
The process of examining many examples and attempting to find a model that minimises loss
\end{definition}
\subsection{Squared loss}
The square of the difference between the label and the prediction
$$(\text{observation}-\text{prediction}(x))^2$$
$$(y-\hat{y})^2$$
\subsection{Mean square error}
$$MSE=\dfrac{1}{N}\sum_{(x,y)\in D}(y-\text{predicition}(x))^2$$
(x,y) is an example where
\begin{itemize}
	\item x is the set of features used by the model to make predictions
	\item y is the example's label
\end{itemize}
prediction(x) is a function of the weights and bias in combination with the set of features x\\
D is the dataset containing many labelled examples\\
N is the number of examples in D
\section{Reducing loss}
\begin{itemize}
	\item Hyperparameters are the configuration settings used to tune how the model is trained
	\item Derivative of loss with respect to weights and biases tells us how loss changes for a given example
	\item So we repeatedly take small steps in the direction that minimises loss, we call these \textbf{Gradient steps}
\end{itemize}
\begin{center}
	\includegraphics[scale=0.5]{"Gradient Descent"}
\end{center}
\subsection{Weight initialisation}
For convex problem, weights can start anywhere forming a graph that looks like $x^2$\\
\\
Foreshadowing: not true for neural networks
\begin{itemize}
	\item More than one minimum
	\item Strong dependency on initial values
\end{itemize}
\subsection{Efficiency of reducing loss}
\begin{itemize}
	\item Could compute gradient over entire dataset on each step, but this turns out to be unnecessary
	\item Computing gradient on small data examples works well
	\item \textbf{Stochastic Gradient Descent} - one example at a time
	\item \textbf{Mini-batch Gradient Descent} - batches of 10-1000
\end{itemize}
\subsection{Learning rate}
The ideal learning rate in one-dimension is
$$\dfrac{1}{f(x)''}$$
The ideal learning rate for 2 or more dimension is the inverse of the Hessian (matrix of second partial derivatives)
\end{document}