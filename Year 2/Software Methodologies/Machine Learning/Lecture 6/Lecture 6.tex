\documentclass{article}[18pt]
\ProvidesPackage{format}
%Page setup
\usepackage[utf8]{inputenc}
\usepackage[margin=0.7in]{geometry}
\usepackage{parselines} 
\usepackage[english]{babel}
\usepackage{fancyhdr}
\usepackage{titlesec}
\hyphenpenalty=10000

\pagestyle{fancy}
\fancyhf{}
\rhead{Sam Robbins}
\rfoot{Page \thepage}

%Characters
\usepackage{amsmath}
\usepackage{amssymb}
\usepackage{gensymb}
\newcommand{\R}{\mathbb{R}}

%Diagrams
\usepackage{pgfplots}
\usepackage{graphicx}
\usepackage{tabularx}
\usepackage{relsize}
\pgfplotsset{width=10cm,compat=1.9}
\usepackage{float}

%Length Setting
\titlespacing\section{0pt}{14pt plus 4pt minus 2pt}{0pt plus 2pt minus 2pt}
\newlength\tindent
\setlength{\tindent}{\parindent}
\setlength{\parindent}{0pt}
\renewcommand{\indent}{\hspace*{\tindent}}

%Programming Font
\usepackage{courier}
\usepackage{listings}
\usepackage{pxfonts}

%Lists
\usepackage{enumerate}
\usepackage{enumitem}

% Networks Macro
\usepackage{tikz}


% Commands for files converted using pandoc
\providecommand{\tightlist}{%
	\setlength{\itemsep}{0pt}\setlength{\parskip}{0pt}}
\usepackage{hyperref}

% Get nice commands for floor and ceil
\usepackage{mathtools}
\DeclarePairedDelimiter{\ceil}{\lceil}{\rceil}
\DeclarePairedDelimiter{\floor}{\lfloor}{\rfloor}

% Allow itemize to go up to 20 levels deep (just change the number if you need more you madman)
\usepackage{enumitem}
\setlistdepth{20}
\renewlist{itemize}{itemize}{20}

% initially, use dots for all levels
\setlist[itemize]{label=$\cdot$}

% customize the first 3 levels
\setlist[itemize,1]{label=\textbullet}
\setlist[itemize,2]{label=--}
\setlist[itemize,3]{label=*}

% Definition and Important Stuff
% Important stuff
\usepackage[framemethod=TikZ]{mdframed}

\newcounter{theo}[section]\setcounter{theo}{0}
\renewcommand{\thetheo}{\arabic{section}.\arabic{theo}}
\newenvironment{important}[1][]{%
	\refstepcounter{theo}%
	\ifstrempty{#1}%
	{\mdfsetup{%
			frametitle={%
				\tikz[baseline=(current bounding box.east),outer sep=0pt]
				\node[anchor=east,rectangle,fill=red!50]
				{\strut Important};}}
	}%
	{\mdfsetup{%
			frametitle={%
				\tikz[baseline=(current bounding box.east),outer sep=0pt]
				\node[anchor=east,rectangle,fill=red!50]
				{\strut Important:~#1};}}%
	}%
	\mdfsetup{innertopmargin=10pt,linecolor=red!50,%
		linewidth=2pt,topline=true,%
		frametitleaboveskip=\dimexpr-\ht\strutbox\relax
	}
	\begin{mdframed}[]\relax%
		\centering
		}{\end{mdframed}}



\newcounter{lem}[section]\setcounter{lem}{0}
\renewcommand{\thelem}{\arabic{section}.\arabic{lem}}
\newenvironment{defin}[1][]{%
	\refstepcounter{lem}%
	\ifstrempty{#1}%
	{\mdfsetup{%
			frametitle={%
				\tikz[baseline=(current bounding box.east),outer sep=0pt]
				\node[anchor=east,rectangle,fill=blue!20]
				{\strut Definition};}}
	}%
	{\mdfsetup{%
			frametitle={%
				\tikz[baseline=(current bounding box.east),outer sep=0pt]
				\node[anchor=east,rectangle,fill=blue!20]
				{\strut Definition:~#1};}}%
	}%
	\mdfsetup{innertopmargin=10pt,linecolor=blue!20,%
		linewidth=2pt,topline=true,%
		frametitleaboveskip=\dimexpr-\ht\strutbox\relax
	}
	\begin{mdframed}[]\relax%
		\centering
		}{\end{mdframed}}
\lhead{Software Methodologies - Machine Learning}


\begin{document}
\begin{center}
\underline{\huge Decision Trees and Random Forests}
\end{center}
\section{Decision Trees}
There might be multiple decision trees for deciding the same thing from different conditions. To decide which is best, we use Gini Impurity
$$\text{Gini Impurity}=1-(\text{the probability of Yes})^2-(\text{the Probability of No})^2$$
The lower the value the better\\
\\
From a raw table of data to a decision tree:
\begin{enumerate}
	\item Calculate all of the Gini Impurity values
	\item If a node itself has the lowest value, leave it as a Leaf node
	\item If separating the data results in an improvement, then pick the separation with the lowest Gini impurity value
\end{enumerate}
\subsection{Numeric Data}
To get impurities
\begin{enumerate}
	\item Sort the values lowest to highest
	\item Calculate the average for adjacent values
	\item Calculate the impurity values for each average weight
\end{enumerate}
To Build a tree:
\begin{enumerate}
	\item Yes/no questions at each step
	\item Numeric data, like patient weight
\end{enumerate}
\subsection{Ranked Data and Multiple Choice Data}
Ranked Data
\begin{center}
	\includegraphics[scale=0.7]{"Ranked Data"}
\end{center}
Multiple Choices Data
\begin{center}
	\includegraphics[scale=0.7]{"Multiple Choices Data"}
\end{center}
\section{Random Forests}
Why Random Forests:
\begin{itemize}
	\item Decision Trees are easy to build, use and interpret, but not flexible when classifying new samples
	\item Random forests combine the simplicity of decision trees with flexibility for better accuracy 
\end{itemize}
\subsection{How to build a random forest}
\textbf{Step 1} - Create a "bootstrapped" dataset:
\begin{itemize}
	\item Same size as the original dataset
	\item Randomly selected samples from the original dataset
	\item Samples can be selected more than once
\end{itemize}
\textbf{Step 2} - Build a decision tree using "bootstrapped" dataset, but only use a random subset of variables, e.g. 2\\
\\
\textbf{Step 3} - Go back to step 1 and repeat: make a new bootstrap  dataset and build a tree considering a subset of variables at each step (ideally 100's of times)
\begin{itemize}
	\item Using a bootstrapped sample and considering only a subset of the variables at each steps results in a wide variety of trees
	\item The variety makes random forests more effective than individual Decision Trees
\end{itemize}
\subsection{How to use a random forest}
\begin{itemize}
	\item Take the data and run it down the first tree we built
	\item Keep track of the result
\end{itemize}
\begin{definition}[Bagging]
	Bootstrapping the data plus using the aggregate to make a decision
\end{definition}


\end{document}
