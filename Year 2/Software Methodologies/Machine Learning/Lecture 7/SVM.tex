\documentclass{article}[18pt]
\ProvidesPackage{format}
%Page setup
\usepackage[utf8]{inputenc}
\usepackage[margin=0.7in]{geometry}
\usepackage{parselines} 
\usepackage[english]{babel}
\usepackage{fancyhdr}
\usepackage{titlesec}
\hyphenpenalty=10000

\pagestyle{fancy}
\fancyhf{}
\rhead{Sam Robbins}
\rfoot{Page \thepage}

%Characters
\usepackage{amsmath}
\usepackage{amssymb}
\usepackage{gensymb}
\newcommand{\R}{\mathbb{R}}

%Diagrams
\usepackage{pgfplots}
\usepackage{graphicx}
\usepackage{tabularx}
\usepackage{relsize}
\pgfplotsset{width=10cm,compat=1.9}
\usepackage{float}

%Length Setting
\titlespacing\section{0pt}{14pt plus 4pt minus 2pt}{0pt plus 2pt minus 2pt}
\newlength\tindent
\setlength{\tindent}{\parindent}
\setlength{\parindent}{0pt}
\renewcommand{\indent}{\hspace*{\tindent}}

%Programming Font
\usepackage{courier}
\usepackage{listings}
\usepackage{pxfonts}

%Lists
\usepackage{enumerate}
\usepackage{enumitem}

% Networks Macro
\usepackage{tikz}


% Commands for files converted using pandoc
\providecommand{\tightlist}{%
	\setlength{\itemsep}{0pt}\setlength{\parskip}{0pt}}
\usepackage{hyperref}

% Get nice commands for floor and ceil
\usepackage{mathtools}
\DeclarePairedDelimiter{\ceil}{\lceil}{\rceil}
\DeclarePairedDelimiter{\floor}{\lfloor}{\rfloor}

% Allow itemize to go up to 20 levels deep (just change the number if you need more you madman)
\usepackage{enumitem}
\setlistdepth{20}
\renewlist{itemize}{itemize}{20}

% initially, use dots for all levels
\setlist[itemize]{label=$\cdot$}

% customize the first 3 levels
\setlist[itemize,1]{label=\textbullet}
\setlist[itemize,2]{label=--}
\setlist[itemize,3]{label=*}

% Definition and Important Stuff
% Important stuff
\usepackage[framemethod=TikZ]{mdframed}

\newcounter{theo}[section]\setcounter{theo}{0}
\renewcommand{\thetheo}{\arabic{section}.\arabic{theo}}
\newenvironment{important}[1][]{%
	\refstepcounter{theo}%
	\ifstrempty{#1}%
	{\mdfsetup{%
			frametitle={%
				\tikz[baseline=(current bounding box.east),outer sep=0pt]
				\node[anchor=east,rectangle,fill=red!50]
				{\strut Important};}}
	}%
	{\mdfsetup{%
			frametitle={%
				\tikz[baseline=(current bounding box.east),outer sep=0pt]
				\node[anchor=east,rectangle,fill=red!50]
				{\strut Important:~#1};}}%
	}%
	\mdfsetup{innertopmargin=10pt,linecolor=red!50,%
		linewidth=2pt,topline=true,%
		frametitleaboveskip=\dimexpr-\ht\strutbox\relax
	}
	\begin{mdframed}[]\relax%
		\centering
		}{\end{mdframed}}



\newcounter{lem}[section]\setcounter{lem}{0}
\renewcommand{\thelem}{\arabic{section}.\arabic{lem}}
\newenvironment{defin}[1][]{%
	\refstepcounter{lem}%
	\ifstrempty{#1}%
	{\mdfsetup{%
			frametitle={%
				\tikz[baseline=(current bounding box.east),outer sep=0pt]
				\node[anchor=east,rectangle,fill=blue!20]
				{\strut Definition};}}
	}%
	{\mdfsetup{%
			frametitle={%
				\tikz[baseline=(current bounding box.east),outer sep=0pt]
				\node[anchor=east,rectangle,fill=blue!20]
				{\strut Definition:~#1};}}%
	}%
	\mdfsetup{innertopmargin=10pt,linecolor=blue!20,%
		linewidth=2pt,topline=true,%
		frametitleaboveskip=\dimexpr-\ht\strutbox\relax
	}
	\begin{mdframed}[]\relax%
		\centering
		}{\end{mdframed}}
\lhead{Software Methodologies - Machine Learning}


\begin{document}
\begin{center}
\underline{\huge Support Vector Machine}
\end{center}
\section{Linear Separable SVM}
\begin{definition}[Hyperplane]
A subspace whose dimension is one less than its ambient space
\end{definition}
For the context of this lecture, the ambient space is defined as the Hilbert space
\textbf{Intuition}\\
Given training data $(x_i,y_i)$ for $i=1,...,n$ with $x_i\in \mathbb{R}^2$ and $y_i\in\{-1,+1\}$ learn a classifier $f(x)$ training such that
$$f\left(x_{i}\right)=\left\{\begin{array}{ll}
\geq 0, & y_{i}=+1 \\
<0, & y_{i}=-1
\end{array}\right.$$
The problem with this solution is that there is no optimal solution of the Hyperplane given the training data points. Hyperplane have can alternately be named decision boundary. Many lines could be just as valid
\begin{definition}[Separating Hyperplane]
Let $S=\{(x_i,y_i)\}^m_{i=1}\in \mathbb{R}^d\times \{-1,+1\}$ be a training set\\
By a hyperplane we mean a set of Hilbert space $H_{w,b}=\{x\in \mathbb{R}^d:w^Tx+b=0\}$ parametrised by $w\in \mathbb{R}^d$ and $b\in \mathbb{R}$\\
We assume that the data are linearly separable, that is, there exist $w\in \mathbb{R}^d$ and $b\in \mathbb{R}$ such that $y_i(w^Tx_i+b)>0,i=1,..,m$\\
In which case we call $H_{w,b}$ a separating hyperplane
\end{definition}
\begin{definition}[Distance]
The distance $\rho_x(w,b)$ of a point x from a hyperplane $H_{w,b}$ is
$$\rho_{x}(w, b)=\frac{\left|w^{T} x+b\right|}{\|w\|}$$
\end{definition}
\begin{definition}[Margin]
If $H_{w,b}$ separates the training set S we define its margin as:
$$\rho_{x}(w,b)=\min_{i=1:m}\rho_{x_i}(w,b)$$
\end{definition}
If $H_{w,b}$ is a hyperplane (separating or not) we also define the margin of a point $x$ as $w^Tx+b$
\subsection{Optimal separating hyperplane}
The separating hyperplane with maximum margin can be solved with the following optimisation problem
$$\rho(S)=\max _{w, b} \min _{i}\left\{\frac{y_{i}\left(w^{T} x_{i}+b\right)}{\|w\|}: y_{i}\left(w^{T} x_{i}+b\right) \geq 0, \quad i=1, \ldots, m\right\}>0$$
A separating hyperplane is parameterised by (w,b), but this choice is not unique (rescaling with a positive constant gives the same separating hyperplane)\\
\\
There are two possible ways to fix the parameterisation:
\begin{itemize}
	\item \textbf{Normalised hyperplane}: set $||w||=1$, in which case $\rho_{x}(w,b)=|w^Tx+b|$ and $\rho_s(w,b)=\min_{i=1:m}y_i(w^Tx_i+b)$
	\item \textbf{Canonical hyperplane}: choose $||w||$ such that $\rho_s(w,b)=\dfrac{1}{||w||}$, i.e. we require that $\min_{i=1:m}y_i(w^Tx_i+b)=1$ 
\end{itemize} 
The problem thus can be defined as\\
\\
Minimise $\frac{1}{2}w^Tw$\\
\\
Subject to $y_i(W^Tx_i+b)\geqslant1, i=1,...,m$
\subsection{Saddle point}
\begin{definition}[Saddle point]
A point on the surface og a graph of a function where the slopes in orthogonal directions are all zero, but which is not a local extremum of the function
\end{definition}
To determine the saddle point of the Lagrangain function
$$L(w, b ; \alpha)=\frac{1}{2} w^{T} w-\sum_{i=1}^{m} \alpha_{i}\left\{y_{i}\left(w^{T} x_{i}+b\right)-1\right\}$$
where $\alpha_{i}\geqslant0$ are the Lagrange multipliers\\
\\
We minimise L over (w,b) and maximise over $\alpha$. Differentiating with respect to w and b we obtain
$$\frac{\partial L}{\partial b}=-\sum_{i=1}^{m} y_{i} \alpha_{i}=0$$
$$\frac{\partial L}{\partial w}=w-\sum_{i=1}^{m} \alpha_{i} y_{i} x_{i}=0 \Rightarrow w=\sum_{i=1}^{m} \alpha_{i} y_{i} x_{i}$$
\end{document}