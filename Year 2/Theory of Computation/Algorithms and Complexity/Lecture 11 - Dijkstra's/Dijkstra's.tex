\documentclass{article}[18pt]
\input{../../../../format}
\lhead{Theory of Computation - Algorithms and Complexity}


\begin{document}
\begin{center}
\underline{\huge Dijkstra's Shortest Path Algorithm}
\end{center}
The shortest path between u and v is denoted $\delta(u,v)$, if there is no path, then $\delta(u,v)=\infty$
\section{Can a shortest path contain a cycle?}
A directed cycle is:
\begin{itemize}
	\item Positive: if its edge weights sum up to a positive number
	\item Negative: if its edge weights sum up to a negative number
\end{itemize}
If there is a positive cycle in the graph, it will not be contained in any shortest path between u and v so we can assume that the shortest paths we find contain no positive cycles.\\
\\
However if there is a negative cycle between u and v, then $\delta(u,v)=-\infty$ so we shall assume that the graphs we consider do not contain negative cycles.
\section{Single-Source Shortest Paths}
\begin{itemize}
	\item Aim: to describe an algorithm that solves the single-source shortest paths problem, i.e. an algorithm that finds the shortest path from a specific source vertex 
	\item This is a generalization of BFS
	\item So the output of the algorithm should be two arrays d, $\pi$ where for each vertex v:
	\begin{itemize}
		\item $d(v)=\delta(s,v)$
		\item $\pi(v)$ is the predecessor of v
	\end{itemize}
\end{itemize}
\section{Relaxation}
\begin{itemize}
	\item Assume that the weight on every edge is non-negative
	\item We do not directly compute the entry $d(v)=\delta(s,v)$ 
	\item Instead, at every step, $d(v)$ is an estimate for $\delta(s,v)$
	\begin{itemize}
		\item Initially, $d(v)=\infty$, and it always remains $d(v)\geqslant \delta(s,v)$
		\item $d(v)$ is updated (i.e. it decreases) as shorter paths are found
		\item At the end of the algorithm we have $d(v)=\delta(s,v)$
	\end{itemize}
\end{itemize}
\begin{lstlisting}[caption=Initialise-Single-Source({G,s})]
for each vertex v$\in$V(g) do
	d(v)=$\infty$
	$\pi$(v)=NIL
d(s)=0
\end{lstlisting}
The process of relaxing an edge (u,v):
\begin{itemize}
	\item Test whether we can improve the shortest path from s to v that we found so far, by going through u
	\item If yes, then update $d(v)$ and $\pi(v)$
	\begin{itemize}
		\item Decrease the estimate $d(v)$
		\item Update the predecessor $\pi(v)$ to u
	\end{itemize}
	\item The algorithm first calls initialise-single-source and then it repeatedly relaxes the appropriate edges (according to the weight function w)
\end{itemize}
\begin{lstlisting}[caption=Relax({u,v,w})]
if d(v)$>$d(u)+w(u,v) then
	d(v)=d(u)+w(u,v)
	$\pi$(v)=u
\end{lstlisting}
\section{Dijkstra's Algorithm}
\begin{itemize}
	\item Initialisation: distance to source A is 0, $S=\varnothing$, Q=V
	\item S stores the vertices v for which we already found $\delta(A,v)$
	\item Q stores all the other vertices
	\item While q is not empty
	\begin{itemize}
		\item Remove from Q the vertex u for which d(u) is minimum
		\item Add this vertex to s
		\item Relax all edges leaving u
	\end{itemize} 
\end{itemize}
\begin{lstlisting}[caption=Dijkstra({G,w,s})]
Initialise-Single-Source(G,s)
S=$\varnothing$
Q=V(G)
while Q$\neq \varnothing$ do
	u=Extract-Min(Q)
	S=S$\cup${u}
	for each vertex v $\in$ Adj(u) do
		Relax(u,v,w)
\end{lstlisting}
\subsection{Runtime}
Initialisation is done in $\mathcal{O}(V)$ time  - two operations per vertex\\
\\
Finding the vertex v in Q with minimum d(v)  takes $\mathcal{O}(V)$ time and this is done v times
\begin{itemize}
	\item To find the minimum, just scan the set Q
	\item To compute the new vertex in S, find the new minimum of Q
\end{itemize}
Relaxation takes in total $\mathcal{O}(E)$ time as every edge is relaxed once\\
\\
The total running time is $\mathcal{O}(V+V^2+E)=\mathcal{O}(V^2)$\\
\\
However using a more sophisticated implementation for extracting the minimum for Q it can run in $\mathcal{O}(V\log V+E)$ time
\section{Properties of shortest paths and relaxation}
\begin{definition}[Triangle inequality]
For all edges (u,v) we have $\delta(s,v)\leqslant \delta(s,u)+w(u,v)$
\end{definition}
\newpage
\begin{definition}[Optimal Substructure]
Any subpath of a shortest path is also a shortest path
\end{definition}
\begin{definition}[Upper bound property]
For every vertex v, we have $d(v)\geqslant \delta(s,v)$
\end{definition}
\begin{definition}[No-path property]
If $\delta(s,v)=\infty$ then we have $d(v)=\infty$ at every iteration
\end{definition}
\begin{definition}[Convergence property]
If there is a shortest path from s to v including the edge (u,v) and if $d(u)=\delta(s,u)$, then we obtain $d(v)=\delta(s,v)$ when $(u,v)$ is relaxed
\end{definition}
\section{Correctness of Dijkstra's algorithm}
We need to prove the loop invariant always remains true.\\
\\
At the start of each iteration of the while loop, $d(v)=\delta(s,v)$ for every $v\in S$
\begin{itemize}
	\item Initialisation: at the start of the algorithm S is empty, so the loop invariant is trivially true
	\item Maintenance: we need to show that $d(u)=\delta(s,u)$ when u is added to S
	\item Termination: at the end, S contains every vertex, which implies that $d(v)=\delta(s,v)$ for all vertices v in the graph
\end{itemize}







\end{document}