\documentclass{article}[18pt]
\input{../../../../format}
\lhead{Theory of Computation - Algorithms and Complexity}


\begin{document}
\begin{center}
\underline{\huge Network flows}
\end{center}
\section{Flow networks, flow, maximum flow}
\begin{itemize}
	\item Material is transferred in a network from a "source" to a "sink"
	\item Source produces material at a steady rate, sink consumes at same rate
\end{itemize}
Edges have a given capacity\\
\\
Vertices (other than source/sink) are junctions
\begin{itemize}
	\item Material flows through them without collecting in them
	\item Entering rate = exiting rate
\end{itemize}
\section{Definitions}
\begin{definition}[Maximum-flow problem]
	We wish to compute the greatest possible rate of transportation from source to sink
\end{definition}

\begin{definition}[Flow network]
\begin{itemize}
	\item G=(V,E)
	\item Two distinguished vertices: source s and sink t
	\item Each edge $(u,v)\in E$ as non-negative capacity $c(u,v)\geqslant 0$
	\item If$(u,v)\notin E$, we assume $c(u,v)=0$
	\item For each $v\in V$, there is a path $s\rightarrow v \rightarrow t$
\end{itemize}
\end{definition}
\subsection{Flow constraints}
\begin{definition}[Capacity constraint]
	For all $u,v\in V$, we require $f(u,v)\leqslant c(u,v)$\\
	Flow from one vertex to another must not exceed given capacity
\end{definition}

\begin{definition}[Skew symmetry]
	For all $u,v\in V$, we require $f(u,v)=-f(v,u)$\\
	Flow from vertex u to vertex v is negative of flow in reverse direction
\end{definition}
\newpage
\begin{definition}[Flow conservation]
For all $u\in V - \{s,t\}$ we require
$$\sum_{c\in V}f(u,v)=0$$
Total flow out of a vertex is 0, likewise for total flow into a vertex (just saying what goes in, comes out), this doesn't apply to the source or drain
\end{definition}
\subsection{Total flows}
\begin{definition}[Total positive flow]
The total positive flow entering vertex v is
$$\sum_{u\in V: f(u,v)>0}f(u,v)$$
The total positive flow leaving vertex u is
$$\sum_{v\in V: f(u,v)>0}f(u,v)$$
\end{definition}

\begin{definition}[Total net flow]
	The total net flow at a vertex v is\\
	total positive flow leaving v - total positive flow entering v
\end{definition}

\begin{definition}[Flow value]
	The value of flow f is defined as the total flow leaving the source (and thus entering the sink)
	$$|f|=\sum_{v\in V}f(s,v)$$
	Note that $|\cdot|$ does not mean absolute value
\end{definition}
If there is an arrow only in one direction on the graph, then the capacity in the other direction is 0. There is no assumption of symmetric capacities.
\section{Technical tools}
\textbf{Implicit summation}\\
Let $X,T\subseteq V$. Then
\[
f(X, Y)=\sum_{x \in X} \sum_{y \in Y} f(x, y)
\]
Commonly occurring identities
\begin{enumerate}
	\item For all $X\subseteq V$, we have $f(X,X)=0$. Because each $f(u,v)$ and $f(v,u)=-f(u,v)$ cancel each other
	\item For all $X,Y\subseteq V$, we have $f(X,Y)=-f(Y,X)$. Generalisation of $f(X,X)=0$, with the same reasoning
	\item For all $X,Y,Z\subseteq V$ with $X\cap Y=\varnothing$, we have
	$$f(X\cup Y, Z)=f(X,Z)+f(Y,Z)$$
	$$f(Z,X\cup Y) = f(Z,X) + f(Z,Y)$$
	Split summation into two: one over X, one over Y
\end{enumerate}
Three important ideas:
\begin{enumerate}
	\item Residual networks
	\item Augmenting paths
	\item Cuts
\end{enumerate}
Method is iterative:
\begin{enumerate}
	\item Start with $f(u,v)=0$ for all $u,v\in V$
	\item At each iteration, increase flow value by finding an augmenting path (a path from source to sink along which we can increase flow) and then augment flow along this path
	\item Repeat until no augmenting paths can be found
\end{enumerate}
\section{Residual networks}
\textbf{Idea:} Residual network consists of edges that can admit more flow\\
\\
\textbf{Formally:} Consider vertices u and v. Amount of additional flow we can push from u to v before exceeding capacity $c(u,v)$ is residual capacity of $(u,v)$
$$c_f(u,v)  = c(u,v) - f(u,v)$$
Note that when flow $f(u,v)$ is negative, then residual capacity $c_f(u,v)$ is greater than $c(u,v)$\\
Interpretation:
\begin{itemize}
	\item Flow of $-x$ from u to v (i.e. $f(u,v)=-x<0$)
	\item Implies flow of x from v to u (i.e. $f(v,u)=x>0$)
	\item Can be cancelled by pushing x units from u back to v
	\item Can then push another $c(u,v)$ from u to v
	\item We can push in total $c_f(u,v)=c(u,v)+x>c(u,v)$ from u to v
\end{itemize}
Given flow network $G=(V,E)$ and flow f, the residual network of G induced by f is $G_f=(V,E_f)$ with
\[
E_{f}=\left\{(u, v) \in V \times V: c_{f}(u, v)>0\right\}
\]
i.e. each residual edge that can admit flow that is strictly positive
$$G: \ u\xrightleftharpoons[1]{3/5} v \ \Rightarrow \ G_f: \ u\xrightleftharpoons[4]{2} v$$
Note that $|E_f| \leqslant 2|E|$\\
\\
\textbf{Lemma}:\\
Let $G=(V,E)$ be a flow network, f be a flow in G, $G_f$ be the residual network of G induced by f, and let $f'$ be a flow in $G_f$. Then the flow sum $f+f'$ with
$$(f+f')(u,v)=f(u,v)+f'(u,v)$$
Is a flow in G with value $|f+f'|=|f|+|f'|$
\textbf{Proof}:\\
Must verify that skew symmetry, capacity constraints and conservation are obeyed\\
\textit{Skew symmetry}:
\[
\begin{aligned}
\left(f+f^{\prime}\right)(u, v) &=f(u, v)+f^{\prime}(u, v) \\
&=-f(v, u)-f^{\prime}(v, u) \\
&=-\left(f(v, u)+f^{\prime}(v, u)\right) \\
&=-\left(f+f^{\prime}\right)(v, u)
\end{aligned}
\]
\textit{Capacity constraint}:\\
Note $f'$ is flow in $G_fr$, so $f'(u,v)\leqslant  c_f(u,v)$. Since by def. $c_f(u,v) = c(u,v)-f(u,v)$
\[
\begin{aligned}
\left(f+f^{\prime}\right)(u, v) &=f(u, v)+f^{\prime}(u, v) \\
& \leq f(u, v)+c_{f}(u, v) \\
&=f(u, v)+(c(u, v)-f(u, v)) \\
&=c(u, v)
\end{aligned}
\]
\textit{Flow conservation}:\\
For all $u\in V-\{s,t\}$
\[
\begin{aligned}
\sum_{v \in V}\left(f+f^{\prime}\right)(u, v) &=\sum_{v \in V}\left(f(u, v)+f^{\prime}(u, v)\right) \\
&=\sum_{v \in V} f(u, v)+\sum_{v \in V} f^{\prime}(u, v) \\
&=0+0 \\
&=0
\end{aligned}
\]
\textit{Value}
\[
\begin{aligned}
\left|f+f^{\prime}\right| &=\sum_{v \in V}\left(f+f^{\prime}\right)(s, v) \\
&=\sum_{v \in V}\left(f(s, v)+f^{\prime}(s, v)\right) \\
&=\sum_{v \in V} f(s, v)+\sum_{v \in V} f^{\prime}(s, v) \\
&=|f|+\left|f^{\prime}\right|
\end{aligned}
\]
\section{Augmenting Paths}
Given a flow network $G(V,E)$ and flow f an augmenting path P is a simple path in residual network $G_f$\\
\\
Recall that each edge $(u,v)$ in $G_f$ admits some additional positive flow, obeying capacity constraint\\
\\
Flow value can be increased by
\[
c_{f}(P)=\min _{(u, v) \in P} c_{f}(u, v)
\]
\begin{center}
	\includegraphics[scale=0.7]{"Augmenting Path"}
\end{center}
\textbf{Lemma}:\\
\\
Let $G=(V,E)$ to be a flow network, f be a flow in G, and let P be an augmenting path in $G_f$. Define $f_p$ by
\[
f_{P}(u, v)=\left\{\begin{array}{ll}
{c_{f}(P)} & {\text { if }(u, v) \text { is on } P} \\
{-c_{f}(P)} & {\text { if }(v, u) \text { is on } P} \\
{0} & {\text { otherwise }}
\end{array}\right.
\]
Then $f_p$ is a flow in $G_f$ with value $|f_p|=c_f(P)>0$
\textbf{Corollary (Improve)}\\
Let $G,f,P,f_p$ be as above, Define $f'=f+f_p$. Then $f'$ is a flow in G with value $|f'|=|f|+|f_p|>|f|$
\section{Basic Ford-Fulkerson algorithm}
\begin{lstlisting}[caption=Ford-Fulkerson({G,s,t})]
for each edge (u,v) $\in$ E do
	f(u,v)=0
	f(v,u)=0
end for
while there exists a path P = $s\rightsquigarrow t$ in the residual network $G_f$ do
	$c_f$(P)=min\{$c_f$(u,v):(u,v)\in P\}
	for each edge $(u,v)\in P$ do
		f(u,v)=f(u,v)+$c_f$(P)
		f(v,u)=-f(u,v)
	end for
end while
\end{lstlisting}
\subsection{Running Time}
The running strongly depends on how the augmenting paths in line 4 are determined\\
\\
\textbf{If chosen poorly:}
\begin{itemize}
	\item Value of flow increases with each iteration
	\item But possibly too slowly
	\item In extreme cases it can never terminate and it can even not converge to the value of maximum flow (can only happen if capacities are irrational numbers)
\end{itemize}
\textbf{In practice}
\begin{itemize}
	\item Capacities are integers so always terminates
	\item If capacities are small then it is an efficient algorithm (polynomial time)
\end{itemize}
\subsubsection{Running time analysis}
Assume integral capacities. Simple bound $\mathcal{O}(E\cdot |f^*|)$ for running time when choosing paths arbitrarily, $|f^*|$ being the value of maximum flow
\begin{itemize}
	\item Initialisation in lines 1-4 take $\mathcal{O}(E)$
	\item While loop in lines 5-11 is executed at most $|f^*|$ times (value of flow increases by at least one unit in each interaction)
	\item In lines 6-10 (within the while loop) we need $\mathcal{O}(V+E)=\mathcal{O}(E)$ time. 
\end{itemize}
\subsubsection{Problematic case}
Algorithm needs $\Omega(E|f^*|)$ time in worst case, because $(u,v)$ is always chosen to be part of augmenting path.\\
\\
Note that the value of the maximum flow f* can be arbitrarily large (might even be exponential on the size of the network)\\
\\
This algorithm has not polynomial running time in the worst case
\section{Edmonds-Karp algorithm}
\begin{itemize}
	\item A special implementation of Ford-Fulkerson algorithm
	\item The augmenting path P is always chosen to be a shortest path from s to t (e.g. using BFS) in the residual network $G_f$
	\item Regardless of the flow that fits in this path P
\end{itemize}
This has running time $\mathcal{O}(V\cdot E^2)$
\section{Cuts of flow networks}
A cut of a flow network $G=(V,E)$ is a partition of V into the sets S and T=V-S such that $s\in S$ and $t\in T$\\
\\
If f is a flow in G, then $f(S,T)$ is the net flow across cut (S,T); its capacity is c(S,T)\\
\\
A minimum cut is a cut with minimum capacity over all cuts\\
\\
\textbf{Lemma: Net Flow}\\
Let f be a flow in a flow network G with source s and sink t, let (S,T) be a cut of G. Then the flow across (S,T) is f(S,T)=$|f|$\\
\\
\textbf{Corollary (Upper Cut)}\\
The value of a flow f in a network G is upper bounded by the capacity of any cut (S,T) in G\\
\\
\textbf{Proof}
\[
|f|=f(S, T)=\sum_{u \in S} \sum_{v \in T} f(u, v) \leq \sum_{u \in S} \sum_{v \in T} c(u, v)=c(S, T)
\]
Therefore max-flow $\leqslant$ min-cut\\
\textbf{Theorem: Max-flow min-cut}\\
If f is a flow in a flow network $G=(V,E)$ with source s and sink t, then the following conditions are equivalent
\begin{enumerate}
	\item f is a maximum flow in G
	\item The residual network $G_f$ contains no augmenting path
	\item $|f|=c(S,T)$ for some cut $(S,T)$ of G
\end{enumerate} 


\end{document}