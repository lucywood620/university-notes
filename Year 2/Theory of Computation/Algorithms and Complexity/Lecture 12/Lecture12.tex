\documentclass{article}[18pt]
\usepackage{../../../../format}
\lhead{Theory of Computation - Algorithms and Complexity}


\begin{document}
\begin{center}
\underline{\huge Network flows}
\end{center}
\section{Flow networks, flow, maximum flow}
\begin{itemize}
	\item Material is transferred in a network from a "source" to a "sink"
	\item Source produces material at a steady rate, sink consumes at same rate
\end{itemize}
Edges have a given capacity\\
\\
Vertices (other than source/sink) are junctions
\begin{itemize}
	\item Material flows through them without collecting in them
	\item Entering rate = exiting rate
\end{itemize}
\section{Definitions}
\begin{definition}[Maximum-flow problem]
	We wish to compute the greatest possible rate of transportation from source to sink
\end{definition}

\begin{definition}[Flow network]
\begin{itemize}
	\item G=(V,E)
	\item Two distinguished vertices: source s and sink t
	\item Each edge $(u,v)\in E$ as non-negative capacity $c(u,v)\geqslant 0$
	\item If$(u,v)\notin E$, we assume $c(u,v)=0$
	\item For each $v\in V$, there is a path $s\rightarrow v \rightarrow t$
\end{itemize}
\end{definition}
\subsection{Flow constraints}
\begin{definition}[Capacity constraint]
	For all $u,v\in V$, we require $f(u,v)\leqslant c(u,v)$\\
	Flow from one vertex to another must not exceed given capacity
\end{definition}

\begin{definition}[Skew symmetry]
	For all $u,v\in V$, we require $f(u,v)=-f(v,u)$\\
	Flow from vertex u to vertex v is negative of flow in reverse direction
\end{definition}
\newpage
\begin{definition}[Flow conservation]
For all $u\in V - \{s,t\}$ we require
$$\sum_{c\in V}f(u,v)=0$$
Total flow out of a vertex is 0, likewise for total flow into a vertex (just saying what goes in, comes out), this doesn't apply to the source or drain
\end{definition}
\subsection{Total flows}
\begin{definition}[Total positive flow]
The total positive flow entering vertex v is
$$\sum_{u\in V: f(u,v)>0}f(u,v)$$
The total positive flow leaving vertex u is
$$\sum_{v\in V: f(u,v)>0}f(u,v)$$
\end{definition}

\begin{definition}[Total net flow]
	The total net flow at a vertex v is\\
	total positive flow leaving v - total positive flow entering v
\end{definition}

\begin{definition}[Flow value]
	The value of flow f is defined as the total flow leaving the source (and thus entering the sink)
	$$|f|=\sum_{v\in V}f(s,v)$$
	Note that $|\cdot|$ does not mean absolute value
\end{definition}
If there is an arrow only in one direction on the graph, then the capacity in the other direction is 0. There is no assumption of symmetric capacities.
\section{Technical tools}
\textbf{Implicit summation}\\
Let $X,T\subseteq V$. Then
\[
f(X, Y)=\sum_{x \in X} \sum_{y \in Y} f(x, y)
\]
Commonly occurring identities
\begin{enumerate}
	\item For all $X\subseteq V$, we have $f(X,X)=0$. Because each $f(u,v)$ and $f(v,u)=-f(u,v)$ cancel each other
	\item For all $X,Y\subseteq V$, we have $f(X,Y)=-f(Y,X)$. Generalisation of $f(X,X)=0$, with the same reasoning
	\item For all $X,Y,Z\subseteq V$ with $X\cap Y=\varnothing$, we have
	$$f(X\cup Y, Z)=f(X,Z)+f(Y,Z)$$
	$$f(Z,X\cup Y) = f(Z,X) + f(Z,Y)$$
	Split summation into two: one over X, one over Y
\end{enumerate}
Three important ideas:
\begin{enumerate}
	\item Residual networks
	\item Augmenting paths
	\item Cuts
\end{enumerate}
Method is iterative:
\begin{enumerate}
	\item Start with $f(u,v)=0$ for all $u,v\in V$
	\item At each iteration, increase flow value by finding an augmenting path (a path from source to sink along which we can increase flow) and then augment flow along this path
	\item Repeat until no augmenting paths can be found
\end{enumerate}
\section{Residual networks}
\textbf{Idea:} Residual network consists of edges that can admit more flow\\
\\
\textbf{Formally:} Consider vertices u and v. Amount of additional flow we can push from u to v before exceeding capacity $c(u,v)$ is residual capacity of $(u,v)$
$$c_f(u,v)  = c(u,v) - f(u,v)$$
Note that when flow $f(u,v)$ is negative, then residual capacity $c_f(u,v)$ is greater than $c(u,v)$\\
Interpretation:
\begin{itemize}
	\item Flow of $-x$ from u to v (i.e. $f(u,v)=-x<0$)
	\item Implies flow of x from v to u (i.e. $f(v,u)=x>0$)
	\item Can be cancelled by pushing x units from u back to v
	\item Can then push another $c(u,v)$ from u to v
	\item We can push in total $c_f(u,v)=c(u,v)+x>c(u,v)$ from u to v
\end{itemize}



\end{document}