\documentclass{article}[18pt]
\usepackage{../../../../format}
\lhead{Theory of Computation - Algorithms and Complexity}
\newtheorem{theorem}{Theorem}


\begin{document}
\begin{center}
\underline{\huge Decision Problems}
\end{center}
\section{Optimisation vs Decision Problems}
\begin{itemize}
	\item A major variation of optimization problems: decision problems
	\item Answer is not a value but YES/NO
	\item Every optimisation -problem has a decision counterpart
	\item An optimisation problem has a fast algorithm iff the corresponding decision problem has a fast algorithm
\end{itemize}
\section{Decision problems and encodings}
The standard way to define a decision problem is to describe a generic instance and a yes-no question about each instance\\
\\
\textbf{Reachability}:
\begin{itemize}
	\item Instance: A finite directed graph G and vertices s and t
	\item Question: Is there a path in G from s to t?
\end{itemize}
To input problems to a computer, each instance must be encoded as a string of symbols over some alphabet. We need an encoding scheme\\
\\
To ensure that encoding the problem does not change its essential nature, an encoding scheme must be concise
\begin{itemize}
	\item Represent numbers efficiently
	\item Not add unnecessary information
\end{itemize}
\section{Languages}
An alphabet, $\Sigma$ is a (finite) set of symbols\\
A string over $\Sigma$ is a finite sequence of symbols from $\Sigma$\\
A language over $\Sigma$ is any set of strings over $\Sigma$\\
\\
For a problem $\Pi$ and an encoding scheme e with alphabet $\Sigma$, the set of all strings corresponding to instances with answer yes is denoted $\mathscr{L}(\Pi,e)$ and is called the language associated with $\Pi$ and e\\
\\
For decision problems, we just want to decide whether the (encoding of a) given instance belongs to the alphabet $\mathscr{L}(\Pi, e)$
\section{Complexity of Problems}
\begin{itemize}
	\item The problems encountered so far in this course have all proved to be tractable (since we have found fast algorithms for all of them)
	\item There are many problems however, which cannot be quickly solved in practice, i.e., which are intractable
	\item There are many difficulties:
	\begin{itemize}
		\item What do we mean by tractable and intractable
		\item Can we define these notions formally
		\item Can we prove that a problem is one but not the other
	\end{itemize}
	\item One technique that will prove very useful: showing that one decision problem can be transformed(reduced) into another
\end{itemize}
\section{Complexity measures}
Every decidable problems has a set of algorithms that solved it. The properties of this set of algorithms:
\begin{itemize}
	\item The difficulty of constructing the algorithm
	\item The length of the shortest possible algorithm
	\begin{itemize}
		\item Static complexity measure
		\item Useful for classifying the complexity of strings, called Kolmogorov complexity
	\end{itemize}
	\item The efficiency of the most possible algorithm?
	\begin{itemize}
		\item Dynamic complexity measure
		\item A numerical function that measures the maximum resources used by an algorithm to compute the answer to a given instance
	\end{itemize}
\end{itemize}
\section{Dynamic complexity measures}
\begin{itemize}
	\item The most critical resources are often time/space
	\item By considering Turing Machines as our model of computation
\end{itemize}
\begin{definition}[Time complexity]
The time complexity of a Turing Machine T is the function $Time_T$ such that $Time_T(x)$ is the number of steps taken by the computation $T(x)$
\end{definition}

\begin{definition}[Space complexity]
The space complexity of a Turing Machine T is the function $Space_T$ such that $Space_T(x)$ is the number of distinct tape cells visited during the computation $T(x)$
\end{definition}

\section{Equivalence}
\begin{theorem}
An n-vertex graph G has an independent set of size k iff G has a vertex cover of size n-k
\end{theorem}
\textbf{Corollary} - The decision problems independent set and vertex cover are equivalent
\section{Clique}
\begin{itemize}
	\item Instance - A graph G and an integer k
	\item Question: Does G have a clique of size at least k - i.e. a set of at least k vertices that are all adjacent to one another
\end{itemize}
\section{Complement}
\begin{definition}[Complement]
The complement of a graph G has the same vertex set, and there is an edge between two vertices u and v in the complement iff there is no edge from u to v in G
\end{definition}
\begin{theorem}
A graph G has an independent set of size k iff its complement $\overline{G}$ has a clique of size k
\end{theorem}
\textbf{Corollary} - The decision problems independent set and clique are equivalent
\section{Set cover}
\begin{itemize}
	\item Instance - A set U of n elements, a collection of subsets $S_1,S_2,...,S_t$ whose union equals I, and an integer k
	\item Question: Does there exist a collection of at most k of these sets whose union is equal to all of U
\end{itemize}
\texttt{VERTEX COVER} can be solved using an algorithm for \texttt{SET COVER}, but not the other way round





\end{document}