\documentclass{article}[18pt]
\input{../../../../format}
\lhead{Theory of Computation - Algorithms and Complexity}


\begin{document}
\begin{center}
\underline{\huge Decision Problems}
\end{center}

As
$$\log_an=\log_ab \times \log_bn$$
$\log_\alpha n$ and $\log_\beta n$ are the same when it comes to $\mathcal{O}$

\begin{definition}[Time complexity]
	For any function f, we say that the time complexity of a decidable language $\mathcal{L}$ is $\mathcal{O}(f)$, or $\mathcal{L}$ is decidable in $\mathcal{O}(f)$ time, if there exists a TM T which decides $\mathcal{L}$, and constants $n_0$ and $c$ such that for all inputs x with $|x|>n_0$
	$$Time_T(x)\leqslant c\cdot f(|x|)$$
\end{definition}
\section{Complexity Classes}
\begin{definition}[Time complexity class {TIME[f]}]
	The class of all problems for which there exists an algorithm with time complexity in $\mathcal{O}(f)$
\end{definition}
This is sometimes called DTIME[f] - for deterministic time
\section{The complexity class P}
\begin{definition}[P]
$$
\mathbf{P}=\bigcup_{k \geq 0} T I M E\left[n^{k}\right]
$$
\end{definition}
The class P is a reasonable mathematical model of the class of problems which are tractable or solvable in practice\\
\\
However, the correspondence is not exact:
\begin{itemize}
	\item When the degree of the polynomial is high then the time grows so fast that in practice the problem is not solvable
	\item The constants may also be very large
\end{itemize}
\section{Different models of computation}
\textbf{Lemma}\\
We can simulate t steps of k-tape TM with an equivalent one tape TM in $\mathcal{O}[t^2]$ steps\\
\\
\textbf{Lemma}\\
We cans simulate t steps of a two-way infinite k-tape machine with an equivalent k-tape TM in $\mathcal{O}[t]$ steps\\
\\
Hence the class P is the same for all of these models of computation (and many others)
\section{Different encodings}
\textbf{Lemma}\\
For any number n, the length of the encoding of n in base $b_1$ and the length of the encoding of n in base $b_2$ are related by a constant factor (provided $b_1,b_2\geqslant 2$)\\
\\
Hence the class P is the same for these encodings (and many others)
\section{Proving a problem is in P}
\begin{itemize}
	\item The class P is said to be robust - it doesn't depend on the exact details of the computational model or encoding so we don't need to specify all the details of the machine model or even the encoding
	\item The most direct way to show that a problem is in P is to give a polynomial time algorithm which solves it 
	\item Even a naive polynomial time algorithm often provides a good insight into how the problem can be solved efficiently
	\item To find such an algorithm we generally need to identify an approach to the problem that is considerably better than brute-force search
\end{itemize}
\section{CNF}
Some of the most important computational problems concern logical formulas\\
\\
A logical formula f is said to be in conjunctive normal form (CNF) if
$$f=C_1\land ... \land C_m$$
where each $C_i$ is a clause, which is a disjunction (OR) of literals
$$C_i=l_{i1}\lor ... \lor l_{ik}$$
and a literal is either a variable or its negation
\begin{definition}[k-CNF]
If a logical formula in CNF has at most k literals per clause then it is in k-CNF
\end{definition}

\section{Satisfiability}
\begin{definition}[Satisfiable]
The logical formula f is satisfiable if there exists an assignment of True and False to the variables of f which makes f true
\end{definition}
\begin{itemize}
	\item f is True iff all the clauses are True
	\item A clause is True iff at least one literal is true
\end{itemize}
\begin{problem}[Satisfiability]
\textbf{Instance} - CNF formula f\\
\textbf{Question} - Is f satisfiable?
\end{problem}

\begin{problem}[k-Satisfiability]
\textbf{Instance} - k-CNF formula f\\
\textbf{Question} - Is f satisfiable?
\end{problem}
\subsection{2 Satisfiability}
\textbf{Proposition} - 2-Satisfiability is in P\\
\\
\textbf{Proof}:
\begin{enumerate}
	\item Declare all clauses unsatisfied and literals unassigned
	\item Select an arbitrary unassigned variable x and assign x the value True and $\lnot x$ the value False
	\item Select an unsatisfied clause $l_i\lor l_j$
	\begin{enumerate}
		\item If both literals are unassigned, ignore the clause
		\item If at least one literal is assigned True, declare the clause satisfied
		\item If both are False, restart the algorithm setting x false and $\lnot x$ True. If a conflict occurs again, declare unsatisfiable
		\item If one literal is False and the other unassigned, set the other to True and its negation to False, and declare the clause satisfied
	\end{enumerate}
	\item Repeat step 3 until either
	\begin{enumerate}
		\item All clauses are satisfied, return satisfiable
		\item Or all clauses remaining not satisfied (yet) have all their variables unassigned. In this case return to step 2
	\end{enumerate}
\end{enumerate}
\section{Polynomial-time reducibility}
Another way to show that a problem is in P is to use a reduction\\
\\
Informally, a problem P is reducible to a problem Q if we can somehow use methods that solve Q in order to solve P
\begin{definition}[Polynomially reducible]
A language $\mathcal{L}_1$ is polynomially reducible to $\mathcal{L}_2$, denoted $\mathcal{L}_1\leqslant \mathcal{L}_2$, if a polynomial-time computable function f exists such that
$$x\in \mathcal{L}_1 \Leftrightarrow f(x)\in \mathcal{L}_2$$
\end{definition}
\textbf{Lemma}\\
$$\mathcal{L}_1\leqslant \mathcal{L}_2 \text{ and } \mathcal{L}_2 \in P \Rightarrow \mathcal{L}_1\in P$$
Main idea - The composition of polynomials is a polynomial\\
\\
\textbf{Proof}
\begin{itemize}
	\item Let $A_2$ be a polynomial-time algorithm that decides $L_2$
	\item Let f be a polynomial-time reduction algorithm from $L_1$ to $L_2$
	\item We construct a polynomial-time algorithm $A_1$ that decides $L_1$
	\begin{enumerate}
		\item Given input $x\in \{0,1\}^*$, compute $f(x)$ in polynomial time (we know that $x\in L_1 \Leftrightarrow f(x)\in L_2$)
		\item Use algorithm $A_2$ to decide whether $f(x)\in L_2$
		\item If $f(x)\in L_2$ then output YES; otherwise output NO
	\end{enumerate}
\end{itemize}
\section{k-Colourability}
\begin{itemize}
	\item Let $G=(V,E)$ be a graph, with vertices V and edges E
	\item Recall that a function $f:V\rightarrow \{1,...,n\}$ is a colouring if adjacent vertices are assigned different values (colours)
\end{itemize}
\begin{problem}[k-Colourability]
\textbf{Instance} - A graph G\\
\textbf{Question} - Is there a colouring of G using at most k colours?
\end{problem}
\subsection{2-Colourability $\leqslant$ 2-Satisfiability}
We can reduce 2-Colourability to 2-Satisfiability
\begin{itemize}
	\item For each vertex $v_i$ of the graph we create a variable $x_i$
	\item For each edge $(v_i,v_j)$ we add two clauses $(x_i\lor x_j)$ and $(\lnot x_i \lor \lnot x_j)$
\end{itemize}
This translation of a 2-colourability problem to a 2-satisfiability problem is computable in polynomial time. now we check if it satisfies the reducibility condition:
\begin{itemize}
	\item [$\Rightarrow$] If the graph is 2-colourable, use 2-colouring to assign truth values to variables (one colour is true, the other false)
	\item If the formula is satisfiable, define the 2-colouring by setting true variables to colour 1 and false to colour 2. If two adjacent vertices get the same colour then one of the associated clauses is not satisfied (contradiction). Thus we have a 2-colouring
\end{itemize}

\end{document}