\documentclass{article}[18pt]
\input{../../../../format}
\lhead{Theory of Computation - Algorithms and Complexity}


\begin{document}
\begin{center}
\underline{\huge Decision Problems}
\end{center}
\begin{definition}[Time complexity]
	For any function f, we say that the time complexity of a decidable language $\mathscr{L}$ is $\mathcal{O}(f)$, or $\mathscr{L}$ is decidable in $\mathcal{O}(f)$ time, if there exists a TM T which decides $\mathscr{L}$, and constants $n_0$ and $c$ such that for all inputs x with $|x|>n_0$
	$$Time_T(x)\leqslant c\cdot f(|x|)$$
\end{definition}
\section{Complexity Classes}
\begin{definition}[Time complexity class {TIME[f]}]
	The class of all problems for which there exists an algorithm with time complexity in $\mathcal{O}(f)$
\end{definition}
\section{The complexity class P}
\begin{definition}[P]
$$
\mathbf{P}=\bigcup_{k \geq 0} T I M E\left[n^{k}\right]
$$
\end{definition}
The class P is a reasonable mathematical model of the class of problems which are tractable or solvable in practice\\
\\
However, the correspondence is not exact:
\begin{itemize}
	\item When the degree of the polynomial is high then the time grows so fast that in practice the problem is not solvable
	\item The constants may also be very large
\end{itemize}
\section{Different models of computation}
\textbf{Lemma}\\
We can simulate t steps of k-tape TM with an equivalent one tape TM in $\mathcal{O}[t^2]$ steps\\
\\
\textbf{Lemma}\\
We cans simulate t steps of a two-way infinite k-tape machine with an equivalent k-tape TM in $\mathcal{O}[t]$ steps\\
\\
Hence the class P is the same for all of these models of computation (and many others)
\section{Different encodings}
\textbf{Lemma}\\
\\
For any number n, the length of the encoding of n in base $b_1$ and the length of the encoding of n in base $b_2$ are related by a constant factor (provided $b_1,b_2\geqslant 2$)



\end{document}