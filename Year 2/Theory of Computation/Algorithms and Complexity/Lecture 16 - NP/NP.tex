\documentclass{article}[18pt]
\ProvidesPackage{format}
%Page setup
\usepackage[utf8]{inputenc}
\usepackage[margin=0.7in]{geometry}
\usepackage{parselines} 
\usepackage[english]{babel}
\usepackage{fancyhdr}
\usepackage{titlesec}
\hyphenpenalty=10000

\pagestyle{fancy}
\fancyhf{}
\rhead{Sam Robbins}
\rfoot{Page \thepage}

%Characters
\usepackage{amsmath}
\usepackage{amssymb}
\usepackage{gensymb}
\newcommand{\R}{\mathbb{R}}

%Diagrams
\usepackage{pgfplots}
\usepackage{graphicx}
\usepackage{tabularx}
\usepackage{relsize}
\pgfplotsset{width=10cm,compat=1.9}
\usepackage{float}

%Length Setting
\titlespacing\section{0pt}{14pt plus 4pt minus 2pt}{0pt plus 2pt minus 2pt}
\newlength\tindent
\setlength{\tindent}{\parindent}
\setlength{\parindent}{0pt}
\renewcommand{\indent}{\hspace*{\tindent}}

%Programming Font
\usepackage{courier}
\usepackage{listings}
\usepackage{pxfonts}

%Lists
\usepackage{enumerate}
\usepackage{enumitem}

% Networks Macro
\usepackage{tikz}


% Commands for files converted using pandoc
\providecommand{\tightlist}{%
	\setlength{\itemsep}{0pt}\setlength{\parskip}{0pt}}
\usepackage{hyperref}

% Get nice commands for floor and ceil
\usepackage{mathtools}
\DeclarePairedDelimiter{\ceil}{\lceil}{\rceil}
\DeclarePairedDelimiter{\floor}{\lfloor}{\rfloor}

% Allow itemize to go up to 20 levels deep (just change the number if you need more you madman)
\usepackage{enumitem}
\setlistdepth{20}
\renewlist{itemize}{itemize}{20}

% initially, use dots for all levels
\setlist[itemize]{label=$\cdot$}

% customize the first 3 levels
\setlist[itemize,1]{label=\textbullet}
\setlist[itemize,2]{label=--}
\setlist[itemize,3]{label=*}

% Definition and Important Stuff
% Important stuff
\usepackage[framemethod=TikZ]{mdframed}

\newcounter{theo}[section]\setcounter{theo}{0}
\renewcommand{\thetheo}{\arabic{section}.\arabic{theo}}
\newenvironment{important}[1][]{%
	\refstepcounter{theo}%
	\ifstrempty{#1}%
	{\mdfsetup{%
			frametitle={%
				\tikz[baseline=(current bounding box.east),outer sep=0pt]
				\node[anchor=east,rectangle,fill=red!50]
				{\strut Important};}}
	}%
	{\mdfsetup{%
			frametitle={%
				\tikz[baseline=(current bounding box.east),outer sep=0pt]
				\node[anchor=east,rectangle,fill=red!50]
				{\strut Important:~#1};}}%
	}%
	\mdfsetup{innertopmargin=10pt,linecolor=red!50,%
		linewidth=2pt,topline=true,%
		frametitleaboveskip=\dimexpr-\ht\strutbox\relax
	}
	\begin{mdframed}[]\relax%
		\centering
		}{\end{mdframed}}



\newcounter{lem}[section]\setcounter{lem}{0}
\renewcommand{\thelem}{\arabic{section}.\arabic{lem}}
\newenvironment{defin}[1][]{%
	\refstepcounter{lem}%
	\ifstrempty{#1}%
	{\mdfsetup{%
			frametitle={%
				\tikz[baseline=(current bounding box.east),outer sep=0pt]
				\node[anchor=east,rectangle,fill=blue!20]
				{\strut Definition};}}
	}%
	{\mdfsetup{%
			frametitle={%
				\tikz[baseline=(current bounding box.east),outer sep=0pt]
				\node[anchor=east,rectangle,fill=blue!20]
				{\strut Definition:~#1};}}%
	}%
	\mdfsetup{innertopmargin=10pt,linecolor=blue!20,%
		linewidth=2pt,topline=true,%
		frametitleaboveskip=\dimexpr-\ht\strutbox\relax
	}
	\begin{mdframed}[]\relax%
		\centering
		}{\end{mdframed}}
\lhead{Theory of Computation - Algorithms and Complexity}
\newtheorem{theorem}{Theorem}

\begin{document}
\begin{center}
\underline{\huge The Complexity Class NP}
\end{center}
\section{Certificates}
Every yes-instance of an NP problem has a short and easily checkable certificate, for example an assignment for satisfiability
\begin{definition}[Certificate]
A potential solution, it may be correct or incorrect
\end{definition}
\section{Verifiers}
\begin{definition}[Verifier]
An acceptor machine (FSM with accepting states) V which halts on all inputs is called a verifier for a language $\mathcal{L}$ if
$$\mathcal{L}=\{w| V \text{accepts "w; c" for some string c}\}$$
\end{definition}
\begin{itemize}
	\item w;c just means a problem certificate pair
	\item The string c is called a certificate (or witness) for w
	\item A verifier is said to be polynomial-time if it is a polynomial-time TM, and there is a polynomial $p(x)$ such that, for any $w\in \mathcal{L}$, there is a certificate c with $|c|\leqslant p(|w|)$
\end{itemize}
\section{The class NP}
\begin{definition}[NP]
	The class of languages that have polynomial-time verifiers is called NP	
\end{definition}
\begin{problem}[Composite Number]
\textbf{Instance} - A positive integer k\\
\textbf{Question} - Are there integers $u,v>1$ such that $u\cdot v=k$
\end{problem}
\begin{problem}[Subset Sum]
\textbf{Instance} - A collection of positive integers $S=\{a_1,...,a_k\}$ and a target integer t\\
\textbf{Question} - Is there a subset $T\subseteq S$ such that $\sum_{i\in T}a_i=t$
\end{problem}
\section{Problems (probably) not in NP}
\begin{problem}[No Hamiltonian Cycle]
\textbf{Instance} - A graph G\\
\textbf{Question} - Is it true that G has no Hamiltonian cycle?
\end{problem}
\begin{problem}[Checkers]
\textbf{Instance} - An integer n and a position in checkers on $n\times n$ board\\
\textbf{Question} - Is it a winning position for white?
\end{problem}
\section{Nondeterministic Machines}
We can get an alternative definition of the class NP by considering non-deterministic machines.\\
\\
Recall that if NT is a non-deterministic Turing Machine, then NT(x) denotes the tree of configurations which can be entered with input x, and NT accepts x if there is some accepting path in NT(x).
\begin{definition}[Time Complexity]
The time complexity of a non-deterministic Turing Machine NT is the function $NTime_{NT}$ such that $NTime_{NT}(x)$ is the number of steps in the shortest accepting path $NT(x)$ is there is one, otherwise it is the number of steps in the shortest rejecting path
\end{definition}
\section{Non-Deterministic time complexity}
\begin{definition}[Non-deterministic time complexity]
For any function f, we say that the non-deterministic time complexity of a decidable language $\mathcal{L}$ is $\mathcal{O}(f)$ is there exists a non-deterministic TM NT which decides $\mathcal{L}$, and constants $n_0$, and $c$ such that for all inputs $x$ with $|x|>n_0$
$$NTime_{NT}(x)\leqslant c\cdot f(|x|)$$
\end{definition}
\begin{definition}[Non-deterministic time complexity class]
The non-deterministic time complexity class NTIME[f] is defined to be the class of all problems (i.e. languages), for which there exists an algorithm with non-deterministic time complexity in $\mathcal{O}(f)$.
\end{definition}
\section{Alternative NP Definition}
\[
\mathrm{NP}=\bigcup_{k \geq 0} N T M E\left[n^{k}\right]
\]
Proof:
\begin{itemize}
	\item If $\mathcal{L}\in NTIME[n^k]$, then there is a non-deterministic machine NT such that $x\in\mathcal{L}$ iff there is an accepting computation path in $NT(x)$. Furthermore, the length of these paths is $\mathcal{O}(|x|^k)$
	\item Using (some encoding of) these computation paths as the certificates, we can construct a polynomial time verifier for $\mathcal{L}$ which simply checks that each step of the computation path is valid
	\item Conversely, if $\mathcal{L}$ has a polynomial-time verifier V, then we can construct a non-deterministic machine that first "guesses" the value of the certificate, and then simulates V with that certificate
	\item Since the length of this certificate is polynomial in the length of the input, this machine is a non-deterministic-polynomial-time decision procedure for $\mathcal{L}$
\end{itemize}
\section{Complete Problems}
\begin{itemize}
	\item Any complexity class can be partitioned into equivalence classes via polynomial time reduction - each class contains problems that are reducible to each other
	\item These equivalence classes are partially ordered by reduction
	\item Problems in the maximal class are called complete
\end{itemize}
\section{NP-completeness}
\begin{itemize}
	\item To show that $\mathcal{L}$ is NP-complete we must show that every language in NP can be reduced to $\mathcal{L}$ in polynomial time
	\item However once we have one NP-complete language $\mathcal{L}_0$, we can show any other language $\mathcal{L}$ is NP complete by showing that $\mathcal{L}_0\leqslant \mathcal{L}$
\end{itemize}
\section{Ladner's theorem}
\begin{theorem}
If $P\neq NP$ then NP contains infinitely many (polynomial time) inequivalent problems
\end{theorem}
\begin{itemize}
	\item This implies that unless P=NP, the class NP contains (infinitely many) problems that are neither in P nor NP-complete. Such problems are called NP-intermediate
\end{itemize}
\section{Linear Programming}
\begin{problem}[Linear Programming]
\textbf{Instance}: Integer vectors $V_i=(v_1^i,...,v_n^i), 1\leqslant i\leqslant m, D=(d_1,...,d_n), C=(c_1,...,c_n)$ and an integer B\\
\textbf{Question}: Is there a rational vector $X=(x_1,...,x_n)$ such that $V_i\cdot X\leqslant d_i$ for all $1\leqslant i\leqslant m$ and such that $C\cdot X\geqslant B$
\end{problem}
\begin{itemize}
	\item This is in P, but the same problems where X is required to be an integer vector is NP-complete
\end{itemize}
\section{Primes/composite}
\begin{problem}[Composite]
\textbf{Instance}: Positive integer K\\
\textbf{Question}: Is K composite?
\end{problem}
\begin{itemize}
	\item Recently proven that it is in P
\end{itemize}
\section{Graph Isomorphism}
\begin{problem}[Graph Isomorphism]
\textbf{Instance}: Two undirected graphs $G=(V_G,E_G)$ and $H=(V_H,E_H)$\\
\textbf{Question}: Are G and H isomorphic, i.e., is there a bijection $f:V_G\rightarrow V_H$ such that $(u,v)\in E_G$ iff $(f(u),f(g))\in E_H$?
\end{problem}
\begin{itemize}
	\item Currently the main candidate for NP-intermediate
\end{itemize}


\end{document}
