\documentclass{article}[18pt]
\ProvidesPackage{format}
%Page setup
\usepackage[utf8]{inputenc}
\usepackage[margin=0.7in]{geometry}
\usepackage{parselines} 
\usepackage[english]{babel}
\usepackage{fancyhdr}
\usepackage{titlesec}
\hyphenpenalty=10000

\pagestyle{fancy}
\fancyhf{}
\rhead{Sam Robbins}
\rfoot{Page \thepage}

%Characters
\usepackage{amsmath}
\usepackage{amssymb}
\usepackage{gensymb}
\newcommand{\R}{\mathbb{R}}

%Diagrams
\usepackage{pgfplots}
\usepackage{graphicx}
\usepackage{tabularx}
\usepackage{relsize}
\pgfplotsset{width=10cm,compat=1.9}
\usepackage{float}

%Length Setting
\titlespacing\section{0pt}{14pt plus 4pt minus 2pt}{0pt plus 2pt minus 2pt}
\newlength\tindent
\setlength{\tindent}{\parindent}
\setlength{\parindent}{0pt}
\renewcommand{\indent}{\hspace*{\tindent}}

%Programming Font
\usepackage{courier}
\usepackage{listings}
\usepackage{pxfonts}

%Lists
\usepackage{enumerate}
\usepackage{enumitem}

% Networks Macro
\usepackage{tikz}


% Commands for files converted using pandoc
\providecommand{\tightlist}{%
	\setlength{\itemsep}{0pt}\setlength{\parskip}{0pt}}
\usepackage{hyperref}

% Get nice commands for floor and ceil
\usepackage{mathtools}
\DeclarePairedDelimiter{\ceil}{\lceil}{\rceil}
\DeclarePairedDelimiter{\floor}{\lfloor}{\rfloor}

% Allow itemize to go up to 20 levels deep (just change the number if you need more you madman)
\usepackage{enumitem}
\setlistdepth{20}
\renewlist{itemize}{itemize}{20}

% initially, use dots for all levels
\setlist[itemize]{label=$\cdot$}

% customize the first 3 levels
\setlist[itemize,1]{label=\textbullet}
\setlist[itemize,2]{label=--}
\setlist[itemize,3]{label=*}

% Definition and Important Stuff
% Important stuff
\usepackage[framemethod=TikZ]{mdframed}

\newcounter{theo}[section]\setcounter{theo}{0}
\renewcommand{\thetheo}{\arabic{section}.\arabic{theo}}
\newenvironment{important}[1][]{%
	\refstepcounter{theo}%
	\ifstrempty{#1}%
	{\mdfsetup{%
			frametitle={%
				\tikz[baseline=(current bounding box.east),outer sep=0pt]
				\node[anchor=east,rectangle,fill=red!50]
				{\strut Important};}}
	}%
	{\mdfsetup{%
			frametitle={%
				\tikz[baseline=(current bounding box.east),outer sep=0pt]
				\node[anchor=east,rectangle,fill=red!50]
				{\strut Important:~#1};}}%
	}%
	\mdfsetup{innertopmargin=10pt,linecolor=red!50,%
		linewidth=2pt,topline=true,%
		frametitleaboveskip=\dimexpr-\ht\strutbox\relax
	}
	\begin{mdframed}[]\relax%
		\centering
		}{\end{mdframed}}



\newcounter{lem}[section]\setcounter{lem}{0}
\renewcommand{\thelem}{\arabic{section}.\arabic{lem}}
\newenvironment{defin}[1][]{%
	\refstepcounter{lem}%
	\ifstrempty{#1}%
	{\mdfsetup{%
			frametitle={%
				\tikz[baseline=(current bounding box.east),outer sep=0pt]
				\node[anchor=east,rectangle,fill=blue!20]
				{\strut Definition};}}
	}%
	{\mdfsetup{%
			frametitle={%
				\tikz[baseline=(current bounding box.east),outer sep=0pt]
				\node[anchor=east,rectangle,fill=blue!20]
				{\strut Definition:~#1};}}%
	}%
	\mdfsetup{innertopmargin=10pt,linecolor=blue!20,%
		linewidth=2pt,topline=true,%
		frametitleaboveskip=\dimexpr-\ht\strutbox\relax
	}
	\begin{mdframed}[]\relax%
		\centering
		}{\end{mdframed}}
\lhead{Theory of Computation - Algorithms and Complexity}


\begin{document}
\begin{center}
\underline{\huge The Complexity Class NP}
\end{center}
\section{Certificates}
Every yes-instance of an NP problem has a short and easily checkable certificate, for example a satisfying assignment for satisfiability
\section{Verifiers}
\begin{definition}[Acceptor]
An acceptor machine V which halts on all inputs is called a verifier for a language $\mathcal{L}$ if
$$\mathcal{L}=\{w| V \text{accepts "w; c" for some string c}\}$$
\end{definition}
\begin{itemize}
	\item The string c is called a certificate (or witness) for w
	\item A verifier is said to be polynomial-time if it is a polynomial-time TM, and there is a polynomial $p(x)$ such that, for any $w\in \mathcal{L}$, there is a certificate c with $|c|\leqslant p(|w|)$
\end{itemize}
\section{The class NP}
\begin{definition}[NP]
	The class of languages that have polynomial-time verifiers is called NP
\end{definition}
\begin{problem}[Composite Number]
\textbf{Instance} - A positive integer k\\
\textbf{Question} - Are there integers $u,v>1$ such that $u\cdot v=k$
\end{problem}
\begin{problem}[Subset Sum]
\textbf{Instance} - A collection of positive integers $S=\{a_1,...,a_k\}$ and a target integer t\\
\textbf{Question} - Is there a subset $T\subseteq S$ such that $\sum_{i\in T}=t$
\end{problem}
\section{Problems (probably) not in NP}
\begin{problem}[No Hamiltonian Cycle]
\textbf{Instance} - A graph G\\
\textbf{Question} - Is it true that G has no Hamiltonian cycle?
\end{problem}
\begin{problem}[Checkers]
\textbf{Instance} - An integer n and a position in checkers on $n\times n$ board\\
\textbf{Question} - Is it a winning position for white?
\end{problem}
\section{Nondeterministic Machines}
We can get an alternative definition of the class NP by considering non-deterministic machines.\\
\\
Recall that if NT is a non-deterministic Turing Machine, then NT(x) denotes the tree of configurations which can be entered with input x, and NT accepts x if there is some accepting path in NT(x).
\begin{definition}[Time Complexity]
The time complexity of a non-deterministic Turing Machine NT is the function $NTime_{NT}$ such that $NTime_{NT}(x)$ is the number of steps in the shortest accepting path $NT(x)$ is there is one, otherwise it is the number of steps in the shortest rejecting path
\end{definition}
\section{Non-Deterministic time complexity}




\end{document}
