\documentclass{article}[18pt]
\ProvidesPackage{format}
%Page setup
\usepackage[utf8]{inputenc}
\usepackage[margin=0.7in]{geometry}
\usepackage{parselines} 
\usepackage[english]{babel}
\usepackage{fancyhdr}
\usepackage{titlesec}
\hyphenpenalty=10000

\pagestyle{fancy}
\fancyhf{}
\rhead{Sam Robbins}
\rfoot{Page \thepage}

%Characters
\usepackage{amsmath}
\usepackage{amssymb}
\usepackage{gensymb}
\newcommand{\R}{\mathbb{R}}

%Diagrams
\usepackage{pgfplots}
\usepackage{graphicx}
\usepackage{tabularx}
\usepackage{relsize}
\pgfplotsset{width=10cm,compat=1.9}
\usepackage{float}

%Length Setting
\titlespacing\section{0pt}{14pt plus 4pt minus 2pt}{0pt plus 2pt minus 2pt}
\newlength\tindent
\setlength{\tindent}{\parindent}
\setlength{\parindent}{0pt}
\renewcommand{\indent}{\hspace*{\tindent}}

%Programming Font
\usepackage{courier}
\usepackage{listings}
\usepackage{pxfonts}

%Lists
\usepackage{enumerate}
\usepackage{enumitem}

% Networks Macro
\usepackage{tikz}


% Commands for files converted using pandoc
\providecommand{\tightlist}{%
	\setlength{\itemsep}{0pt}\setlength{\parskip}{0pt}}
\usepackage{hyperref}

% Get nice commands for floor and ceil
\usepackage{mathtools}
\DeclarePairedDelimiter{\ceil}{\lceil}{\rceil}
\DeclarePairedDelimiter{\floor}{\lfloor}{\rfloor}

% Allow itemize to go up to 20 levels deep (just change the number if you need more you madman)
\usepackage{enumitem}
\setlistdepth{20}
\renewlist{itemize}{itemize}{20}

% initially, use dots for all levels
\setlist[itemize]{label=$\cdot$}

% customize the first 3 levels
\setlist[itemize,1]{label=\textbullet}
\setlist[itemize,2]{label=--}
\setlist[itemize,3]{label=*}

% Definition and Important Stuff
% Important stuff
\usepackage[framemethod=TikZ]{mdframed}

\newcounter{theo}[section]\setcounter{theo}{0}
\renewcommand{\thetheo}{\arabic{section}.\arabic{theo}}
\newenvironment{important}[1][]{%
	\refstepcounter{theo}%
	\ifstrempty{#1}%
	{\mdfsetup{%
			frametitle={%
				\tikz[baseline=(current bounding box.east),outer sep=0pt]
				\node[anchor=east,rectangle,fill=red!50]
				{\strut Important};}}
	}%
	{\mdfsetup{%
			frametitle={%
				\tikz[baseline=(current bounding box.east),outer sep=0pt]
				\node[anchor=east,rectangle,fill=red!50]
				{\strut Important:~#1};}}%
	}%
	\mdfsetup{innertopmargin=10pt,linecolor=red!50,%
		linewidth=2pt,topline=true,%
		frametitleaboveskip=\dimexpr-\ht\strutbox\relax
	}
	\begin{mdframed}[]\relax%
		\centering
		}{\end{mdframed}}



\newcounter{lem}[section]\setcounter{lem}{0}
\renewcommand{\thelem}{\arabic{section}.\arabic{lem}}
\newenvironment{defin}[1][]{%
	\refstepcounter{lem}%
	\ifstrempty{#1}%
	{\mdfsetup{%
			frametitle={%
				\tikz[baseline=(current bounding box.east),outer sep=0pt]
				\node[anchor=east,rectangle,fill=blue!20]
				{\strut Definition};}}
	}%
	{\mdfsetup{%
			frametitle={%
				\tikz[baseline=(current bounding box.east),outer sep=0pt]
				\node[anchor=east,rectangle,fill=blue!20]
				{\strut Definition:~#1};}}%
	}%
	\mdfsetup{innertopmargin=10pt,linecolor=blue!20,%
		linewidth=2pt,topline=true,%
		frametitleaboveskip=\dimexpr-\ht\strutbox\relax
	}
	\begin{mdframed}[]\relax%
		\centering
		}{\end{mdframed}}
\lhead{Theory of Computation - Algorithms and Complexity}
\newtheorem{theorem}{Theorem}

\begin{document}
\begin{center}
\underline{\huge Coping with Intractability}
\end{center}
If an optimisation problem is NP-hard we generally regard it as intractable\\
\\
We traditionally cope with this in the following ways
\begin{enumerate}
	\item Restrict the input
	\item Use heuristics
	\item Use approximation algorithms
	\item Random, fixed parameter, exact algorithms, average case etc
\end{enumerate}
\section{Restricting the input}
A planar graph is a graph that can be drawn in the plane without edge crossings
\begin{theorem}
	Every planar graph is 4-colourable
\end{theorem}
So 4 colouring is trivial for planar graphs, the answer is always yes
\begin{theorem}
3 colouring is NP complete for planar graphs
\end{theorem}
\begin{theorem}
Every triangle free planar graph is 3 colourable
\end{theorem}
Hence 3 colouring is trivial for triangle free planar graphs
\section{Using Heuristics}
A well know heuristic for colouring is first fit
\begin{itemize}
	\item Order the vertices of an n-vertex graph G as $v_1,...,v_n$
	\item Colour the vertices one by one in that order assigning the smallest available colour
	\item So, $v_i$ gets the smallest colour $x$ not used on $N(v_i)\cap\{v_1,...,v_{i-1}\}$ where $N(v_i)$ denotes the neighbourhood of $v_i$ in $G$
	\item Every tree is bipartite so can eb coloured with at most 2 colours
\end{itemize}
\section{Approximation algorithms}
\begin{itemize}
	\item An algorithm is a k-approximation if it always finds a solution that is a factor of k within the optimum
\end{itemize}
\subsection{Vertex cover}
Here is an approximation algorithm for vertex cover
\begin{lstlisting}[caption=Approx-Vertex-Cover({G=(V,E)})]
C=$\varnothing$
E'=E(G)
while E'$\neq\varnothing$ do
	let (u,v) be an arbitrary edge of E'
	C=C$\cup${u,v}
	remove from E' every edge incident with either u or v
return C
\end{lstlisting}
\begin{theorem}
	The algorithm is a 2 approximation for vertex cover
\end{theorem}
\textbf{Proof}
\begin{itemize}
	\item Let $C=\{u_1,v_1,u_2,v_2,...,u_p,v_p\}$ be the output, where the edges $(u_i,v_i)$ were chosen in executions of step 3, so $|C|=2p$
	\item By construction, $G-C$ has no edges so C is a vertex cover
	\item Let $C^*$ be a minimum vertex cover of G. As $(u_i,v_i)$ is an edge, at least one of $u_i,v_i$ belongs to $C^*$. So $|C^*|\geqslant p$. Hence $|C|=2p\leqslant 2|C^*|$
\end{itemize}


\end{document}