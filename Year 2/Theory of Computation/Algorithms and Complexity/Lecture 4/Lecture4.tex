\documentclass{article}[18pt]
\usepackage{../../../../format}
\lhead{Theory of Computation - Algorithms and Complexity}
\lstset{language=C,
	basicstyle=\ttfamily,
	keywordstyle=\bfseries,
	showstringspaces=false,
	morekeywords={if, else, then, print, end, for, do, while},
	tabsize=4,
	mathescape=true,
	escapechar=£,
	numbers=left,
	stepnumber=1,
}

\usepackage{caption}
\DeclareCaptionFont{white}{\color{white}}
\DeclareCaptionFormat{listing}{\colorbox{gray}{\parbox{\textwidth}{#1#2#3}}}
\captionsetup[lstlisting]{format=listing,labelfont=white,textfont=white}

\begin{document}
\begin{center}
\underline{\huge Matrices and Strassen's Algorithm}
\end{center}
\section{Master Method}
The master method depends on the master theorem\\
\\
Let $T(n)$ be a function on the natural numbers defined by
$$T(n)=aT(n/b)+f(n)$$
(Interpret $n/b$ as either $\lfloor n/b \rfloor$ or $\lceil n/b \rceil$)\\
\\
We will compare $f(n)$ with the function $n^{\log_ba}$\\
\\
Let $T(n)$ be a function on the natural numbers defined by
$$T(n)=aT(n/b)+f(n)$$
for some constant $a\geqslant 1$ and $b>1$
\begin{enumerate}
	\item If $f(n)=\mathcal{O}(n^{\log_ba-\epsilon})$ for some constant $\epsilon>0$, then $T(n)=\Theta(n^{\log_ba})$
	\item If $f(n)=\Theta(n^{\log_ba})$ then $T(n)=\Theta(n^{\log_ba}\log n)$
	\item If $f(n)=\Omega(n^{\log_ba+\epsilon})$ for some constant $\epsilon>0$, and if there exist constant $c<1$ and $n_0$ such that $af(n/b)\leqslant cf(n)$ for all $n\geqslant n_0$, then $T(n)=\Theta(f(n))$
\end{enumerate}
\section{Divide-and-conquer for matrix multiplication}
\subsection{Multiplying two matrices}
We are interested in the following problem:
\begin{itemize}
	\item Input: two square matrices $A,B \in \mathbb{R}^{n\times n}$
	\item Output: their product $C\in \mathbb{R}^{n\times n}$
\end{itemize}
We can do it in time $\Theta(n^3)$ by using the definition
\[
c_{i j}=\sum_{k=1}^{n} a_{i k} b_{k j}
\]
(for simplicity, we assume n is a power of two)\\
$n^2$ entries, computing one entry takes $\Theta(n)$, so $\Theta(n^3)$\\
\\
Partition A, B and C into four parts (the matrix doesn't need to be 2x2, $A_{11}$ etc can be a matrix, doesn't have to be a number):
\[
A=\left(\begin{array}{ll}{A_{11}} & {A_{12}} \\ {A_{21}} & {A_{22}}\end{array}\right), \quad B=\left(\begin{array}{ll}{B_{11}} & {B_{12}} \\ {B_{21}} & {B_{22}}\end{array}\right), \quad C=\left(\begin{array}{ll}{C_{11}} & {C_{12}} \\ {C_{21}} & {C_{22}}\end{array}\right)
\]
Thus
\[
\begin{array}{l}{C_{11}=A_{11} B_{11}+A_{12} B_{21}} \\ {C_{12}=A_{11} B_{12}+A_{12} B_{22}} \\ {C_{21}=A_{21} B_{11}+A_{22} B_{21}} \\ {C_{22}=A_{21} B_{12}+A_{22} B_{22}}\end{array}
\]
\newpage
\subsection{Pseudo-code}
\begin{lstlisting}[caption=R-MULT({A, B})]
let C be a new $n\times n$ matrix
if n=1 then
	$c_{11}\leftarrow a_{11}b+{11}$
else
	partition A,B and C
	$C_{11}\leftarrow$ R-MULT($A_{11},B_{11}$)+R-MULT($A_{12},B_{21}$)
	$C_{12}\leftarrow$ R-MULT($A_{11},B_{12}$)+R-MULT($A_{12},B_{22})$
	$C_{21}\leftarrow$ R-MULT($A_{21},B_{11}$)+R-MULT($A_{22},B_{21})$
	$C_{22}\leftarrow$ R-MULT($A_{21},B_{12}$)+R-MULT($A_{22},B_{22})$
return C
\end{lstlisting}
\subsection{Running time of simple approach}
The running time of the R-MULT satisfies
$$T(n)=8T(n/2)+\Theta(n^2)$$
Since we do 8 multiplications ($8T(n/2)$) and four additions (of time $\Theta(n^2)$ each)
$$f(n)=\Theta(n^2)=\mathcal{O}(n^{3-1})$$
$$=\mathcal{O}(n^{\log_ba-\epsilon})$$
$$T(n)=\Theta(n^{\log_ba})=\Theta(n^3)$$
By the master theorem $T(n)=\Theta(n^3)$
\section{Strassen's algorithm}
Strassen's algorithm uses a sophisticated divide-and-conquer approach. It has four steps:
\begin{enumerate}
	\item Divide A, B and C into four parts as before\\
	Running time $\Theta(1)$
	\item Create 10 Matrices $S_1,...,S_{10}\in \mathbb{R}^{n/2\times n/2}$, obtained by sums or differences of $A_{11},...,B_{22}$\\
	Running time: $\Theta(n^2)$
	\item Using all the matrices available, recursively compute seven product matrices $P_1,...,P_7\in \mathbb{R}^{n/2\times n/2}$\\
	Running time $7T(n/2)$
	\item Compute $C_{11},...,C_{22}$ by adding and subtracting combinations of $P_1,...,P_7$\\
	Running time $\Theta(n^2)$
\end{enumerate}
\subsection{Running time of Strassen's}
Thus, the worst-case running time of Strassen's algorithm satisfies
\[
T(n) \leq 7 T(n / 2)+\Theta\left(n^{2}\right)
\]
By the master method we obtain
\[
T(n)=\Theta\left(n^{\log _{2} 7}\right)=\Theta\left(n^{2.8074}\right)
\]
\subsection{Description of $S_1,...,S_{10}$}
\[
\begin{aligned} S_{1} &=B_{12}-B_{22} \\ S_{2} &=A_{11}+A_{12} \\ S_{3} &=A_{21}+A_{22} \\ S_{4} &=A_{21}-B_{11} \\ S_{5} &=A_{11}+A_{22} \\ S_{7} &=A_{12}-A_{22} \\ S_{8} &=B_{21}+B_{22} \\ S_{9} &=A_{11}-A_{21} \\ S_{10} &=B_{11}+B_{12} \end{aligned}
\]
\subsection{Description of $P_1,...,P_7$}
\[
\begin{array}{l}{P_{1}=A_{11} S_{1}=A_{11} B_{12}-A_{11} B_{22}} \\ {P_{2}=S_{2} B_{22}=A_{11} B_{22}+A_{12} B_{22}} \\ {P_{3}=S_{3} B_{11}=A_{21} B_{11}+A_{22} B_{11}} \\ {P_{4}=A_{22} S_{4}=A_{22} B_{21}-A_{22} B_{11}} \\ {P_{5}=S_{5} S_{6}=A_{11} B_{11}+A_{11} B_{22}+A_{22} B_{11}+A_{22} B_{22}} \\ {P_{6}=S_{7} S_{8}=A_{12} B_{21}+A_{12} B_{22}-A_{22} B_{21}-A_{22} B_{22}} \\ {P_{7}=S_{9} S_{10}=A_{11} B_{11}+A_{11} B_{12}-A_{21} B_{11}-A_{21} B_{12}}\end{array}
\]
\subsection{Expressing C}
\[
\begin{array}{l}{C_{11}=P_{5}+P_{4}-P_{2}+P_{6}} \\ {C_{12}=P_{1}+P_{2}} \\ {C_{21}=P_{3}+P_{4}} \\ {C_{22}=P_{5}+P_{1}-P_{3}-P_{7}}\end{array}
\]
\end{document}