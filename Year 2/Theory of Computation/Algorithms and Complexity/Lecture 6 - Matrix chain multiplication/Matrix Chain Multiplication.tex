\documentclass{article}[18pt]
\usepackage{../../../../format}
\lhead{Theory of Computation - Algorithms and Complexity}



\begin{document}
\begin{center}
\underline{\huge Dynamic Programming II  - Matrix chain multiplication}
\end{center}
\section{Matrix chain multiplication problem}
\subsection{Matrix multiplication}
Let A be a $p\times q$ matrix, and let B be a $q'\times r$ matrix, i.e.
\[
A=\left(\begin{array}{cccc}{a_{1,1}} & {a_{1,2}} & {\dots} & {a_{1, q}} \\ {a_{2,1}} & {a_{2,2}} & {\dots} & {a_{2, q}} \\ {\vdots} & {\vdots} & {\dots} & {\vdots} \\ {a_{p, 1}} & {a_{p, 2}} & {\dots} & {a_{p, q}}\end{array}\right), \quad B=\left(\begin{array}{cccc}{b_{1,1}} & {b_{1,2}} & {\dots} & {b_{1, r}} \\ {b_{2,1}} & {b_{2,2}} & {\dots} & {b_{2, r}} \\ {\vdots} & {\vdots} & {\dots} & {\vdots} \\ {b_{q^{\prime}, 1}} & {b_{q^{\prime}, 2}} & {\dots} & {b_{q^{\prime}, r}}\end{array}\right)
\]
The if $q=q'$ their product $C=AB$ is the $p\times r$ matrix where
\[
c_{i, j}=\sum_{k=1}^{q} a_{i, k} b_{k, j}
\]
If the number of columns of A is not equal to the number of rows of B, the product is not defined
\subsection{Matrix multiplication algorithm}
Input: a $p\times q$ matrix A ad a $q'\times r$ matrix B\\
Output: the product $C=AB$ if $q=q'$
\begin{lstlisting}[caption=Multiply({A,B})]
if $q\neq q'$ then
	return ERROR: incompatible dimensions
else
	Let C be a new $p\times r$ matrix
	for $i \leftarrow$ 1 to p do
		for $j \leftarrow 1$ to r do
			$c_{ij}\leqslant 0$
			for $k\leftarrow 1$ to q do
				$c_{ij}=c_{ij}=a_{ik}b_{kj}$
	return C
\end{lstlisting}
Number of scalar multiplications: pqr
\subsection{Parenthesizing matrix multiplications}
Matrix multiplication is associative:
\[
A_{1} A_{2} A_{3}=\left(A_{1}\left(A_{2} A_{3}\right)\right)=\left(\left(A_{1} A_{2}\right) A_{3}\right)
\]
Let $n=3$ and $\langle A_1, A_2, A_3 \rangle$ have the following dimensions:
\[
A_{1}: 10 \times 100, \quad A_{2}: 100 \times 5, \quad A_{3}: 5 \times 50
\]
There are two ways of parenthesizing:
\begin{enumerate}
	\item $((A_1A_2)A_3)$ requires
\[
(10 \times 100 \times 5)+(10 \times 5 \times 50)=7,500 \text { multiplications. }
\]
	\item $(A_1(A_2A_3))$ requires
\[
(100 \times 5 \times 50)+(10 \times 100 \times 50)=75,000 \text { multiplications. }
\]
\end{enumerate}
\subsection{Matrix chain multiplication}
We are given a sequence (chain) of $\langle A_1,...,A_n\rangle$ of n compatible matrices to be multiplied: we wish to compute
$$A_1A_2...A_n$$
$A_i$ is a $p_{i-1}\times p_{i}$ matrix\\
\\
\textbf{Problem} Where should we place the parentheses so as to minimise the number of operations?\\
\\
Note: We are not actually computing the product here!
\subsection{How many ways to parenthesize?}
Let $P(n)$ denote the number of ways to parenthesize a product of n matrices for $n\geqslant 2$\\
$$P(2) = 1, \qquad P(3) = 2, \qquad P(4)=5, ...$$
We have $P(n)=C_{n-1}$, where $C_n$ is the n-th Catalan number:
\[
C_{n}=\frac{1}{n+1}\left(\begin{array}{c}{2 n} \\ {n}\end{array}\right) \sim \frac{4^{n}}{\sqrt{\pi} n^{3 / 2}}
\]
Hence brute force has an exponential running time
\section{Applying dynamic programming}
To use dynamic programming, we follow a four step sequence:
\begin{enumerate}
	\item Characterise the structure of an optimal solution
	\item Recursively define the value of an optimal solution
	\item Compute the value of an optimal solution
	\item Construct an optimal solution from the computed information
\end{enumerate} 
\subsection{Step 1: The structure of an optimal parenthesization}
Denote $A_{i..j}=A_iA_{i+1} \cdots A_j$ for $i\leqslant j$\\
\\
To parenthesize $A_{i..j}$ we must split the product between $A_k$ and $A_{k+1}$ for some $i\leqslant k< j$\\
\\
\textbf{Optimal substructure}: Suppose that to optimally parenthesize $A_{i..j}$ we split the product between $A_k$ and $A_{k+1}$. Then both parenthesizations of $A_{1..k}$ and $A_{k+1...j}$ must be optimal
\subsection{Step 2: A recursive solution}
Let $m[i,j]$ be the minimum number of multiplications required to compute $A[i..j]$\\
\\
Our goal is to compute $m[1,n]$\\
\\
We have
\[
m[i, j]=\min \left\{m[i, k]+m[k+1, j]+p_{i-1} p_{k} p_{j}: i \leq k<j\right\} \quad \text { for all } i<j
\]
and of course
\[
m[i, i]=0 \quad \text { for all } i
\]
\subsection{Step 3: Computing the optimal costs}
Let us use the bottom-up approach\\
\\
Input: a sequence $p=\langle p_0,p_1,...,p_n\rangle$\\
\\
Output: The minimum number of multiplications required to compute the matrix chain multiplication $A_1\cdots A_n$ where $A_i$ has the dimension $p_{i-1}\times p_i$\\
\\
\textbf{Idea}: Computing $m[i,j]$ for a product of $j-i+1$ matrices only uses values for products of fewer than $j-1+1$ matrices
\begin{lstlisting}[caption=MATRIX-CHAIN-ORDER(p)]
Let m[1..n,1..n] be a new table
for $i\leftarrow 1$ to n do
	$m[i,i]\leftarrow 0$
for $I \leftarrow 2$ to n do
	for $i\leftarrow 1$ to $n_I+1$ do
		$j\leftarrow i+I-1$
		$m[i.j]\leftarrow \infty$
		for $k\leftarrow i$ to $j-1$ do
		 $m[i, j] \leftarrow \min \left\{m[i, k]+m[k+1, j]+p_{i-1} p_{k} p_{j}: i \leq k<j\right\}$
return m
\end{lstlisting}
We can determine the minimum number of multiplications, but we also want to know HOW to parenthesize!\\
\\
Good news: easy to modify MATRIX-CHAIN-ORDER to keep track of where we place the parentheses\\
\\
We use an additional array $s[i.j]$ which keeps track of that k where we cut the product $A_{i..j}$ into $A_{i..k}$ and $A_{k_1...j}$
\begin{lstlisting}[caption=MATRIX-CHAIN-ORDER'(p)]
Let m[1..n,1..n] and s[1..n-1, 2..n ]be new tables
for $i\leftarrow 1$ to n do
	$m[i,i]\leftarrow 0$
for $I \leftarrow 2$ to n do
	for $i\leftarrow 1$ to $n_I+1$ do
		$j\leftarrow i+I-1$
		$m[i.j]\leftarrow \infty$
		for $k= i$ to $j-1$ do
			$q\leftarrow m[i,k] + m[k+1, j]+p_{i-1}p_kp_j$
			if $q<m[i,j]$ then
				$m[i,j]\leftarrow q$
				$s[i,j]\leftarrow k$
return m and s
\end{lstlisting}
\subsubsection{Running time}
Three nested loops: worst case running time of $\mathcal{O}(n^3)$\\
\\
By using the formula
$$\sum_{k=1}^{n}k^2=\dfrac{n(n+1)(2n+1)}{6}$$
one can show that the bound is tight, that is, that the worst-case running time is $\Theta(n^3)$
\subsection{Step 4: Constructing an optimal solution}
The table s[1..n-1,2..n] gives us the information to print out the right parenthesization
\begin{lstlisting}[caption=PRINT-PAR({s,i,j})]
if i==j then
	print $"A"_i$
else
	print "("
	PRINT-PAR(s,i,s[i,j])
	PRINT-PAR(s,s[i,j+1],j)
	print ")"
\end{lstlisting}
Initial call: PRINT-PAR(s,1,n)
\end{document}