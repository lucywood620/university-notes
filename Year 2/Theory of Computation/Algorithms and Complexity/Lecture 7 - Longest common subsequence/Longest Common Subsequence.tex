\documentclass{article}[18pt]
\input{../../../../format}
\lhead{Theory of Computation - Algorithms and Complexity}


\begin{document}
\begin{center}
\underline{\huge Dynamic Programming III - Longest common subsequence}
\end{center}
\section{Longest common subsequence problem}
A strand of DNA can be represented as a string over the finite set \texttt{\{A,C,G,T\}}\\
\\
We want to know how similar two strings of DNA are. Our measure is the length of the longest common subsequence\\
\\
In this problem we allow gaps between letters to form a common subsequence, so\\
A\textbf{BC}BD\textbf{A}B and \textbf{B}D\textbf{CA}BA have a common subsequence of BCA
\subsection{Formal definition}
\begin{defin}[Subsequence]
Given a sequence $X=\langle x_1,...,x_m \rangle$, another sequence $Z=\langle z_1,...,z_k \rangle$ is a subsequence of X if $z_1=x_{i_1},...,z_k=x_{i_k}$ for some $i_1<i_2<...<i_k$
\end{defin}
\begin{defin}[Common subsequence]
A common subsequence of X and Y is a subsequence of both X and Y
\end{defin}
\section{Dynamic programming for longest common subsequence}
\subsection{Step 1: Characterizing a longest common subsequence}
\textbf{Theorem: Optimal substructure}:\\
Let $Z=\langle z_1,...,z_k \rangle$ be an LCS of $X=\langle x_1,...,x_m \rangle$ and $Y=\langle y_1,...,y_n \rangle$
\begin{enumerate}
	\item If $x_m=y_n$, then $z_k=x_m=y_n$ and $Z[1..k-1]$ is an LCS of $X[1..m-1]$ and $Y[1..n-1]$
	\item If $x_m\neq y_n$, then $z_k\neq x_m$ implies that Z is an LCS of $X[1..m-1]$ and $Y$
	\item If $x_m\neq y_n$, then $z_k\neq y_n$ implies that Z is an LCS of $X$ and $Y[1..n-1]$
\end{enumerate}
\subsubsection{Proof}
\begin{enumerate}
	\item If $z_k\neq x_m$ then appending $x_m=y_n$ to Z yields a common subsequence longer than Z. This is a contradiction, this $z_k=x_m=y_n$\\
	\\
	Then $Z[1..k-1]$ is a common subsequence of $X[1..m-1]$ and $Y[1..n-1]$. Suppose there is a longer one, way W. Again appending $x_m$ to W wields a common subsequence of X and Y longer than Z. This is a contradiction, thus $Z[1..k-1]$ is an LCS of $X[1..m-1]$ and $Y[1..n-1]$
	\item If $z_k\neq x_m$, then Z is a common subsequence of $X[1..m-1]$ and Y. Suppose there is a longer one, say W. Then W is also a common subsequence of X and Y but is longer than Z. This is a contradiction, thus Z is an LCS of $X[1..m-1]$ and Y
	\item By symmetry
\end{enumerate}
\subsection{Step 2: A recursive solution}
Let $c[i,j]$ be the length of an LCS of $X[1..i]$ and $Y[1..j]$\\
\\
The theorem then yields
\[
c[i, j]=\left\{\begin{array}{ll}{0} & {\text { if } i=0 \text { or } j=0} \\ {c[i-1, j-1]+1} & {\text { if } i, j>0 \text { and } x_{i}=y_{j}} \\ {\max \{c[i, j-1], c[i-1, j]\}} & {\text { if } i, j>0 \text { and } x_{i} \neq y_{j}}\end{array}\right.
\]
Unlike rod cutting and matrix chain multiplication problems, this time we can readily rule out some subproblems (those where $x_i=y_j$)
\subsection{Step 3: Computing the length of an LCS}
Let us use a bottom up approach\\
Input: $X=\langle x_1,..,x_m \rangle$ and $Y=\langle y_1,...,y_n \rangle$\\
\\
The algorithm stores the values $c[0..m,0..n]$ and also maintains the table $b[1..m,1..n]$ where $b[i,j]$ "points" to the next pair (i,j) to consider while reconstructing the LCS
\section{Algorithm}
\begin{lstlisting}[caption=LCS({X,Y})]
Let b[1..m,1..n] and c[0..m, 0..n] be new tables
for i = 1 to m do
	c[i,0] = 0
for j = 0 to n do
	c[0,j] = 0
for i = 1 to m do
	for i = 1 to n do
		if $x_i==y_i$ then
			c[i,j]=c[i-1,j-1]+1
			b[i,j]="$\nwarrow$"
		else if c[i-1,j]$\geqslant$c[i,j-1] then
			c[i,j] = c[i-1,j]
			b[i,j]="$\uparrow$"
		else
			c[i,j]=c[i,j-1]
			n[i,j]="$\leftarrow$"
return c and b
\end{lstlisting}
\begin{center}
	\includegraphics[scale=0.7]{algorithm}
\end{center}
\newpage
\section{Constructing an LCS}
\begin{lstlisting}[caption=PRINT-LCS({b,X,i,j})]
if i==0 or j==0 then
	return
if b[i,j] == "$\nwarrow$" then
	PRINT-LCS(b,X,i-1,j-1)
	print $x_i$
else if b[i,j] == "$\uparrow$" then
	PRINT-LCS(b,X,i-1,j)
else
	PRINT-LCS(b,X,i,j-1)
\end{lstlisting}
Initial call: PRINT-LCS(b,X,m,n)


\end{document}