\documentclass{article}[18pt]
\input{../../../../format}
\lhead{Theory of Computation - Algorithms and Complexity}


\begin{document}
\begin{center}
\underline{\huge Graph Algorithms - Matchings in Graphs}
\end{center}

\section{Problem Definition}
Let $G=(V,E)$ be an undirected graph\\
\\
A set $M\subseteq E$ is a matching in G if no two edges in M have an end-vertex in common\\
\\
A matching M is maximum if there is no matching of G with more edges than M\\
\\
The matching number $\nu_G$ of G is the size of a maximum matching in G
\begin{defin}[Matching]
Instance: a graph G\\
Task: determine $\nu_G$
\end{defin}
\section{Alternating Paths and Cycles}
Let $G=(V,E)$ be a graph and M be a matching inG\\
\\
A path P is alternating with respect to M iff among every two consecutive edges of P, exactly one edge belongs to M\\
\\
A cycle C is alternating with respect to M iff among every two consecutive edges of C, exactly one edge belongs to M.\\
\\
A vertex u is matched by a matching M is it is the end-vertex of an edge of M; otherwise u is unmatched by M\\
\\
For two sets A and B, the symmetric difference is the set
$$A\varnothing B = (A \backslash B)\cup (B \backslash A)$$
\textbf{Lemma}\\
Let G be a graph with a matching M and an alternating path P with respect to M. If each end-point of P is either unmatched by M or matched by $M\cap P$, then $M \varnothing P$ is another matching
\section{Augmenting Paths} 
An alternating path P with respect to a matching M is augmenting if both endpoints of P are unmatched by M.\\
\\
\textbf{Lemma}\\
Let G be a graph with a matching M and an augmenting path P with respect to M. Then $M \otimes P$ is a matching of size $|M|+1$\\
\\
Hence, M is not maximum if there exists an augmenting path P with respect to M\\
\\
We now prove that the converse holds as well.\\
\\
That is, M is maximum if there exists no augmenting path P with respect to M\\
\\
\textbf{Lemma}\\
Let $G=(V,E)$ be an undirected graph, and let M and M* be matchings in G. Then the subgraph $(V, M\otimes M*)$ is the disjoint union of
\begin{itemize}
	\item Isolated vertices
	\item Alternating cycles with respect to both M and M*
	\item Alternating paths with respect to both M and M*
\end{itemize}
\textbf{Lemma}\\
Let $G=(V,E)$ be an undirected graph, and let M and M* be matchings in G such that $|M*|=|M|+k$ for some $k\geqslant 1$. Then G has at least k pairwise vertex-disjoint augmenting paths with respect to M\\
\\
\textbf{Proof}
The subgraph $G'=(V,M\otimes M*)$ consists of alternating paths and cycles, with respect to M and M* (and isolated vertices)\\
\\
The set $M\otimes M*$ has exactly k more edges from M* than M\\
\\
Hence G' contains k pairwise vertex-disjoint paths that start and end with edges of M*. These paths are augmenting paths of G with respect to M\\
\\
From this lemma we find that a matching M is maximum if there exists no augmenting path P with respect to M.\\
\\
We already proved that M is not maximum if there exists an augmenting path P with respect to M. We conclude:\\
\\
\textbf{Theorem}\\
A matching M in a graph G is maximum iff G has no augmenting path P with respect to M\\
\\
This suggests the following approach for solving matching:
\begin{enumerate}
	\item Let M be the empty set
	\item Check if there is an augmenting path P with respect to M
	\item If so, let $M:= M\otimes P$ and go to step 2. Otherwise output M
\end{enumerate}
\section{Alternating Trees}
Let M be a matching in bipartite graph $G=(R\cup B, E)$. If all red vertices are matched, then M is maximum, Otherwise we try to find an augmenting path P with respect to M (if it exists):
\begin{enumerate}
	\item Start with an unmatched red vertex v
	\item Consider all blue neighbours to v. If one of then is unmatched, STOP (P has been found)
	\item Consider all matched red neighbours of the newly found blue vertices
	\item Consider all new blue neighbours of red vertices from step 3. If one of then is unmatched, STOP (P has been found). If no new blue neighbour is detected, STOP. Otherwise go to strp 3
\end{enumerate}
If the algorithm terminated because no new blue neighbour was found in step 4, then we found an alternating tree T.\\
\\
Consider the (bipartite) graph G-T and repeat the process until an augmenting path P is found or the set of unmatched red vertices has become empty
\section{Correctness}
Suppose no augmenting path (with respect to M) is found in steps 2 or 4\\
\\
So we have found an alternating tree T. By construction, v is not an end-vertex of an augmenting path P\\
\\
We claim that no other vertex in T is used for such a P either:
\begin{enumerate}
	\item Observe that there are no edges from red vertices to (blue) vertices outside T (as our algorithm has terminated)
	\item Any edge from a blue vertex in T to a red vertex outside T cannot be on such a P either, else we can modify P into a new augmenting path P' (with respect to M) that starts in v.
\end{enumerate}
So we can safely remove the vertices of T and consider G-T
\section{Running time}
Since we only explore each edge once, the augmenting path algorithm runs in $\mathcal{O}(|E|)$ time\\
\\
Each occasion results in a matching that has one more edge than the previous matching. Hence we run the augmenting path algorithm at most $\frac{|V|}{2}$ times\\
\\
\textbf{Conclusion}: The total running time of the algorithm for finding a maximum matching in a bipartite graph is $\mathcal{O}(|E||V|)$ time\\
\\
\textbf{General Graphs}: Edmond's algorithm for matching based on the bipartite algorithm, takes $\mathcal{O}(|E||V|^2)$ time
\section{Connections to other problems: vertex cover}
Suppose you have only a small number of cameras that you want to place at crossroads in a city so that every road is "covered"
\begin{defin}[Vertex cover]
A vertex cover of a graph G=(V,E) is a set $S\subseteq V$ so that each edge G has at least one end-vertex in S
\end{defin}
\begin{defin}[Vertex cover number]
The vertex cover number $\tau_G$ of G is the size of a smallest vertex cover of G
\end{defin}
\section{Vertex cover for Bipartite Graphs}
\textbf{Theorem}\\
Let G be a bipartite graph. Then $\tau_G=\nu_G$\\
\\
So we can use the matching algorithm for bipartite graphs to compute $\tau_G$ as well\\
\\
\textbf{Proof}\\
\\
Let M be a matching and C be a vertex cover of G. At least one end-vertex of every edge in M belong to C, so $|M|\leqslant |C|$\\
\\
Let M* be a maximum matching of G. We construct a vertex cover C* with $|M*|=|C*|$, so C* has minimum size\\
\\
As M* has maximum size, the matching algorithm for bipartite graphs constructs a forest F with alternating trees.\\
\\
Let $M_1$ be the set of edges of M in F. Let $M_2=M \backslash M_1$\\
\\
We let C* consist of all blue end-vertices of edges in $M_1$ and all red end-vertices of edges in $M_2$\\
\\
Then $|C*|=|M_1|+|M_2|=|M*|$. Moreover C* is a vertex cover. Otherwise, there is an edge vw where v is red, w is blue and neither of them belongs to C*. This implies that v is in some tree of F and w is not. However, then out algorithm would have discovered w at some point in step 4, a contradiction
\section{Vertex cover for General Graphs}
Instance: a graph G and integer $k\geqslant 0$\\
Task: Does G have a vertex cover of size at most k?
\begin{itemize}
	\item Recall that this problem is computationally hard
	\item Let n and m be the number of vertices and edges of G. Brute force requires checking $\mathcal{O}(n^k)$ possible choices for S
	\item Now assume k is relatively small compared to n. Can we find an algorithm that runs in $f(k)+\mathcal{O}(n+m)$ time?
\end{itemize}
\textbf{Rule 1}. If v is not an end-vertex of an edge, then remove v, so consider (G-v,k)\\
\textbf{Rule 2}. If v has more than k neighbours then put v in the vertex cover and consider (G-v,k-1)\\
\\
At some point, rules 1 and 2 can't be applied any more. Let (G',k') be the resulting instance. Note that we found (G',k') in $\mathcal{O}(n+m)$ time
\begin{itemize}
	\item If G' has more tan $k+k^2$ vertices, then the answer is no. For contradiction, assume S is vertex cover of G' with $|S|\leqslant k' \leqslant k$. By rule 2, there are at most $k|S|\leqslant k^2$ vertices that are not in S but that have a neighbour in S. Hence, there exists a vertex u in G' that has no neighbour S. BY rule 1, u has a neighbour v, so u or v must be in S, a contradiction
	\item Otherwise G' has at most $k+k^2$ vertices. We solve the problem on G' using brute force. This takes $f(k)$ time for some function f that only depends on k
\end{itemize}



\end{document}