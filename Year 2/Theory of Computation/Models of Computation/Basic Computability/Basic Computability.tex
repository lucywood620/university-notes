\documentclass{article}[18pt]
\usepackage{../../../../format}
\lhead{Theory of Computation - Models of Computation}


\begin{document}
\begin{center}
\underline{\huge Basic Computability}
\end{center}
\section{m-reducibility}
\textbf{Definition}. Let A and B be languages over the same alphabet $\Sigma$. A is a many-to-one reducible to B (write $A\leqslant B$) if there is a Turing machine F that terminates on every input $u\in \Sigma^*$, and such that
$$A\{u\in \Sigma^*|F(u)\in B\}$$
Informally: checking $u\in A$ is no harder than checking $w\in B$
\subsection{Properties of m-reducibility}
\textbf{Proposition}. Suppose $A\leqslant B$
\begin{enumerate}
	\item If B is Turing-decidable, so is A
	\item If B is Turing-recognisable, so is A
	\item If $A\leqslant B$ and $B\leqslant C$, then $A\leqslant C$
\end{enumerate}
\textbf{Definition}. Denote $A\equiv B$ to mean that $A\leqslant B$ and $B\leqslant A$\\
\\
Informally: A and B are equally difficult
\section{m-completeness}
\textbf{Definition}. A language A is m-complete if
\begin{enumerate}
	\item A is Turing-recognisable
	\item For every Turing-recognisable language B, $B\leqslant A$
\end{enumerate}
Informally: If A is m-complete then A is as hard as any other Turing-recognisable language\\
\\
\textbf{Corollary} If A is m-complete and $A\leqslant B$, then B is m-complete\\
\\
\textbf{Definition} - The Halting language H consists of the words $\langle M \rangle \circ w$ (over some fixed alphabet) such that the Turing machine M terminates on $w$\\
\\
\textbf{Theorem} H is M complete\\
\\
\textbf{Proof}: Generic reduction. Pick any Turing-recognisable language A. It is recognised by some machine $M_A$. Reduce it to H by mapping any word $w$ onto the word $\langle M_A \rangle \circ w$. It is obvious that the reduction is computable and $w\in A$ iff $\langle M_A \rangle \circ w \in H$\\
\\
\textbf{Definition}: $H_0$ is the "diagonal" of H, i.e. the language $\langle M \rangle \circ \langle M \rangle$ such that M terminates on $\langle M \rangle$\\
\\
\textbf{Theorem}: $H_0$ is m-complete\\
\\
\textbf{Proof}: Reduction from H. Given a word $\langle M \rangle \circ w$, create a Turing machine $N_{M,w}$ that simulates M on w (and note that it ignores the input) - this can be done using a universal Turing machine. Now, $N_{M,w}$ terminates on any input iff M terminates on w. In particular $N_{M,w}$ terminates on $\langle N_{M,w}$ iff M terminates on w
\section{Oracle Turing Machine and t-reducibility}
\textbf{Definition}
\begin{enumerate}
	\item An oracle for a language A is a black-box that takes a word w as an input and instantly (and correctly) replies if $w\in A$
	\item An oracle Turing machine M, denotes by $M^A$ is a Turing machine that has an additional capability of making calls to an oracle for the language A
\end{enumerate}
\textbf{Definition}: A language A is t-reducible to a language B is A is decidable by some oracle Turing machine $M^B$\\
\\
\textbf{Theorem}: If $A\leqslant_t B$ and B is Turing-decidable, then A is Turing-decidable
\section{Computable and Partially Computable Functions}
\textbf{Definition}. A total function $f: \Sigma^*\rightarrow \Sigma^*$ is computable if there is a TM $\mathscr{F}$ such that on any input $x\in \Sigma^*$, $\mathscr{F}$ produces $f(x)$ as the output\\
\\
\textbf{Definition}. A partial function $g:\Sigma^*\rightarrow \Sigma^*$ is partially computable if there is a TM $\mathscr{G}$ such that on any input $x\in \text{dom}(g)$, $\mathscr{G}$ produces $g(x)$ as the output and if $x \notin \text{dom}(g)$, $\mathscr{G}$ doesn't terminate\\
\\
\textbf{Proposition}. A language (set) $S\subseteq \Sigma^*$ is Turing-recognisable iff it is:
\begin{itemize}
	\item The domain of a partially computable function
	\item The range of a computable function
	\item The range of a partially computable function
\end{itemize}
\section{Parameter Theorem}
\textbf{Theorem}. Let $\mathscr{M}(x,y)$ be a TM that expects a two-part input $x\sqcup y$. There is a TM $\mathscr{SMN} (t,x)$ that on inputs $\langle \mathscr{M}\rangle$ and $x$, produces a (description of a) TM $\langle \mathscr{M}_x \rangle$ such that for every y, $\mathscr{M}_x(y)=\mathscr{M}(x,y)$
\section{Recursion theorem}
\textbf{Theorem}. Let $\mathscr{M}(x,y)$ be a TM that expects a two-part input $x\sqcup y$. There is a TM $\mathscr{R}(y)$ such that for every y, $\mathscr{R}(y)=\mathscr{M}(\langle \mathscr{R} \rangle, y)$
\section{Partially Computable Functions w/o Machines}
We consider functions on the set of natural numbers $\mathbb{N}$\\
\\
Definition. The initial functions are
\begin{enumerate}
	\item The successor: $s(x)=x+1$ (returns one more than what you give it)
	\item The zero: $n(x)=0$ (returns 0)
	\item The projections $u_i^n(x_1,x_2,...,x_n)=x_i$ for every $n\in \mathbb{N}$, $1\leqslant i \leqslant n$ (takes n numbers, returns ith one)
\end{enumerate}
\section{Primitive Recursive functions}
\textbf{Definition}. A function is called \textbf{primitive recursive} if it can be obtained from the initial functions by a finite number of applications of composition and primitive recursion (defined below)\\
\\
\textbf{Definition} Let $f$ be a function of $k$ variables and let $g_1,g_2,...,g_k$ be functions of n variables. The function $h$ of $n$ variables if obtained from $f$ and $g_1,g_2,...,g_k$ by composition if
$$h(x_1,x_2,...,x_n)=^{def}f(g_1(x_1,x_2,...,x_n), g_2(x_1,x_2,...x_n),... g_k(x_1,x_2,...,x_n))$$
\textbf{Definition}. Let $f$ and $g$ be total functions of $n$ and $n+1$ variables, respectively. The function $h$ of $n+1$ variables is obtained from $f$ and $g$ by primitive recursion if
$$h(x_1,x_2,..,x_n,0)=^{def} f(x_1,x_2,...,x_n)$$
$$h(x_1,x_2,...,x_n,t+1)=^{def} g(t,h(x_1,x_2,...,x_n,t),x_1,x_2,...,x_n)$$
Addition can be defined as follows:
$$a(x,y)=x+y$$
$$a(x,t+1)=s(a(x,t))$$
Multiplication can be defined as follows:
$$m(x,t+1)=a(m(x,t),x)$$
\section{Gödel Numbers}
Given a sequence of numbers $x_1,x_2,...,x_n$ encode it by a single number\\
\\
Pick the first n prime numbers and raise each to the respective value of x, so the first prime raised to $x_1$ etc, apart from the last one, which is raised to $x_n+1$ and multiply them all together. This will generate the Gödel number of this sequence\\
\\
You can recover the sequence through factorisation of the Gödel number.\\
\\
1 is added to the last exponent as it allows you to know where to stop





\end{document}