\documentclass{article}[18pt]
\input{../../../../format}
\lhead{Theory of Computation - Models of Computation}


\begin{document}
\begin{center}
\underline{\huge Finite-state Automata over infinite words}
\end{center}
A Finite-sate Automaton (FA) consists of
\begin{itemize}
	\item A finite input alphabet $\Sigma$
	\item A finite set of states Q
	\item A transition relation $\Delta\subseteq Q\times \Sigma \times Q$
	\item A start state $q_0\in Q$
	\item A set of accepting states $F\subseteq Q$
\end{itemize}
Then
\begin{enumerate}
	\item If the input is finite, i.e. in $\Sigma^*$, we have a non-deterministic FA (with no $\epsilon$ transitions)
	\item If the input is infinite, i.e. in $\Sigma^\omega$, we have A Buchi Automaton
	\item If $\Delta$ is a partial function $Q\times \Sigma \rightarrow Q$, we have a Deterministic Automaton
\end{enumerate}
\section{Acceptance conditions}

\begin{minipage}{0.4\textwidth}
NFA or DFA accepts a finite word $w_1w_2...w_n\in \Sigma^*$ if there is a sequence of states $r_0,r_1,r_2,...,r_n$ satisfying the following conditions
\begin{enumerate}
	\item $r_0=q_0$
	\item $(r_i,w_{i+1},r_{i+1})\in \Delta$ for every $i,-\leqslant i\leqslant n-1$
	\item $r_n\in F$
\end{enumerate}
\end{minipage}
\begin{minipage}{0.1\textwidth}
\vline
\end{minipage}
\begin{minipage}{0.4\textwidth}
Buchi Automaton accepts $w_1w_2...\in \Sigma^\omega$ if there is a sequence of states $r_0, r_1,r_2,...\in Q^\omega$ satisfying the following conditions
\begin{enumerate}
	\item $r_0=q_0$
	\item $(r_i,w_{i+1},r_{i+1})\in \Delta$ for every $i,i>0$
	\item There are infinitely many $r_i$'s in F
\end{enumerate}
\end{minipage}
\section{Regular Languages}
\textbf{Regular language} - Some DFA/NFA recognises it\\
\\
\textbf{Theorem} - A language is regular iff it could be described by a regular expression\\
\\
A regular language/expression is built upon the basic ones, which are any $s\in \Sigma$, the regular symbol $\epsilon$ or the empty language $\varnothing$, using the following operations (where A and B are regular)
\begin{enumerate}
	\item $A\cup B$, which is the set-theoretic union
	\item $A\circ B$ (or simply AB) which is $\{ab| a\in A, b\in B\}$
	\item A*, which is $\{a_1...a_n|a_i\in A, n\geqslant 0\}$
\end{enumerate}
\section{$\omega$-regular languages}
\begin{definition}[$\omega$ regular language]
An $\omega$-regular language/expression is built upon regular languages, using the following expressions
\begin{enumerate}
	\item $A\cup B$, where both A and B are $\omega$-regular
	\item $AB$, where A is regular and B is $\omega$-regular
	\item $A^\omega$, which is $\{a_1...| a_i\in A\}$, i.e. an infinite sequence of words from A, where A is regular and doesn't contain the empty word
\end{enumerate}
\end{definition}
\textbf{Theorem} - An $\omega$-language is $\omega$-regular iff some non-deterministic Buchi Automaton recognises it
\section{Limits of Regular Languages}
\begin{definition}[Limit of a regular language]
Let A be a regular language. The limit of A limA is the language $\{a\in \Sigma^\omega| \text{a has infinitely many prefixes in A}\}$
\end{definition}

\textbf{Theorem}: An $\omega$-language is a limit of a regular language iff some deterministic Buchi Automaton recognises it





\end{document}