\documentclass{article}[18pt]
\input{../../../../format}
\lhead{Theory of Computation - Models of Computation}

\lstdefinelanguage{Lambda}{%
	morekeywords={%
		if,then,else,fix % keywords go here
	},%
	morekeywords={[2]int},   % types go here
	otherkeywords={:}, % operators go here
	literate={% replace strings with symbols
		{->}{{$\to$}}{2}
		{lambda}{{$\lambda$}}{1}
	},
	basicstyle={\sffamily},
	keywordstyle={\bfseries},
	keywordstyle={[2]\itshape}, % style for types
	keepspaces,
	mathescape,
	numbers=none,
	frame=none
}[keywords,comments,strings]%

\begin{document}
\begin{center}
\underline{\huge Church's $\lambda$-calculus}
\end{center}
\section{Syntax}
Assume we have a countable set of (variable) names, which we shall denote by (possibly indexed) small letters - $a,b,c, ..., x,y,z,a_0,a_1,a_2,...$
\begin{defin}[$\lambda-term$]
\begin{lstlisting}[language=Lambda]
<term> := <name>
	    | ($\lambda$ <name>.<term>)
	    | (<term><term)
\end{lstlisting}
\end{defin}
\textbf{Conventions}
\begin{enumerate}
	\item Function application (3rd line) is left-associative, so $(((A_1A_2)A_3)...A_k)$
	\item Nested abstractions (2nd line) $(\lambda x_1.(\lambda x_2.(...\lambda x_k.A)...))$ can be abbreviated as $\lambda x_1x_2...x_k.A$
\end{enumerate}
\section{Free and Bound Variables}
\textbf{Free variables}:
\begin{enumerate}
	\item $<name>$ is free in $<name>$
	\item $<name>$ is free in $\lambda <name'>.<term>$ if $<name>\neq <name'>$ and $<name>$ is free in $<term>$
	\item $<name>$ is free in $<term'><term''>$ if $<name>$ is free in $<term'>$ or $<name>$ is free in $<term''>$
\end{enumerate}
\textbf{Bound variables}
\begin{enumerate}
	\item $<name>$ is bound in $\lambda<name'>.<term>$ if $<name>=<name'>$ or $<name>$ is bound in $<term>$
	\item $<name>$ is bound in $<term'><term''>$ if $<name>$ is bound in $<term'>$ or $<name>$ is bound in $<term''>$
\end{enumerate}
\section{Reductions}
Denote by $T[x:= R]$ the term in which for every free occurrence of $x$ is replaced by E\\
\\
\textbf{$\alpha$-conversion}: Bound variables can be renamed: $\lambda x.F \equiv \lambda t.F[x:=t]$ provided t is not free in F and t is not bound by $\lambda$ in F whenever it replaces an $x$. Example: $\lambda x.yx(\lambda x.xx)zx \equiv \lambda t.yt(\lambda x.xx)zt.$\\
\\
\textbf{$\beta$-reduction}: Applying a function to the argument $(\lambda x.F)A \equiv F[x:=A]$ provided all free occurrences in A remain free in $F[x:=A]$
\begin{defin}[Normal Form]
A $\lambda$-term is in normal form if no $\beta$ reduction can be applied to it
\end{defin}
\textbf{Theorem} If a $\lambda$-term has a normal form then the formal for is unique (up to renaming of bound variables)\\
\\
\textbf{Computing the normal form}: Keep on replacing the leftmost bound variable by the actual argument until no further reduction is possible. This does not terminate iff the initial term has no normal form.
\section{Church Numerals}
The Church Numerals $C_0,C_1,C_2,...$ are defined as follows
\[
\begin{array}{l}{C_{0} \equiv \lambda s z . z} \\ {C_{1} \equiv \lambda s z . s(z)} \\ {C_{2} \equiv \lambda s z . s(s(z))} \\ {\ldots \quad \ldots}\end{array}
\]
The successor can be defined as
$$S=\lambda uvw.v(uvw)$$
\textbf{Lemma}. For every two terms in F and A, $C_nFA=F^{(n)}A$\\
\textbf{Corollary}. Doing addition in $\lambda$-calculus: $C_nSC_m=C_{n+m}$
\section{Predecessor is hard}
Define true and false by
\[
\begin{array}{l}{T \equiv \lambda x y . x} \\ {F \equiv \lambda x y . y}\end{array}
\]
and represent a pair $(a,b)$ by $\lambda z.zab$ Define
\[
\Phi \equiv \lambda p z . z(S(p T))(p T)
\]
And finally the predecessor is defined as
\[
P \equiv \lambda n . n \Phi\left(\lambda z . z C_{0} C_{0}\right) F
\]






\end{document}