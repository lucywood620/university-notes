\documentclass{article}[18pt]
\input{../../../../format}
\lhead{Theory of Computation - Models of Computation}


\begin{document}
\begin{center}
\underline{\huge Model Checking LTL by Buchi Automata}
\end{center}
\section{Setup and high-level solution}
We are given a Transition System $\tau$ and an LTL $\varphi$, both over the same set of atomic propositions AP. The task is to decide if $\tau\models\varphi$\\
\\
That is, if all runs of $\tau$ satisfy $\varphi$, i.e. $Traces(\tau)\subseteq Words(\varphi)$.\\
 Equivalently
$$Traces(\tau)\cap \big((2^{AP})^\omega \backslash Words(\varphi)\big)=\varnothing$$
This is the same as
$$Traces(\tau)\cap Words(\lnot \phi)=\varnothing$$
So if both the runs of $\tau$ and the models of $\varphi$ are represented by Buchi Automata, we can construct the intersection and check for emptiness. If it is empty, reply that $\tau\models\varphi$, otherwise return $a,b\in (2^{AP})^*$ such that $ab^\omega$ is a run of $\tau$ that falsifies $\varphi$
\section{Difficulties of operations on BA}
\begin{enumerate}
	\item \textbf{Transforming a TS into a Buchi Automaton} - easy - move each label from the state onto all outgoing transitions and introduce a new start state (of the automaton) instead of multiple ones (of the TS)
	\item \textbf{Constructing the intersection of two BA} - easy - take the product of the two BA and making sure that we alternate infinitely often between accepting states from the two
	\item \textbf{Checking a BA for emptiness} - easy - check if there is a non-trivial strongly connected component that contains an accepting state and is reachable from the start state
	\item \textbf{Transforming an LTL formulae into an equivalent BA} - difficult - Translate it into a generalised BA that has multiple sets of accepting states and the acceptance condition is that a state form each set is visited infinitely often. A GBA is however easily transformed into an equivalent BA
	\item \textbf{Negating a BA} - Difficult
\end{enumerate}
\section{LTL to Buchi}
\subsection{States}
A state has two components
\begin{enumerate}
	\item A subset of the AP (that are true; all the other are false) that records the current world, which is the last input seen
	\item A subset of all subformulae of the $\varphi$ that should be true in the future, from this state onward
\end{enumerate}
\begin{itemize}
	\item A state must be propositionally consistent. This takes care of all boolean connectives
\end{itemize}
\subsection{Transitions}
A transition from $s_1$ into $s_2$ labelled by $a\in 2^{AP}$ is added if
\begin{enumerate}
	\item The label $a$ matches the first component of the state $s_2$
	\item The state $s_1$ has the subformula $\circ\psi$ iff the state $s_2$ has the subformula (or AP) $\psi$ - this takes care of the Next operator
	\item Whether the state $s_2$ has the subformula $\varphi_1\cup \varphi_2$
	\begin{enumerate}
		\item The state $s_1$ has $\varphi_2$, or
		\item The state $s_1$ has $\varphi_1$ and the state $s_2$ has $\varphi_1\cup \varphi_2$
	\end{enumerate}
	This partly takes care of the until operator as it can be expanded as follows
	\[
	\varphi_{1} \cup \varphi_{2} \rightarrow \varphi_{2} \vee\left(\varphi_{1} \wedge \bigcirc\left(\varphi_{1} \cup \varphi_{2}\right)\right)
	\]
\end{enumerate}
\subsection{Acceptance and Start}
The expansion \(
\varphi_{1} \cup \varphi_{2} \rightarrow \varphi_{2} \vee\left(\varphi_{1} \wedge \bigcirc\left(\varphi_{1} \cup \varphi_{2}\right)\right)
\)
doesn't guarantee that $\varphi_2$ eventually happens. Thus, any infinite run that always has $\varphi_{1}\cup\varphi_{2}$ but never $\varphi_{2}$ is inconsistent. We can prevent it, though, by insisting that every run has states that have $\varphi_{2}$ or haven't $\varphi_{1}\cup \varphi_{2}$ infinitely often.




\end{document}
