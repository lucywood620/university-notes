\documentclass{article}[18pt]
\ProvidesPackage{format}
%Page setup
\usepackage[utf8]{inputenc}
\usepackage[margin=0.7in]{geometry}
\usepackage{parselines} 
\usepackage[english]{babel}
\usepackage{fancyhdr}
\usepackage{titlesec}
\hyphenpenalty=10000

\pagestyle{fancy}
\fancyhf{}
\rhead{Sam Robbins}
\rfoot{Page \thepage}

%Characters
\usepackage{amsmath}
\usepackage{amssymb}
\usepackage{gensymb}
\newcommand{\R}{\mathbb{R}}

%Diagrams
\usepackage{pgfplots}
\usepackage{graphicx}
\usepackage{tabularx}
\usepackage{relsize}
\pgfplotsset{width=10cm,compat=1.9}
\usepackage{float}

%Length Setting
\titlespacing\section{0pt}{14pt plus 4pt minus 2pt}{0pt plus 2pt minus 2pt}
\newlength\tindent
\setlength{\tindent}{\parindent}
\setlength{\parindent}{0pt}
\renewcommand{\indent}{\hspace*{\tindent}}

%Programming Font
\usepackage{courier}
\usepackage{listings}
\usepackage{pxfonts}

%Lists
\usepackage{enumerate}
\usepackage{enumitem}

% Networks Macro
\usepackage{tikz}


% Commands for files converted using pandoc
\providecommand{\tightlist}{%
	\setlength{\itemsep}{0pt}\setlength{\parskip}{0pt}}
\usepackage{hyperref}

% Get nice commands for floor and ceil
\usepackage{mathtools}
\DeclarePairedDelimiter{\ceil}{\lceil}{\rceil}
\DeclarePairedDelimiter{\floor}{\lfloor}{\rfloor}

% Allow itemize to go up to 20 levels deep (just change the number if you need more you madman)
\usepackage{enumitem}
\setlistdepth{20}
\renewlist{itemize}{itemize}{20}

% initially, use dots for all levels
\setlist[itemize]{label=$\cdot$}

% customize the first 3 levels
\setlist[itemize,1]{label=\textbullet}
\setlist[itemize,2]{label=--}
\setlist[itemize,3]{label=*}

% Definition and Important Stuff
% Important stuff
\usepackage[framemethod=TikZ]{mdframed}

\newcounter{theo}[section]\setcounter{theo}{0}
\renewcommand{\thetheo}{\arabic{section}.\arabic{theo}}
\newenvironment{important}[1][]{%
	\refstepcounter{theo}%
	\ifstrempty{#1}%
	{\mdfsetup{%
			frametitle={%
				\tikz[baseline=(current bounding box.east),outer sep=0pt]
				\node[anchor=east,rectangle,fill=red!50]
				{\strut Important};}}
	}%
	{\mdfsetup{%
			frametitle={%
				\tikz[baseline=(current bounding box.east),outer sep=0pt]
				\node[anchor=east,rectangle,fill=red!50]
				{\strut Important:~#1};}}%
	}%
	\mdfsetup{innertopmargin=10pt,linecolor=red!50,%
		linewidth=2pt,topline=true,%
		frametitleaboveskip=\dimexpr-\ht\strutbox\relax
	}
	\begin{mdframed}[]\relax%
		\centering
		}{\end{mdframed}}



\newcounter{lem}[section]\setcounter{lem}{0}
\renewcommand{\thelem}{\arabic{section}.\arabic{lem}}
\newenvironment{defin}[1][]{%
	\refstepcounter{lem}%
	\ifstrempty{#1}%
	{\mdfsetup{%
			frametitle={%
				\tikz[baseline=(current bounding box.east),outer sep=0pt]
				\node[anchor=east,rectangle,fill=blue!20]
				{\strut Definition};}}
	}%
	{\mdfsetup{%
			frametitle={%
				\tikz[baseline=(current bounding box.east),outer sep=0pt]
				\node[anchor=east,rectangle,fill=blue!20]
				{\strut Definition:~#1};}}%
	}%
	\mdfsetup{innertopmargin=10pt,linecolor=blue!20,%
		linewidth=2pt,topline=true,%
		frametitleaboveskip=\dimexpr-\ht\strutbox\relax
	}
	\begin{mdframed}[]\relax%
		\centering
		}{\end{mdframed}}
\lhead{Theory of Computation - Models of Computation}


\begin{document}
\begin{center}
\underline{\huge Modelling Concurrent Systems}
\end{center}
\section{Peterson's Algorithm for Mutual Exclusion}
Two processes $P_0$ and $P_1$, with shared variables: boolean $flag_i,i\in\{0,1\}$ and $turn\in\{0,1\}$ 
\begin{lstlisting}[caption=Process $P_i$]
Non critical:
$flag_i$=true
turn=1-i
while $flag_{1-i}\land (turn=1-i)$
	do
		wait
end
Critical ection:
$flag_i$=false
\end{lstlisting}
Program Graph
\begin{center}
	\includegraphics[scale=0.7]{"Program Graph"}
\end{center}
\section{Program Graph}
\begin{definition}[Program Graph]
A program graph over a finite set of Boolean variables has
\begin{enumerate}
	\item A finite set of states S called locations
	\item A deterministic transition relation $\rightarrow \subseteq S\times S \times (Act\cup Cond)$ where
	\begin{enumerate}
		\item Act is a set of atomic actions that change the values of some variables - if a transition labelled by an action is taken, the respective variables are updated accordingly
		\item Cond is a set of formulae over the variables - such a transition can be taken only if the respective condition (formula) is true under the current valuation of the variables
	\end{enumerate}
\end{enumerate}
\end{definition}
\section{Interleaved program graphs as a transition system}
Given two program graphs $(S_1,\rightarrow_1)$ and $(S_2,\rightarrow_2)$ over a joint set of Boolean variables Vars, we create a transition system with states $S_1\times S_2\times 2^{vars}$ that simulates a concurrent execution of the two programs.\\
\\
The transition relation of the system is defined as follows. Either
$$(s_1,s_2,v)\rightarrow (S_1',s_2,v_1)$$
Where $s_1\rightarrow_1s_1'$ and $\rightarrow_1$ changes/checks the values $v$ into $v_1$ accordingly or
$$(s_1,s_2,v)\rightarrow (s_1,s_2',v_2)$$
where $s_2\rightarrow_2s_2'$ and $\rightarrow_2$ changes/checks the values $v$ into $v_2$ accordingly. In words - one of the programs makes a step while the other stands still.

\end{document}