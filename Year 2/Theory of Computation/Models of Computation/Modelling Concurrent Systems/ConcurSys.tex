\documentclass{article}[18pt]
\input{../../../../format}
\lhead{Theory of Computation - Models of Computation}


\begin{document}
\begin{center}
\underline{\huge Modelling Concurrent Systems}
\end{center}
\section{Peterson's Algorithm for Mutual Exclusion}
Two processes $P_0$ and $P_1$, with shared variables: boolean $flag_i,i\in\{0,1\}$ and $turn\in\{0,1\}$ 
\begin{lstlisting}[caption=Process $P_i$]
Non critical:
$flag_i$=true
turn=1-i
while $flag_{1-i}\land (turn=1-i)$
	do
		wait
end
Critical ection:
$flag_i$=false
\end{lstlisting}
Program Graph
\begin{center}
	\includegraphics[scale=0.7]{"Program Graph"}
\end{center}
\section{Program Graph}
\begin{definition}[Program Graph]
A program graph over a finite set of Boolean variables has
\begin{enumerate}
	\item A finite set of states S called locations
	\item A deterministic transition relation $\rightarrow \subseteq S\times S \times (Act\cup Cond)$ where
	\begin{enumerate}
		\item Act is a set of atomic actions that change the values of some variables - if a transition labelled by an action is taken, the respective variables are updated accordingly
		\item Cond is a set of formulae over the variables - such a transition can be taken only if the respective condition (formula) is true under the current valuation of the variables
	\end{enumerate}
\end{enumerate}
\end{definition}
\section{Interleaved program graphs as a transition system}
Given two program graphs $(S_1,\rightarrow_1)$ and $(S_2,\rightarrow_2)$ over a joint set of Boolean variables Vars, we create a transition system with states $S_1\times S_2\times 2^{vars}$ that simulates a concurrent execution of the two programs.\\
\\
The transition relation of the system is defined as follows. Either
$$(s_1,s_2,v)\rightarrow (S_1',s_2,v_1)$$
Where $s_1\rightarrow_1s_1'$ and $\rightarrow_1$ changes/checks the values $v$ into $v_1$ accordingly or
$$(s_1,s_2,v)\rightarrow (s_1,s_2',v_2)$$
where $s_2\rightarrow_2s_2'$ and $\rightarrow_2$ changes/checks the values $v$ into $v_2$ accordingly. In words - one of the programs makes a step while the other stands still.

\end{document}